%
%	Plan de Garantía de Calidad (SQA)
%

\documentclass[11pt, a4paper, twoside, titlepage]{article}
\usepackage[utf8x]{inputenc}
\usepackage[T1]{fontenc}
\usepackage[spanish]{babel}
\usepackage{lmodern}
\usepackage{anysize}
\usepackage{fancyhdr}
\usepackage[none]{hyphenat}
\usepackage[colorlinks, linkcolor=red]{hyperref}
\usepackage{glossaries}
\usepackage{glossaries-babel}
\usepackage[doc=plancalidad]{isdiedral}

% Nombre del documento (para futuras referencias)
\newcommand*{\doctitle}{Plan de calidad}


%%% Configuraciones %%%
\marginsize{2.5cm}{2cm}{2cm}{2cm}

% Usa como familia tipográfica por defecto "Sans"
\renewcommand{\familydefault}{\sfdefault}

% Establece la profundidad hasta la cual se numeran los elementos de sección
\setcounter{secnumdepth}{4}

% Establece la profundidad de niveles de sección que aparece en el TOC
\setcounter{tocdepth}{4}

% Fija que la entrada del glosario se comporte como una subsección
\setglossarysection{subsection}

% Configuración de los encabezados
\encabezadodiedral{\doctitle}
\pagestyle{fancy}

\renewcommand*{\thepage}{\sffamily \roman{page}}


% Modelo copiado de los apuntes del tema 8 (páginas 93 a 95) IEEE Std. 730-2002

\title{\doctitle\\\textsl{Airline Common Environment}}
\author{Grupo Diedral}

% Metadatos del pdf
\hypersetup{
pdfinfo={
	Author={Grupo Diedral},
	Title={\doctitle},
	Subject={Airline Common Environment},
	Keywords={SQA;Airline Common Environment;Ingeniería del Software}
}
}

% Inclusión del glosario (gracias a David Peñas)
%
%	Plan SQA: Glosario
%

\PrerenderUnicode{ñ}
\PrerenderUnicode{ó}
\PrerenderUnicode{í}

\newglossaryentry{Gestor_superior}{
	name=Gestor superior del proyecto,
	description={Definen los aspectos de negocios que a menudo tienen una significativa influencia en el proyecto.},
}
\newglossaryentry{Gestor_tecnico}{
	name=Gestor técnico del proyecto,
	description={Deben planificar, organizar y controlar a los profesionales que realizan el trabajo del software.},
}
\newglossaryentry{Profesional}{
	name=Profesional,
	description={Proporcionan las capacidades técnicas para la ingeniería de un producto.},
}
\makeglossaries

\begin{document}
	% Tabla de cambios
	\begin{tablacambios}
		0.0 & 5 de marzo de 2013 & Todos & Iniciados
	\end{tablacambios}

	% Cita inicial
	\fijacitainicial{La calidad de la tela, ya una hilacha la revela}{Refrán popular}

	% Portada
	\portadaace{\doctitle}{2.0}

	\tableofcontents
	\newpage

	\iniciarnumeraciondiedral
		
	\section{Propósito}
		El propósito de este documento es establecer unos planes de control de calidad de software para así poder entender las expectativas del cliente en términos de calidad. Se basa en la determinación y puesta en marcha de las políticas de calidad de la empresa. \\
		%lista de los nombres de los elementos software cubiertos por el plan SQA??
		Este Plan de Calidad del Software, cubre solo la parte correspondiente al desarrollo del software, no cubre la parte del ciclo de vida que corresponde al mantenimiento.
		
	\section{Documentos de referencia}
		\begin{itemize}
			\item \textit{IEEE 730-2002 - Standard for Software Quality Assurance Plans.}
			\item \textit{IEEE 1028-1997 - Standard for Software Reviews.}
			\item \textit{IEEE 1058-1998 - Standard for Software Project Management Plans.}
			\item \textit{IEEE 1012-1998 - Standard for Software Verification and Validation.}
			\item \textit{Pressman, R.S.Ingeniería del Software. Un Enfoque Práctico.}
		\end{itemize}
	\section{Gestión}
		Las tareas desarrolladas en la SQA deberán reflejar los estándares a seguir, los procedimientos correctos que hay que seguir para la elaboración de los productos, los productos a revisar e informar de los fallos encontrados y realizar un seguimiento de los mismos hasta su corrección. Así, las actividades que se van a llevar a cabo son: \\
			\begin{enumerate}
				\item Realización de Revisiones Técnicas Formales (RTF).
				\item Revisión de cada producto.
				\item Asegurar que los fallos de los productos y sus respectivas modificaciones son documentadas.
				\item Revisión de los productos que se ajustan al proceso Software.
			\end{enumerate}
		\subsection{Ciclo de vida del software} %Etapas más importantes del software que cubre el plan
			Las etapas más importantes que cubre este Plan son el análisis, la especificación de los requisitos y la gestión de riesgos. De la calidad de estas depende la intensidad con la que se tendrán que realizar las revisiones de calidad. \\
			Productos que tendrán revisión de calidad: \\
				\begin{itemize}
					\item Casos de Uso
					\item SRS
					\item Plan de Proyecto
						\begin{itemize}
							\item Planificación temporal
							\item Gestión de Riesgos
						\end{itemize}
				\end{itemize}
				
		\subsection{Organización interna del equipo de trabajo}
		
			
			Para una mejor organización y control de la calidad del software, el equipo de desarrollo ha quedado estructurado de la siguiente manera:
			
			\begin{center}
				\begin{tabular}{|l |c |r|}
				\hline
				\textbf{Nombre} & \textbf{Linea de trabajo}\\
				\hline
				Rubén Rafael & \gls{coordinador}, \gls{desarrollador}\\
				\hline
				Juan Andrés & \gls{ing_software} y \gls{analista_sistemas}\\
				\hline
				Sandra & \gls{Gestor_tecnico} y Analista de Sistemas\\
				\hline
				Cristina & \gls{Gestor_superior} y \gls{profesional}\\
				\hline
				Natalia & Ingeniero de Software y Profesional\\
				\hline
				\end{tabular}
			\end{center}
			
	\section{Documentación} %Natalia
		\subsection{Propósito} 
			El propósito de la documentación generada por las revisiones del proyecto es de proporcionar una consistencia y 
			calidad a éste. \\
			El criterio de supervisión será organizar a los analistas y supervisores para que realicen las modificaciones 
			necesarias. Además se elaborarán reuniones para que miembros de otros proyectos junto con el cliente den su punto
			de vista sobre los errores y modificaciones que se deberían de hacer para así elaborar una documentación consistente.\\
			La documentación generada en el proceso de desarrollo será: \\
			Documento de Casos de Uso, SRS, Plan de proyecto (Plan de riesgos, Planificación temporal), Plan de diseño, Plan de pruebas 
			Verifiación y Validación) y la documentación relativa al uso y mantenimiento del software.
			
		\subsection{Requisitos mínimos de documentación} 
			Para asegurarse de que la documentación cumpla los requisitos técnicos especificados, realizaremos las revisiones nombradas
			anteriormente. Esto es necesario dado que la futura implementación también deberá cumplir dichos requisitos, y esta está basada
			en esta documentación generada.\\
			Los criterios principales a seguir para la corrección de la documentación serán: establecer una linea secuencial que sea
			coherente en todos los documentos generados y que no haya por tanto partes contradictorias; comprobar que se van cumpliendo tal
			y como se indican todos los requisitos detallados y que se sigue el modelo de documentación especificado (Estándares, normas de
			documentación...), es decir, que la información es verificable; y por último, la información debe ser completa, esto es, no
			debe quedar ningún elemento sin especificar y debe cumplir todo lo acordado con el cliente.
			
				\subsubsection{Especificación de requisitos del software}
					El documento de los Requisitos del Sistema Software (SRS) deberá describir de forma clara y detallada todos los requisitos
					necesarios del software y las Interfaces Externas. Así el cliente obtendrá una especificación que cubra las necesidades en
					el área de alcance del proyecto y todo lo acordado anteriormente con los desarrolladores.
		
		\subsection{Otra documentación} 
			Otros documentos que influyen directamente en la calidad del software a desarrollar son:
			\begin{itemize}
				\item Plan de desarrollo software.
				\item Plan de proyecto.
					\subitem Planificación temporal.
					\subitem Plan de Gestión de Riesgos.
				\item Estándares y manuales para generar la documentación.
				\item Plan de Gestión de Configuración del Software.
			\end{itemize}
			
	\section{Estándares, prácticas, convenciones y métricas}%Sandra
		\subsection{Propósito}
			Esta sección trata de identificar los estándares, prácticas, convenciones y métricas que se aplicarán para evaluar la calidad del software. Además, se realizará un seguimiento de su cumplimiento, el cual será verificado en el proceso de  \textit{Verificación y Validadación del Software}.
		\subsection{Contenido}
			\begin{itemize}
				\item Estándar de documentación
					Los documentos generados por el equipo de desarrollo deben ser precisos, completos, no muy extensos y claros, de tal forma que cualquier persona ajena al proyecto pueda encontrar la información relevante con facilidad.
Para los distintos documentos se han creado plantillas en {\rmfamily\LaTeX{}}, a partir de las cuales generar los documentos a entregar a nuestro cliente. \\
					Estas plantillas deben incluir: portada (en la que aparecerán el título del documento, la versión, la fecha de entrega, el nombre del proyecto (\textit{Airline Common Environment}) y de los integrantes del equipo de desarrollo y el icono del \textit{Grupo Diedral}). A continuación se deben incluir el control de cambios y el índice del documento. La bibliografía aparecerá al final del mismo. Las páginas estarán numeradas y en el encabezado de cada una debe figurar la sección del documento que se está tratando. En el pie de página aperecerán el nombre del proyecto y del equipo de desarrollo. El formato de texto será común a todos los documentos generados.

				\item Estándar de  codificación %duda, preguntar a Rubén
					

				\item Estándar de verificación y prácticas
					Se utilizarán las prácticas definidas en el estándar \textit{IEEE 1012-1998 - Standard for Software Verification and Validation}.

				\item Métricas elegidas
					\begin{itemize}
						\item Defectos por cada página de documento generado
						\item Defectos por cada 1000 líneas de código
					\end{itemize}
			\end{itemize}			
	\section{Revisiones del software}%natalia
		\subsection{Propósito}
			Detallar las revisiones y auditorías que se realizarán y especificar cómo se van a llevar a cabo.
		\subsection{Requisitos mínimos}
			\subsubsection{Revisiones de Requisitos}
				Antes de pasar a la \textit{Fase de Diseño} del producto software, se realizarán revisiones para asegurar que se
				cumplieron los requisitos establecidos por el Cliente. Estas revisiones servirán para tener una buena base de 
				Requisitos del Sistema y poder diseñar el producto desde una base sólida. Además servirán para poder establecer una línea 
				base entre el Cliente y los Desarrolladores. 
				
			\subsubsection{Revisiones de Gestión}
			Algunas personas ajenas a nuestro proyecto deberán revisar los documentos realizados hasta la fecha. Para ello se ha elaborado 
			la siguiente planificación de reuniones: \\
			
			\begin{center}
				\begin{tabular}{|c |c |c |c |c| c|}
				\hline
				\textbf{Documento} & \textbf{Fecha} &\textbf{Hora inicio} & \textbf{Hora fin}& \textbf{Equipo Revisor} & \textbf{Encargados}\\
				\hline
				Casos de Uso & 8/03/2013 & 13:00 & 13:30 & Nameless & Juan Andrés\\
										 &					 &			 &			 &					& Rubén\\
										 &					 &			 &			 &					& Cristina\\
				\hline
				SRS & 8/03/2013 & 12:30 & 13:00 & Cauchy Team & Natalia\\
				    &						&				&				&							& Sandra\\
				\hline
				Plan de Proyecto & 8/03/2013 & 12:00 & 12:30 & PKT & Juan Andrés\\
												 &					 &			 &			 &		 & Rubén\\
												 &					 &			 &			 &		 & Cristina\\
				\hline
				\end{tabular}
			\end{center}
			
			
		\subsection{Otras revisiones y auditorías}
		Los miembros del grupo encargados de las correcciones de los documentos realizados hasta la fecha tendrán que encargarse
		de comprobar que dichos documentos quedan completos, claros y que son correctos. Además deberán corregir los fallos detectados
		en las reuniones de la Revisión de Gestión. \\
		
			\begin{center}
				\begin{tabular}{|c |c |c |c |}
				\hline
				\textbf{Documento} & \textbf{Fecha inicio} & \textbf{Fecha fin} &\textbf{Revisores encargados}\\
				\hline
				Casos de Uso & 5/03/2013 & 12/03/2013 & Cristina y Juan Andrés\\
				\hline
				SRS & 5/03/2013 & 12/03/2013 & Rubén\\
				\hline
				Plan de gestión de riesgos & 5/03/2013 & 12/03/2013 & Cristina y Juan Andrés\\
				\hline
				Plan de proyecto y Planificación temporal & 5/03/2013 & 12/03/2013 & Rubén\\
				\hline
				\end{tabular}
			\end{center}
		
	\section{Informe de errores y acciones correctoras}%sandra
		A continuación, se muestran los informes sumarios de las revisiones técnicas formales llevadas a cabo el día 8 de marzo de 2013 en la \textit{Facultad de Informática} de la UCM.
		\begin{itemize}
			\item Revisión de casos de uso
				\begin{itemize}
					\item Revisó: Nameless
					\item Errores encontrados
						\begin{itemize}
							\item Notas generales 
								Hay muchos casos de uso con los verbos en tiempos no consistentes y con secuencias alternativas a las que no se llega nunca. Las secuencias han sido marcadas en los casos de uso, en cambio los verbos únicamente en los casos que más llamaban la atención hasta el punto de hacer incómoda la lectura. \\
								También sugerimos en lugar de dejar campos de las tablas en blanco poner no procede, o no hay ninguna salida. Otra sugerencia es poner como precondición una conexión estable con la base de datos, porque ahorraría numerosas secuencias alternativas.
							\item Diagramas de gestión interna 
								Falta la descripción de editar cliente, y muestra que registrar empleado y dar de baja empleado extienden a acceder, cuando no hay ninguna relación entre esos casos de uso. Los casos de uso que aparecen en los diagramas modificar items inventario y realizar entrada de material en las tablas se llaman modificar inventario y registrar entrada de material respectivamente.
							\item Caso de uso 3: Verificar el registro de un empleado 
								No figura en el diagrama, y pensamos que puede ser incluido en la secuencia del caso de uso 2: Registrar empleado, o incluso ser eliminado, porque da más trabajo a los empleados y en la vida real probablemente fuesen aceptados todos los nuevos empleados sistemáticamente sin comprobarse que son nuevos empleados que deben ser registrados.
							\item Casos de uso 4 y 5: Consultar ficha empleado y Modificar ficha de empleado 
								Son unos de los casos de uso en los que la inconsistencia entre los tiempos verbales es más acusada.
							\item Caso de uso 7: Configurar sistema general 
								Permitís hacer cambios en la configuración del sistema, pero no detalláis que cambios.
							\item Caso de uso 8: Establecer organización laboral 
								No se llega nunca a S-1
							\item Caso de uso 9: Acceder horarios 
								No se llega nunca a S-2. Además pensamos que debería estar incluido en el caso de uso 4, si no ser el mismo caso de uso, porque lo lógico es que desde la ficha del empleado se pueda acceder también a sus horarios.
							\item Caso de uso 10: Consultar el plan de vuelo 
								Nunca se alcanza S-3.
							\item Caso de uso 12: Introducir plan de vuelo 
								En S-1 no entendemos por qué motivo no iba a poder ser introducida la información de vuelo, en caso de haber un error al volcar la información en la base de datos se haría la secuencia alternativa S-2.
							\item Caso de uso 18: Consultar ficha cliente 
								Falta en el diagrama.
							\item Caso de uso 19: Modificar inventario 
								En salidas la última palabra, "postexito", no procede, suponemos que es la etiqueta que se encuentra mal ubicada. Asimismo no creemos que en S-1 sea correcto disculparse por las molestias y dar las gracias por el aviso, en 	S-3 no se especifica a que parte de la secuencia normal lleva el reintento, y sugerimos que en lugar de en S-4 que haya que reintroducir los datos en caso de error o esperar para volver a intentar comunicarse con la base de datos que haya una cola de tareas pendientes que lo gestione.
							\item Caso de uso 20: realizar mantenimiento 
								En salidas la última palabra, "postexito", no procede, suponemos que es la etiqueta que se encuentra mal ubicada. En 4 hay un error ortográfico, "el cantidad".
							\item Caso de uso 21: Programar revisión 
								En precondiciones hay un error ortográfico, "de del vehículo", y no se llega nunca a S-2.
							\item Caso de uso 22: Registrar entrada de material 
								No se entiende bien si se refiere a material ya existente o a nuevo material, y pensamos que puede ser redundante con el caso de uso 19. La parte del adhesivo es un problema de hardware, luego no debería figurar aquí.
							\item Caso de uso 24: Programar oferta 
								No especifica que datos hay que introducir para crear una nueva oferta.
							\item Caso de uso 25: Efectuar embarque 
								No se sabe qué ocurre si nunca llega a efectuarse el embarque. Asimismo numerosas compañías aéreas procesan de forma diferente a los clientes que no han embarcado en uno de sus vuelos, así que aunque no sea este caso de uso en concreto pensamos que debería explicarse en este documento.
							\item Caso de uso 26: Ver incidencias del sistema 
								Las secuencias alternativas deben ser concretadas en este documento.
							\item Diagramas de gestión externa 
								No creemos que acceder incluya a registrarse ni que sea extendido por modificar datos de cliente ni por comprar billete. Tampoco pensamos que acceder a una oferta extienda a iniciar pago billetes de vuelo, porque antes habría que comprar el billete. El nombre en la tabla de a iniciar pago billetes de vuelo es a iniciar el pago de los billetes de vuelo.
							\item Caso de uso 29: Editar cliente 
								Creemos que es el que figura en el primer diagrama de gestión interna, en cualquier caso no figura en el de gestión externa.
							\item Caso de uso 34: Comprar billete 
								Permite imprimir la tarjeta de embarque sin haber realizado el pago.
							\item Caso de uso 35: Iniciar el pago de los billetes de vuelo 
								Es uno de los casos de uso en los que la inconsistencia entre los tiempos verbales es más acusada.
							\item Caso de uso 36: realizar pago con tarjeta 
								Parece que se da la opción de imprimir la tarjeta de embarque dos veces.
						\end{itemize}
				\end{itemize}	
			\item Revisión de SRS
				\begin{itemize}
					\item Revisó: Cauchy Team
					\item Errores encontrados 
						\begin{itemize}
							\item Notas generales 
								En líneas generales, hay algún detalle de demasiado bajo nivel (expresiones como “botón”, “array” (ver 3.4)), así como inespecificación de actores en la mayor parte de las funcionalidades (se los llama de forma genérica “usuarios”). Además, a lo largo de la sección 3.2 aparece y desaparece constantemente el campo Prioridad, llegando incluso a que en una funcionalidad se especifica la precondición. Se debe intentar ser más homogéneo. Definir alfabetización informática.
							\item 3.1 Interfaces externas 
								\begin{itemize}
								\item Las faltas ortográficas aparecen escritas sobre la copia impresa que se nos entregó. 
								\item 2.5 (Supuestos y dependencias): Demasiadas vueltas sin llegar a algo concreto. 
								\item 3.1 (Interfaces externos): Cuando dice “… se admitirán caracteres alfabéticos latinos…” definir una codificación concreta o no decir nada. 
								\item 3.1.1.1 (Fecha): Los días y meses no empiezan en 0. La definición formal de año bisiesto es excesiva. 
								\item 3.1.1.3 (NIF): Es innecesario detallar el algoritmo para hallar las letras de la identificación fiscal en España. Con poner únicamente la referencia al RD 1065/2007 y citar los tipos admitidos sería suficiente. 
								\item 3.1.1.4 (Dirección postal): No hay calle en los tipos de vía admitidos. Si la aplicación sólo opera en España, el ZIP es innecesario.
								\end{itemize}
							\item  3.2 Funciones 
								\begin{itemize}
								\item 3.2.1 Gestión interna
									\begin{itemize}
									\item 3.2.1.1 (Acceder): ¿No puede recuperarse la contraseña? 
										Quizá sea mejor, debido a las características de la compañía exigir conexión a la base de datos central y especificarlo en algún lado. ¿Dónde quedan almacenadas las contraseñas?
									\item 3.2.1.2 (Consultar plan de vuelo): La denominación plan de vuelo puede llevar a equívoco, puesto que en realidad lo que se quiere hacer en consultar una ruta aérea, no un plan de vuelo concreto. 
									\item 3.2.1.4 (Establecer organización laboral): Caso redundante. Si lo que se quiere es establecer la jerarquía de la empresa, puede hacerse a medida que se van dando de alta a los trabajadores en el sistema. Si en algún momento debe modificarse el escalafón de algún empleado, la Editar empleado (3.2.1.16) puede encargarse de ello sin mayores problemas. \\
									 2b) Excesiva tecnicidad, 2d) ¿Las horas trabajadas no pueden ser decimales también?, ¿Qué usuarios consultan y qué usuarios modifican? 
									\item 3.2.1.5 (Consultar ficha de clientes): ¿No debería existir una función que sea Modificar ficha de clientes? 
									\item 3.2.1.6 (Configurar el sistema): Demasiado general. Especificar un poco más. Definir qué funcionalidades/herramientas/parámetros son los que pueden modificarse aquí. 
									\item 3.2.1.7 (Dar de baja cliente): La expresión “nuestro sistema” está de más. 2b) Pregunta: ¿Alguna consideración especial por tratarse de datos sensibles?
									\item 3.2.1.8 (Editar información económica): 3e) No puede ordenarse por orden alfabético ¿Por qué? 
									\item 3.2.1.11 (Programar revisión): A la hora de programar el mantenimiento de una aeronave es imposible predecir con exactitud qué piezas van a usarse/cambiarse. Especificar las herramientas quizá sea también excesivo (hay que conocer muy muy muy bien la composición de las partes de una aeronave para saber qué herramientas deben usarse) 
									\item 3.2.1.12 (Realizar mantenimiento): Falta algún campo/formulario/cuadro que indique qué ha pasado con el vehículo que se tiene en mantenimiento. Si se sobrepasa la fechaque se tenía programada y la nave sigue en mantenimiento ¿se alarga la revisión? 
									\item 3.2.1.13 (Registrar empleado): 2c) Basta decir que la dirección postal sigue la especificación anterior. 
									\item 3.2.1.21 (Consultar ficha empleado) ¿En qué se mide Tiempo trabajado en la empresa? (ver figura 4) 
									\item 3.2.1.22 (Dar de baja empleado): ¿Cola de supervisión? ¿Qué es esto? 
									\item 3.2.1.24 (Introducir plan de vuelo): La expresión plan de vuelo puede llevar a confusión. Y el verbo introducir probablemente no sea el más adecuado. 
										 La representación interna del aeropuerto puede hacerse con el código IATA (¡que lo identifica unívocamente!). Además, creo que esto debería estar especificado en algún apartado anterior, no en una funcionalidad. 
									\item 3.2.1.25 (Facturar) y 3.2.1.26 (Efectuar embarque): ¿Qué es número de reserva?
									\end{itemize}
								\item 3.1.2 Gestión externa 
									\begin{itemize}
									\item 3.2.2.1 (Registrarse): 2c) Para verificar el correo, un mail de confirmación a la cuenta indicada por el cliente. 2e) Basta decir que la dirección postal sigue los criterios de la especificación anterior. 
									\item 3.2.2.3 (Ver información de vuelo): Para distinguirlo de 3.2.2.10 (Consultar vuelos), sugiero cambiar el nombre de la funcionalidad a algo similar a Ver información de vuelo comprado. 
									\item 3.2.2.4 (Mostrar ofertas): 4a) Si no pueden mostrarse las ofertas, avisar al cliente indicado específicamente que no están disponibles y que los precios que se muestran son otros (la tarifa normal, precios a los que se les ha aplicado otra oferta…). 
									\item 3.2.2.7 (Realizar pago con tarjeta): El IBAN es para cuentas corrientes bancarias y no guarda relación alguna con las tarjetas de crédito/débito. 
									\item 3.2.2.8 (Presentar reclamación): Presentante como que no suena muy bien. 
									\item 3.2.2.9 (Comprar billete): ¿Y el DNI para imprimirlo en la tarjeta de embarque? ¿No se pide? Si se está comprando el billete del usuario registrado, sería interesante ofrecer una opción Autorrellenar con datos de usuario.
									\end{itemize}
								\end{itemize}
							\item  3.3 Requisitos de rendimiento 
								¿Cuáles son las circunstancias en las que agrupaciones de terminales se comportan como uno solo? Si la aerolínea es de tamaño medio (aerolíneas regionales, tipo AirNostrum de Iberia) conexión simultánea de 1.000 personas en la parte interna de la aplicación puede ocasionar embotellamientos y caídas graves que en sistemas sensibles como los del transporte aéreo no puede permitirse. Además, debe definirse qué es KeepAlive.
						\end{itemize}
				\end{itemize}
			\item Revisión de Plan de Proyecto 
				\begin{itemize}
					\item Revisó: PTK
					\item Errores encontrados 
						\begin{itemize}
							\item Notas generales 
								Revisar erratas y errores en el formato (anotadas a lápiz). Revisar ortografía  (tildes). En el encabezado pone “Especificación de requisitos software”. El control de cambios del Índice está vacío.  
							\item Introducción
								\begin{itemize}
								\item Pág  1:  El  punto  1.2.1  y  el  1.2.2  son  muy  parecidos.  
								\item Pág  2:  En  el  apartado  1.2.4  sería  bueno  hablar  un  poco  de  la  
seguridad  del  software,  ya  que  al  tratarse  de  un  aeropuerto  debe   estar  a  salvo  de  terroristas  etc.  
								\end{itemize}
							\item Estimación del proyecto
								\begin{itemize}
								\item Pág.  2:  Los  datos  históricos  que  se  aportan  pertenecen  a  la  parte  de   riesgos,  no  a  las  estimaciones.   
								\item Pág  4:  La  dirección  la  contáis  como  6  DETs,  pero  no  decís  de dónde sale.  	
								\item Pág  5:  Cuando  habláis  de  FIE,  pone  FTR,  cuando  seía  RET  (se  repite  en  más  sitios).  
								\item Pág  6:  Si  no  hay  EO,  o  se  dice  o  se  quita,  pero  no  lo  dejéis  vacio  (se  repite  en  más  sitios).   
								\item Pág  8:  De  repente,  dejáis  de  decir  la  complejidad,  y  luego  volvéis  a  
ponerla.  
								\item En general: 
								Calculáis  los  factores  de  ajuste  pero  luego  no  ajustáis  los  PF  totales,  ni   siquiera  la  suma  de  los  PFs  totales.  Además,  decís  en  el  apartado  2.2   que  utilizáis  el  COCOMO  II,  pero  no  reflejáis  los  resultados  que  éste   da  (tiempo,  dinero,  esfuerzo),  que  es  el  fin  de  este  apartado.  También   revisad  el  formato,  pues  en  unas  partes  aparece  lo  de  FLI,  FIE  etc.   numerado  y  en  otras  aparece  en  modo  lista.  
								\end{itemize}
							\item Gestión de riesgos
								\begin{itemize}
								\item Pág.  16:  Poco  claro  el  uso  de  iniciales.  Aclarar  a  qué  palabra   corresponde  cada  inicial.  
								\item Pág.  18:  No  está  explicado  a  qué  probabilidad  corresponde  la  letra  
"B".  
								\item Pág.  18-23:  En  cada  riesgo  tratado  hay  un  apartado  titulado  
"Detonantes  y  acción  de  contingencia",  sin  embargo  solo  se  explica  la   acción  de  contingencia.
								\end{itemize}  
							\item Planificación temporal
								Mejorar  el  formato  (disposición  de  tablas,  títulos…).  Si  se  convierte   en  un  apartado  extenso  se  puede  sacar  a  un  nuevo  documento   (también  se  puede  sacar  la  estimación  a  otro  documento).   Pág  29:  Falta  la  tabla  de  recursos.    
							\item Recursos
								\begin{itemize}
								\item Pág.  30:  Describir  qué  tipo  de  licencia  es  la  LPPL.   
								\item Pág.  31:  Describir  qué  tipo  de  licencia  es  la  GPL.  
								\end{itemize}
						\end{itemize}
				\end{itemize}
		\end{itemize}	
	\section{Herramientas, técnicas y metodologías}%natalia
	\section{Control de medios}%sandra
		Los documentos generados durante el proceso de desarrollo están disponibles en \textit{Google Drive}. Para acceder a estos documentos es necesario tener una cuenta de correo de \textit{Google} y contar con el permiso los integrantes del equipo de desarrollo. El propietario de la carpeta (el coordinador del proyecto) permitirá el acceso a los documentos. \\
		En \url{https://drive.google.com/a/ucm.es/?tab=mo#folders/0B-DSueJcTSlPLUVwS0VLdGF1Sms} se pueden consultar, entre otros, la documentación interna, los documentos de Casos de Uso, SRS, Plan de Proyecto, Plan de Gestión y Configuración del Software y Plan de Garantía de Calidad del Software.

	%\section{Control de proveedor}%natalia NO HAY PROVEEDORES
	\section{Colección de registros, mantenimiento y conservación}%sandra





	\section{Formación}%natalia
	Las actividades de formación necesarias que tienen que tener los desarrolladores del proyecto  para satisfacer
		y realizar correctamente el Plan de Calidad son:
		\begin{enumerate}
			\item Comprender el proceso de desarrollo del producto y saber identificar las desviaciones surgidas en la documentación.
			\item Conocimiento sobre la Verificación y Validación de Requisitos.
			\item Comunicación a través de auditorías con el Cliente para conocer a fondo sus necesidades y saber establecer correctamente
			los requisitos.
		\end{enumerate}
		
	\section{Gestión del riesgo}
		Véase \textit{Plan de Proyecto}.
	%\section{Glosario}
		\printglossaries
	\section{Procedimiento de cambio e historial del plan de SQA}%sandra


	\newpage
	\bibliography{plancalidad}
	\bibliographystyle{plain}
\end{document}
