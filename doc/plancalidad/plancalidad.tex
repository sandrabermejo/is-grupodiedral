%
%	Plan de Garantía de Calidad (SQA)
%

\documentclass[11pt, a4paper, twoside, titlepage]{article}
\usepackage[utf8x]{inputenc}
\usepackage[T1]{fontenc}
\usepackage[spanish]{babel}
\usepackage{lmodern}
\usepackage{anysize}
\usepackage{fancyhdr}
\usepackage[none]{hyphenat}
\usepackage[colorlinks, linkcolor=red]{hyperref}
\usepackage{glossaries}
\usepackage{glossaries-babel}
\usepackage[doc=plancalidad]{isdiedral}

% Nombre del documento (para futuras referencias)
\newcommand*{\doctitle}{Plan de calidad}


%%% Configuraciones %%%
\marginsize{2.5cm}{2cm}{2cm}{2cm}

% Usa como familia tipográfica por defecto "Sans"
\renewcommand{\familydefault}{\sfdefault}

% Establece la profundidad hasta la cual se numeran los elementos de sección
\setcounter{secnumdepth}{4}

% Establece la profundidad de niveles de sección que aparece en el TOC
\setcounter{tocdepth}{4}

% Fija que la entrada del glosario se comporte como una subsección
\setglossarysection{subsection}

% Configuración de los encabezados
\encabezadodiedral{\doctitle}
\pagestyle{fancy}

\renewcommand*{\thepage}{\sffamily \roman{page}}


% Modelo copiado de los apuntes del tema 8 (páginas 93 a 95) IEEE Std. 730-2002

\title{\doctitle\\\textsl{Airline Common Environment}}
\author{Grupo Diedral}

% Metadatos del pdf
\hypersetup{
pdfinfo={
	Author={Grupo Diedral},
	Title={\doctitle},
	Subject={Airline Common Environment},
	Keywords={SQA;Airline Common Environment;Ingeniería del Software}
}
}

% Inclusión del glosario (gracias a David Peñas)
%
%	Plan SQA: Glosario
%

\PrerenderUnicode{ñ}
\PrerenderUnicode{ó}
\PrerenderUnicode{í}

\newglossaryentry{Gestor_superior}{
	name=Gestor superior del proyecto,
	description={Definen los aspectos de negocios que a menudo tienen una significativa influencia en el proyecto.},
}
\newglossaryentry{Gestor_tecnico}{
	name=Gestor técnico del proyecto,
	description={Deben planificar, organizar y controlar a los profesionales que realizan el trabajo del software.},
}
\newglossaryentry{Profesional}{
	name=Profesional,
	description={Proporcionan las capacidades técnicas para la ingeniería de un producto.},
}
\makeglossaries

\begin{document}
	% Tabla de cambios
	\begin{tablacambios}
		0.0 & 5 de marzo de 2013 & Todos & Iniciados
	\end{tablacambios}

	% Cita inicial
	\fijacitainicial{La calidad de la tela, ya una hilacha la revela}{Refrán popular}

	% Portada
	\portadaace{\doctitle}{2.0}

	\tableofcontents
	\newpage

	\iniciarnumeraciondiedral
		
	\section{Propósito}	% ¿?
		El propósito de este documento es establecer unos planes de control de calidad de software para así poder entender las expectativas del cliente en términos de calidad. Se basa en la determinación y puesta en marcha de las políticas de calidad de la empresa. \\
		% lista de los nombres de los elementos software cubiertos por el plan SQA??
		Este Plan de Calidad del Software, cubre solo la parte correspondiente al desarrollo del software, no cubre la parte del ciclo de vida que corresponde al mantenimiento.
		
	\section{Documentos de referencia}
		\nocite{IEEE730-2002}
		\nocite{IEEE1028-1997}
		\nocite{IEEE1058-1998}
		\nocite{IEEE1012-1998}
		\nocite{PSMAN}

		Véase la sección {\itshape Referencias} al final del documento.

	\section{Gestión} % ¿?
		Las tareas desarrolladas en la SQA deberán reflejar los estándares a seguir, los procedimientos correctos que hay que seguir para la elaboración de los productos, los productos a revisar e informar de los fallos encontrados y realizar un seguimiento de los mismos hasta su corrección. Así, las actividades que se van a llevar a cabo son: \\

			\begin{enumerate}
				\item Realización de Revisiones Técnicas Formales (RTF).
				\item Revisión de cada producto.
				\item Asegurar que los fallos de los productos y sus respectivas modificaciones son documentadas.
				\item Revisión de los productos que se ajustan al proceso Software.
			\end{enumerate}


		\subsection{Ciclo de vida del software} % Etapas más importantes del software que cubre el plan
			Las etapas más importantes que cubre este plan son el análisis, la especificación de los requisitos y la gestión de riesgos. De la calidad de éstas depende la intensidad con la que se tendrán que realizar las revisiones de calidad. % ¿Qué?
 \\

			Productos que tendrán revisión de calidad: \\

				\begin{itemize}
					\item Casos de Uso
					\item SRS
					\item Plan de Proyecto
						\begin{itemize}
							\item Planificación temporal
							\item Gestión de Riesgos
						\end{itemize}
				\end{itemize}
				
		\subsection{Organización interna del equipo de trabajo}
			Para una mejor organización y control de la calidad del software, el equipo de desarrollo ha quedado estructurado de la siguiente manera:
			
			\begin{center}
				\begin{tabular}{| l | c | r |}
				\hline
				\bfseries Nombre 	& \bfseries Línea de trabajo			\\ \hline
				Rubén Rafael 		& \gls{coordinador} y \gls{desarrollador} 	\\ \hline
				Juan Andrés 		& \gls{ing_software} y \gls{analista_sistemas}	\\ \hline
				Sandra 			& \gls{Gestor_tecnico} y analista de Sistemas	\\ \hline
				Cristina 		& \gls{Gestor_superior} y \gls{profesional}	\\ \hline
				Natalia 		& Ingeniero de Software y Profesional		\\ \hline
				\end{tabular}
			\end{center}
			
	\section{Documentación} % Natalia
		\subsection{Propósito} 
			El propósito de la documentación generada por las revisiones del proyecto es de proporcionar una consistencia y calidad a éste.\\
			El criterio de supervisión será organizar a los analistas y supervisores para que realicen las modificaciones necesarias. Además se elaborarán reuniones para que miembros de otros proyectos junto con el cliente den su punto de vista sobre los errores y modificaciones que se deberían de hacer para así elaborar una documentación consistente.\\

			La documentación generada en el proceso de desarrollo será: \\
			Documento de Casos de Uso, SRS, Plan de proyecto (Plan de riesgos, Planificación temporal), Plan de diseño, Plan de pruebas Verificación y Validación) y la documentación relativa al uso y mantenimiento del \software.
			
		\subsection{Requisitos mínimos de documentación}
			Para asegurarse de que la documentación cumpla los requisitos técnicos especificados, realizaremos las revisiones nombradas
anteriormente. Esto es necesario dado que la futura implementación también deberá cumplir dichos requisitos, y esta está basada en esta documentación generada.\\

			Los criterios principales a seguir para la corrección de la documentación serán: establecer una linea secuencial que sea
coherente en todos los documentos generados y que no haya por tanto partes contradictorias; comprobar que se van cumpliendo tal y como se indican todos los requisitos detallados y que se sigue el modelo de documentación especificado (estándares, normas de documentación...), es decir, que la información es verificable; y por último, la información debe ser completa, esto es, no debe quedar ningún elemento sin especificar y debe cumplir todo lo acordado con el cliente.
			
				\subsubsection{Especificación de requisitos del software}
					El documento de los Requisitos del Sistema {\itshape software} (SRS) deberá describir de forma clara y detallada todos los requisitos necesarios del software y las interfaces externas. Así el cliente obtendrá una especificación que cubra las necesidades en el área de alcance del proyecto y todo lo acordado anteriormente con los desarrolladores.
		
		\subsection{Otra documentación} 
			Otros documentos que influyen directamente en la calidad del software a desarrollar son:

			\begin{itemize}
				\item Plan de desarrollo software.
				\item Plan de proyecto.
					\subitem Planificación temporal.
					\subitem Plan de Gestión de Riesgos.
				\item Estándares y manuales para generar la documentación.
				\item Plan de Gestión de Configuración del Software.
			\end{itemize}
			
	\section{Estándares, prácticas, convenciones y métricas} % Sandra
		\subsection{Propósito}
			Esta sección trata de identificar los estándares, prácticas, convenciones y métricas que se aplicarán para evaluar la calidad del software. Además, se realizará un seguimiento de su cumplimiento, el cual será verificado en el proceso de \textit{Verificación y Validadación del Software}.

		\subsection{Contenido}
			\begin{itemize}
				\item Estándar de documentación\\
					Los documentos generados por el equipo de desarrollo deben ser precisos, completos, no muy extensos y claros, de tal forma que cualquier persona ajena al proyecto pueda encontrar la información relevante con facilidad. % ¿?
Para los distintos documentos se han creado plantillas en {\rmfamily\LaTeX{}}, a partir de las cuales generar los documentos a entregar a nuestro cliente.\\
					Estas plantillas deben incluir: portada (en la que aparecerán el título del documento, la versión, la fecha de entrega, el nombre del proyecto (\textit{Airline Common Environment}) y de los integrantes del equipo de desarrollo y el icono del \textit{Grupo Diedral}). A continuación se deben incluir el control de cambios y el índice del documento. La bibliografía aparecerá al final del mismo. Las páginas estarán numeradas y en el encabezado de cada una debe figurar la sección del documento que se está tratando. En el pie de página aparecerán el nombre del proyecto y del equipo de desarrollo. El formato de texto será común a todos los documentos generados.

				\item Estándar de  codificación %duda, preguntar a Rubén
					El código de los prototipos y las versiones finales se están de acuerdo al estándar C++ 2003. Cuando no ocasione conflicto con el anterior se tomará como referencia el estándar ISO/IEC 14882:2011\footnote{Borrador de trabajo disponible en \url{www.open-std.org/jtc1/sc22/wg21/docs/papers/2012/n3337.pdf}. La versión final es accesible gratuitamente.}.
					
				\item Estándar de verificación y prácticas
					Se utilizarán las prácticas definidas en el estándar \textit{IEEE 1012-1998 - Standard for Software Verification and Validation} \cite{IEEE1012-1998}.

				\item Métricas elegidas
					\begin{itemize}
						\item Defectos por cada página de documento generado
						\item Defectos por cada 1000 líneas de código
					\end{itemize}
			\end{itemize}

	\section{Revisiones del software} % Natalia

		\subsection{Propósito}
			Detallar las revisiones y auditorías que se realizarán y especificar cómo se van a llevar a cabo.

		\subsection{Requisitos mínimos}
			\subsubsection{Revisiones de Requisitos}
				Antes de pasar a la fase de diseño del producto software, se realizarán revisiones para asegurar que se cumplieron los requisitos establecidos por el cliente. Estas revisiones servirán para tener una buena base de Requisitos del Sistema y poder diseñar el producto desde una base sólida. % Demasiadas bases: home run
 Además servirán para poder establecer una línea base entre el cliente y los desarrolladores. 
				
			\subsubsection{Revisiones de Gestión}
			Algunas personas ajenas a nuestro proyecto deberán revisar los documentos realizados hasta la fecha. Para ello se ha elaborado la siguiente planificación de reuniones: \\
			
			\begin{center}
				\begin{tabular}{| c | c | c | c | c | c |}
				\hline
				\bfseries Documento	& \bfseries Fecha & \bfseries Hora inicio & \bfseries Hora fin & \bfseries Equipo Revisor &  \bfseries Encargados \\ \hline
				Casos de Uso 		& 8/03/2013	& 13:00	& 13:30	& Nameless	& Juan Andrés	\\ 								&		&	&	&		& Rubén		\\
							&		&	&	&		& Cristina	\\ \hline
				SRS 			& 8/03/2013	& 12:30	& 13:00 & Cauchy Team 	& Natalia	\\
							&		&	&	&		& Sandra	\\ \hline
				Plan de Proyecto	& 8/03/2013	& 12:00 & 12:30 & PKT		& Juan Andrés	\\
							&		&	&	&		& Rubén		\\
							&		&	&	&		& Cristina	\\ \hline
				\end{tabular}
			\end{center}
			
			
		\subsection{Otras revisiones y auditorías}
		Los miembros del grupo encargados de las correcciones de los documentos realizados hasta la fecha tendrán que encargarse 	de comprobar que dichos documentos quedan completos, claros y que son correctos. Además deberán corregir los fallos detectados en las reuniones de la Revisión de Gestión. \\
		
			\begin{center}
			\begin{tabular}{| c | c | c | c |}
				\hline
				\bfseries Documento 	& \bfseries Fecha inicio & \bfseries Fecha fin & \bfseries Revisores encargados	\\ \hline
				Casos de Uso		& 5/03/2013	& 12/03/2013	& Cristina y Juan Andrés	\\ \hline
				SRS 			& 5/03/2013 	& 12/03/2013 	& Rubén				\\ \hline
				Plan de gestión de riesgos & 5/03/2013	& 12/03/2013 	& Cristina y Juan Andrés	\\ \hline
				Plan de proyecto y planificación temporal & 5/03/2013 	& 12/03/2013 & Rubén		\\ \hline
			\end{tabular}
			\end{center}
		
	\section{Informe de errores y acciones correctoras} % Sandra
		De nuevo vacío.
	
	\section{Herramientas, técnicas y metodologías} % Natalia
	\section{Control de medios} % Sandra
		Los documentos generados durante el proceso de desarrollo están disponibles en \textit{Google Drive}. Para acceder a estos documentos es necesario tener una cuenta de correo de \textit{Google} y contar con el permiso los integrantes del equipo de desarrollo. El propietario de la carpeta (el coordinador del proyecto) permitirá el acceso a los documentos. \\
		Entre otros documentos se puede consultar, la documentación interna, los documentos de Casos de Uso, SRS, Plan de Proyecto, Plan de Gestión y Configuración del Software y Plan de Garantía de Calidad del Software.

	%\section{Control de proveedor} % Natalia: no hay proveedores

	\section{Colección de registros, mantenimiento y conservación} % Sandra

	\section{Formación} % Natalia
	Las actividades de formación necesarias que tienen que tener los desarrolladores del proyecto  para satisfacer y realizar correctamente el Plan de Calidad son:
		\begin{enumerate}
			\item Comprender el proceso de desarrollo del producto y saber identificar las desviaciones surgidas en la documentación.
			\item Conocimiento sobre la Verificación y Validación de Requisitos.
			\item Comunicación a través de auditorías con el Cliente para conocer a fondo sus necesidades y saber establecer correctamente
los requisitos.
		\end{enumerate}
		
	\section{Gestión del riesgo}
		Véase \textit{Plan de Proyecto}.

	\section{Glosario}
		\printglossaries
	\section{Procedimiento de cambio e historial del plan de SQA} % Sandra

	\appendix
	\newpage
	\bibliography{plancalidad}
	\bibliographystyle{plain}
\end{document}
