%
%	Plan de Garantía de Calidad (SQA)
%

\documentclass[11pt, a4paper, twoside, titlepage]{article}
\usepackage[utf8x]{inputenc}
\usepackage[T1]{fontenc}
\usepackage[spanish]{babel}
\usepackage{lmodern}
\usepackage{anysize}
\usepackage{fancyhdr}
\usepackage[none]{hyphenat}
\usepackage[colorlinks, linkcolor=red]{hyperref}
\usepackage{glossaries}
\usepackage{glossaries-babel}
\usepackage{isdiedral}

% Nombre del documento (para futuras referencias)
\newcommand*{\doctitle}{Plan de calidad}


%%% Configuraciones %%%
\marginsize{2.5cm}{2cm}{2cm}{2cm}

% Usa como familia tipográfica por defecto "Sans"
\renewcommand{\familydefault}{\sfdefault}

% Establece la profundidad hasta la cual se numeran los elementos de sección
\setcounter{secnumdepth}{4}

% Establece la profundidad de niveles de sección que aparece en el TOC
\setcounter{tocdepth}{4}

% Fija que la entrada del glosario se comporte como una subsección
\setglossarysection{subsection}

% Configuración de los encabezados
\encabezadodiedral{\doctitle}
\pagestyle{fancy}

\renewcommand*{\thepage}{\sffamily \roman{page}}


% Modelo copiado de los apuntes del tema 8 (páginas 93 a 95) IEEE Std. 730-2002

\title{\doctitle\\\textsl{Airline Common Environment}}
\author{Grupo Diedral}

% Metadatos del pdf
\hypersetup{
pdfinfo={
	Author={Grupo Diedral},
	Title={\doctitle},
	Subject={Airline Common Environment},
	Keywords={SQA;Airline Common Environment;Ingeniería del Software}
}
}

% Inclusión del glosario (gracias a David Peñas)
%
%	Plan SQA: Glosario
%

\PrerenderUnicode{ñ}
\PrerenderUnicode{ó}
\PrerenderUnicode{í}

\newglossaryentry{Gestor_superior}{
	name=Gestor superior del proyecto,
	description={Definen los aspectos de negocios que a menudo tienen una significativa influencia en el proyecto.},
}
\newglossaryentry{Gestor_tecnico}{
	name=Gestor técnico del proyecto,
	description={Deben planificar, organizar y controlar a los profesionales que realizan el trabajo del software.},
}
\newglossaryentry{Profesional}{
	name=Profesional,
	description={Proporcionan las capacidades técnicas para la ingeniería de un producto.},
}
\makeglossaries

\begin{document}
	% Tabla de cambios
	\begin{tablacambios}
		0.0 & 5 de marzo de 2013 & Todos & Iniciados
	\end{tablacambios}

	% Cita inicial
	\fijacitainicial{La calidad de la tela, ya una hilacha la revela}{Refrán popular}

	% Portada
	\portadaace{\doctitle}{2.0}

	\tableofcontents
	\newpage

	\iniciarnumeraciondiedral
		
	\section{Propósito}
		El propósito de este documento es establecer unos planes de control de calidad de software para así poder entender las expectativas del cliente en términos de calidad. Se basa en la determinación y puesta en marcha de las políticas de calidad de la empresa. \\
		%lista de los nombres de lso elementos software cubiertos por el plan SQA??
		Este Plan de Calidad del Software, cubre solo la parte correspondiente al desarrollo del software, no cubre la parte del ciclo de vida que corresponde al mantenimiento.
		
	\section{Documentos de referencia}
		\begin{itemize}
			\item \textit{IEEE 730-2002 - Standard for Software Quality Assurance Plans.}
			\item \textit{IEEE 1028-1997 - Standard for Software Reviews}
			\item \textit{IEEE 1058-1998 - standard for Software Project Management Plans}
			\item \textit{Pressman, R.S.Ingeniería del Software. Un Enfoque Práctico.}
		\end{itemize}
	\section{Gestión}
		Las tareas desarrolladas en la SQA deberán reflejar los estándares a seguir, los procedimientos correctos que hay que seguir para la elaboración de los productos, los productos a revisar e informar de los fallos encontrados y realizar un seguimiento de los mismos hasta su corrección. Así, las actividades que se van a llevar a cabo son: \\
			\begin{enumerate}
				\item Realización de Revisiones Técnicas Formales (RTF).
				\item Revisión de cada producto.
				\item Asegurar que los fallos de los productos y sus respectivas modificaciones son documentadas.
				\item Revisión de los productos que se ajustan al proceso Software.
			\end{enumerate}
		\subsection{Ciclo de vida del software} %Etapas más importantes del software que cubre el plan
			Las etapas más importantes que cubre este Plan son el análisis, la especificación de los requisitos y la gestión de riesgos. De la calidad de estas depende la intensidad con la que se tendrán que realizar las revisiones de calidad. \\
			Productos que tendrán revisión de calidad: \\
				\begin{itemize}
					\item Casos de Uso
					\item SRS
					\item Plan de Proyecto
						\begin{itemize}
							\item Planificación temporal
							\item Gestión de Riesgos
						\end{itemize}
				\end{itemize}
				
		\subsection{Organización interna del equipo de trabajo}
		
			Para una mejor organización y control de la calidad del software, el equipo de desarrollo ha quedado estructurado de la siguiente manera:
			
			\begin{center}
				\begin{tabular}{|l |c |r|}
				\hline
				\textbf{Nombre} & \textbf{Linea de trabajo}\\
				\hline
				Rubén Rafael & Coordinador, Desarrollador y Jefe de Proyecto\\
				\hline
				Juan Andrés & \gls{Gestor_tecnico} y Supervisor\\
				\hline
				Sandra & \gls{Gestor_superior} y Supervisor\\
				\hline
				Cristina & Gestor superior del proyecto y \gls{Profesional}\\
				\hline
				Natalia & Gestor técnico del proyecto y Profesional\\
				\hline
				\end{tabular}
			\end{center}
			
	\section{Documentación} %Natalia
		\subsection{Propósito} 
		\subsection{Requisitos mínimos de documentación} 
		\subsection{Otra documentación} 
	\section{Estándares, prácticas, convenciones y métricas}%Sandra
		\subsection{Propósito}
		\subsection{Contenido}
	\section{Revisiones del software}%natalia
		\subsection{Propósito}
		\subsection{Requisitos mínimos}
		\subsection{Otras revisiones y auditorías}
		\subsection{Pruebas}
	\section{Informe de errores y acciones correctoras}%sandra
	\section{Herramientas, técnicas y metodologías}%natalia
	\section{Control de medios}%sandra
	\section{Control de proveedor}%natalia
	\section{Colección de registros, mantenimiento y conservación}%sandra
	\section{Formación}%natalia
	\section{Gestión del riesgo}
		Véase \textit{Plan de Proyecto}.
	%\section{Glosario}
		\printglossaries
	\section{Procedimiento de cambio e historial del plan de SQA}%sandra


	\newpage
	\bibliography{plancalidad}
	\bibliographystyle{plain}
\end{document}
