%
%	Plan de Garantía de Calidad (SQA)
%

\documentclass[11pt, a4paper, twoside]{report}
\usepackage[utf8x]{inputenc}
\usepackage[T1]{fontenc}
\usepackage[spanish]{babel}
\usepackage{lmodern}
\usepackage{anysize}
\usepackage{fancyhdr}
\usepackage[none]{hyphenat}
\usepackage[colorlinks, linkcolor=red]{hyperref}
\usepackage{glossaries}
\usepackage{glossaries-babel}
\usepackage{isdiedral}

%%% Configuraciones %%%
\marginsize{2.5cm}{2cm}{2cm}{2cm}

% Usa como familia tipográfica por defecto "Sans"
\renewcommand{\familydefault}{\sfdefault}

% Establece la profundidad hasta la cual se numeran los elementos de sección
\setcounter{secnumdepth}{4}

% Establece la profundidad de niveles de sección que aparece en el TOC
\setcounter{tocdepth}{4}

% Para las clases "book" o semejantes desactiva la impresión del número de capítulo
\renewcommand*\thesection{\arabic{section}}

% Cambia el nombre del TOC a "Índice" (pues en "report" se le denomina por defecto "Índice general")
\addto{\captionsspanish}{\renewcommand*{\contentsname}{Índice}}

% Fija que la entrada del glosario se comporte como una subsección
\setglossarysection{subsection}

% Configuración de los encabezados
\encabezadodiedral{Plan de garantía de calidad}
\pagestyle{fancy}

\renewcommand*{\thepage}{\sffamily \roman{page}}


% Modelo copiado de los apuntes del tema 8 (páginas 93 a 95) IEEE Std. 730-2002

\title{Plan de garantía de calidad\\\textsl{Airline Common Environment}}
\author{Grupo Diedral}

% Metadatos del pdf
\hypersetup{
pdfinfo={
	Author={Grupo Diedral},
	Title={Plan de garantía de calidad},
	Subject={Airline Common Environment},
	Keywords={SQA;Airline Common Environment;Ingeniería del Software}
}
}

% Inclusión del glosario (gracias a David Peñas)
%
%	Plan SQA: Glosario
%

\PrerenderUnicode{ñ}
\PrerenderUnicode{ó}
\PrerenderUnicode{í}

\newglossaryentry{Gestor_superior}{
	name=Gestor superior del proyecto,
	description={Definen los aspectos de negocios que a menudo tienen una significativa influencia en el proyecto.},
}
\newglossaryentry{Gestor_tecnico}{
	name=Gestor técnico del proyecto,
	description={Deben planificar, organizar y controlar a los profesionales que realizan el trabajo del software.},
}
\newglossaryentry{Profesional}{
	name=Profesional,
	description={Proporcionan las capacidades técnicas para la ingeniería de un producto.},
}
\makeglossaries

\begin{document}
	% Portada
	\portadaace{Plan de garantía de calidad}

	\tableofcontents
	\newpage
	\thispagestyle{plain}

	% Tabla de cambios
	\begin{scriptsize}
	\begin{tablacambios}

	\end{tablacambios}
	\end{scriptsize}
	\newpage
	\iniciarnumeraciondiedral
		
	\section{Propósito}
	\section{Documentos de referencia}
	\section{Gestión}
	\section{Documentación}
		\subsection{Propósito}
		\subsection{Requisitos mínimos de documentación}
		\subsection{Otra documentación}
	\section{Estándares, prácticas, convenciones y métricas}
		\subsection{Propósito}
		\subsection{Contenido}
	\section{Revisiones del software}
		\subsection{Propósito}
		\subsection{Requisitos mínimos}
		\subsection{Otras revisiones y auditorías}
		\subsection{Pruebas}
	\section{Informe de errores y acciones correctoras}
	\section{Herramientas, técnicas y metodologías}
	\section{Control de medios}
	\section{Control de proveedor}
	\section{Colección de registros, mantenimiento y conservación}
	\section{Formación}
	\section{Gestión del riesgo}
	\section{Glosario}
		\printglossaries
	\section{Procedimiento de cambio e historial del plan de SQA}


	\newpage
	\bibliography{plancalidad}
	\bibliographystyle{plain}
\end{document}
