%
% Plantilla in situ para la tabla de gestión de riesgos
%

\documentclass[11pt,a4paper]{article}
\usepackage[utf8x]{inputenc}
\usepackage[T1]{fontenc}
\usepackage[spanish]{babel}
\usepackage{lmodern}
\usepackage{fancyhdr}
\usepackage{lscape}
\usepackage{colortbl}
\usepackage{tabu}
\usepackage{xcolor}
\usepackage{anysize}

% Tamaño de los márgenes
\marginsize{1cm}{1cm}{1cm}{1cm}

% Margin interno (vertical) de las celdas
\tabulinesep = 1mm

% Estilo de página
\pagestyle{headings}

\begin{document}
	\section{EDT}
	
	% VERSIÓN PARCIAL
	
	\begin{landscape}
	\begin{table} \centering
		
	\begin{tabu} to .9\linewidth {| X[1, l] | X[2, l] | X[2, l] | X[2, l] | X[2, l] | X[2, l] | X[2, l] | X[2, l] |} \hline
		% Encabezado general
		\rowcolor{green!25}
		AE & \multicolumn{3}{c|}{Comunicación con el cliente} & \multicolumn{2}{c|}{Planificación} & \multicolumn{2}{c|}{Análisis de riesgos}\\ \hline
		
		% Encabezado particular
		\rowcolor{green!10}
		Acc. & TUE & Prototipo & SRS & Estimación & Planificación & Valoración & Planificación \\ \hline
		
		% Celdas con contenido
		\rowfont{\itshape} & T: Reunión cliente & T: Proto. Interna & T: Casos de uso & T: Estimación con PF & T: Plan proyecto & T: Id Riesgos & T: Tabla Riesgos \\
		& r: Todos & r: Rubén & r: Cris, Juanan, Natalia & r: Todos & r: Juanan, Rubén & r: Cris, Juanan y Sandra & r: Rubén y Natalia \\ 
		&i: 6/11/12  & i: 8/12/12& i: 26/11/12 & i: 20/02/13 &i: 14/02/13  &i: 14/02/13  &i: 18/02/13\\
		& f: 22/11/12 & f: 20/12/12 & f: 12/12/12 & f: 25/02/13 & f: 28/02/13 & f: 19/02/13 & f: 21/02/13\\
		&v: 0.1 &v: 1.2 &v: 1.0 &v: 2.4 &v: 2.0 &v: 2.1 &v: 2.2\\ \hline
	
		\rowfont{\itshape} &  & T: Proto. Externa & T: Revisión casos de uso &  & T: Gráfico de Gantt & T: Análisis de Riesgo &  \\
		&	 & r: Natalia y Juanan & r: Rubén y Sandra &  & r: Juanan & r: Rubén &  \\ 
		&  & i: 14/12/12 & i: 21/12/12 &  &i: 27/02/13  &i: 19/02/13  & \\
		&  & f: 19/12/12 & f: 04/01/13 &  & f: 27/02/13 & f: 20/02/13 & \\
		& &v: 1.4 &v: 1.6 & &v: 2.5 &v: 2.3 & \\ \hline
		
		\rowfont{\itshape} &  &  & T: Diagramas & & T: Red de tareas &  & \\
		&  &  & r: Cris, Sandra &  & r: Cris &  &  \\ 
		&  & & i: 12/12/12 &  &i: 27/02/13  &  & \\
		&  &  & f: 17/12/12 &  & f: 27/02/13 &  & \\
		& & &v: 1.1 & &v: 2.6 & & \\ \hline
		
		\rowfont{\itshape} &  &  & T: SRS &  & T: Tabla uso de recursos & &  \\
		&  &  & r: Cris, Rubén y Sandra &  & r:Sandra &  &  \\ 
		& & & i: 05/12/12  &  &i: 27/02/13  &  & \\
		&  &  & f: 20/12/12 &  & f: 27/02/13 & & \\
		&  & &v: 1.3 & &v: 2.7 & & \\ \hline
		
		\rowfont{\itshape} &  &  & T: SRS revisión &  & T: EDT & &  \\
		&  &  & r: Juanan, Natalia y Rubén &  & r: Natalia &  &  \\ 
		& & & i: 21/12/12  &  & i: 24/02/13  & &\\
		&  &  & f: 09/01/13 &  & f: 27/02/13 &  & \\
		&  & &v: 1.5 & &v: 2.0.1 & & \\ \hline
		
		\rowfont{\itshape} &  &  &  &  & T: Ensamblación Plan de Proyecto & &  \\
		&  &  &  &  & r: Rubén y Sandra &  &  \\ 
		& & &  &  & i: 27/02/13  & &\\
		&  &  &  &  & f: 28/02/13 &  & \\
		&  & & & &v: 2.8 & & \\ \hline
		
		\rowfont{\itshape} &  &  &  &  & T: Revisión Plan de Proyecto & &  \\
		&  &  &  &  & r: Juanan, Sandra, Cris y Natalia &  &  \\ 
		& & &  &  & i: 01/03/13  & &\\
		&  &  &  &  & f: 09/03/13 &  & \\
		&  & & & &v: 2.9 & & \\ \hline

		% Segunda celda con contenido	
		\rowfont{\itshape}
		& T: Reunión cliente & T: Proto. Interna & T: Casos de uso & T: Estimación con PF & T: Plan proyecto & T: Id Riesgos & T: Tabla Riesgos \\ \hline
	\end{tabu}
	\caption{EDT desglosada}
	\end{table}
	
	\end{landscape}
\end{document}