% Hoja de descripción de riesgo: Pérdida temporal o permanente de alguno de los miembros del equipo

\hojariesgo{Pérdida temporal o permanente de alguno de los miembros del equipo}{
	% Identificador del riesgo
	id=S2,
	% Persona o departamento que ha identificado el riesgo
	identificador=Departamento de personal.,
	% Fecha de identificación del riesgo
	fecha=21 de enero de 2013,
	% Descripción
	descripcion={La pérdida temporal o permanente de algún miembro del equipo podría deberse a diversas causas como enfermedades, viajes de trabajo o vacaciones, etc, que provocarían que, durante un período definido o no de tiempo, el equipo se viera obligado a trabajar sin alguno de sus miembros.},
	% Consecuencia (C->coste, S->calendario, R->rendimiento, Q->calidad)
	consecuencia={S,R},
	% Impacto (C->crítico, S->serio, M->moderado, N->menor)
	impacto=S,
	% Probabilidad (A->alta, M->media, B->baja, %)
	probabilidad=B,
	% Periodo de previsión (C->corto plazo, L->largo plazo)
	periodo=L,
	% Estrategia de prevención
	estrategia={Para las pérdidas que puedan preveerse con antelación, se asignará al miembro del grupo en cuetión algo de trabajo extra antes de su marcha, para que mientras ésta dure el resto del equipo pueda seguir trabajando sin sobrecarga de trabajo. Respecto a las que no se puedan preveer, se intentara que sean las menos posibles o que no afecten en los períodos más criticos del proyecto.},
	% Plan de contingencia ante el acontecimiento
	contingencia={El resto del equipo tendrá que realizar un sobresfuerzo para realizar el trabajo del miembro no disponible en caso de que éste no pueda posponerse, o se dejará su parte del trabajo para su regreso, en caso de que el período de inactividad sea definido.},
	% Grupo de riesgos al que pertenece
	grupo=De proyecto
}