% Hoja de descripción de riesgo: Planificaciones poco realistas

\hojariesgo{Riesgo de ejemplo}{
	% Identificador del riesgo
	id=S1,
	% Persona o departamento que ha identificado el riesgo
	identificador=Departamento de Control Temporal,
	% Fecha de identificación del riesgo
	fecha=19 de febrero de 2013,
	% Descripción
	descripcion={El cliente es muy exigente y nos pondrá objetivos poco realistas, difíciles de llevar a cabo en el tiempo acordado. Además, en muchas ocasiones la planificación estimada del proyecto no será fiel a la realidad debido a la descompensación entre la cantidad de trabajo a realizar y el tiempo disponible por el equipo de desarrollo. Por todo ello, se producirán notables cambios en el calendario, repercutiendo así al coste del proyecto.},
	% Consecuencia (C->coste, S->calendario, R->rendimiento, Q->calidad)
	consecuencia={C, S},
	% Impacto (C->crítico, S->serio, M->moderado, N->menor)
	impacto=S,
	% Probabilidad (A->alta, M->media, B->baja, %)
	probabilidad=A,
	% Periodo de previsión (C->corto plazo, L->largo plazo)
	periodo=L,
	% Estrategia de prevención
	estrategia={Intentar llegar a un acuerdo con el cliente, negociando en la medida de lo posible para poder ajustarnos a la fecha de entrega. En cuanto al equipo de desarrollo, llevaremos a cabo una organización detallada teniendo en cuenta la situacioón de cada integrante del equipo con el objetivo de satisfacer al cliente.},
	% Plan de contingencia ante el acontecimiento
	contingencia={En caso de no cumplir con la planificación propuesta por el cliente, daremos preferencia  a la calidad frente a la cantidad entregada, es decir, si vemos que no nos ajustamos a la fecha de entrega, entregaremos todo lo que podamos con la mayor calidad posible y dejaríamos para la siguiente entrega lo que haya quedado pendiente.},
	% Grupo de riesgos al que pertenece
	grupo=De Proyecto
}
