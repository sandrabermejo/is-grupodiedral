% Hoja de descripción de riesgo: Falta de comunicación con el cliente

\hojariesgo{Falta de comunicación con el cliente}{
	% Identificador del riesgo
	id=N1,
	% Persona o departamento que ha identificado el riesgo
	identificador=Departamento de Ingeniería de Requisitos,
	% Fecha de identificación del riesgo
	fecha=19 de febrero de 2013,
	% Descripción
	descripcion={No hay una gran comunicación entre el cliente y los desarrolladores, lo que conlleva a posibles
	ambiguedades en el desarrollo del proyecto y sobre todo, a que no se traten con exactitud los requisitos que
	el cliente desea obtener. Además, la pérdida de esta vía de comunicación trae consigo peligrosas consecuencias
	que pondrán completamente en riesgo a todo el trabajo que los desarrolladores hayan realizado sobre el proyecto.},
	% Influencia (C->coste, S->calendario, R->rendimiento, Q->calidad)
	influye={C, Q, S, R},
	% Consecuencia (C->crítico, S->serio, M->moderado, N->menor)
	consecuencia=C,
	% Impacto 
	impacto=Intolerable,
	% Probabilidad (A->alta, M->media, B->baja, %)
	probabilidad=M,
	% Periodo de previsión (C->corto plazo, L->largo plazo)
	periodo=L,
	% Estrategia de prevención
	estrategia={Se deberá establecer una vía de comunicación cliente-desarrollador sólida y estable en la que se reflejen
	todos los requisitos software esperados por ambas partes.},
	% Plan de contigenicia ante el acontecimiento
	contingencia={Si este riesgo afectase al proyecto en algún momento, se deberá actuar lo más rápido posible antes de
	seguir con cualquier desarrollo de éste y ponerse en contacto con el cliente por todos los medios posibles. Si por algún
	motivo no pudiese darse dicho contacto, los desarrolladores deberán trabajar con precacución intentando no "`innovar"'
	demasiado hasta que puedan hablar con el cliente y mitigar la inestable situación.},
	% Grupo de riesgos al que pertenece
	grupo=De negocio
}
