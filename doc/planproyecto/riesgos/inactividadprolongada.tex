% Hoja de descripción de riesgo: inactividad prolongada por periodo no lectivo o exámenes

\hojariesgo{Inactividad prolongada por periodo no lectivo o exámenes}{
	% Identificador del riesgo
	id=P9,
	% Persona o departamento que ha identificado el riesgo
	identificador=Departamento de cronología. :),
	% Fecha de identificación del riesgo
	fecha=19 de enero de 2013,
	% Descripción
	descripcion={Habrá un periodo de tiempo en el que el proyecto estará totalmente parado debido a que los desarrolladores no dispondrán de tiempo por épocas no lectivas, periodos de exámenes o por otros trabajos a realizar.},
	% Consecuencia (C->coste, S->calendario, R->rendimiento, Q->calidad)
	consecuencia=S,
	% Impacto (C->crítico, S->serio, M->moderado, N->menor)
	impacto=M,
	% Probabilidad (A->alta, M->media, B->baja, %)
	probabilidad=A,
	% Periodo de previsión (C->corto plazo, L->largo plazo)
	periodo=L,
	% Estrategia de prevención
	estrategia={Realizar una buena planificación temporal previniendo las épocas conocidas en las que el proyecto quedará temporalmente pausado.},
	% Plan de contigenicia ante el acontecimiento
	contingencia={Se deberán ajustar las planificaciones de todos los miembros del equipo, teniendo dos opciones:
o bien, exigir un mayor rendimiento de los desarrolladores para ajustarse a la fecha de entrega intentando cumplir los objetivos anteriomente fijados, o bien, dejar una parte del proyecto sin entregar ajustando así la siguiente planificación para la próxima entrega, asumiendo en esta opción que podemos sufrir de nuevo este riesgo y no finalizar el proyecto en las fechas previstas.},
	% Grupo de riesgos al que pertenece
	grupo=De proyecto
}
