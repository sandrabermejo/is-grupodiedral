% Hoja de descripción de riesgo: Cambios en los requisitos por el cliente.

\hojariesgo{Cambios en los requisitos por el cliente}{
	% Identificador del riesgo
	id=P3,
	% Persona o departamento que ha identificado el riesgo
	identificador=Departamento de captura de requisitos.,
	% Fecha de identificación del riesgo
	fecha=19 de febrero de 2013,
	% Descripción
	descripcion={Carencias e imprecisiones en la documentación causadas por la inexpeiencia y desconocimiento de los miembros del departamento de los estándares y la relevancia de datos.},
	% Influencia (C->coste, S->calendario, R->rendimiento, Q->calidad)
	influye={S,C},
	% Consecuencia (C->crítico, S->serio, M->moderado, N->menor)
	consecuencia=M,
	% Impacto (Tolerable – Bajo – Medio – Alto - Intolerable)
	impacto=Alto,
	% Probabilidad (F->Frecuente,P->Probable, O->Ocasional, R->Remoto, I->Improbable)
	probabilidad=S,
	% Periodo de previsión (C->corto plazo, L->largo plazo)
	periodo=L,
	% Estrategia de prevención
	estrategia={Se mantendrá un canal abierto y fluido de comunicación con el cliente mediante el cual se intentaran resolver las dudas que aparezcan sobre la especificación tanto por el cliente como por el equipo de desarrollo, dejando siempre copia por escrito de lo pactado.},
	% Plan de contigenicia ante el acontecimiento
	contingencia={Previamente a los cambios necesarios bajo petición del cliente, éstos se estudiaran, definiendo sus consecuencias e impacto sobre el proyecto y encontrando la mejor solución para acometer los cambios.},
	% Grupo de riesgos al que pertenece
	grupo=De proyecto
}
