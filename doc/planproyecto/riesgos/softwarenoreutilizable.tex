% Hoja de descripción de riesgo: ejemplo

\hojariesgo{Riesgo de ejemplo}{
	% Identificador del riesgo
	id=A1,
	% Persona o departamento que ha identificado el riesgo
	identificador=Departamento de desarrollo de software.,
	% Fecha de identificación del riesgo
	fecha=19 de febrero de 2013,
	% Descripción
	descripcion={No es posible la reutilización del software en etapas posteriores, lo que supone la generación de
	nuevo software y unas pérdidas del antiguo (pérdidas de costes, rendimiento y tiempo por ese software que no se
	va a poder reutilizar y que queda inservible).},
	% Consecuencia (C->coste, S->calendario, R->rendimiento, Q->calidad)
	consecuencia={C, S, R},
	% Impacto (C->crítico, S->serio, M->moderado, N->menor)
	impacto=N,
	% Probabilidad (A->alta, M->media, B->baja, %)
	probabilidad=M,
	% Periodo de previsión (C->corto plazo, L->largo plazo)
	periodo=L,
	% Estrategia de prevención
	estrategia={Los desarrolladores se ceñirán a los requisitos que pida el cliente para intentar evitar la producción
	de software no reutilizable. Además, los desarrolladores intentarán llegar lo más pronto posible a la línea base con 
	el cliente en el documento de Especificación de Requisitos para que así se puedan centrar en el desarrollo de 
	software que sea necesario y no tener cambios imprevisibles de los que pueda surgir este riesgo.},
	% Plan de contigenicia ante el acontecimiento
	contingencia={Pese a que el software no pueda aprovecharse por completo, se intentará hacer todas las pequeñas
	reutilizaciones que sean posibles en todas las partes de desarrollo que tengan algo de parecido al software anterior. 
	Con esto, se intentará reducir el coste del nuevo software y mejorar el rendimiento de los desarrolladores.},
	% Grupo de riesgos al que pertenece
	grupo=Técnicos
}
