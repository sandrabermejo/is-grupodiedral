% Hoja de descripción de riesgo: conflictos o ambigüedad en la especificación de requisitos

\hojariesgo{Conflictos o ambigüedad en la especificación de requisitos}{
	% Identificador del riesgo
	id=P8,
	% Persona o departamento que ha identificado el riesgo
	identificador=Departamento de Ingeniería de Requisitos,
	% Fecha de identificación del riesgo
	fecha=20 de febrero de 2013,
	% Descripción
	descripcion={El documento de requisitos es un documento extenso (trata una larga lista de funciones y aspectos) e intenso (contiene información relevante y vinculante en cada una de las partes). Por ello es posible, si bien no deseable, que algunos asertos resulten ambiguos o, por posible desorganización de los autores, se incurra en conflictos o contradicciones. Sus efectos son gravemente perjudiciales para el desarrollo del proyecto. En un primer lugar la ambigüedad posiblemente de lugar a interpretaciones divergentes por partes de los miembros del equipo, por lo que podrían aparecer incompatibilidades, y por parte del cliente, que podría pretender otra interpretación para el requisito mal especificado. El grado de inconveniencia de las inconsistencias no merece mayor comentario.},
	% Influencia (C->coste, S->calendario, R->rendimiento, Q->calidad)
	influye={C, S, Q},
	% Consecuencia
	consecuencia=S,
	% Impacto
	impacto=S,
	% Probabilidad (A->alta, M->media, B->baja, %)
	probabilidad=B,
	% Periodo de previsión (C->corto plazo, L->largo plazo)
	periodo=L,
	% Estrategia de prevención
	estrategia={Revisar los requisitos recogidos en la especificación a fin de cerciorarse de su verificabilidad. Elaboración de cuadros de relaciones entre epígrafes y funcionalidades especificadas a fin de prevenir la introducción de inconsistencias. Extraer las especificaciones globales a emplazamientos comunes para evitar duplicidades y problemas derivados. Revisar --en virtud de lo anterior-- el trabajo de los otros miembros del equipo y puesta en común de los aspectos conflictivos como consecuencia de la distribución del trabajo.},
	% Plan de contingencia ante el acontecimiento
	contingencia={En el caso de incompatibilidades o contradicciones, será necesario resolverlas a costa del calendario y el coste. Los conflictos asociados a las diferentes interpretaciones del cliente y los desarrolladores serán en parte responsabilidad del cliente, por lo que habrá que llegar a un acuerdo.},
	% Grupo de riesgos al que pertenece
	grupo=Calendario
}
