% Hoja de descripción de riesgo: Qt desfasado
% TODO: Permitir notas al pie
\hojariesgo{Obsolescencia Qt}{
	% Identificador del riesgo
	id=3,
	% Persona o departamento que ha identificado el riesgo
	identificador=Departamento de Desarrollo Tecnológico,
	% Fecha de identificación del riesgo
	fecha=1 de enero de 2013,
	% Descripción
	descripcion={La versión de la biblioteca gráfica sobre la que se pretende desarrollar la aplicación (4.8.4) ha sido remplazada por una nueva versión (5.0). Es posible que en el momento de la entrega la nueva versión ya esté implantada y la versión empleada haya quedado obsoleta. La documentación de la biblioteca indica que los programas codificados con la versión escogida pueden ser casi directamente trasladados a la versión 5 {\scriptsize (\url{http://qt-project.org/doc/qt-5.0/qtdoc/qt5-intro.html})}. La nueva versión dispone de ciertas características mejoradas, muchas de las cuales no son aplicables a este desarrollo. Las que sí son aplicables (tratamiento de señales más seguro, compatibilidad con C++11...) son relativamente poco relevantes.},
	% Consecuencia (C->coste, S->calendario, R->rendimiento, Q->calidad)
	consecuencia=C,
	% Impacto (C->crítico, S->serio, M->moderado, N->menor)
	impacto=M,
	% Probabilidad (A->alta, M->media, B->baja, %)
	probabilidad=M,
	% Periodo de previsión (C->corto plazo, L->largo plazo)
	periodo=C,
	% Estrategia de prevención
	estrategia={Vigilar activamente qué parte del código introducido es compatible o mejorable ante un posible cambio a la versión recientemente publicada.},
	% Plan de contigenicia ante el acontecimiento
	contingencia={Si finalmente, a la entrega, la versión empleada quedase desfasada (improbable) se consideran dos opciones: adaptar el desarrollo a la versión 5.0 o realizar la instalación en el cliente con la versión antigua de la biblioteca. {\scriptsize Véase \url{http://qt-project.org/doc/qt-5.0/qtdoc/portingguide.html}}	.},
	% Grupo de riesgos al que pertenece
	grupo=Tecnologías
}
