%% Estimación. Función "Generación de gráficas".

% Tabla de la estimación

\begin{tablapf}{Generación de gráficas}{10}
	FLI	& 0 	& 7 	& 0 	& 10 	& 0 	& 15 	& 0	\\ \hline
	FIE	& 1	& 5 	& 0 	& 7 	& 0 	& 10 	& 5	\\ \hline
	EI	& 0	& 3	& 0	& 4	& 0	& 6	& 0	\\ \hline
	EO	& 0	& 4	& 1	& 5	& 0	& 7	& 5	\\ \hline
	EQ 	& 0	& 3	& 0	& 4	& 0	& 6	& 0
\end{tablapf}


% Detalles de la estimación

\estimacionfunc	% Ficheros Lógicos Internos
{}
% Ficheros de Interfaz Externos
{

	\estfie{Informes dinámicos de datos analizados}{
		\rets{2}{ficheros internos, ficheros externos}
		\dets{12}{clientes, empleados, vuelos, ofertas, nominas, inventarios, económica, horarios, países, ciudades, aeropuertos, organigrama 				laboral}
		% más concreto ¿de qué haría el programa gráficas?
	}{baja}

}
%	Entradas externas
{}
%	Salidas externas
{
	\estse{Informes dinámicos de datos analizados}{
		\dets{2}{ficheros internos, ficheros externos}
		\ftrs{12}{clientes, empleados, vuelos, ofertas, nominas, inventarios, económica, horarios, países, ciudades, aeropuertos, organigrama 				laboral}
	}{media}

	% La salida sería un gráfico
}
%	Consultas externas
{}

% fin
