% Caso de uso: registrar empleado.
% Obs: para escribir comas en el texto del primer parámetro se han de encerrar entre {}.

\casodeuso{
	% Nombre del caso de uso
	nombre=Registrar empleado,
	% Objetivo
	objetivo=Añadir un nuevo usuario a la base de datos del personal de la compañía con unos determinados permisos de acceso.,
	% Entradas
	entradas={Nombre de usuario, datos personales y puesto de trabajo al que se incorpora dentro de la compañía.},
	% Precondiciones
	precondiciones={Que el futuro usuario haya sido contratado por la empresa y que éste haya firmado la LOPD (Ley Orgánica de Protección de Datos), y que el encargado del registro, personal de Recursos Humanos, haya accedido al sistema y a la opción \textit{Registrar empleado}.},
	% Salidas
	salidas={Confirmación de la creación de la cuenta o exposición de los datos incorrectos, según proceda.},
	% Postcondiciones en caso de éxito
	postexito=El usuario queda registrado en la base de datos pero su cuenta no estará activada hasta que sea verificada por el departamento de intervención.,
	% Postcondiciones en caso de error
	posterror=La base de datos no ha sido alterada.,
	% Actores
	actores={El personal administrativo encargado de registrar nuevos empleados, el usuario a registrar y la base de datos.},
}{
	% Tabla de secuencia normal del caso de uso
	\begin{tablasecuencias}
		1 & El usuario registrador inserta los datos del futuro usuario siendo el nombre de usuario el que se asignará a la cuenta de correo de la compañía. Si error S-1.\\
		2 & El sistema genera una contraseña aleatoria. \\
		3 & Se crea una cuenta de correo. \\
		4 & Se vuelcan los datos a la base de datos. Si error S-2.\\
		5 & Se muestra un mensaje de confirmación del registro. \\
		6 & Se imprime un documento con los datos de acceso para el usuario.
	\end{tablasecuencias}
}{
	% Tabla de secuencia con errores del caso de uso
	\begin{tablasecuencias}
		S-1 & Alguno de los datos no es válido. El sistema vuelve al paso 1 de la secuencia normal de uso e indica los campos erróneos.\\
		S-2 & No se puede conectar con la base de datos. Se cancela la operación, se muestra un mensaje de error por pantalla y se vuelve a la página principal del sistema.
	\end{tablasecuencias}
}


