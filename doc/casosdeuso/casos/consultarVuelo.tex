% Caso de uso: consultar vuelo.
% Obs: para escribir comas en el texto del primer parámetro se han de encerrar entre {}.

% Revisado por Cristina el día 11/03/2013

\casodeuso{
	% Nombre del caso de uso
	nombre=Consultar vuelo,
	% Objetivo
	objetivo={Mostrar al cliente la relación de vuelos operados por la compañía, permitiendo buscar y filtrar resultados, así como información detallada de un vuelo en particular.},
	% Entradas
	entradas={Opcionalmente, campos de búsqueda(aeropuertos de origen y destino, número de escalas, fecha y hora, precio del billete\ldots).},
	% Precondiciones
	precondiciones=No hay precondiciones.,
	% Salidas
	salidas=Lista de vuelos e información detallada sobre el seleccionado.,
	% Postcondiciones en caso de éxito
	postexito=No se realiza ningún cambio en el sistema.,
	% Postcondiciones en caso de error
	posterror=No se realiza ningún cambio en el sistema.,
	% Actores
	actores=Los clientes de la compañía y la base de datos.,
}{
	% Tabla de secuencia normal del caso de uso
	\begin{tablasecuencias}
		1 & Se extrae de la base de datos del sistema el listado de vuelos. Si error S-1.\\
		2 & Se muestra la lista de vuelos ordenada por un criterio asignado por defecto.\\
		3 & El usuario puede filtrar los resultados y buscar vuelos según diferentes criterios.\\
		4 & Se selecciona un vuelo.\\
		5 & Se muestra la información detallada del vuelo. Si error S-1.
	\end{tablasecuencias}
}{
	% Tabla de secuencia con errores del caso de uso
	\begin{tablasecuencias}
		S-1 &No se puede conectar con la base de datos, se muestra un mensaje de error por pantalla dando la opción de reintentar o volver al menú principal de la aplicación.
	\end{tablasecuencias}
}
