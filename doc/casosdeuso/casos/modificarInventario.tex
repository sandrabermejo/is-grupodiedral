% Caso de uso: modificar items inventario.
% Obs: para escribir comas en el texto del primer parámetro se han de encerrar entre {}.

\casodeuso{
	% Nombre del caso de uso
	nombre=Modificar inventario,
	% Objetivo
	objetivo=Permite modificar diversos datos sobre el material registrado en el inventario de la empresa.,
	% Entradas
	entradas=La nueva información sobre el material.,
	% Precondiciones
	precondiciones={El operador de la aplicación está debidamente registrado y posee credenciales que le habilitan para realizar esta operación. El servidor que hospeda la base de datos de inventario está operativo.},
	% Salidas
	salidas=El inventario de la empresa modificado en función de los cambios registrados.
	% Postcondiciones en caso de éxito
	postexito=El material disponible de la empresa en el inventario se habrá actualizado de acuerdo a los datos introducidos.,
	% Postcondiciones en caso de error
	posterror={El sistema central no ha sufrido cambios y, por tanto, no se actualiza el material disponible de la empresa.},
	% Actores
	actores=El personal de la compañía con permisos para realizar estas modificaciones y la base de datos.,
}{
	% Tabla de secuencia normal del caso de uso
	\begin{tablasecuencias}
		1 & Extraer de la base de datos el inventario con los elementos disponibles. Si error S-1. \\
		2 & Mostrar una lista ordenada según el criterio configurado y permitir la búsqueda. Si error S-2. \\
		3 & Permitir añadir o eliminar elementos del inventario. Si error S-3.\\
		4 & Al seleccionar un objeto, mostrar una vista editable de sus propiedades. Si error S-4.
	\end{tablasecuencias}
}{
	% Tabla de secuencia con errores del caso de uso
	\begin{tablasecuencias}
		S-1 & La base de datos está dañada y no se han podido extraer los datos, o ha habido un error en la aplicación. Mostrar por pantalla un mensaje para que el usuario se ponga en contacto con el personal técnico de la empresa y le manifieste el error, disculparse por las molestias y dar las gracias por el aviso.\\
		S-2 & Si no se ha podido generar la lista, mostrar información del error al usuario y abortar la operación. \\
		S-3 & Si no se puede acometer la transacción con la base de datos, informar al usuario y permitir reintento. \\
		S-4 & Si no se puede obtener la información de la base de datos, informar al usuario y cancelar la pantalla de edición. Si no se puede introducir el contenido editado, S-3.
	\end{tablasecuencias}
}

