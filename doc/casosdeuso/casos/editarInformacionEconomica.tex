% Caso de uso: editar información económica.
% Obs: para escribir comas en el texto del primer parámetro se han de encerrar entre {}.

\casodeuso{
	% Nombre del caso de uso
	nombre=Editar información económica,
	% Objetivo
	objetivo=Editar la información económica de la compañía de acuerdo a los resultados del último ejercicio (cuentas anuales).,
	% Entradas
	entradas=La nueva información económica de la compañía.,
	% Precondiciones
	precondiciones={Haber accedido a la aplicación con un usuario válido, tener permisos para editar la información requerida y elegir la opción \textit{Editar información económica.}},
	% Salidas
	salidas={La información económica de la compañía modificada, si procede.},
	% Postcondiciones en caso de éxito
	postexito=La información económica de la compañía ha sido modificada correctamente.,
	% Postcondiciones en caso de error
	posterror=El sistema no ha sufrido ningún cambio.,
	% Actores
	actores={El personal administrativo de la compañía, con permisos para editar la información económica, y la base de datos.},
}{
	% Tabla de secuencia normal del caso de uso
	\begin{tablasecuencias}
		1 & Extraer de la base de datos la información económica de la compañía. Si error S-1.\\
		2 & Mostrar la información por pantalla.\\
		3 & Seleccionar la información a modificar.\\
		4 & Introducir los datos económicos actualizados de la compañía. Si error S-2.\\
		5 & Almacenar los cambios en la base de datos. Si error S-3.\\
		6 & Mostrar un mensaje de confirmación de la operación.
	\end{tablasecuencias}
}{
	% Tabla de secuencia con errores del caso de uso
	\begin{tablasecuencias}
		S-1 & No se ha podido conectar con la base de datos para obtener la información requerida. Mostrar un mensaje de error por pantalla y volver a la página principal del sistema.\\
		S-2 & Alguno de los datos introducidos no es válido. Volver al paso 4 de la secuencia normal de uso indicando los campos erróneos.\\
		S-3 & No se han podido almacenar los cambios en la base de datos. Se cancela la operación, se muestra un mensaje de error por pantalla informando de que la información económica no ha podido actualizarse y se vuelve a la página principal del sistema.
	\end{tablasecuencias}
}

