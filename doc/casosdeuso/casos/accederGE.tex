% Caso de uso: Acceder web
% Obs: para escribir comas en el texto del primer parámetro se han de encerrar entre {}.

% Revisado por Juanan el día 12/03/2013

\casodeuso{
	% Nombre del caso de uso
	nombre=Acceder web,
	% Objetivo
	objetivo=Identificarse como cliente de la compañia en el servicio web.,
	% Entradas
	entradas=Nombre de usuario y contraseña.,
	% Precondiciones
	precondiciones=El usuario dispone de una cuenta de cliente de la compañía aérea.,
	% Salidas
	salidas=Menú principal de la aplicación de gestión externa de la compañía.,
	% Postcondiciones en caso de éxito
	postexito=No se realiza ningún cambio en el sistema.,
	% Postcondiciones en caso de error
	posterror=No se realiza ningún cambio en el sistema.,
	% Actores
	actores=Cliente-usuario de interfaz web,
}{
	% Tabla de secuencia normal del caso de uso
	\begin{tablasecuencias}
		1 & El cliente introduce su nombre y contraseña y elige la opción \textit{Acceder}. Si error S-1.
	\end{tablasecuencias}
}{
	% Tabla de secuencia con errores del caso de uso
	\begin{tablasecuencias}
		S-1 & Si el nombre o la contraseña no son válidos, el sistema vuelve a la secuencia normal de uso e indica que los datos son erróneos.
	\end{tablasecuencias}
}


