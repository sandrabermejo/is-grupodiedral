% Caso de uso: Acceder Gestión Externa
% Obs: para escribir comas en el texto del primer parámetro se han de encerrar entre {}.

\casodeuso{
	% Nombre del caso de uso
	nombre=Acceder web,
	% Objetivo
	objetivo=Abrir la aplicación de gestión externa de la compañía.,
	% Entradas
	entradas=Nombre del usuario y contraseña.,
	% Precondiciones
	precondiciones=Que el usuario esté registrado como cliente de la compañía aérea.,
	% Salidas
	salidas=Las diversas opciones de la aplicación de gestión externa de la compañía.,
	% Postcondiciones en caso de éxito
	postexito=El cliente tendrá acceso como usuario a los servicios ofertados por la compañia a través de la Web.,
	% Postcondiciones en caso de error
	posterror=No se realizan cambios en el sistema.,
	% Actores
	actores=Cliente-usuario de interfaz web,
}{
	% Tabla de secuencia normal del caso de uso
	\begin{tablasecuencias}
		1 & El cliente introduce su nombre y contraseña y elige la opción \textit{Acceder}. Si error S-1.
	\end{tablasecuencias}
}{
	% Tabla de secuencia con errores del caso de uso
	\begin{tablasecuencias}
		S-1 & Si el nombre o la contraseña no son válidos, el sistema vuelve a la secuencia normal de uso e indica los campos erróneos.
	\end{tablasecuencias}
}


