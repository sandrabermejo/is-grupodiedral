% Caso de uso: Facturar
% Obs: para escribir comas en el texto del primer parámetro se han de encerrar entre {}.

% Revisado por Cristina el día 13/03/2013

\casodeuso{
	% Nombre del caso de uso
	nombre=Facturar,
	% Objetivo
	objetivo=Controlar la facturación del equipaje del viajero.,
	% Entradas
	entradas={El número de vuelo, el número de reserva del pasajero y el peso del equipaje.},
	% Precondiciones
	precondiciones=El operador de la aplicación tiene credenciales que le habilitan para realizar dicha operación.,
	% Salidas
	salidas=Confirmación de que el equipaje ha sido facturado.,
	% Postcondiciones en caso de éxito
	postexito=Se registra la operación en la base de datos del sistema.,
	% Postcondiciones en caso de error
	posterror=No se realiza ningún cambio en el sistema.,
	% Actores
	actores=El personal de la compañía presente en el aeropuerto y la base de datos.
}{
	% Tabla de secuencia normal del caso de uso
	\begin{tablasecuencias}
		1 & El usuario introduce el número de vuelo y el número de reserva del pasajero. Si algún dato no es válido S-1.\\
		2 & El usuario introduce el peso de la maleta del pasajero. Si error S-2.\\
		3 & En caso de exceso de peso, se muestra la cantidad de dinero a pagar en el momento.
	\end{tablasecuencias}
}{
	% Tabla de secuencia con errores del caso de uso
	\begin{tablasecuencias}
		S-1 & El sistema vuelve a 1 de la secuencia normal de uso e indica los campos erróneos.\\
		S-2 & El peso introducido no está en el rango admitido por la compañía. Se muestra un mensaje de error y se pide de nuevo la introducción del peso.
	\end{tablasecuencias}
}

