% Caso de uso: Consultar inventario
% Obs: para escribir comas en el texto del primer parámetro se han de encerrar entre {}.

% Revisado por Juanan el día 11/03/2013

\casodeuso{
	% Nombre del caso de uso
	nombre=Consultar inventario,
	% Objetivo
	objetivo={Mostrar el inventario de material de la empresa, permitiendo buscar y filtrar resultados.},
	% Entradas
	entradas={Opcionalmente, campos de búsqueda.},
	% Precondiciones
	precondiciones=El operador de la aplicación tiene credenciales que le habilitan para realizar dicha operación.,
	% Salidas
	salidas=El registro de material en el inventario de la empresa e información detallada sobre el seleccionado.,
	% Postcondiciones en caso de éxito
	postexito=No se realiza ningún cambio en el sistema.,
	% Postcondiciones en caso de error
	posterror= No se realiza ningún cambio en el sistema.,
	% Actores
	actores={Personal administrativo, mecánico y la base de datos.}
}{
	% Tabla de secuencia normal del caso de uso
	\begin{tablasecuencias}
		1 & Se extrae de la base de datos el inventario. Si error S-1. \\
		2 & Se muestra por pantalla listado de material. \\
		3 & El usuario puede filtrar los resultados según diferentes criterios(Nombre, fabricante, vehiculo, \ldots),
	\end{tablasecuencias}
}{
	% Tabla de secuencia con errores del caso de uso
	\begin{tablasecuencias}
		S-1 & No se puede conectar con la base de datos. Informar al usuario y volver al menú principal.
	\end{tablasecuencias}
}

