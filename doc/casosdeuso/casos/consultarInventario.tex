% Caso de uso: Consultar inventario
% Obs: para escribir comas en el texto del primer parámetro se han de encerrar entre {}.

\casodeuso{
	% Nombre del caso de uso
	nombre=Consultar inventario,
	% Objetivo
	objetivo=Muestra el inventario de la empresa. Este material es tanto para reparaciones como de otros tipos que necesite la empresa.,
	% Entradas
	entradas=,
	% Precondiciones
	precondiciones={El operador de la aplicación está debidamente registrado y posee credenciales que le habilitan para realizar esta operación. El servidor que hospeda la base de datos de inventario está operativo.},
	% Salidas
	salidas=El registro de material en el inventario de la empresa.,
	% Postcondiciones en caso de éxito
	postexito=El empleado quedará informado sobre el material del que dispone la empresa en el inventario.,
	% Postcondiciones en caso de error
	posterror=El empleado no habrá podido recibir la información del inventario de la empresa y desconocerá el material disponible.,
	% Actores
	actores=El empleado con permisos para acceder a la información requerida y la base de datos.
}{
	% Tabla de secuencia normal del caso de uso
	\begin{tablasecuencias}
		1 & Extraer de la base de datos el inventario. Si error S-1. \\
		2 & Realizar una lista con la información organizada por el criterio configurado. Si error S-2. \\
		3 & Mostrar por pantalla estos datos permitiendo ser filtrados.\\
		4 & Al seleccionar un elemento, mostrar información detallada del mismo. Si error S-1.
	\end{tablasecuencias}
}{
	% Tabla de secuencia con errores del caso de uso
	\begin{tablasecuencias}
		S-1 & No se pudo conectar con la base de datos o no se pudo obtener la información. Informar al usuario y abandonar la carga.\\
		S-2 & Si no se encuentra configuración sobre criterios de ordenación, ordenar por orden lexicográfico o numérico del primer campo si lo hubiera.
	\end{tablasecuencias}
}

