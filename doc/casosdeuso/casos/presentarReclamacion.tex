% Caso de uso: Presentar reclamacion
% Obs: para escribir comas en el texto del primer parámetro se han de encerrar entre {}.

% Revisado por Juanan el día 12/03/2013

\casodeuso{
	% Nombre del caso de uso
	nombre=Presentar reclamaciones,
	% Objetivo
	objetivo= Presentar una reclamación por parte del cliente.,
	% Entradas
	entradas= Motivo de la queja y detallado de la misma.,
	% Precondiciones
	precondiciones=No hay precondiciones,
	% Salidas
	salidas=Confirmación del envio y número asociado a la reclamación.,
	% Postcondiciones en caso de éxito
	postexito=La notificación se envia al departamento correspondiente.,
	% Postcondiciones en caso de error
	posterror=No se realiza ningún cambio en el sistema.,
	% Actores
	actores=Cliente-usuario de interfaz web.,
}{
	% Tabla de secuencia normal del caso de uso
	\begin{tablasecuencias}
		1 & El usuario completa los datos requeridos. Si error S-1. \\
		2 & Muestra número asignado a la reclamación junto a la confirmación de que la reclamación ha sido tramitada y en breve será atendia por el departamento correspondiente. Si error S-2. 
		
	\end{tablasecuencias}
}{
	% Tabla de secuencia con errores del caso de uso
	\begin{tablasecuencias}
		S-1 & Los campos de datos están incompletos. Se vuelve a 1 de la secuencia normal de uso y se incican los campos erróneos o vacíos.\\
		S-2 & Si no puede transmitir los datos de la reclamación a la base de datos se indica en un mensaje y se insta a que se vuelva a intentar.
	\end{tablasecuencias}
}

