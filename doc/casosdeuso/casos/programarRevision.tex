% Caso de uso: Programar revisión.
% Obs: para escribir comas en el texto del primer parámetro se han de encerrar entre {}.

% Revisado por Juanan el día 12/03/2013

\casodeuso{
	% Nombre del caso de uso
	nombre=Programar revisión,
	% Objetivo
	objetivo=Programar una revisión para un vehiculo determinado.,
	% Entradas
	entradas={Equipo, herramientas , material y personal necesario. Además de la fecha, hora y vehiculo.},
	% Precondiciones
	precondiciones={Disponibilidad del vehiculo, así como del peresonal necesario.},
	% Salidas
	salidas=Confirmación adjuntando informe de la revisión programada.,
	% Postcondiciones en caso de éxito
	postexito={Queda asignado un hangar para la revisión, así como registrados los datos correspondientes y reservados para el propósito los recursos humanos y materiales indicados.},
	% Postcondiciones en caso de error
	posterror=No se realiza ningún cambio en el sistema.,
	% Actores
	actores=Personal de mantenimiento autorizado.
}{
	% Tabla de secuencia normal del caso de uso
	\begin{tablasecuencias}
		1 & Se escoge fecha y hora.\\
		2 & Se selecciona vehiculo.\\
		3 & Se especifica el material y las herramientas necesarias. Si error S-1.\\
		4 & Se elige al personal de entre los disponibles.\\
		5 & Se da la opción de confirmar operación.
	\end{tablasecuencias}
}{
	% Tabla de secuencia con errores del caso de uso
	\begin{tablasecuencias}
		S-1 & Si alguno de los elementos no se encuentra disponible en el almacén se notifica y vuelve a 3 de la secuencia normal de uso.\\
		S-2 & Si no se puede conectar con la base de datos, se muestra un mensaje de error. Se ofrece la posibilidad de reintentar o cancelar operación.
	\end{tablasecuencias}
}
