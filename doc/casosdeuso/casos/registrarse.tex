% Caso de uso: Registrarse
% Obs: para escribir comas en el texto del primer parámetro se han de encerrar entre {}.

% Revisado por Cristina el día 11/03/2013

\casodeuso{
	% Nombre del caso de uso
	nombre=Registrarse,
	% Objetivo
	objetivo=Registrar un nuevo cliente de la compañía en la base de datos.,
	% Entradas
	entradas={Datos personales del cliente (nombre y apellidos, NIF o equivalente, dirección, teléfono y una dirección de correo electrónico).},
	% Precondiciones
	precondiciones=El visitante que pretende registrarse no está previamente registrado.,
	% Salidas
	salidas=Confirmación del registro del nuevo cliente.,
	% Postcondiciones en caso de éxito
	postexito=El cliente queda registrado en la base de datos.,
	% Postcondiciones en caso de error
	posterror=No se realiza ningún cambio en el sistema.,
	% Actores
	actores=El visitante de la interfaz externa y la base de datos.,
}{
	% Tabla de secuencia normal del caso de uso
	\begin{tablasecuencias}
		1 & El usuario introduce sus datos personales. Si error S-1.\\
		2 & Se almacenan los datos del nuevo cliente en la base de datos. Si error S-2.
	\end{tablasecuencias}
}{
	% Tabla de secuencia con errores del caso de uso
	\begin{tablasecuencias}
		S-1 & El sistema vuelve al paso 1 de la secuencia normal de uso e indica los campos erróneos.\\
		S-2 & No se puede conectar con la base de datos, se muestra un mensaje de error por pantalla dando la opción de reintentar o volver al menú principal de la aplicación.
	\end{tablasecuencias}
}
