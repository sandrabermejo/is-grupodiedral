% Caso de uso: Acceder a una oferta.
% Obs: para escribir comas en el texto del primer parámetro se han de encerrar entre {}.

\casodeuso{
	% Nombre del caso de uso
	nombre=Acceder a una oferta.,
	% Objetivo
	objetivo=Mostrar detalladamente una oferta elegida por el cliente.,
	% Entradas
	entradas=El cliente podrá comprar la oferta.,
	% Precondiciones
	precondiciones=Haber accedido a la página web de la compañía.,
	% Salidas
	salidas=Los detalles de una oferta específica.,
	% Postcondiciones en caso de éxito
	postexito=Muestra las ofertas actuales de la compañía.,
	% Postcondiciones en caso de error
	posterror=No se modifica la página web.,
	% Actores
	actores=Cualquier usuario que acceda a la página web y la base de datos.
}{
	% Tabla de secuencia normal del caso de uso
	\begin{tablasecuencias}
		1 & Acceder a la base de datos para obtener los datos completos de la oferta. Si error S-1.\\
		2 & Muestra por pantalla los detalles de la oferta. \\
		3 & Se dará la opción de comprar los productos ofertados.
	\end{tablasecuencias}
}{
	% Tabla de secuencia con errores del caso de uso
	\begin{tablasecuencias}
		S-1 & No se puede acceder a la base de datos. No se carga las oferta y se muestra un mensaje de error indicando que la oferta no está disponible por algún error del sistema. 
	\end{tablasecuencias}
}
