% Caso de uso: iniciar pago billetes de vuelo
% Obs: para escribir comas en el texto del primer parámetro se han de encerrar entre {}.

% Revisado por Juanan el día 12/03/2013

\casodeuso{
	% Nombre del caso de uso
	nombre=Iniciar pago billetes de vuelo,
	% Objetivo
	objetivo=Iniciar el proceso de pago del billete para finalizar el proceso de compra.,
	% Entradas
	entradas=No hay entradas.,
	% Precondiciones
	precondiciones=El usuario completa el proceso de \textit{Comprar billete} correctamente.,
	% Salidas
	salidas=No hay salidas.,
	% Postcondiciones en caso de éxito
	postexito=El proceso de compra se redirecciona a \textit{Realizar Pago con Tarjeta} para finalizar el pago mediante tarjeta de crédito o débito.,
	% Postcondiciones en caso de error
	posterror=No se realiza ningún cambio en el sistema.,
	% Actores
	actores={El usuario, la base de datos y las entidades financieras.},
}{
	% Tabla de secuencia normal del caso de uso
	\begin{tablasecuencias}
		1 & Se muestran todos los datos de la compra. Si error S-1.\\
		2 & Se indican con claridad las claúsulas de las leyes de protección de datos para que el usuario las acepte y pueda seguir con la compra. Si rechazo S-2.
	\end{tablasecuencias}
}{
	% Tabla de secuencia con errores del caso de uso
	\begin{tablasecuencias}
		S-1 & La reserva de los billetes ya no es válida. Se muestra un mensaje de error y se da la opción de abortar la operación o volver al menú de compra de billete.\\
		S-2 & El cliente no ha aceptado las claúsulas de leyes de protección de datos. Se aborta la operación, se muestra un mensaje por pantalla indicándolo y se vuelve a la página principal de la aplicación.
	\end{tablasecuencias}
}

