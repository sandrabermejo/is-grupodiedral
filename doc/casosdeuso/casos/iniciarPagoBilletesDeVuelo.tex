% Caso de uso: iniciar pago billetes de vuelo.
% Obs: para escribir comas en el texto del primer parámetro se han de encerrar entre {}.

\casodeuso{
	% Nombre del caso de uso
	nombre=Iniciar el pago de los billetes de vuelo,
	% Objetivo
	objetivo=Iniciar el proceso de pago del billete para finalizar el proceso de compra.,
	% Entradas
	entradas=La información detallada de los billetes de vuelo.,
	% Precondiciones
	precondiciones={Haberse registrado en la página web de la compañía aérea y haber seleccionado todos los campos detallados de los billetes.},
	% Salidas
	salidas={Se procede a realizar el pago mediante tarjeta de crédido o débito, si procede}.,
	% Postcondiciones en caso de éxito
	postexito=El proceso de compra será redirigido para finalizar el pago mediante tarjeta de crédito o débito.,
	% Postcondiciones en caso de error
	posterror=La compra no se ha realizado y la base de datos no ha sido modificada.,
	% Actores
	actores={El usuario, la base de datos y las entidades financieras.},
}{
	% Tabla de secuencia normal del caso de uso
	\begin{tablasecuencias}
		1 & Mostrar todos los datos de la compra. Si error S-1.\\
		2 & Indicar con claridad las claúsulas de las leyes de protección de datos para que el usuario las acepte y pueda seguir con la compra. Si error S-2.\\ 
		3 & Se redirecciona según a \textbf{Realizar Pago con Tarjeta}.
	\end{tablasecuencias}
}{
	% Tabla de secuencia con errores del caso de uso
	\begin{tablasecuencias}
		S-1 & La reserva de los billetes ha expirado porque ha pasado mucho tiempo desde que se añadieron los billetes al carrito de la compra o por algún error en su almacenamiento. Se mostrará un mensaje indicando que se vuelva a realizar la compra desde el principio.\\3
		S-2 & El cliente no ha aceptado las claúsulas de leyes de protección de datos. Se aborta la operación, se muestra un mensaje por pantalla indicándolo y se vuelve a la página principal de la aplicación.
	\end{tablasecuencias}
}

