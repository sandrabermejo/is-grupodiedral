% Caso de uso: registrar entrada de material.
% Obs: para escribir comas en el texto del primer parámetro se han de encerrar entre {}.

\casodeuso{
	% Nombre del caso de uso
	nombre=Registrar entrada de material,
	% Objetivo
	objetivo={Registrar una entrada de material en el inventario, declarando en que sección y ubicación física se encuentra.},
	% Entradas
	entradas={Nombre del producto o identificador si es un material previamente registrado, descripción, sección que lo recibe (si procede) y ubicación final.},
	% Precondiciones
	precondiciones={El operador de la aplicación está debidamente registrado y tiene credenciales que le habilitan para realizar dicha operación. Existen configuraciones sobre ubicaciones, productos y secciones que previamente han sido introducidas.},
	% Salidas
	salidas=Adhesivo de inventario con la clave de identificación del material en el sistema (si procede).,
	% Postcondiciones en caso de éxito
	postexito=El material introducido aparecerá en el inventario.,
	% Postcondiciones en caso de error
	posterror=El sistema central no ha sufrido cambio alguno.,
	% Actores
	actores=El personal de la compañía cuyo rol implique registro de entrada de material dentro de su ámbito y la base de datos.,
}{
	% Tabla de secuencia normal del caso de uso
	\begin{tablasecuencias}
		1 & En el formulario de registro de inventario, introduce los datos necesarios del ingreso. Si error S-1.\\
		2 & Se registran los cambios en el sistema correspondiente. Si error S-2.\\
		3 & Obtiene una adhesivo impreso para identificar el objeto catalogado (si procede).
	\end{tablasecuencias}
}{
	% Tabla de secuencia con errores del caso de uso
	\begin{tablasecuencias}
		S-1 & Se reitera la opción de insertar los datos o abandonar.\\
		S-2 & Da la opción de escoger entre abortar la operación o reintentar.
	\end{tablasecuencias}
}
