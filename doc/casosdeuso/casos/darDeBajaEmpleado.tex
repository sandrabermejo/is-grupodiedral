% Caso de uso: dar de baja a un empleado.
% Obs: para escribir comas en el texto del primer parámetro se han de encerrar entre {}.
% Revisado por Cristina y Juanan el día 11/03/2013
%  Incluir o extender de consultar ficha empleado?
\casodeuso{
	% Nombre del caso de uso
	nombre=Dar de baja a un empleado.,
	% Objetivo
	objetivo={Dar de baja a un empleado, eliminando su información personal de acuerdo a la legislación vigente.},
	% Entradas
	entradas=No hay entradas.,
	% Precondiciones
	precondiciones=El operador de la aplicación tiene credenciales que le habilitan para realizar dicha operación y tiene una ficha seleccionada.,	
	% Salidas
	salidas=Se confirma la operación.,
	% Postcondiciones en caso de éxito
	postexito=Los datos del usuario se eliminan del sistema exceptuando aquellos datos que la ley fije como de obligada conservación.,
	% Postcondiciones en caso de error
	posterror=No se realiza ningún cambio en el sistema.,
	% Actores
	actores={El personal administrativo y la base de datos.},
}{
	% Tabla de secuencia normal del caso de uso
	\begin{tablasecuencias}
		1 & El administrativo confirma la operación. Si no S-1.\\
		2 & Se almacenan los cambios en la base de datos. Si error S-2.
	\end{tablasecuencias}
}{
	% Tabla de secuencia con errores del caso de uso
	\begin{tablasecuencias}
		S-1 & Se cancela la operación y se vuelve a la ficha del empleado.\\
		S-2 & No se puede conectar con la base de datos. Se cancela la operación, se muestra un mensaje de error por pantalla y se vuelve a la página principal del sistema.
	\end{tablasecuencias}
}


