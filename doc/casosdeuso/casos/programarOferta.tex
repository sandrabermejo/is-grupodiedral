% Caso de uso: Programar oferta.
% Obs: para escribir comas en el texto del primer parámetro se han de encerrar entre {}.

\casodeuso{
	% Nombre del caso de uso
	nombre=Programar oferta,
	% Objetivo
	objetivo=Poder crear una nueva oferta para poder mostrarla y darle acceso posteriomente a los clientes.,
	% Entradas
	entradas=Los datos concretos de la oferta.,
	% Precondiciones
	precondiciones=Haber accedido al sistema.,
	% Salidas
	salidas=Crear una oferta que la empresa desea mostrar.,
	% Postcondiciones en caso de éxito
	postexito=Se crea una nueva oferta para promocionar un item y motivar su venta.,
	% Postcondiciones en caso de error
	posterror=No se modifica la lista de ofertas de la empresa.,
	% Actores
	actores=La base de datos y el personal administrativo cuyo rol sea programar., 
}{
	% Tabla de secuencia normal del caso de uso
	\begin{tablasecuencias}
		1 & Rellenar el formulario de una oferta con todos los datos requeridos.\\
		2 & Añadir la oferta a la lista de ofertas de la base de datos de la empresa. Si error S-1.
	\end{tablasecuencias}
}{
	% Tabla de secuencia con errores del caso de uso
	\begin{tablasecuencias}
		S-1 & No se ha podido añadir la oferta a la base de datos, mostrar un mensaje de error y permitir que el usuario pueda volver a intentar añadir la oferta.
	\end{tablasecuencias}
}
