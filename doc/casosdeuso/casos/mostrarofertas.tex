% Caso de uso: Mostrar ofertas.
% Obs: para escribir comas en el texto del primer parámetro se han de encerrar entre {}.

\casodeuso{
	% Nombre del caso de uso
	nombre=Mostrar ofertas.,
	% Objetivo
	objetivo=Mostrar las ofertas actuales a los clientes con el fin de incentivar su compra.,
	% Entradas
	entradas=El cliente podrá elegir una oferta entre las que se muestran.,
	% Precondiciones
	precondiciones=Haber accedido a la página web de la compañía.,
	% Salidas
	salidas=Las ofertas actuales.,
	% Postcondiciones en caso de éxito
	postexito=Muestra las ofertas actuales de la compañía.,
	% Postcondiciones en caso de error
	posterror=No se modifica la página web.,
	% Actores
	actores=Cualquier usuario que acceda a la página web y la base de datos.
}{
	% Tabla de secuencia normal del caso de uso
	\begin{tablasecuencias}
		1 & Acceder a la base de datos. Si error S-1.\\
		2 & Muestra por pantalla las ofertas de la compañía.\\
		3 & Si el cliente accede a una oferta en concreto, se rediccionará a esa oferta a través de \textit{Acceder a la oferta}. Si error S-2.
	\end{tablasecuencias}
}{
	% Tabla de secuencia con errores del caso de uso
	\begin{tablasecuencias}
		S-1 & No se puede acceder a la base de datos. No se cargan las ofertas y se muestra la pantalla de la página web sin las ofertas.\\
		S-2 & Si no se ha podido rediccionar a la oferta elegida se mostrará un mensaje por pantalla indicando que la oferta no es accesible en estos momentos.
	\end{tablasecuencias}
}
