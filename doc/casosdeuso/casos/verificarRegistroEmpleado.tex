% Caso de uso: verificar registro empleado.
% Obs: para escribir comas en el texto del primer parámetro se han de encerrar entre {}.

\casodeuso{
	% Nombre del caso de uso
	nombre=Verificar el registro de un empleado,
	% Objetivo
	objetivo=Establece una nueva cuenta de usuario como válida y totalmente operativa.,
	% Entradas
	entradas={Los datos del usuario previamente registrado, incluyendo su futuro puesto de trabajo en la compañía.},
	% Precondiciones
	precondiciones={El nuevo usuario ha sido registrado correctamente en el sistema, el empleado de Recursos Humanos tiene los permisos necesarios para realizar la operación y ha accedido a la opción \textit{Verificar registro empleado}.},
	% Salidas
	salidas={Confirmación de la operación, si procede.},
	% Postcondiciones en caso de éxito
	postexito={La cuenta del usuario queda verificada y, por tanto, pasa a estar activa y totalmente operativa.},
	% Postcondiciones en caso de error
	posterror=La cuenta del usuario permanece inactiva.,
	% Actores
	actores=El personal del departamento de intervención y la base de datos.,
}{
	% Tabla de secuencia normal del caso de uso
	\begin{tablasecuencias}
		1 & Se extrae de la base de datos del sistema la información de la cuenta del usuario. Si error S-1. \\
		2 & Se muestran los datos por pantalla y se comprueba que sean correctos. Si error S-2.\\
		3 & Se verifica la cuenta del usuario y se modifica en la base de datos. Si error S-3.\\
		4 & Se muestra por pantalla un mensaje de confirmación de la operación.
	\end{tablasecuencias}
}{
	% Tabla de secuencia con errores del caso de uso
	\begin{tablasecuencias}
		S-1 & No se ha podido conectar con la base de datos. Mostrar un mensaje de error por pantalla y volver a la página principal de la aplicación.\\
		S-2 & Alguno de los datos de la cuenta es erróneo, o se han concedido al usuario permisos que no se corresponden a su función en la compañía. El empleado del departamento de intervención cancela la operación e informa al personal administrativo para que contacten con el futuro empleado y se proceda de nuevo a su registro.\\
		S-3 & No se ha podido acceder a la base de datos. Se cancela la operación, se muestra un mensaje de error por pantalla y vuelve a la página principal de la aplicación.
	\end{tablasecuencias}
}


