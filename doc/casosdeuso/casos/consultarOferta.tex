% Caso de uso: Consultar oferta.
% Obs: para escribir comas en el texto del primer parámetro se han de encerrar entre {}.

% Revisado por Cristina el día 11/03/2013
% Nombre antiguo: accederaUnaOferta

\casodeuso{
	% Nombre del caso de uso
	nombre=Consultar oferta,
	% Objetivo
	objetivo=Consultar una oferta elegida por el cliente.,
	% Entradas
	entradas=No hay entradas.,
	% Precondiciones
	precondiciones=No hay precondiciones.,
	% Salidas
	salidas=Los detalles de una oferta específica.,
	% Postcondiciones en caso de éxito
	postexito=No se realiza ningún cambio en el sistema.,
	% Postcondiciones en caso de error
	posterror=No se realiza ningún cambio en el sistema.,
	% Actores
	actores=Los clientes de la compañía y la base de datos.,
}{
	% Tabla de secuencia normal del caso de uso
	\begin{tablasecuencias}
		1 & Se extrae de la base de datos la información completa de la oferta. Si error S-1.
	\end{tablasecuencias}
}{
	% Tabla de secuencia con errores del caso de uso
	\begin{tablasecuencias}
		S-1 & No se puede conectar con la base de datos, se muestra un mensaje de error por pantalla dando la opción de reintentar o volver al menú principal de la aplicación.
	\end{tablasecuencias}
}
