% Caso de uso: Configurar nómina
% Obs: para escribir comas en el texto del primer parámetro se han de encerrar entre {}.

\casodeuso{
	% Nombre del caso de uso
	nombre=Configurar nómina,
	% Objetivo
	objetivo=Confeccionar y almacenar las nóminas mensuales de todos empleados.,
	% Entradas
	entradas={Incidencias relativas a la nómina de un empleado correspondientes a un mes (horas extras, comisiones, sustituciones\ldots)},
	% Precondiciones
	precondiciones=Haber accedido al sistema como un personal de Recursos Humanos y acceder a la opción \textit{Configurar nónima}.,
	% Salidas
	salidas=Confirmación de la operación,
	% Postcondiciones en caso de éxito
	postexito=La nómina queda archivada en la base de datos.,
	% Postcondiciones en caso de error
	posterror=La base de datos no se modifica,
	% Actores
	actores=Personal de administración y la base de datos.
}{
	% Tabla de secuencia normal del caso de uso
	\begin{tablasecuencias}
		1 & Seleccionar el empleado correspondiente y obtener los datos de la base de datos. Si error S-1. \\
		2 & Se introducen los conceptos e incidencias correspondientes. Si error S-2. \\
		3 & Petición de confirmación de la corrección de los datos introducidos.
	\end{tablasecuencias}
}{
	% Tabla de secuencia con errores del caso de uso
	\begin{tablasecuencias}
		S-1 & Si no se puede conectar con la base de datos, mostrar un mensaje de error y volver a la página principal del sistema. \\
		S-2 & Mostrar un mensaje de error y volver a la página principal del sistema.
	\end{tablasecuencias}
}
