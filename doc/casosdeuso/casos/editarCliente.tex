% Caso de uso: editar cliente.
% Obs: para escribir comas en el texto del primer parámetro se han de encerrar entre {}.

% Revisado por Juanan el día 12/03/2013

\casodeuso{
	% Nombre del caso de uso
	nombre=Editar cliente,
	% Objetivo
	objetivo={Editar la información de un cliente, bien él mismo o desde la gestión interna de la aplicación.},
	% Entradas
	entradas=Los datos que se quieren modificar.,
	% Precondiciones
	precondiciones={El operador de la aplicación tiene credenciales que le habilitan para realizar dicha operación y tiene una ficha seleccionada.},
	% Salidas
	salidas=La información de perfil actualizada.,
	% Postcondiciones en caso de éxito
	postexito=Los cambios efectuados se guardan en la base de datos.,
	% Postcondiciones en caso de error
	posterror=No se realiza ningún cambio en el sistema.,
	% Actores
	actores={Cliente-usuario de interfaz web o el personal administrativo y la base de datos},
}{
	% Tabla de secuencia normal del caso de uso
	\begin{tablasecuencias}
		1 & Muestra los campos de datos personales de usuario. Si error S-1.\\
		2 & El usuario modifica los datos deseados. Si error S-2.\\
		3 & Se almacenan los cambios en la base de datos. Si error S-3.
	\end{tablasecuencias}
}{
	% Tabla de secuencia con errores del caso de uso
	\begin{tablasecuencias}
		S-1 & Si no se puede conectar con la base de datos se muestra mensaje  de tipo \textit{información no disponible temporalmente} y se vuelve a 1 de la secuencia normal de uso.\\
		S-2 & Alguno de los datos introducidos no es válido. Vuelve a 1 de la secuencia normal de uso indicando los campos erróneos.\\
		S-3 & No se puede conectar con la base de datos. Se cancela la operación, se muestra un mensaje de error por pantalla y se vuelve a la ficha del cliente.
	\end{tablasecuencias}
}

