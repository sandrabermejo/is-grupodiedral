% Caso de uso: realizar mantenimiento.
% Obs: para escribir comas en el texto del primer parámetro se han de encerrar entre {}.

\casodeuso{
	% Nombre del caso de uso
	nombre=Realizar mantenimiento.,
	% Objetivo
	objetivo=Permite registrar la información sobre un mantenimiento realizado.,
	% Entradas
	entradas=Los datos del mantenimiento realizado.,
	% Precondiciones
	precondiciones=Haber accedido al sistema con un usuario válido y elegir la opción \textit{Realizar mantenimiento}.,
	% Salidas
	salidas=El mantenimiento queda registrado.
	% Postcondiciones en caso de éxito
	postexito={Se ha registrado el mantenimiento y, si procede, se ha actualizado el material disponible de la empresa en el inventario.},
	% Postcondiciones en caso de error
	posterror=El mantenimiento no ha quedado registrado y el sistema central no ha sufrido cambios.,
	% Actores
	actores=El personal mecánico de la compañía y la base de datos.,
}{
	% Tabla de secuencia normal del caso de uso
	\begin{tablasecuencias}
		1 & Extraer de la base de datos el listado de los mantenimientos programados para el usuario. Si error S-1. \\
		2 & Seleccionar un mantenimiento.\\
		3 & Introducir la información detallada del mantenimiento realizado (informe de la operación, y si ha podido completarse o no). Si error S-2.\\
		4 & Se muestra el listado del material mecánico disponible en el inventario. Si error S-3 \\
		5 & El usuario selecciona el material empleado en el mantenimiento, así como el cantidad de items de cada tipo utilizados. Si error S-4 \\
		6 & Mostrar un mensaje confirmando el registro del mantenimiento.
	\end{tablasecuencias}
}{
	% Tabla de secuencia con errores del caso de uso
	\begin{tablasecuencias}
		S-1 & No se ha podido extraer la información de la base de datos. Mostrar un mensaje de error por pantalla y volver a la página principal del sistema.\\
		S-2 & No se ha podido almacenar la información introducida. Mostrar un mensaje por pantalla indicándolo y regresar a la página anterior. \\
		S-3 & No se ha podido cargar el listado del material del inventario. Mostrar un mensaje de error por pantalla y volver a la página anterior. \\
		S-4 & No se ha podido almacenar los cambios introducidos. Mostrar un mensaje por pantalla indicándolo y regresar a la página anterior.
	\end{tablasecuencias}
}
