% Caso de uso: editar información económica.
% Obs: para escribir comas en el texto del primer parámetro se han de encerrar entre {}.

% Revisado por Cristina el día 12/03/2013

\casodeuso{
	% Nombre del caso de uso
	nombre=Editar información económica,
	% Objetivo
	objetivo=Modificar la información económica de la compañía debido a algún cambio relevante en la misma.,
	% Entradas
	entradas={La nueva información económica de la compañía (nuevos activos, pasivos, inversiones de la empresa\dots).},
	% Precondiciones
	precondiciones=El operador de la aplicación tiene credenciales que le habilitan para realizar dicha operación.,
	% Salidas
	salidas=La información económica de la compañía actualizada.,
	% Postcondiciones en caso de éxito
	postexito=Los cambios efectuados se guardan en la base de datos.,
	% Postcondiciones en caso de error
	posterror=No se realiza ningún cambio en el sistema.,
	% Actores
	actores=El personal administrativo de la compañía y la base de datos.,
}{
	% Tabla de secuencia normal del caso de uso
	\begin{tablasecuencias}
		1 & El empleado modifica los datos económicos de la compañía. Si error S-1.\\
		2 & Se almacenan los cambios en la base de datos. Si error S-2.
	\end{tablasecuencias}
}{
	% Tabla de secuencia con errores del caso de uso
	\begin{tablasecuencias}
		S-1 & Alguno de los datos introducidos no es válido. Vuelve a 1 de la secuencia normal de uso indicando los campos erróneos.\\
		S-2 & No se puede conectar con la base de datos, se muestra un mensaje de error por pantalla dando la opción de reintentar o volver al menú principal de la aplicación.
	\end{tablasecuencias}
}

