% Caso de uso: ver incidencias del sistema.
% Obs: para escribir comas en el texto del primer parámetro se han de encerrar entre {}.

% Cristina solicita mayor información acerca de este caso de uso

\casodeuso{
	% Nombre del caso de uso
	nombre=Ver incidencias del sistema,
	% Objetivo
	objetivo={Permite a los supervisores informáticos del sistema inspeccionar el correcto funcionamiento del \software y responder a comportamientos erróneos del mismo que hayan sido detectados, a partir de los registros que este genera.},
	% Entradas
	entradas={El nombre de registro que se quiere consultar.},
	% Precondiciones
	precondiciones={El operador de la aplicación está debidamente registrado y posee credenciales que le habilitan para realizar esta operación.},
	% Salidas
	salidas={Archivos de registro del sistema solicitados.},
	% Postcondiciones en caso de éxito
	postexito={El usuario habrá accedido a los registros del sistema.},
	% Postcondiciones en caso de error
	posterror={El sistema central no habrá sufrido cambios.},
	% Actores
	actores={Personal de \textit{Servicios Informáticos} y supervisores del sistema con autorización para ello.},
}{
	% Tabla de secuencia normal del caso de uso
	\begin{tablasecuencias}
		1 & Los archivos de registro del servidor central y los errores reportados por las aplicaciones cliente componen una serie de archivos de texto en el servidor central. Esos archivos podrán ser consultados dando acceso a su ubicación en el sistema de archivos del servidor central.
		% De momento, ¿para qué complicarlo?
	\end{tablasecuencias}
}{
	% Tabla de secuencia con errores del caso de uso
	\begin{tablasecuencias}
		S-1 & Las secuencias alternativas son las que determine el método de acceso al servidor.
	\end{tablasecuencias}
}
