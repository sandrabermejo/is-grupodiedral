% Caso de uso: Ver incidencias del sistema.
% Obs: para escribir comas en el texto del primer parámetro se han de encerrar entre {}.

% Revisado por Cristina el día 12/03/2013

% Los archivos de registro del servidor central y los errores reportados por las aplicaciones cliente componen una serie de archivos de texto en el servidor central. 
% Esos archivos podrán ser consultados dando acceso a su ubicación en el sistema de archivos del servidor central.

\casodeuso{
	% Nombre del caso de uso
	nombre=Ver incidencias del sistema,
	% Objetivo
	objetivo={Comprobar el correcto funcionamiento del \software y solucionar los posibles comportamientos erróneos del mismo que hayan sido detectados a partir de los registros que éste genera.},
	% Entradas
	entradas=El nombre de registro que se quiere consultar.,
	% Precondiciones
	precondiciones=El operador de la aplicación tiene credenciales que le habilitan para realizar esta operación.,
	% Salidas
	salidas=Archivos de registro del sistema solicitados.,
	% Postcondiciones en caso de éxito
	postexito=No se realiza ningún cambio en el sistema.,
	% Postcondiciones en caso de error
	posterror=No se realiza ningún cambio en el sistema.,
	% Actores
	actores=Personal de \textit{Servicios Informáticos} y supervisores del sistema.,
}{
	% Tabla de secuencia normal del caso de uso
	\begin{tablasecuencias}
		1 & El empleado consulta los archivos de registro del servidor central. Si error S-1.
	\end{tablasecuencias}
}{
	% Tabla de secuencia con errores del caso de uso
	\begin{tablasecuencias}
		S-1 & Las secuencias alternativas son las que determine el método de acceso al servidor.
	\end{tablasecuencias}
}
