% Caso de uso:Consultar nómina.
% Obs: para escribir comas en el texto del primer parámetro se han de encerrar entre {}.

\casodeuso{
	% Nombre del caso de uso
	nombre=Consultar nómina.,
	% Objetivo
	objetivo=Permite al personal de la compañía aérea consultar su nómina.,
	% Entradas
	entradas=Mes del que se desea obtener la nómina,
	% Precondiciones
	precondiciones=Haber accedido al sistema con un usuario válido y acceder a la opción \textit{Consultar nómina},
	% Salidas
	salidas=El desglose detallado de la nómina.,
	% Postcondiciones en caso de éxito
	postexito=El cliente puede acceder a la nómina seleccionada.,
	% Postcondiciones en caso de error
	posterror={Una pantalla de notificación de error, en la medida de lo posible.},
	% Actores
	actores={Empleados de la compañía, base de datos.}.
}{
	% Tabla de secuencia normal del caso de uso
	\begin{tablasecuencias}
		1 & Accede a los datos de empleado en la base de datos. Si error S-1.\\
		2 & Seleccionar el mes del que se desea consultar la nómina. Si error S-2.\\
		3 & Muestra	por pantalla el desglose de la nómina seleccionada por el empleado.
	\end{tablasecuencias}
}{
	% Tabla de secuencia con errores del caso de uso
	\begin{tablasecuencias}
		S-1 & No se puede acceder a la base de datos. Muestra un mensaje de error por pantalla y vuelve a la página principal del sistema.\\
		S-2 & No se han encontrado los datos de la nómina seleccionada. Muestra un mensaje de error por pantalla y vuelve al paso 2 de la secuencia normal.
	\end{tablasecuencias}
}
