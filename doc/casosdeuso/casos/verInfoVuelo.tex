% Caso de uso: ver información de vuelo
% Obs: para escribir comas en el texto del primer parámetro se han de encerrar entre {}.

\casodeuso{
	% Nombre del caso de uso
	nombre=Ver información de vuelo contratado,
	% Objetivo
	objetivo=Muestra al usuario la información del sus vuelos contratados.,
	% Entradas
	entradas=,
	% Precondiciones
	precondiciones={El usuario ha iniciado sesión correctamente en la interfaz web. Para que se muestre información, tener plaza reservada o comprada en un vuelo operado por la compañía.},
	% Salidas
	salidas={Información de la reserva como puede ser número de vuelo, fecha y hora, número de plaza, terminales de salida y de llegada.},
	% Postcondiciones en caso de éxito
	postexito=No se realizan cambios en el sistema.,
	% Postcondiciones en caso de error
	posterror=No se realizan cambios en el sistema.,
	% Actores
	actores=Cliente-usuario de interfaz web.
}{
	% Tabla de secuencia normal del caso de uso
	\begin{tablasecuencias}
		1 & Muestra una sucesión detalla de información sobre los vuelos contratados. Si no disponibles S-1.\\
		2 & Ofrece la opción de imprimir la información de los vuelos mostrados o de alguno de ellos en particular.
	\end{tablasecuencias}
}{
	% Tabla de secuencia con errores del caso de uso
	\begin{tablasecuencias}
		S-1 & Si no se puede conectar con la base de datos se muestra mensaje  de tipo \textit{información no disponible temporalmente} y se vuelve al paso 1 de la secuencia normal.
	\end{tablasecuencias}
}
