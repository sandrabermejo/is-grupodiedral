%
%	Plan de Gestión de Configuración del Software (GCS)
%

\documentclass[11pt, a4paper, twoside, titlepage]{article}
\usepackage[utf8x]{inputenc}
\usepackage[T1]{fontenc}
\usepackage[spanish]{babel}
\usepackage{lmodern}
\usepackage{anysize}
\usepackage[none]{hyphenat}
\usepackage[colorlinks, linkcolor=red]{hyperref}
\usepackage{glossaries}
\usepackage{glossaries-babel}
%\usepackage{lscape}
\usepackage[doc=gcs]{isdiedral}
%\usepackage{multicol}
%\usepackage{amsmath}
%\usepackage{float}

% Nombre del documento (para futuras referencias)
\newcommand*{\doctitle}{Plan de gestión de configuración del software}


%%% Configuraciones %%%
\marginsize{2.5cm}{2cm}{2cm}{2cm}

% Usa como familia tipográfica por defecto "Sans"
\renewcommand{\familydefault}{\sfdefault}

% Establece la profundidad hasta la cual se numeran los elementos de sección
\setcounter{secnumdepth}{4}

% Establece la profundidad de niveles de sección que aparece en el TOC
\setcounter{tocdepth}{4}

% Fija que la entrada del glosario se comporte como una subsección
\setglossarysection{subsection}

% Configuración de los encabezados
\encabezadodiedral{\doctitle}
\pagestyle{fancy}

\renewcommand*{\thepage}{\sffamily \roman{page}}


% Modelo copiado de los apuntes del tema 8 (páginas 93 a 95) IEEE Std. 730-2002

\title{\doctitle\\\textsl{Airline Common Environment}}
\author{Grupo Diedral}

% Metadatos del pdf
\hypersetup{
pdfinfo={
	Author={Grupo Diedral},
	Title={\doctitle},
	Subject={Airline Common Environment},
	Keywords={SQA;Airline Common Environment;Ingeniería del Software}
}
}

% Inclusión del glosario (gracias a David Peñas)
%
%	Plan GCS: Glosario
%

\PrerenderUnicode{ñ}
\PrerenderUnicode{ó}
\PrerenderUnicode{í}


\makeglossaries

\begin{document}

	% Cita inicial
	\fijacitainicial{Nada es permanente a excepción del cambio}{Heráclito de Éfeso, $\sim$ 500 a.C.}

	% Portada
	\portadaace{\doctitle}{2.0}

	\tableofcontents
	\newpage

	\iniciarnumeraciondiedral
	
	\section{Introducción} % Natalia
		\subsection{Propósito}
			El propósito del Plan de \gls{config_software} es establecer y mantener la integridad de los productos a desarrollar a través del proceso de software asociado. 
		\subsection{Alcance}
		\begin{itemize}
			\item Se \textit{identificarán y definirán los elementos del sistema}, controlando el posible cambio de estos durante todo su ciclo de vida, y se verificarán que sean correctos y completos.
			\item Se establecerá un protocolo de \textit{gestión de cambios} de estos elementos.
			\item Será la base de partida para que se pueda generar una buena \textit{calidad de software}.
			\item Ayudará a la comunicación y organización en el grupo de desarrollo o con personas ajenas al proyecto.
		\end{itemize}
			
		%\subsection{Definición de términos clave} Esto es el Glosario
		\subsection{Referencia}
			\nocite{IEEE828-1998}
			\nocite{PSMAN}
			Ver sección de {\itshape Referencias} al final del documento.
		
	\section{Gestión de la GCS} % Sandra
		\subsection{Organización}
		\subsection{Responsabilidades GCS}
		\subsection{Políticas, directivas y procedimientos aplicables}
	\section{Actividades de la GCS}
		\subsection{Identificación de la configuración} % Natalia
			En la siguiente sección se va a realizar la Identificación, nombrado y adquisición de \gls{ECS}.

			\subsubsection{Casos de Uso}
				Completar.

			\subsubsection{Especificación de requisitos Software}

				\begin{enumerate}
					\item {\ithshape \bfseries Descripción.}
						\begin{itemize}
							\item \textit{Tipo de ECS:} Documento.
							\item \textit{Identificador proyecto:} SRS.
							\item \textit{Información de la versión y/o cambio:} Se han cambiado principalmente errores derivados de modificaciones en el documento de Casos de Uso. Además, también se han corregido errores de redacción, pequeñas contradicciones entre algunas partes del proyecto y se ha mejorado el documento gracias a revisiones de terceras personas desde otro punto de vista diferente.
						\end{itemize}

					\item {\itshape \bfseries Lista de recursos.}
						Principalmente el documento de la Especificación de Requisitos Software tiene como entidades requeridas al documento de Casos de Uso y al Prototipo, es decir, depende directamente de ellos.
				\end{enumerate}

			\subsubsection{Prototipo Gestión Externa}
				Completar.

			\subsubsection{Prototipo Gestión Interna}
				Completar.

			\subsubsection{Plan de Proyecto}
				Completar.

			\subsubsection{Plan de Gestión de Configuración}
				Completar.

			\subsubsection{Plan de Gestión de Calidad}
				Completar.

			\subsubsection{Otros}
				Algunos ECS importantes que no se van han desarrollado aún y que por tanto no se van a detallar en este documento son: Diagramas de colaboración de análisis, el documento de diseño, el manual de usuario, el plan de pruebas del sistema, documentos de diseño de base de datos, especificaciones de prueba del sistema\ldots

		\subsection{Contabilidad de estado de configuración} % Sandra
	\section{Recursos de la GCS} % Natalia
	\section{Glosario}
		\printglossaries

	\newpage
	\bibliography{gcs}
	\bibliographystyle{plain}
\end{document}
