%
%	Documento de especificación de requisitos software (SRS).
%

\documentclass[11pt, a4paper, twoside]{report}
\usepackage[utf8x]{inputenc}
\usepackage[T1]{fontenc}
\usepackage[spanish]{babel}
\usepackage{lmodern}
\usepackage{anysize}
\usepackage{fancyhdr}
\usepackage{titletoc}
\usepackage{amsmath}
\usepackage[none]{hyphenat}
\usepackage{graphicx}
\usepackage[colorlinks, linkcolor=red]{hyperref}
\usepackage{glossaries}
\usepackage{glossaries-babel}
\usepackage{isdiedral}
\usepackage{todonotes}

%%% Configuraciones %%%
\marginsize{2.5cm}{2cm}{2cm}{2cm}

% Usa como familia tipográfica por defecto "Sans"
\renewcommand{\familydefault}{\sfdefault}

% Establece la profundidad hasta la cual se numeran los elementos de sección
\setcounter{secnumdepth}{4}

% Establece la profundidad de niveles de sección que aparece en el TOC
%\setcounter{tocdepth}{4}

% Para las clases "book" o semejantes desactiva la impresión del número de capítulo
\renewcommand*\thesection{\arabic{section}}

% Cambia el nombre del TOC a "Índice" (pues en "report" se le denomina por defecto "Índice general")
\addto{\captionsspanish}{\renewcommand*{\contentsname}{Índice}}

% Fija que la entrada del glosario se comporte como una subsección
\setglossarysection{subsection}

% Configuración de los encabezados
\encabezadodiedral{Especificación de requisitos software}
\pagestyle{fancy}

\renewcommand*{\thepage}{\sffamily \roman{page}}


% Modelo copiado de los apuntes del tema 3 (páginas 69 a 98)

\title{Especificación de requisitos software\\\textsl{Airline Common Environment}}
\author{Grupo Diedral}

% Metadatos del pdf
\hypersetup{
pdfinfo={
	Author={Grupo Diedral},
	Title={Especificación de requisitos software},
	Subject={Airline Common Environment},
	Keywords={SRS;Airline Common Environment;Ingeniería del Software}
}
}

% Inclusión del glosario (gracias a David Peñas)
%
%	SRS: Glosario
%

\PrerenderUnicode{ñ}
\PrerenderUnicode{ó}
\PrerenderUnicode{í}

\newglossaryentry{la_app}{
	name=la aplicación,
	description={producto sofware descrito en el documento: \textit{Airline Common Environment}.},
	sort=aplicación
}

\newglossaryentry{el_programa}{
	name=el programa,
	description={véase \gls{la_app}.},
	sort=aplicación
}

\newglossaryentry{Internet}{
	name=Internet,
	description={conjunto descentralizado y globalizado de redes de comunicación interconectadas, cuyo origen se sitúa en la puesta en marcha de la \textit{Advanced Research Projects Agency Network} (ARPANET) en 1969 y que se popularizó a finales del siglo XX y principios de XXI}
}

\newglossaryentry{CNED_3}{
	name=superior a educación secundaria,
	description={\textit{referido al nivel de estudios}, en términos de la \gls{CNED} y la \gls{CINE} (niveles 3 y 4), habiendo cursado 2ª etapa de educación secundaria o postsecundaria no superior. En el texto, salvo que se diga lo contrario, incluye también niveles superiores}
}

\newglossaryentry{CNED_5}{
	name=educación superior o doctorado,
	description={\textit{referido al nivel de estudios}, en términos de la \gls{CNED} y la \gls{CINE} (niveles 5 y 6), habiendo cursado 1º o 2º ciclo de educación superior, o doctorado}
}

\newglossaryentry{CVV2}{
	name=código CVV2/CVC2,
	description={El CVV2/CVC2 es un código de seguridad de tres dígitos que se encuentra impreso al dorso de las tarjetas de crédito. Este código se utiliza para determinar que usted está en posesión de la tarjeta empleada para el pago. Todas las tarjetas MasterCard y Visa, tanto de crédito como de débito, deben llevar un CVV2.}
}

\newglossaryentry{base_de_datos}{
	name=base de datos,
	description={Una base de datos o banco de datos es un conjunto de datos pertenecientes a un mismo contexto y almacenados sistemáticamente para su posterior uso}	
}

\newglossaryentry{numero_de_vuelo}{
	name=número de vuelo,
	description={código de vuelo acorde el formato definido por la \gls{IATA} en \cite{SSIM} omitiendo la identificación de la compañía}	
}

\newglossaryentry{combobox}{
	name=cuadro desplegable,
	description={control de interfaz gráfica que muestra un valor comprendido en una colección finita, editable o no, y permite escoger otros valores de la colección mostrando una lista desplegable}
}

\newglossaryentry{captcha}{
	name=captcha,
	description={(\textit{Completely Automated Public Turing test to tell Computers and Humans Apart}, Prueba de Turing pública y automática para diferenciar máquinas y humanos) prueba para comprobar que un usuario es humano que consiste en hacerle escribir el texto representado en una imagen distorsionada de forma que para una máquina sería difícil descifrar}
}

\newglossaryentry{Portable_Document_Format}{
	name=Portable Document Format,
	description={(Formato de Documento Portátil) formato de almacenamiento de documentos digitales independiente de plataformas de \software o \hardware diseñado originalmente por \textit{Adobe Systems} y estandarizado en el ISO 32000-1:2008}
}
\newglossaryentry{C++}{
	name=C++,
	description={C++ es un lenguaje de programación diseñado a mediados de los años 1980 por Bjarne Stroustrup. La intención de su creación fue el extender al exitoso lenguaje de programación C con mecanismos que permitan la manipulación de objetos}
}
\newglossaryentry{Qt}{
	name=Qt,
	description={Qt es una biblioteca multiplataforma ampliamente usada para desarrollar aplicaciones con interfaz gráfica de usuario, así como también para el desarrollo de programas sin interfaz gráfica, como herramientas para la línea de comandos y consolas para servidores},
}
\newglossaryentry{HTML5}{
	name=HTML5,
	description={HTML5 (HyperText Markup Language, versión 5) es la quinta revisión importante del lenguaje básico de la World Wide Web, HTML}
}
\newglossaryentry{Usabilidad}{
	name=Usabilidad,
	description={ Neologismo usabilidad (del inglés usability -facilidad de uso-)}
}
\newglossaryentry{arrays}{
	name=array,
	description={En programación, una matriz o vector (llamados en inglés arrays) es una zona de almacenamiento continuo, que contiene una serie de elementos del mismo tipo, los elementos de la matriz}
}
\newglossaryentry{RAE}{
	name=RAE,
	description={(Real Academia Española) es una institución cultural con sede en Madrid. Junto con otras veintiuna Academias correspondientes en sendos países donde se habla español, conforman la Asociación de Academias de la Lengua Española}
}
\newglossaryentry{PayPal}{
	name=PayPal,
	description={PayPal es una empresa estadounidense, propiedad de eBay, perteneciente al sector del comercio electrónico por Internet que permite la transferencia de dinero entre usuarios que tengan correo electrónico, una alternativa al tradicional método en papel como los cheques o giros postales.}
}
\newglossaryentry{QuickPay}{
	name=Quick Pay,
	description={Quick Pay es un servicio que permite a clientes nacionales e internacionales pagar directamente por vía electrónica, transferiendo pagos electrónicamente y dando la opción, cuando sea posible, de convertir automáticamente el dinero a la moneda local}
}
\newglossaryentry{WesternUnion}{
	name=Western Union,
	description={Western Union es una compañía estadounidense que ofrece servicios financieros y de comunicación}
}
\newglossaryentry{Login}{
	name=Login,
	description = {En el ámbito de seguridad informática, login o logon (en español ingresar o entrar) es el proceso mediante el cual se controla el acceso individual a un sistema informático mediante la identificación del usuario utilizando credenciales provistas por el usuario}
}

\newglossaryentry{Linux}{
	name=Linux,
	description = {En el texto, Linux no hace referencia al núcleo Linux, sino a distribuciones basadas en Linux generalmente utilizadas (frecuentemente con el \software del proyecto GNU), como Debian, Mandrake, Red Hat o SuSE.}
}

\newglossaryentry{Microsoft Windows}{
	name={Microsoft Windows},
	description = {Familia de sistemas operativos desarrollados y comercializados por multinacional estadounidense Microsoft, que comienza con la versión 1.0 (1985) y actualmente termina con la versión 8 (2012)}
}


\newacronym{ACE}{ACE}{Airline Common Environment}
\newacronym{CNED}{CNED}{\textit{Clasificación Nacional de Educación} (España)}
\newacronym{IATA}{IATA}{\textit{Asociación Internacional de Transporte Aéreo}}
\newacronym{INE}{INE}{\textit{Instituto Nacional de Estadística} (España)}
\newacronym{CINE}{CINE-97}{\textit{Clasificación Internacional Normalizada de la Educación}}
\newacronym{NIC}{NIC}{\textit{Normas Internacionales de Contabilidad}}
\newacronym{PTLA}{PTLA}{Piloto de Tranporte de Línea Aérea}
\newacronym{ATPL}{ATPL}{\textit{Airline Transport Pilot License}, veáse \gls{PTLA}}
\newacronym{UTF8}{UTF-8}{\textit{Unicode Tranformation Format 8-bit}}
\newacronym{DNI}{DNI}{\textit{Documento Nacional de Identidad} (España)}
\newacronym{NIF}{NIF}{\textit{Número de Identificación Fiscal} (España)}
\newacronym{NIE}{NIE}{\textit{Número de Identidad de Extranjero} (España)}
\newacronym{PDF}{PDF}{\textit{\gls{Portable_Document_Format}}}
\newacronym{GCC}{GCC}{\textit{GNU Compiler Collection}}

\makeglossaries

\begin{document}
	% Portada
	\portadaace{Especificación de requisitos software}

	\tableofcontents
	\newpage
	\thispagestyle{plain}

	% Tabla de cambios
	\begin{scriptsize}
	\begin{tablacambios}
		1 & 15 de Noviembre de 2012 & \begin{tabular}{l}Cristina Alonso\\Natalia Angulo\\Sandra Bermejo\\Juan Andrés Claramunt\\Rubén Rafael Rubio \\\textit{(Todo el grupo)}\end{tabular} & Realización completa de la Introducción al proyecto.\\ \hline
		2 & 19 de Noviembre de 2012 & Todo el grupo & Realización de los primeros diagramas de los casos de uso y preparar la charla sobre el proyecto del 22 de noviembre de 2012.\\ \hline
		3 & 27 de Noviembre de 2012 & Rubén Rafael Rubio & Preparación y distribución de la plantilla de \LaTeX{} para los Casos de Uso.\\ \hline
		4 & 28 de Noviembre de 2012 & Todo el grupo & Sorteo de los Casos de Uso y reparto del trabajo.\\ \hline
		5 & 6 de Diciembre de 2012 & Rubén Rafael Rubio & Preparación y distribución de la plantilla de \LaTeX{} para los SRS.\\ \hline
		6 & 11 de Diciembre de 2012 & Todo el grupo & Puesta en común de los Casos de Uso y reparto de estos para realizar una revisión desde un punto de vista diferente al que los ha hecho. Discusión sobre el entorno y lenguaje en el que vamos a realizar el prototipo (aplicación).\\ \hline
		7 & 12 de Diciembre de 2012 & Todo el grupo & Revisión desde casa de los Casos de Uso y posterior modificación de algunos para mejorarlos. Obtención de nuevos Casos de Uso mientras se hacía esta revisión y realización de ellos.\\ \hline
		8 & 13 de Diciembre de 2012 & Todo el grupo & Puesta en común de ideas y explicaciones de los SRS. Planificación de los SRS y de las pantallas para el prototipo (aplicación). Reparto del trabajo.\\ \hline
		9 & 14 de Diciembre de 2012 & Todo el grupo & Realización en casa de las pantallas para el prototipo y del documento de los SRS.\\ \hline
		10 & 15 de Diciembre de 2012 & Todo el grupo & (Reunión durante todo el día). Puesta en común de las pantallas para el prototipo y organización del mismo para que Rubén Rafael Rubio las juntase todas en el prototipo (aplicación). Revisión general de los SRS y reparto de trabajo para una segunda revisión individual de estos. Diagramas con el nuevo programa recomendado por el profesor ARGO UML realizados por Cristina Alonso. Discusión sobre el prototipo (página Web) y reparto de dibujos de las diversas páginas correspondientes con los Casos de Uso.\\ \hline
		11 & 16 de Diciembre de 2012 & Todo el grupo & Realización individual de todo el trabajo repartido el día 15 de diciembre de 2012. \\ \hline
		12 & 17 de Diciembre de 2012 & Todo el grupo & Revisión de todo el grupo al prototipo (aplicación) para ver cómo ha quedado y proposición de nuevas ideas y sugerencias para la futura aplicación. Puesta en común de lo nuevo de los SRS. Nueva revisión de los SRS realizada por Juan Andrés Claramunt y Sandra Bermejo. Vista común de los dibujos de las pantallas para el prototipo Web. Diseño de dicho prototipo (página Web) realizado por Natalia Angulo. \\ \hline
		13 & 18 de Diciembre de 2012 & Todo el grupo & Realización de la presentación en Beamer (\LaTeX) y organización y pequeña simulación de dicha presentación. Realización individual de las tareas pendientes.\\ \hline
		14 & 19 de Diciembre de 2012 & Todo el grupo & Revisión de todo el grupo al prototipo (página Web) para ver cómo ha quedado y proposición de nuevas ideas y sugerencias para la futura página Web. Última revisión general antes de la exposición y preparación individual de ésta.\\ \hline
		15 & 20 de Diciembre de 2012 & Todo el grupo & Presentación de los Casos de Uso, los SRS y los prototipos (aplicación y página Web).
	\end{tablacambios}
	\end{scriptsize}
	\newpage
	\iniciarnumeraciondiedral
		
	\section{Introducción}
		\subsection{Propósito}
			El objetivo de este documento es especificar la naturaleza y detalles de realización de software, con el fin de organizar el proceso de desarrollo y servir como referencia a los clientes.

		\subsection{Alcance}
			El ámbito de \textit{Airline Common Environment} \glsunset{ACE} (\gls{ACE}) comprende la gestión interna y externa de una compañía aérea dedicada al transporte de viajeros. El sistema de gestión interna estará destinado a facilitar y optimizar la actividad laboral del empleado, así como permitir a la compañía el control de recursos tanto materiales, financieros como humanos. Sin embargo, no corresponde a este proyecto sustituir al empleado en procesos como pueden ser la la planificación de rutas de vuelo, horarios o revisiones. \\

			Por otro lado, el sistema de gestión externa abre un canal de comunicación fluido entre la empresa y el viajero, manteniendo actualizada la información referente a vuelos, billetes, ofertas y dando la posibilidad de realizar descuentos personalizados en sus compras.\\

			El uso de esta aplicación proporcionará a los clientes una mayor eficiencia y organización a la par que una interfaz cómoda y completa para desarrollar su actividad como empresa. 
			
		\subsection{Definiciones acrónimos y abreviaturas}
			Véase~\nameref{srs:glosario}.
		\subsection{Referencias}
			Véase sección Referencias al final del texto. Los estándares de IEEE se pueden obtener de \url{http://standards.ieee.org/findstds/standard/}.
		\subsection{Resumen}
			La especificación de requisitos trata, por tanto, de describir completamente el comportamiento de la aplicación que se va a desarrollar. Este documento incluye la especificación de los casos de uso, mostrando así todas las interacciones que se puedan llevar a cabo entre el cliente y el software. \\

			Además, este documento contiene la descripción general del producto y requisitos específicos como: funcionalidades que pueda realizar la aplicación en un futuro, requisitos de rendimiento y de la base de datos y restricciones de diseño o implementación.
			
	\section{Descripción general}
		\subsection{Perspectiva del producto}
			ACE permitirá la gestión completa, tanto de la parte interna como de la parte externa, de una compañía aérea, sin que sea necesario, en un principio, integrarla dentro de otro sistema. \\
			
			La interfaz interna de ACE facilitará al personal de la compañía aérea acceder a sus datos laborales actualizados, ya que éstos estarán informatizados y podrán ser consultados a través de la aplicación, así como a ciertos datos de la compañía de interés general. Además, a través de su cuenta personal y en función del cargo que ejerza dentro de la compañía, la aplicación permitirá al usuario realizar unas u otras tareas para mejorar la programación laboral, optimizando así el tiempo y el esfuerzo del trabajador. \\
			
			En cuanto a la parte externa, el sitio web de la compañía permitirá al usuario la compra de billetes, ofreciéndole la información de todos los vuelos disponibles según sus criterios de búsqueda, así como mostrándole las últimas ofertas y permitiendo el pago a través de la web.
			
		\subsection{Funciones del producto}
			Las principales funciones del producto están destinadas a gestionar de forma centralizada todo lo relativo a las actividades normales de una compañía aérea. Básicamente éstas se pueden clasificar en dos grandes grupos, que serían los siguientes. \\
			
			\begin{itemize}
				\item Funciones relacionadas con el funcionamiento interno de la compañía:
					\begin{itemize}
						\item \textit{Financieras}: información de ventas, gastos, valoración de inventario.
						\item \textit{Administrativas}: confección y consultas de nóminas de empleados, pedidos a proveedores.
						\item \textit{Mantenimiento}: programación de revisiones de vehículos, inventariado de herramientas y repuestos.
						\item \textit{Planificación}: creación y consultas de planes y rutas de vuelos, cuadrar horarios de personal.			
					\end{itemize}
				\item Funciones relativas a la interacción con el cliente final:
					\begin{itemize}
						\item \textit{Venta}: Venta de billetes.
						\item \textit{Oferta}: Ofertas, promociones y descuentos personalizados.
					\end{itemize}
			\end{itemize}
			
		\subsection{Características del usuario}
			El perfil del usuario de esta aplicación no es homogéneo. Una primera clasificación surge por la mayor o menor relación con la compañía.\\
			En cuanto a los usuarios ajenos a la empresa se cuenta con:
	
			\begin{itemize}
				\item \textit{Cliente}: no hay restricción alguna en el tipo de cliente que puede acceder a la aplicación externa, aunque sí a la hora de realizar el pago, pues este cliente debería de tener como mínimo la mayoría de edad. Se puede suponer cierto conocimiento sobre realización de operaciones en \gls{Internet}, pero no se puede asumir dominio sobre la materia. El perfil educacional del cliente, en general, no incluye conocimiento sobre el marco legal y organizativo del transporte aéreo, así como tampoco se puede asumir un nivel educativo mínimo. Si bien, el nivel socio-cultural y económico de los usuarios del transporte aéreo, especialmente aquellos que hacen uso frecuente del mismo, es estadísticamente medio o alto.
			\end{itemize}

			El personal de la empresa es previsiblemente un usuario seleccionado por la empresa por disponer de suficiente conocimiento o experiencia en la materia de su ocupación. Adicionalmente, este usuario puede haber recibido formación por parte de la compañía sobre su función o la de otros componentes de la organización. Como consecuencia el perfil de los usuarios internos está más controlado y definido:
			\begin{itemize}
				\item \textit{Personal administrativo}: es en general un usuario versátil, es decir, puede cambiar de ocupación con cierta facilidad. El nivel educativo usualmente será \gls{CNED_3}.
				\item \textit{Personal de a bordo}: la formación y requisitos del personal de a bordo no sólo está definida por la compañía, sino que está regulada por organismos internacionales y nacionales de los países donde la compañía ofrece sus servicios. Los pilotos \glsunset{PTLA} (\gls{PTLA}) están sometidos a una serie de requisitos como edad mínima de 21 años, haber conseguido una licencia mediante examen oficial regulado y disponer de un mínimo de experiencia de vuelo. El nivel educativo será \gls{CNED_3}.
				\item \textit{Personal de mantenimiento}: la formación del personal de mantenimiento en tierra suele ser \gls{CNED_3}. En las exigencias de acceso de algunos empleados se habrá solicitado \gls{CNED_5}.
				\item \textit{Directivos}: el perfil de los directivos presume cierta formación y experiencia en la gestión empresarial y conocimiento de la actividad de la empresa. El nivel educativo usual es \gls{CNED_3} o \gls{CNED_5}.
			\end{itemize}

			Con generalidad, la \textit{alfabetización informática} de los usuarios será suficiente para hacer uso eficiente de la aplicación.

		\subsection{Restricciones}
			% Descripción general de cualquier ¿otro? elemento que pueda limtar las opciones de los desarrolladores. En la primera versión incluimos -erróneamente- estas opciones.

			La aplicación de la interfaz interna debe poder estar disponible para al menos los siguientes plataformas: \gls{Linux} (32 y 64 bits); \gls{Microsoft Windows} \textit{XP},  \textit{Vista} y \textit{7} (32 y 64 bits). Se considerará la disponibilidad para \textit{Apple Mac OS X 10.X} y \textit{Solaris}. Será necesario que el diseño de la aplicación cumpla unos mínimos de reutilización y facilidad de adaptación, por lo cual deberá ser codificado de acuerdo a los estándares reconocidos y los componentes no estándar o con riesgo de obsolescencia a corto plazo habrán de estar debidamente aislados del resto. El programa a su vez debe proporcionar los mecanismos necesarios para la realización de auditorías y controles de su funcionamiento, así como de su uso, a la par que garantizar el secreto de los datos manipulados, protegiendo el sistema frente a intentos de acceso externo o espionaje industrial y respetando las regulaciones que protegen los datos de carácter personal.\\

			El sistema \software central debe asegurar la posibilidad de realizar operaciones en paralelo hasta un máximo establecido. La fiabilidad de los componentes críticos debe ser verificable y deben incluir mecanismos de autodiagnóstico.\\

			Para lograr lo pretendido en el \software de \textit{Gestión Interna} se considera utilizar \gls{C++} (C++98) con la biblioteca de interfaces gráficas \gls{Qt} que ofrece soporte para las plataformas requeridas\footnote{Información sobre soporte de Qt en \url{http://doc.qt.digia.com/qt/supported-platforms.html}}. El estándar C++11 puede no resultar conveniente debido a que no es soportado por todos los compiladores en la actualidad. La opción de utilizar \textit{Java} también es compatible con las restricciones de plataforma anteriormente especificadas. La interfaz de \textit{Gestión Externa} puede ser programada en \gls{HTML5}, que si bien está en fase experimental, es soportado en gran medida por los principales navegadores, disponibles en todas las plataformas requeridas.

		\subsection{Supuestos y dependencias}
			Los requisitos presuponen diversos factores que pueden cambiar en cualquier momento y, por tanto, afectar a dichos requisitos. Por ejemplo, si cambian ciertos detalles técnicos del sistema operativo, será necesario revisar y modificar los requisitos de nuestro proyecto. El sistema también depende del formato de la base de datos donde se almacena la información. \\
			
			Además, podría darse la situación de que el cliente solicitara alguna modificación sobre cierto aspecto de la aplicación o incluso una nueva funcionalidad de ésta. Para ello, el proyecto tiene que estar preparado para aceptar cualquier modificación con relativa facilidad, dependiendo de la magnitud y complejidad de la misma. Si llegase a producirse esta petición por parte del cliente, también se deberán modificar todos los documentos actualizando la información y los nuevos cambios producidos.

		\subsection{Requisitos futuros}
			Futuras versiones de la aplicación ACE podrán integrar mejoras para la gestión interna de la compañía tales como modificar el puesto dentro de la empresa de un empleado y, en consecuencia, el nivel de permisos del usuario, o crear estadísticas basadas en los datos de entrada y salida del inventario que permitan prever las necesidades de comprar material y optimizar los recursos. \\
		
			Para la gestión externa, ACE podrá incorporar más opciones para la búsqueda de billetes tales como optimizar la ruta según el menor número de escalas, menor duración total del viaje o menor número de kilómetros recorridos. También integrará opciones para añadir o eliminar aeropuertos, orígenes y destinos de vuelo. Además, permitirá al usuario nuevas formas de pago como la transferencia bancaria, \gls{PayPal} o el \gls{QuickPay} de \gls{WesternUnion}.

	\section{Requisitos específicos}
		\subsection{Interfaces externas}
			Las interfaces del usuario, tanto para la gestión interna como para la externa, se ajustarán al perfil del usuario promedio, así como a las restricciones específicas que imponga la empresa. En la gestión interna, cada usuario podrá acceder a una parte de la interfaz que será específica para él según el trabajo que realice en la compañía. Esta interfaz admitirá diferentes datos introducidos por el usuario, los cuales deberán ser válidos según los estándares establecidos por la empresa y que dependerán en gran medida del país donde se vaya a utilizar este software. Por defecto, en esta primera versión se utilizará el euro como unidad monetaria, las entradas numéricas deberán estar dentro de un rango determinado y admitirán un máximo de 2 dígitos decimales, y, para las entradas de texto, se admitirán carácteres alfabéticos latinos, con posibles carácteres especiales añadidos. Para la interfaz externa, se aplicarán estos mismos criterios y se comprobará que los datos introducidos por el cliente para efectuar la compra se ajustan a ellos. El usuario deberá identificarse obligatoriamente cuando llegue el momento de efectuar el pago.
			

		\subsubsection{Formato de entrada y salida}
			\subsubsubsection{Fecha}

				A efectos de visualización y entrada la fecha ha de ajustarse, según configuración, a los patrones \verb|DD/MM/AAAA| o \verb|AAAA/MM/DD|\footnote{De acuerdo al estándar ISO 8601.}. En el caso de la visualización se puede incluir como prefijo el día de la semana. Siendo los campos enteros positivos, se ha de cumplir:
				\begin{itemize}
					\item $0 \leq \verb|MM| \leq 12$
					\item $\verb|MM| \in \{1, 3, 5, 7, 8, 10, 12\} \Rightarrow 0 \leq \verb|DD| \leq 31$
					\item $\verb|MM| \in \{4, 6, 9, 11\} \Rightarrow 0 \leq \verb|DD| \leq 30$ 
					\item $\verb|MM| = 2 \wedge ((4 | \verb|AAAA| \wedge 100 \not| \verb|AAAA|) \vee (100 | \verb|AAAA| \wedge 400 | \verb|AAAA|)) \Rightarrow 0 \leq \verb|DD| \leq 29$ (año bisiesto)
					\item $\verb|MM| = 2$ y en caso contrario $\Rightarrow 0 \leq \verb|DD| \leq 28$
				\end{itemize}
			La aplicación también puede producir expresiones de fecha \textit{redactadas} en el idioma configurado tales como \verb|DD de MM de AAAA| pudiendo ser \verb|MM| el nombre del mes en lenguaje natural.\\

			A efectos de representación interna la fecha seguirá la directrices del estándar ISO 8601, se denotará mediante el formato de texto \verb|AAAAMMDD| en las condiciones anteriores. También se podrá utilizar un formato interno de fecha orientado a objetos~\footnote{Por ejemplo el tipo {\texttt Date} de las bibliotecas \textit{Boost}.}.

				% Código de identificación personal
				\subsubsubsection{Código de identificación personal}
					En lo concerniente a este programa se considerará como código identificador personal el \nameref{srs:nif}, referido en esta sección. El \software deberá reconocer los NIF correspondientes a personas físicas. Estas secuencias no necesariamente habrán de corresponder a un identificador oficial (si bien cuando sea posible sería recomendable). Además de los formatos que define la especificación de NIF, se define el siguiente formato \verb|L0000000A| (letra \verb|L| + 7 números + letra de control) para uso interno, validado según las mismas reglas. \label{srs:idpersonal}

				% NIF
				\subsubsubsection{\gls{NIF}} \label{srs:nif}
					Secuencia de identificación para las personas físicas y jurídicas en España. Está regulado por el Real Decreto 1065/2007. Las diferentes formatos son los siguientes:

				\begin{itemize}
					\item \verb|00000000A| (8 números + dígito de control) para españoles con \gls{DNI}. El dígito de control es un carácter alfabético.
					\item \verb|A0000000A| (letra inicial + 7 números + dígito de control) para personas físicas salvo las del apartado anterior. La letra inicial indica diferentes clasificaciones de personas físicas. Las letras disponibles son \verb|K| (españoles menores de 14 años sin DNI); \verb|L|; \verb|M| y; \verb|X| , \verb|Y| y \verb|Z| (extranjeros identificados residentes en España, denominado \gls{NIE}).
					\item \verb|X00000000A| (\verb|X| + 8 números + dígito de control) con el mismo significado que el correspondiente de 7 dígitos, mantenido por razones históricas. 
					\item \verb:A0000000(A|0): (letra inicial + 7 números + dígito de control) para personas jurídicas. La letra inicial es una de las siguientes $\{ \verb|A|, \verb|B|, \verb|C|, \verb|D|, \verb|E|, \verb|F|, \verb|G|, \verb|H|, \verb|J|, \verb|P|, \verb|Q|, \verb|R|, \verb|S|, \verb|U|, \verb|V|, \verb|N|, \verb|W| \}$ que determina la naturaleza jurídica de la persona. El dígito de control puede ser letra o número dependiendo de la letra inicial.\\
				 \end{itemize}
				
					Existen reglas de correspondencia para determinar el dígito de control:
				\begin{itemize}
					\item {\itshape DNI y NIE}: sea $n$ el número que aparece en NIF.

						\begin{equation*}
							\text{Siendo } s := \sum_{i=1}^8 \left( \frac{n}{10^{i-1}} \mod 10 \right) \text{ la suma de las cifras}
						\end{equation*}

						\begin{equation*}
							\text{Se define } m := 
							\begin{cases}
								s & \text{si es DNI o NIE-{\texttt X}}\\
								s + 1 & \text{si es NIE-{\texttt Y}}\\
								s + 2 & \text{si es NIE-{\texttt Z}}\\
							\end{cases}
						\end{equation*}

						Entonces el dígito de control es la letra que corresponde al número obtenido $m \mod 23$ en la siguiente tabla\footnote{La definición oficial no es la misma pero es equivalente.}: 

						{\small
						\begin{equation*}
							\setcounter{MaxMatrixCols}{12}
							\begin{pmatrix}
								0 & 1 & 2 & 3 & 4 & 5 & 6 & 7 & 8 & 9 & 10 & 11\\
								T & R & W & A & G & M & Y & F & P & D & X  & B  \\
								\\
								12 & 13 & 14 & 15 & 16 & 17 & 18 & 19 & 20 & 21 & 22 &\\
								N  & J  & Z  & S  & Q  & V  & H  & L  & C  & K  & E &\\ 
							\end{pmatrix}
						\end{equation*}
						}

					\item {\itshape Otros NIF}: sea $n$ el número del NIF (correspondiente a los 7 primeros caracteres numéricos). Se procede de la siguiente forma:	
						% Fuente Wikipedia.
						\begin{enumerate}
							\item Se suman las dígitos de las posiciones pares (se considera que la posición de las unidades es impar\footnote{Téngase en cuenta que así, las posiciones pares son las que están multiplicadas por una potencia de 10 impar.}).
								\begin{equation*}
									a_1 := \sum_{i=0}^3 \left( \frac{n}{10^{2i + 1}} \mod 10 \right)
								\end{equation*}
							\item Para cada uno de los dígitos de la posiciones impares, se multiplica el dígito por dos y se suman las cifras obtenidas. Se acumulan todas las sumas anteriores.
								\begin{equation*}
									a_2 := \sum_{i=0}^3 \left( \left( 2 \left( \frac{n}{10^{2i}} \mod 10\right) \right) \mod 10 + \frac{ 2 \left( n / 10^{2i} \mod 10 \right)}{10} \right)
								\end{equation*}

							\item Se suman los resultados de los epígrafes anteriores.
								\begin{equation*}
									a_3 := a_1 + a_2
								\end{equation*}
							\item Se resta a 10 el dígito de las unidades del número obtenido en el apartado anterior si es mayor que 0.
								\begin{equation*}
									a_4 := (10 - a_3 \mod 10) \mod 10
								\end{equation*}
						\end{enumerate}

	
					El resultado del último apartado es el dígito de control si éste ha de ser un número. En caso contrario es una de las siguientes letras según la siguiente correspondencia:
						{\small
						\begin{equation*}
							\setcounter{MaxMatrixCols}{10}
							\begin{pmatrix}
								1 & 2 & 3 & 4 & 5 & 6 & 7 & 8 & 9 & 0 \\
								A & B & C & D & E & F & G & H & I & J
							\end{pmatrix}
						\end{equation*}
						}
				\end{itemize}

				% Dirección
				\subsubsubsection{Dirección postal}				
					En lo referente a este proyecto de \software una dirección postal constará de los siguientes campos:
					\begin{itemize}
						\item \textbf{Tipo de vía}: será una opción entre un rango de opciones disponibles (configurables). Para un correcto funcionamiento deberían considerarse los siguientes tipos admitidos por Correos: \textit{alameda}, \textit{autopista}, \textit{autovía}, \textit{avenida}, \textit{bulevar}, \textit{camino}, \textit{carretera}, \textit{glorieta}, \textit{paseo}, \textit{plaza}, \textit{pasaje}, \textit{ronda}, \textit{sector}, \textit{travesía}, \textit{otros}, \textit{vía} (numerados por ese orden a partir de 1).
						\item \textbf{Nombre de la vía} o simplemente dirección. En principio limitada a 30 caracteres alfanuméricos.
						\item \textbf{Número}: número dentro de la vía (opcional).
						\item \textbf{Portal}: portal de la dirección. 2 caracteres alfanuméricos (opcional).
						\item \textbf{Bloque}: bloque de la dirección. 2 caracteres alfanuméricos (opcional).
						\item \textbf{Escalera}: escalera de la dirección. 2 caracteres alfanuméricos (opcional).
						\item \textbf{Piso}: piso de la dirección. 2 caracteres alfanuméricos (opcional).
						\item \textbf{Puerta}: piso de la dirección. 2 caracteres alfanuméricos (opcional).
						\item \textbf{Localidad}: en principio limitada a 30 caracteres alfanuméricos.
						\item \textbf{Provincia/Estado}: en principio limitado a 30 caracteres alfanuméricos.
						\item \textbf{Código postal}: 5 caracteres.
						\item \textbf{ZIP}: código postal del país correspondiente\footnote{Referencia en \url{www.upu.int/en/resources/postcodes/} - Universal Postal Union.} (no necesariamente Estados Unidos). Hasta un máximo de 10 caracteres.
						\item \textbf{Región}: hasta un máximo de 50 caracteres (opcional).
						\item \textbf{País}: código ISO 3166\footnote{Información y tabla de decodificación en \url{www.iso.org/iso/country_codes.htm}} del país (2 caracteres alfabéticos).
						\item \textbf{Apartado postal}: 10 caracteres.
					\end{itemize}

				A efectos de impresión el formato será el siguiente {\texttt {} $\langle$ tipo de vía $\rangle$ $\langle$ nombre de la vía $\rangle$ $\langle$ número | s/n $\rangle$ [``Portal ''  portal] [``Bloque ''  bloque] [``Escalera ''  escalera] [piso ``º''] [puerta] [``Apartado~'' apartado postal] [localidad ] [provincia ] $\langle$ código postal / ZIP $\rangle$ [región] $\langle$ país $\rangle$}, denotando los corchetes parámetros opcionales.\\
	
				Un ejemplo sería {\itshape Avenida Pennsylvania 1600 Washington D.C. 20500 United States of America}.
		
		% Inicia un índice parcial para las funciones
		\startcontents[tocfunciones]

		\subsection{Funciones}
			\subsubsection{Gestión interna}
				% FUNCIÓN: ACCEDER
				\input{funciones/acceder}

				% FUNCIÓN: CONSULTAR PLAN DE VUELO
				\srsfuncion{Consultar plan de vuelo}
	\todo[inline]{¿De qué manera depende de ``Introducir plan de vuelo''?}
	Esta función debe mostrar una formulario de búsqueda de vuelos y devolver al usuario una lista de vuelos según sus restricciones para así obtener el plan de vuelo detallado de cada servicio.

\begin{enumerate}
	\item \textit{Prioridad}: alta.
	\item \textit{Entradas}
	\begin{enumerate}
		\item Las opciones que admite el formulario de búsqueda son: \gls{numero_de_vuelo}, fecha y hora de origen y llegada, aeropuerto de origen y destino, y personal que forma parte de la tripulación.
		\item El número de vuelo caracteriza e identifica unívocamente a todos los vuelos operados por la compañía. Si dicho parámetro es introducido, todos los demás deben ser ignorados en la búsqueda. Se ha de comprobar que el formato del número de vuelo se corresponde con una sucesión de 4 caracteres numéricos (si el número de caracteres es menor que 4 se completará con \verb|0| por la izquierda).
		\item No se debe permitir introducir fechas u horas no válidas.
		\item Los aeropuertos de origen y destino son un conjunto finito y han de haber sido configurados previamente. Internamente se componen de nombre, ciudad y código \gls{IATA}; externamente se visualizan como una secuencia de texto configurada. El usuario podrá seleccionar uno entre ellos para cada entrada (operación que se puede abreviar introduciendo el código IATA).
		\item En los circunstancias que se estime oportuno, aparecerá una lista editable de personal que formará parte de la tripulación del vuelo para filtrar resultados.
	\end{enumerate}
	\item \textit{Flujo de operaciones}
	\begin{enumerate}
		\item Se muestra un formulario con los campos anteriormente descritos y un botón \verb|Buscar|.
		\item El usuario completa al menos uno de los diferentes campos. Ningún campo permite entradas erróneas por definición, salvo el de código de aeropuertos, que tras introducir un código de aeropuerto válido lo selecciona en el \gls{combobox} de aeropuertos y si no es válido no tiene efecto alguno, restaurándose su valor original.
		\item Si no se obtiene ningún resultado se informa de ello con un cuadro de diálogo. Si el resultado es único se muestra el plan de vuelo. Si se producen varias coincidencias se muestra una lista (indicando número de vuelo, fechas y aeropuertos) que permite la selección de alguno de ellos.
		\item Se accede a una nueva pantalla donde la información del plan de vuelo es distribuida de forma organizada. Desde esta pantalla se puede volver a la pantalla anterior de resultados y búsqueda.
	\end{enumerate}
	\item \textit{Respuesta a situaciones no previstas}
	\begin{enumerate}
		\item Si no se puede acceder a la base de datos de configuraciones: no mostrar la pantalla de inicio de la función e informar del error al usuario.
		\item Si la respuesta de búsqueda en la base datos no tiene lugar o es errónea: informar al usuario y permanecer en la pantalla actual como si no se hubiese buscado.
		\item Si la información de vuelos no puede ser obtenida: informar al usuario y permanecer en la pantalla actual como si no se hubiese buscado o seleccionado un vuelo.
	\end{enumerate}
	\item \textit{Relación con otras funciones}
		Depende indirectamente de la función \verb|Introducir plan de vuelo|.
\end{enumerate}

\begin{figure}[ht]\centering
\includegraphics[scale=.6]{imagenes/BuscarPlanVuelo.png}
\caption{Pantalla aproximada de la búsqueda del plan de vuelo}
\end{figure}


				% FUNCIÓN: OBTENER INFORMACIÓN ECONÓMICA
				% Caso de uso: obtener información económica
% Obs: para escribir comas en el texto del primer parámetro se han de encerrar entre {}.

\casodeuso{
	% Nombre del caso de uso
	nombre=Obtener información económica,
	% Objetivo
	objetivo={Mostrar la información económica de la empresa, de acuerdo a las funciones del usuario. Entre la información recogida se encuentran los balances de la empresa, cuentas de resultados, inversiones en bolsa\ldots},
	% Entradas
	entradas=,
	% Precondiciones
	precondiciones={El operador de la aplicación está debidamente registrado y ocupar el puesto de directivo o empleado de asuntos económicos e infraestructura. La información económica ha sido introducida y procesada con anterioridad.},
	% Salidas
	salidas={La información correspondiente debidamente organizada.},
	% Postcondiciones en caso de éxito
	postexito=El usuario puede explorar los datos económicos mostrados.,
	% Postcondiciones en caso de error
	posterror={El sistema central no ha sufrido cambios.},
	% Actores
	actores={El usuario (personal de asuntos económicos e infraestructura, directivos\dots) y la base de datos.},
}{
	% Tabla de secuencia normal del caso de uso
	\begin{tablasecuencias}
		1 & Extraer de la base de datos la información económica de la empresa. Si error S-1. \\
		2 & Mostrar la visualización de los datos obtenidos (podría incluir gráficos). Si error S-2.
	\end{tablasecuencias}
}{
	% Tabla de secuencia con errores del caso de uso
	\begin{tablasecuencias}
		S-1 & No se puede acceder a la base de datos o no se pueden obtener los datos necesarios. Informar al usuario y abortar la operación.\\
		S-2 & No se ha podido generar la visualización de los datos por algún error imprevisto o por la carencia de datos de la información económica. Infomar al usuario, omitir los elementos erróneos en la medida de lo posible o abortar la operación.
	\end{tablasecuencias}
}

			
				% FUNCIÓN: ESTABLECER ORGANIZACIÓN LABORAL
				% Caso de uso: establecer organización laboral
% Obs: para escribir comas en el texto del primer parámetro se han de encerrar entre {}.

% Revisado por Cristina el día 12/03/2013

\casodeuso{
	% Nombre del caso de uso
	nombre=Establecer organización laboral,
	% Objetivo
	objetivo={Establecer las configuraciones generales sobre la organización laboral de la compañía, como secciones y puestos de trabajo, detallando la información relacionada (salario base\dots) y fijando los privilegios de acceso dentro de la aplicación.},
	% Entradas
	entradas=Los valores a configurar.,
	% Precondiciones
	precondiciones=El operador de la aplicación tiene credenciales que le habilitan para realizar esta operación.,
	% Salidas
	salidas=El valor final de los parámetros configurados.,
	% Postcondiciones en caso de éxito
	postexito=Los cambios efectuados se guardan en la base de datos.,
	% Postcondiciones en caso de error
	posterror=No se realiza ningún cambio en el sistema.,
	% Actores
	actores={Personal administrativo, \textit{Recursos Humanos} y \textit{Servicios Informáticos}.},
}{
	% Tabla de secuencia normal del caso de uso
	\begin{tablasecuencias}
		1 & Se muestran sendas listas de categorías obtenidas del servidor central, permitiendo su edición general con precaución.\\
		2 & El usuario edita los detalles de cada configuración. Si error S-1.\\
		3 & Se almacenan los cambios en la base de datos. Si error S-2.
	\end{tablasecuencias}
}{
	% Tabla de secuencia con errores del caso de uso
	\begin{tablasecuencias}
		S-1 & Alguno de los datos introducidos no es válido. Vuelve a 1 de la secuencia normal de uso indicando los campos erróneos.\\
		S-2 & No se puede conectar con la base de datos, se muestra un mensaje de error por pantalla dando la opción de reintentar o volver al menú principal de la aplicación.
	\end{tablasecuencias}
}


				% FUNCIÓN: CONSULTAR FICHA DE CLIENTES
				\input{funciones/consultarFichaClientes}

				% FUNCIÓN: CONFIGURAR SISTEMA
				% Caso de uso: Configurar sistema general.
% Obs: para escribir comas en el texto del primer parámetro se han de encerrar entre {}.
% A peticionario de Cristina ->
% Revisado por Rubén el día xx/0x/2013
\casodeuso{
	% Nombre del caso de uso
	nombre=Configurar sistema general,
	% Objetivo
	objetivo=Configurar el sistema.,
	% Entradas
	entradas={Cambios que se quieran realizar en la configuración del sistema.},
	% Precondiciones
	precondiciones={Que un personal administrativo autorizado acceda a la configuración del sistema para realizar los cambios que desee dentro de las opciones posibles.},
	% Salidas
	salidas={Cambiar la configuración.},
	% Postcondiciones en caso de éxito
	postexito={La configuración que ha elegido el personal administrativo habrá sido restablecida por lo que haya indicado.},
	% Postcondiciones en caso de error
	posterror={No se ha realizado ningún cambio en el sistema y la configuración sigue igual que estaba.},
	% Actores
	actores=El personal administrativo cuyo rol implique cambiar la configuración del sistema de la empresa.,
}{
	% Tabla de secuencia normal del caso de uso
	\begin{tablasecuencias}
		1 & El usuario registrado accede a la configuración del sistema.\\
		2 & Introduce las modificaciones que desea realizar en el sistema.\\
		3 & Se modifican los datos de la configuración indicados previamente. Si error S-1.\\
		3 & Se confirma que los datos han sido modificados con éxito.
	\end{tablasecuencias}
}{
	% Tabla de secuencia con errores del caso de uso
	\begin{tablasecuencias}
		S-1 & Los datos indicados no se pueden eliminar. Si quiere modificar esos datos asegúrese de que son datos modificables. Si lo son, ha ocurrido un fallo en el sistema que debe de comunicar al personal técnico para su revisión.
	\end{tablasecuencias}
}




				% FUNCIÓN: DAR DE BAJA CLIENTE
				% Caso de uso: Dar de baja cliente.
% Obs: para escribir comas en el texto del primer parámetro se han de encerrar entre {}.

\casodeuso{
	% Nombre del caso de uso
	nombre= Dar de baja cliente,
	% Objetivo
	objetivo={Permitir a un cliente que pueda darse de baja en nuestro sistema borrando todos sus datos de él. Esta acción es obligatoria en cualquier sistema donde se pueda registrar un cliente por la LOPD \textit{(Ley Orgánica de Protección de Datos)}.},
	% Entradas
	entradas={Un personal administrativo acceda a la opción del sistema para dar de baja a un cliente.},
	% Precondiciones
	precondiciones={Que el cliente se haya puesto en contacto con la empresa indicándole que quiere darse de baja en el sistema y haberse registrado como un usuario válido de la aplicación perteneciente al Personal de atención al cliente. Elegir la opción \textit{Dar de baja cliente}.},
	% Salidas
	salidas= {Eliminar al cliente de la base de datos de la empresa.},
	% Postcondiciones en caso de éxito
	postexito={Los datos del cliente habrán desaparecido de la base de datos del sistema, borrándose todo por completo.},
	% Postcondiciones en caso de error
	posterror={Los datos del cliente no se habrán eliminado y seguirán estando en nuestro sistema.},
	% Actores
	actores={El Personal de atención al cliente, el cliente a dar de baja y la base de datos.},
}{
	% Tabla de secuencia normal del caso de uso
	\begin{tablasecuencias}
		1 & El cliente notifica a la empresa que desea darse de baja en su sistema.\\
		2 & El personal administrativo accede al sistema y da a la opción de dar de baja al cliente.\\
		3 & El sistema accede a la base de datos para encontrar al cliente. Si error S-1\\
		4 & Se borran del sistema los datos del cliente. Si error S-2.\\
		5 & Se notifica al cliente de que ha sido dado de baja con éxito.
	\end{tablasecuencias}
}{
	% Tabla de secuencia con errores del caso de uso
	\begin{tablasecuencias}
		S-1 & No se ha encontrado al cliente en la base de datos, por lo que se muestra un mensaje de que el cliente no existe.\\
		S-2 & Si no se ha podido eliminar los datos del cliente se le notifica indicándole que si quiere darse de baja vuelva a realizar la operación.
	\end{tablasecuencias}
}




				% FUNCIÓN: EDITAR INFORMACIÓN ECONÓMICA
				\srsfuncion{Editar información económica} \label{fun:editareconomica}
	Esta función debe permitir añadir, editar o eliminar elementos patrimoniales que forman parte de los gastos e ingresos de la compañía aérea, de acuerdo a los resultados del último ejercicio.

\begin{enumerate}
	\item \textit{Entradas}
	\begin{enumerate}
		\item Los nombres de los conceptos deberán contener únicamente carácteres alfabéticos latinos, acentuados o no, y espacios.
		\item Los datos tanto de los gastos como de los ingresos de la compañía aérea en el último ejercicio serán números reales positivos, con dos decimales a lo sumo separados por una coma (se completará con \verb|0| los dos dígitos decimales en caso de no especificarse).
		\item Se utilizará el euro como unidad monetaria.
	\end{enumerate}
	\item \textit{Flujo de operaciones}
	\begin{enumerate}
		\item Se mostrarán dos tablas: la primera de ella con los gastos de la compañía y la segunda con los ingresos. Ambas tablas aparecerán ordenadas por defecto en orden alfabético por concepto. Se dará la opción al usuario de ordenarlas según diferentes criterios (de mayor a menor importe y de mayor a menor porcentaje que suponen dentro de la masa patrimonial correspondiente), los cuales aparecen en pestañas en la parte superior de la tabla.
		\item Al final de cada tabla aparecerá una fila adicional con el importe total. En un cuadro separado en la parte inferior aparecerá el resultado del ejercicio, el cual se calculará como la resta de ingresos menos gastos, y se mostrará en negro si tiene saldo positivo o cero y en rojo si tiene saldo negativo.
		\item En cada tabla, aparecerá en la parte superior un botón \verb|Añadir| que permitirá añadir un nuevo elemento a la masa patrimonial. Al añadirlo, deberá indicarse un nombre y un importe válidos. Además, cada elemento patrimonial de la tabla podrá modificarse (debiendo introducirse nuevos datos válidos) y eliminarse de la tabla. Cada vez que se produzca una modificación en los datos, se actualizará el total de la tabla así como el resultado del ejercicio.
	\end{enumerate}
	\item \textit{Respuesta a situaciones no previstas}
	\begin{enumerate}
		\item Si no se puede conectar con la base de datos para obtener la información económica: se muestra un mensaje de error por pantalla y regresa a la página principal del sistema.
		\item Si no se puede conectar con la base de datos para almacenar la información: se muestra un mensaje de error por pantalla informando de que la información económica no ha podido actualizarse y se vuelve a la página principal del sistema.
		\item El nombre introducido no es válido: se muestra un mensaje avisando del error y se da la opción de editarlo de nuevo.
		\item El importe introducido no es válido: se muestra un mensaje avisando del error y se da la opción de editarlo de nuevo.
		\item Si no se ha podido ordenar en orden alfabético: mostrar la información desordenada e indicar que no se ha podido ordenar.
	\end{enumerate}

\end{enumerate}


				% FUNCIÓN: MODIFICAR ITEMS INVENTARIO
				% Caso de uso: modificar items inventario.
% Obs: para escribir comas en el texto del primer parámetro se han de encerrar entre {}.

\casodeuso{
	% Nombre del caso de uso
	nombre=Modificar inventario,
	% Objetivo
	objetivo=Permite modificar diversos datos sobre el material registrado en el inventario de la empresa.,
	% Entradas
	entradas=La nueva información sobre el material.,
	% Precondiciones
	precondiciones={El operador de la aplicación está debidamente registrado y posee credenciales que le habilitan para realizar esta operación. El servidor que hospeda la base de datos de inventario está operativo.},
	% Salidas
	salidas=El inventario de la empresa modificado en función de los cambios registrados.
	% Postcondiciones en caso de éxito
	postexito=El material disponible de la empresa en el inventario se habrá actualizado de acuerdo a los datos introducidos.,
	% Postcondiciones en caso de error
	posterror={El sistema central no ha sufrido cambios y, por tanto, no se actualiza el material disponible de la empresa.},
	% Actores
	actores=El personal de la compañía con permisos para realizar estas modificaciones y la base de datos.,
}{
	% Tabla de secuencia normal del caso de uso
	\begin{tablasecuencias}
		1 & Extraer de la base de datos el inventario con los elementos disponibles. Si error S-1. \\
		2 & Mostrar una lista ordenada según el criterio configurado y permitir la búsqueda. Si error S-2. \\
		3 & Permitir añadir o eliminar elementos del inventario. Si error S-3.\\
		4 & Al seleccionar un objeto, mostrar una vista editable de sus propiedades. Si error S-4.
	\end{tablasecuencias}
}{
	% Tabla de secuencia con errores del caso de uso
	\begin{tablasecuencias}
		S-1 & La base de datos está dañada y no se han podido extraer los datos, o ha habido un error en la aplicación. Mostrar por pantalla un mensaje para que el usuario se ponga en contacto con el personal técnico de la empresa y le manifieste el error, disculparse por las molestias y dar las gracias por el aviso.\\
		S-2 & Si no se ha podido generar la lista, mostrar información del error al usuario y abortar la operación. \\
		S-3 & Si no se puede acometer la transacción con la base de datos, informar al usuario y permitir reintento. \\
		S-4 & Si no se puede obtener la información de la base de datos, informar al usuario y cancelar la pantalla de edición. Si no se puede introducir el contenido editado, S-3.
	\end{tablasecuencias}
}



				% FUNCIÓN: PROGRAMAR HORARIOS
				% Caso de uso: programar horarios.
% Obs: para escribir comas en el texto del primer parámetro se han de encerrar entre {}.

\casodeuso{
	% Nombre del caso de uso
	nombre=Programar horarios.,
	% Objetivo
	objetivo=Añadir a la base de datos del sistema los horarios de los empleados.,
	% Entradas
	entradas=Los nuevos horarios y el personal al que afectan.,
	% Precondiciones
	precondiciones={Haber accedido al sistema con un usuario válido perteneciente al Personal de planificación de operaciones y elegir la opción \textit{Programar horarios}.},
	% Salidas
	salidas={Los horarios de los empleados quedan registrados o modificados, si procede.},
	% Postcondiciones en caso de éxito
	postexito=Los horarios de los empleados han sido actualizados en la base de datos.,
	% Postcondiciones en caso de error
	posterror=El sistema no ha sufrido ningún cambio.,
	% Actores
	actores=El Personal de planificación de operaciones y la base de datos.,
}{
	% Tabla de secuencia normal del caso de uso
	\begin{tablasecuencias}
		1 & Extraer de la base de datos el listado de personal de la compañía cuyo horario pueda ser modificado. Si error S-1. \\
		2 & Seleccionar el empleado cuyo horario vaya a ser modificado. \\
		3 & Seleccionar un nuevo horario para el empleado y una fecha a partir de la cual será vigente. Si error S-2. \\
		4 & Almacenar los cambios en la base de datos. Si error S-3. \\
		5 & Mostrar un mensaje de confirmación de la operación.
	\end{tablasecuencias}
}{
	% Tabla de secuencia con errores del caso de uso
	\begin{tablasecuencias}
		S-1 & No se ha podido extraer la información de la base de datos. Mostrar un mensaje de error por pantalla y volver a la página principal del sistema.\\
		S-2 & El horario o la fecha introducidos no son válido por algún motivo. Vuelve al paso 3 de la secuencia normal de uso indicando que el horario no es válido. \\
		S-3 & No se ha podido conectar con la base de datos. Se cancela la operación, se muestra el error por pantalla y vuelve a la página principal del sistema.
	\end{tablasecuencias}
}



				% FUNCIÓN: PROGRAMAR REVISIÓN
				
% Revisado por Juanan el día 12/03/2013

\srsfuncion{Programar revisión}
	Esta función permite programar una revisión a un vehículo determinado.

\begin{enumerate}
	\item \textit{Prioridad}: alta.
	\item \textit{Entradas}
	\begin{enumerate}
		\item Fecha y hora que será coherente con el horario del personal seleccionado, personal, material y herramientas necesarias.
	\end{enumerate}
	\item \textit{Flujo de operaciones}
	\begin{enumerate}
		\item Se elige fecha y hora para la revisión. 
		\item Se selecciona el vehículo a reparar. Además, se rellena el resto del formulario indicando el material y las herramientas necesarias para la reparación, así como el motivo por el que necesita una reparación especificando qué se va a realizar en ella.
		\item Se muestra por pantalla el listado de personal mecánico disponible en esos momentos, ordenados por orden alfabético según el primer apellido. 
		\item Se elegien los trabajadores para que realice la tarea. Para ello se seleccionan los empleados de la lista.
		\item Cuando los datos se han modificado, se muestra confirmación detallada y se envia automáticamente notificación al personal afectado.
	\end{enumerate}
	\item \textit{Respuesta a situaciones no previstas}
	\begin{enumerate}
		\item Si no se puede acceder a la base de datos del personal: se muestra un mensaje de error por pantalla y se vuelve a la página principal del sistema.
		\item Si no se puede conectar con la base de datos para almacenar la información de la revisión: se muestra un mensaje de error por pantalla informando de que la revisión no ha podido darse de alta y se vuelve a la página principal del sistema.
		\item Si no existe ningún empleado disponible para la fecha y hora indicadas: Mostrar un mensaje indicando que la revisión no se puede programar por falta de personal disponible en esa fecha elegida. Dar la opción de modificar la fecha y hora elegidas.
	\end{enumerate}

\end{enumerate}

				
				% FUNCIÓN: REALIZAR MANTENIMIENTO
				% Caso de uso: realizar mantenimiento.
% Obs: para escribir comas en el texto del primer parámetro se han de encerrar entre {}.

\casodeuso{
	% Nombre del caso de uso
	nombre=Realizar mantenimiento,
	% Objetivo
	objetivo=Permite registrar la información sobre un mantenimiento realizado.,
	% Entradas
	entradas=Los datos del mantenimiento realizado.,
	% Precondiciones
	precondiciones=Haber accedido al sistema con un usuario válido y elegir la opción \textit{Realizar mantenimiento}.,
	% Salidas
	salidas=El mantenimiento queda registrado.
	% Postcondiciones en caso de éxito
	postexito={Se ha registrado el mantenimiento y, si procede, se ha actualizado el material disponible de la empresa en el inventario.},
	% Postcondiciones en caso de error
	posterror=El mantenimiento no ha quedado registrado y el sistema central no ha sufrido cambios.,
	% Actores
	actores=El personal mecánico de la compañía y la base de datos.,
}{
	% Tabla de secuencia normal del caso de uso
	\begin{tablasecuencias}
		1 & Extraer de la base de datos el listado de los mantenimientos programados para el usuario. Si error S-1. \\
		2 & Seleccionar un mantenimiento.\\
		3 & Introducir la información detallada del mantenimiento realizado (informe de la operación, y si ha podido completarse o no). Si error S-2.\\
		4 & Se muestra el listado del material mecánico disponible en el inventario. Si error S-3 \\
		5 & El usuario selecciona el material empleado en el mantenimiento, así como el cantidad de items de cada tipo utilizados. Si error S-4 \\
		6 & Mostrar un mensaje confirmando el registro del mantenimiento.
	\end{tablasecuencias}
}{
	% Tabla de secuencia con errores del caso de uso
	\begin{tablasecuencias}
		S-1 & No se ha podido extraer la información de la base de datos. Mostrar un mensaje de error por pantalla y volver a la página principal del sistema.\\
		S-2 & No se ha podido almacenar la información introducida. Mostrar un mensaje por pantalla indicándolo y regresar a la página anterior. \\
		S-3 & No se ha podido cargar el listado del material del inventario. Mostrar un mensaje de error por pantalla y volver a la página anterior. \\
		S-4 & No se ha podido almacenar los cambios introducidos. Mostrar un mensaje por pantalla indicándolo y regresar a la página anterior.
	\end{tablasecuencias}
}


				% FUNCIÓN: REGISTRAR EMPLEADO
				% Caso de uso: registrar empleado.
% Obs: para escribir comas en el texto del primer parámetro se han de encerrar entre {}.

\casodeuso{
	% Nombre del caso de uso
	nombre=Registrar empleado,
	% Objetivo
	objetivo=Añadir un nuevo usuario a la base de datos del personal de la compañía con unos determinados permisos de acceso.,
	% Entradas
	entradas={Nombre de usuario, datos personales y puesto de trabajo al que se incorpora dentro de la compañía.},
	% Precondiciones
	precondiciones={Que el futuro usuario haya sido contratado por la empresa y que éste haya firmado la LOPD (Ley Orgánica de Protección de Datos), y que el encargado del registro, personal de Recursos Humanos, haya accedido al sistema y a la opción \textit{Registrar empleado}.},
	% Salidas
	salidas={Confirmación de la creación de la cuenta o exposición de los datos incorrectos, según proceda.},
	% Postcondiciones en caso de éxito
	postexito=El usuario queda registrado en la base de datos pero su cuenta no estará activada hasta que sea verificada por el departamento de intervención.,
	% Postcondiciones en caso de error
	posterror=La base de datos no ha sido alterada.,
	% Actores
	actores={El personal administrativo encargado de registrar nuevos empleados, el usuario a registrar y la base de datos.},
}{
	% Tabla de secuencia normal del caso de uso
	\begin{tablasecuencias}
		1 & El usuario registrador inserta los datos del futuro usuario siendo el nombre de usuario el que se asignará a la cuenta de correo de la compañía. Si error S-1.\\
		2 & El sistema genera una contraseña aleatoria. \\
		3 & Se crea una cuenta de correo. \\
		4 & Se vuelcan los datos a la base de datos. Si error S-2.\\
		5 & Se muestra un mensaje de confirmación del registro. \\
		6 & Se imprime un documento con los datos de acceso para el usuario.
	\end{tablasecuencias}
}{
	% Tabla de secuencia con errores del caso de uso
	\begin{tablasecuencias}
		S-1 & Alguno de los datos no es válido. El sistema vuelve al paso 1 de la secuencia normal de uso e indica los campos erróneos.\\
		S-2 & No se puede conectar con la base de datos. Se cancela la operación, se muestra un mensaje de error por pantalla y se vuelve a la página principal del sistema.
	\end{tablasecuencias}
}




				% FUNCIÓN: REGISTRAR ENTRADA MATERIAL
				
\srsfuncion{Registrar entrada material}
	Esta función debe permitir registrar un nuevo material en el inventario de la empresa.

\begin{enumerate}
	\item \textit{Entradas}
	\begin{enumerate}
		\item Al añadirlo, los items del inventario deberán seguir quedando ordenados por defecto por orden alfabético, ordenando posteriormente el sistema si se desea por otros campos como fecha de entrada en almacén, cantidad de items, destino de uso (oficina, mecánica, edificios, aeronáutico\ldots) y valor de adquisición.
		\item El número de registro del item debe de ser un número mayor o igual que cero (numeración de los items).
	\end{enumerate}
	\item \textit{Flujo de operaciones}
	\begin{enumerate}
		\item Se muestra por pantalla el formulario a rellenar con los datos específicos del item que se van a introducir en la base de datos para poder identificarlo posteriormente y poder mostrar la información sobre él.
		\item Habrá un botón \verb|Añadir|, que registrará la entrada de material en el sistema.
		\item Si el item existe se incrementará la cantidad de items en el inventario.
		\item Si el item no existe, cuando se añada se modificará la lista añadiéndolo en el lugar correcto.
	\end{enumerate}
	\item \textit{Respuesta a situaciones no previstas}
	\begin{enumerate}
		\item Si se ha introducido algún campo incorrecto en el formulario a rellenar de los datos específicos del item, marcar los campos erróneos y mostrar un mensaje por pantalla indicando cuáles de estos campos son erróneos (por escrito) y una nota con una breve descripción diciendo el por qué.
		\item Si no se ha podido ordenar en orden alfabético: mostrar la información desordenada e indicar que no se ha podido ordenar.
	\end{enumerate}

\end{enumerate}
\label{fun:EntrMat}


				% FUNCIÓN: VERIFICAR REGISTRO EMPLEADO
				% Caso de uso: verificar registro empleado.
% Obs: para escribir comas en el texto del primer parámetro se han de encerrar entre {}.

% Revisado por Cristina y Juanan el día 11/03/2013

\casodeuso{
	% Nombre del caso de uso
	nombre=Verificar el registro de un empleado,
	% Objetivo
	objetivo=Establecer una nueva cuenta de usuario como válida.,
	% Entradas
	entradas={Los datos del usuario previamente registrado.},
	% Precondiciones
	precondiciones={El operador de la aplicación tiene credenciales que le habilitan para realizar dicha operación. El nuevo usuario está registrado correctamente en el sistema.},
	% Salidas
	salidas=Se confirma la operación.,
	% Postcondiciones en caso de éxito
	postexito={La cuenta del usuario queda verificada y, por tanto, pasa a estar activa y totalmente operativa.},
	% Postcondiciones en caso de error
	posterror=La cuenta del usuario permanece inactiva.,
	% Actores
	actores=El personal del departamento de intervención y la base de datos.,
}{
	% Tabla de secuencia normal del caso de uso
	\begin{tablasecuencias}
		1 & Se extrae de la base de datos del sistema el listado de usuarios pendientes de verificación. Si error S-1.\\
		2 & El empleado selecciona el usuario.\\
		3 & Se muestran los datos por pantalla.\\
		4 & Se comprueba que sean correctos. Si alguno de los datos no es correcto, S-2.\\
		5 & Se verifica la cuenta del usuario y se modifica en la base de datos. Si error S-3.
	\end{tablasecuencias}
}{
	% Tabla de secuencia con errores del caso de uso
	\begin{tablasecuencias}
		S-1 & No se puede conectar con la base de datos, se muestra un mensaje de error por pantalla dando la opción de reintentar o volver al menú principal de la aplicación.\\
		S-2 & Se cancela la operación y vuelve al menú principal.\\
		S-3 & No se puede acceder a la base de datos. Se cancela la operación, se muestra un mensaje de error por pantalla dando la opción de reintentar o volver al menú principal de la aplicación.
	\end{tablasecuencias}
}



				
				% FUNCIÓN: EDITAR EMPLEADO
				% Caso de uso: editar empleado.
% Obs: para escribir comas en el texto del primer parámetro se han de encerrar entre {}.

\casodeuso{
	% Nombre del caso de uso
	nombre=Modificar ficha de empleado.,
	% Objetivo
	objetivo=Permite actualizar la información referente a uno de los empleados.,
	% Entradas
	entradas=Los datos nuevos a actualizar.,
	% Precondiciones
	precondiciones={Haber accedido al sistema con un usuario válido y los permisos necesarios.},
	% Salidas
	salidas=La ficha de empleado con los datos modificados actualizados.,
	% Postcondiciones en caso de éxito
	postexito=Los cambios efectuados se guardan en la base de datos.,
	% Postcondiciones en caso de error
	posterror={El sistema central no ha sufrido cambios y, por tanto, no se modifica la ficha del empleado.},
	% Actores
	actores={El personal administrativo de la compañía con permisos para realizar estas modificaciones, base de datos.},
}{
	% Tabla de secuencia normal del caso de uso
	\begin{tablasecuencias}
		1 & Muestra lista de empleados. Si error S-1. \\
		2 & Selecciona ficha de empleado. \\
		3 & Muestra ficha seleccionada.\\
		4 & Selecciona campos a modificar. \\
		5 & Introducir datos nuevos. \\
		6 & Comprueba corrección en los datos introducidos. Si error S-2. \\
		7 & Almacenar los cambios en la base de datos. Si error S-3. \\
		8 & Muestra confirmación y a continuación vuelve al paso 3.
	\end{tablasecuencias}
}{
	% Tabla de secuencia con errores del caso de uso
	\begin{tablasecuencias}
		S-1 & Si no se puede conectar con la base de datos se muestra mensaje  de tipo \textit{información no disponible temporalmente} y se vuelve al paso 1 de la secuencia normal.\\
		S-2 & Alguno de los datos introducidos no es válido. Volver al paso 3 de la secuencia normal de uso indicando los campos erróneos. \\
		S-3 & No se han podido almacenar los cambios en la base de datos. Se cancela la operación, se informa y se vuelve a la página principal del sistema.
	\end{tablasecuencias}
}



				% FUNCIÓN: CONSULTAR NÓMINA
				
% Revisado por Juanan el día 12/03/2013

\srsfuncion{Consultar nómina} \label{fun:consultarnomina}
	Esta función muestra la nómina del usuario correspondiente a un mes seleccionado.

\begin{enumerate}
	\item \textit{Prioridad}: media.
	\item \textit{Entradas}
	\begin{enumerate}
		\item Mes del que se quiere realizar la consulta.
		\item El sueldo del mes (cobrado o a cobrar) aparece expresado en la moneda que haya sido configurada.
	\end{enumerate}
	\item \textit{Flujo de operaciones}
	\begin{enumerate}
		\item Se selecciona el mes.
		\item Se muestra en pantalla la información requerida.
	\end{enumerate}
	\item \textit{Respuesta a situaciones no previstas}
	\begin{enumerate}
		\item Si no existe nómina del mes seleccionado o no se ha podido acceder a ella se informa de lo sucedido y se permite seleccionar otro.
	\end{enumerate}
\end{enumerate}	


				% FUNCIÓN: CONSULTAR HORARIOS
				\srsfuncion{Acceder horarios}
	Esta función permite al empleado consultar sus horarios.

\begin{enumerate}
	\item \textit{Prioridad}: media.
	\item \textit{Entradas}
	\begin{enumerate}
		\item Fecha o rango de fechas del que se quiere consultar la nómina.
		\item Se comprobará que estas fechas marcadas sean válidas.
	\end{enumerate}
	\item \textit{Flujo de operaciones}
	\begin{enumerate}
		\item Se seleccionarán fechas en las que se desea consultar la nómina.
		\item Se mostrará una plantilla con los turnos de trabajo del empleado distinguiendo turnos de mañana, tarde o noche, y resaltando los días festivos.
	\end{enumerate}
	\item \textit{Respuesta a situaciones no previstas}
	\begin{enumerate}
		\item Si no se encuentran en el sistema datos relativos a la búsqueda se notifica al usuario y se vuelve al menú principal del sistema.
	\end{enumerate}
\end{enumerate}

				
				% FUNCIÓN: PROGRAMAR UNA OFERTA
				\srsfuncion{Programar oferta}
	Esta función debe permitir crear una oferta de un item a promocionar que posteriormente se mostrará a los clientes y podrán disponer de ella.
	
\begin{enumerate}
	\item \textit{Entradas}
	\begin{enumerate}
		\item Se programarán las ofertas que la empresa crea convenientes.
		\item Todas las palabras del lenguaje en el que se programe la oferta deberán estar en la \gls{RAE} para posteriormente poder traducirlas de un idioma a otro.
		\item El título y resumen de la oferta deberán tener un máximo de caracteres de 60 y 200 respectivamente.
	\end{enumerate}
	\item \textit{Flujo de operaciones}
	\begin{enumerate}
		\item El empleado con el rol de programar las ofertas introducirá los siguientes datos de la oferta:
		
			\begin{enumerate}
				\item Título
				\item Resumen
				\item Descripción detallada
				
					\begin{itemize}
						\item Artículo/s promocionado/s
						\item Precio anterior de cada artículo
						\item Precio de cada artículo con esta oferta
						\item Fecha límite de caducidad de la oferta
					\end{itemize}
				
			\end{enumerate}
		\item El programador validará pulsando el botón \verb|Validar| la oferta para introducirla en la base de datos del programa.
	\end{enumerate}
	\item \textit{Respuesta a situaciones no previstas}
	\begin{enumerate}
		\item En caso de no haber completado algún campo del formulario, marcar los campos erróneos e indicar una breve descripción indicando que no se han completado el formulario entero.
		\item Si al intentar validar la oferta esta no puede ser incluida en la base de datos, mostrar un mensaje indicando el error por no poder ser añadida a la base de datos del sistema. El programador podrá volver a pulsar el botón \verb|Validar| para intentar volver a ivalidar la oferta.
	\end{enumerate}
\end{enumerate}


				% FUNCIÓN: CONSULTAR INVENTARIO
				% Caso de uso: Consultar inventario
% Obs: para escribir comas en el texto del primer parámetro se han de encerrar entre {}.

\casodeuso{
	% Nombre del caso de uso
	nombre=Consultar inventario,
	% Objetivo
	objetivo=Muestra el inventario de la empresa. Este material es tanto para reparaciones como de otros tipos que necesite la empresa.,
	% Entradas
	entradas=,
	% Precondiciones
	precondiciones={El operador de la aplicación está debidamente registrado y posee credenciales que le habilitan para realizar esta operación. El servidor que hospeda la base de datos de inventario está operativo.},
	% Salidas
	salidas=El registro de material en el inventario de la empresa.,
	% Postcondiciones en caso de éxito
	postexito=El empleado quedará informado sobre el material del que dispone la empresa en el inventario.,
	% Postcondiciones en caso de error
	posterror=El empleado no habrá podido recibir la información del inventario de la empresa y desconocerá el material disponible.,
	% Actores
	actores=El empleado con permisos para acceder a la información requerida y la base de datos.
}{
	% Tabla de secuencia normal del caso de uso
	\begin{tablasecuencias}
		1 & Extraer de la base de datos el inventario. Si error S-1. \\
		2 & Realizar una lista con la información organizada por el criterio configurado. Si error S-2. \\
		3 & Mostrar por pantalla estos datos permitiendo ser filtrados.\\
		4 & Al seleccionar un elemento, mostrar información detallada del mismo. Si error S-1.
	\end{tablasecuencias}
}{
	% Tabla de secuencia con errores del caso de uso
	\begin{tablasecuencias}
		S-1 & No se pudo conectar con la base de datos o no se pudo obtener la información. Informar al usuario y abandonar la carga.\\
		S-2 & Si no se encuentra configuración sobre criterios de ordenación, ordenar por orden lexicográfico o numérico del primer campo si lo hubiera.
	\end{tablasecuencias}
}



				% FUNCIÓN: CONSULTAR FICHA DE EMPLEADOS
				% Caso de uso: consultar ficha empleado
% Obs: para escribir comas en el texto del primer parámetro se han de encerrar entre {}.

% Revisado por Cristina y Juanan el día 11/03/2013

\casodeuso{
	% Nombre del caso de uso
	nombre=Consultar ficha empleado,
	% Objetivo
	objetivo={Mostrar la lista de empleados de la compañía, permitiendo buscar y filtrar resultados, así como información detallada de cada empleado en particular.},
	% Entradas
	entradas={Opcionalmente, campos de búsqueda.},
	% Precondiciones
	precondiciones=El operador de la aplicación tiene credenciales que le habilitan para realizar dicha operación.,
	% Salidas
	salidas=Lista de empleados e información detallada sobre el seleccionado.,
	% Postcondiciones en caso de éxito
	postexito=No se realiza ningún cambio en el sistema.,
	% Postcondiciones en caso de error
	posterror=No se realiza ningún cambio en el sistema.,
	% Actores
	actores={Los directivos, empleados de recursos humanos y la base de datos.}.
}{
	% Tabla de secuencia normal del caso de uso
	\begin{tablasecuencias}
		1 & Se extrae de la base de datos del sistema el listado de empleados. Si error S-1.\\
		2 & Se muestra por pantalla la lista de empleados.\\
		3 & El usuario puede filtrar los resultados y buscar empleados según diferentes criterios, como tipo de empleado, duración en la empresa, departamentos, etc.\\
		4 & Se selecciona un empleado.\\
		5 & Se muestra la información detallada del empleado. Si error S-1.
	\end{tablasecuencias}
}{
	% Tabla de secuencia con errores del caso de uso
	\begin{tablasecuencias}
		S-1 & No se puede conectar con la base de datos, se muestra un mensaje de error por pantalla dando la opción de reintentar o volver al menú principal de la aplicación.
	\end{tablasecuencias}
}

			
				% FUNCIÓN:  DAR DE BAJA EMPLEADO
				\srsfuncion{Dar de baja empleado}
	Función que debe permitir dar de baja a un empleado, eliminando su información personal de acuerdo a la legislación vigente y derogando las autorizaciones de acceso al sistema y las instalaciones.
							
	\begin{enumerate}
		\item \textit{Prioridad}: alta.
		\item \textit{Entradas}
			\begin{enumerate}
				\item El usuario debe introducir en el campo de causa los motivos por los cuales se ha decidido llevar a cabo esta operación.
			\end{enumerate}
		\item \textit{Flujo de operaciones}
			\begin{enumerate}
				\item Una vez que el usuario ha pulsado el botón de \verb|Dar de baja| en la información detallada del empleado, rellenará de forma obliglatoria el campo de causa y confirmará que quiere realizar la operación dar de baja con el cliente seleccionado.
				\item El empleado dado de baja pierde inmediatamente todo acceso al sistema y su información personal se elimina de la base de datos de acuerdo a la legislación vigente.
				\item Por último, se registra la operación en la \textit{Cola de Supervisión}.
			\end{enumerate}
		\item \textit{Respuesta a situaciones no previstas}
			\begin{enumerate}
				\item Si no se puede establecer conexión con la base de datos: se muestra un mensaje de error y se da la opción de reintentar o abortar el proceso.
				\item Si no se puede registrar  la operación en la \textit{Cola de Supervisión} se anula la operación o se solicita al operador que informe a dicho departamento manualmente.
			\end{enumerate}
	\end{enumerate}
								

			
				% FUNCIÓN: CONFIGURAR NÓMINA
				
% Revisado por Juanan el día 12/03/2013

\srsfuncion{Configurar nómina}
	Función que permite confeccionar y almacenar las nóminas mensuales de cada empleado de la compañía según las incidencias que se hayan producido en el último mes.
						
	\begin{enumerate}
		\item \textit{Prioridad}: alta.
		\item \textit{Entradas}
			\begin{enumerate}
				\item El usuario introduce en el campo de incidencias los motivos por los cuales este mes se produce una modificación en la nómina como, por ejemplo, aumentos o disminuciones del salario a causa de horas extras, comisiones, sustituciones, huelgas\ldots
				\item El sueldo del mes será el resultante de sumar el sueldo base y el correspondiente a las incidencias.
			\end{enumerate}
		\item \textit{Flujo de operaciones}
			\begin{enumerate}
				\item Mediante \verb{Consultar empleado} se accede a la configuración de la nómina de cada empleado. 
				\item A continuación, el usuario rellena de forma obliglatoria el campo de incidencias y confirma que quiere aplicar los cambios en la nómina del cliente seleccionado.
				\item Por último, se registra la nómina en la base de datos.
			\end{enumerate}
		\item \textit{Respuesta a situaciones no previstas}
			\begin{enumerate}
				\item Si no se puede establecer conexión con la base de datos: se muestra un mensaje de error y se da la opción de reintentar o abortar el proceso.
				\item Si no se puede registrar la nómina: anular la operación y volver a la página principal del sistema.
			\end{enumerate}				
		\item \textit{Relación con otras funciones}\\
		La función está relacionada con \nameref{fun:consultarnomina} y \nameref{fun:consultarempleado}.
	\end{enumerate}
								


				% FUNCIÓN: INTRODUCIR PLAN DE VUELO
				\srsfuncion{Introducir plan de vuelo}
	Esta función debe añadir un nuevo vuelo a la lista de vuelos de la compañía aérea.

	\begin{enumerate}
		\item \textit{Entradas}
			\begin{enumerate}
				\item Los campos a introducir por el usuario son: \gls{numero_de_vuelo}, fecha y hora de origen y llegada, aeropuerto de origen y destino, el modelo de avión y el  precio total según la clase (turista, turista superior, business y primera) y tipo de pasajero (adulto, niño y bebé).
				\item El número de vuelo caracteriza e identifica unívocamente a todos los vuelos operados por la compañía. Si dicho parámetro es introducido, todos los demás deben ser ignorados en la búsqueda. Se ha de comprobar que el formato del número de vuelo se corresponde con una sucesión de 4 carácteres numéricos (si el número de caracteres es menor que 4 se completará con \verb|0| por la izquierda).
				\item No se debe permitir introducir fechas u horas no válidas.
				\item Los aeropuertos de origen y destino son un conjunto finito y han de haber sido configurados previamente. Internamente se componen de nombre, ciudad y código \gls{IATA}; externamente se visualizan como una secuencia de texto configurada. El usuario podrá seleccionar uno entre ellos para cada entrada (operación que se puede abreviar introduciendo el código IATA).
				\item El modelo de avión tiene que estar disponible en las fechas indicadas, así como en un correcto estado para su funcionamiento. Por supuesto, dicho modelo debe estar registrado en el inventario.
			\end{enumerate}
		\item \textit{Flujo de operaciones}
			\begin{enumerate}
				\item Se muestra un formulario con los campos anteriormente descritos.
				\item El usuario completa obligatoriamente todos los campos. Ningún campo permite entradas erróneas por definición, salvo el de código de aeropuertos, que tras introducir un código de aeropuerto válido lo selecciona en el \gls{combobox} de aeropuertos y si no es válido no tiene efecto alguno, restaurándose su valor original.
				\item Para cada tipo de clase y de pasajero se mostrará un botón \verb|Calcular precio| que permitirá determinar el precio del vuelo en función del pasajero.
				\item Se crea el nuevo vuelo pulsando sobre el botón \verb|Añadir vuelo|.
			\end{enumerate}
		\item \textit{Respuesta a situaciones no previstas}
			\begin{enumerate}
				\item Si no se puede acceder a la base de datos de configuraciones: no mostrar la pantalla de inicio de la función e informar del error al usuario.
				\item Si alguno de los datos es erróneo: informar al usuario y volver a solicitar la información.
			\end{enumerate}
	\end{enumerate}
	\begin{figure}[ht]\centering
	\includegraphics[scale=.6]{imagenes/introducirPlanDeVueloImagen.png}
	\caption{Pantalla aproximada de como introducir un nuevo vuelo}
\end{figure}

				
				% FUNCIÓN: FACTURAR
				
% Revisado por Juanan el día 12/03/2013

\srsfuncion{Facturar} \label{fun:facturar}
	Esta función registra y controla la facturación del equipaje del viajero.

	\begin{enumerate}
		\item \textit{Prioridad}: alta.
		\item \textit{Entradas}
			\begin{enumerate}
				\item Los campos a introducir por el usuario son: número de reserva y peso del equipaje.
				\item El número de reserva caracteriza e identifica unívocamente tanto al usuario como al vuelo.
				\item El peso del equipaje debe ser expresado en kilogramos. En caso de que dicha cantidad supere el máximo permitido por billete, el pasajero deberá abonar la cantidad calculada por indique el sistema. En cualquier caso, el peso del equipaje deberá estar dentro de un rango establecido previamente por la compañía.
			\end{enumerate}
		\item \textit{Flujo de operaciones}
			\begin{enumerate}
				\item Se muestra un formulario con el número de vuelo, número de reserva y peso del equipaje.
				\item El usuario completa obligatoriamente todos los campos. Si el peso excede el permitido por billete, se indicará la cantidad a pagar y se habilitará un botón \verb|Confirmar pago| para indicar al sistema que el cliente ha abonado el dinero por el exceso de equipaje. No se dejará continuar mientras no se pulse dicha opción o no se modifique el peso del equipaje.
				\item Cuando los datos son correctos, el usuario confirma la operación.
			\end{enumerate}
		\item \textit{Respuesta a situaciones no previstas}
			\begin{enumerate}
				\item Si alguno de los datos introducidos no es válido: se muestran los campos erróneos y se da la opción de modificarlos.
				\item Si no se puede acceder a la base de datos: se muestra un mensaje de error y vuelve a la página anterior.
			\end{enumerate}
		\item \textit{Relación con otras funciones}\\
		La función está relacionada con \nameref{fun:efectuarembarque}.		
	\end{enumerate}	

			
				% FUNCIÓN: EFECTUAR EMBARQUE
				% Caso de uso: Efectuar embarque
% Obs: para escribir comas en el texto del primer parámetro se han de encerrar entre {}.

% Comentario de Aitor, si un cliente no se presenta a un embarque se cancelan todos los billetes asociados a ese cliente.
% Revisado por Juanan el día 12/03/2013

\casodeuso{
	% Nombre del caso de uso
	nombre=Efectuar embarque,
	% Objetivo
	objetivo=Registrar el embarque del pasajero.,
	% Entradas
	entradas=El número de vuelo y el número de reserva del pasajero.,
	% Precondiciones
	precondiciones=El operador de la aplicación tiene credenciales que le habilitan para realizar dicha operación y un billete válido,
	% Salidas
	salidas=Confirmación de que el pasajero pasa el control de seguridad con éxito y ha embarca en el avión.,
	% Postcondiciones en caso de éxito
	postexito=Se registra el embarque del pasajero.,
	% Postcondiciones en caso de error
	posterror=No se realiza ningún cambio en el sistema.,
	% Actores
	actores=El personal de la compañía presente en el aeropuerto y la base de datos.
}{
	% Tabla de secuencia normal del caso de uso
	\begin{tablasecuencias}
		1 & El usuario introduce el número de vuelo y el número de reservar del pasajero. Si error S-1.\\
		2 & Se comprueba la reserva y se actualiza la base de datos. Si error S-2.		
	\end{tablasecuencias}
}{
	% Tabla de secuencia con errores del caso de uso
	\begin{tablasecuencias}
		S-1 & Alguno de los campos introducidos por el usuario no es válido. Se muestra un mensaje de error y se vuelve a 1 de la secuencia normal de uso indicando los datos erroneos. \\
		S-2 & No se ha podido conectar con la base de datos. Se muestra un mensaje de error y se ofrece la posibilidad de reintentar.
	\end{tablasecuencias}
}



				% FUNCIÓN: EFECTUAR EMBARQUE
				% Caso de uso: ver incidencias del sistema.
% Obs: para escribir comas en el texto del primer parámetro se han de encerrar entre {}.

\casodeuso{
	% Nombre del caso de uso
	nombre=Ver incidencias del sistema,
	% Objetivo
	objetivo={Permite a los supervisores informáticos del sistema inspeccionar el correcto funcionamiento del \software y responder a comportamientos erróneos del mismo que hayan sido detectados, a partir de los registros que este genera.},
	% Entradas
	entradas={El nombre de registro que se quiere consultar.},
	% Precondiciones
	precondiciones={El operador de la aplicación está debidamente registrado y posee credenciales que le habilitan para realizar esta operación.},
	% Salidas
	salidas={Archivos de registro del sistema solicitados.},
	% Postcondiciones en caso de éxito
	postexito={},
	% Postcondiciones en caso de error
	posterror={El sistema central no habrá sufrido cambios.},
	% Actores
	actores={Personal de \textit{Servicios Informáticos} y supervisores del sistema con autorización para ello.},
}{
	% Tabla de secuencia normal del caso de uso
	\begin{tablasecuencias}
		1 & Los archivos de registro del servidor central y los errores reportados por las aplicaciones cliente componen una serie de archivos de texto en el servidor central. Esos archivos podrán ser consultados dando acceso a su ubicación en el sistema de archivos del servidor central.
		% De momento, ¿para qué complicarlo?
	\end{tablasecuencias}
}{
	% Tabla de secuencia con errores del caso de uso
	\begin{tablasecuencias}
		S-1 & Las secuencias alternativas son las que determine el método de acceso al servidor.
	\end{tablasecuencias}
}


			\subsubsection{Gestión externa}

				% FUNCIÓN: REGISTRARSE
				% Caso de uso: Registrarse
% Obs: para escribir comas en el texto del primer parámetro se han de encerrar entre {}.

% Revisado por Cristina el día 11/03/2013

\casodeuso{
	% Nombre del caso de uso
	nombre=Registrarse,
	% Objetivo
	objetivo=Registrar un nuevo cliente de la compañía en la base de datos.,
	% Entradas
	entradas={Datos personales del cliente (nombre y apellidos, NIF o equivalente, dirección, teléfono y una dirección de correo electrónico).},
	% Precondiciones
	precondiciones=El visitante que pretende registrarse no está previamente registrado.,
	% Salidas
	salidas=Confirmación del registro del nuevo cliente.,
	% Postcondiciones en caso de éxito
	postexito=El cliente queda registrado en la base de datos.,
	% Postcondiciones en caso de error
	posterror=No se realiza ningún cambio en el sistema.,
	% Actores
	actores=El visitante de la interfaz externa y la base de datos.,
}{
	% Tabla de secuencia normal del caso de uso
	\begin{tablasecuencias}
		1 & El usuario introduce sus datos personales. Si error S-1.\\
		2 & Se almacenan los datos del nuevo cliente en la base de datos. Si error S-2.
	\end{tablasecuencias}
}{
	% Tabla de secuencia con errores del caso de uso
	\begin{tablasecuencias}
		S-1 & El sistema vuelve al paso 1 de la secuencia normal de uso e indica los campos erróneos.\\
		S-2 & No se puede conectar con la base de datos, se muestra un mensaje de error por pantalla dando la opción de reintentar o volver al menú principal de la aplicación.
	\end{tablasecuencias}
}


				% FUNCIÓN: EDITAR CLIENTE
				% Caso de uso: editar cliente.
% Obs: para escribir comas en el texto del primer parámetro se han de encerrar entre {}.

% Revisado por Juanan el día 12/03/2013

\casodeuso{
	% Nombre del caso de uso
	nombre=Editar cliente,
	% Objetivo
	objetivo={Editar la información de un cliente, bien él mismo o desde la gestión interna de la aplicación.},
	% Entradas
	entradas=Los datos que se quieren modificar.,
	% Precondiciones
	precondiciones={El operador de la aplicación tiene credenciales que le habilitan para realizar dicha operación y tiene una ficha seleccionada.},
	% Salidas
	salidas=La información de perfil actualizada.,
	% Postcondiciones en caso de éxito
	postexito=Los cambios efectuados se guardan en la base de datos.,
	% Postcondiciones en caso de error
	posterror=No se realiza ningún cambio en el sistema.,
	% Actores
	actores={Cliente-usuario de interfaz web o el personal administrativo y la base de datos},
}{
	% Tabla de secuencia normal del caso de uso
	\begin{tablasecuencias}
		1 & Muestra los campos de datos personales de usuario. Si error S-1.\\
		2 & El usuario modifica los datos deseados. Si error S-2.\\
		3 & Se almacenan los cambios en la base de datos. Si error S-3.
	\end{tablasecuencias}
}{
	% Tabla de secuencia con errores del caso de uso
	\begin{tablasecuencias}
		S-1 & Si no se puede conectar con la base de datos se muestra mensaje  de tipo \textit{información no disponible temporalmente} y se vuelve a 1 de la secuencia normal de uso.\\
		S-2 & Alguno de los datos introducidos no es válido. Vuelve a 1 de la secuencia normal de uso indicando los campos erróneos.\\
		S-3 & No se puede conectar con la base de datos. Se cancela la operación, se muestra un mensaje de error por pantalla y se vuelve a la ficha del cliente.
	\end{tablasecuencias}
}



				% FUNCIÓN: VER INFORMACIÓN DE VUELO
				
% Revisado por Juanan el día 12/03/2013

\srsfuncion{Ver información de vuelo contratado}
	
	Esta función muestra a los clientes que han adquirido billetes información sobre sus reservas. Muestra una lista de todos los vuelos pendientes contratados con la compañía, permitiendo imprimir un informe sobre cada uno de ellos o sobre la totalidad. No muestra información sobre vuelos pasados ni permite filtrar los resultados obtenidos.

	\begin{enumerate}
		\item \textit{Prioridad}: media.
		\item \textit{Entradas}\\
			Esta función no acepta ningún parámetro.

		\item \textit{Precondiciones}: el usuario ha iniciado sesión correctamente en la interfaz externa.
		
		\item \textit{Flujo de operaciones}
			\begin{enumerate}
				\item Se muestra una sucesión de cuadros describiendo cada uno de los vuelos contratados. Si no hay ningún vuelo contratado mostrar ese mensaje.
				\item El usuario puede seleccionar --en cada vuelo o en general-- la opción imprimir que formateará la información adecuadamente para ser impresa o generará un documento \gls{PDF} para ser descargado.
			\end{enumerate}
		\item \textit{Respuesta a situaciones no previstas}
			\begin{enumerate}
				\item Si no se puede conectar con la base de datos u obtener la información: informar del error al usuario.
				\item Si no se puede generar la vista de impresión o el documento \gls{PDF}: informar al usuario e indicarle que puede volver a intentarlo en otro momento.
			\end{enumerate}
		\item \textit{Relación con otras funciones}\\
			Esta función requiere indirectamente de las funciones \verb|Comprar billete| y \verb|Acceder|.
	\end{enumerate}


				% FUNCIÓN: MOSTRAR OFERTAS
				
% Revisado por Cristina el día 12/03/2013

\srsfuncion{Mostrar ofertas} \label{fun:mostrarofertas}
	Esta función muestra las ofertas actuales de la compañía.
	
	\begin{enumerate}
		\item \textit{Prioridad}: media.
		\item \textit{Entradas}
		\begin{enumerate}
			\item Las ofertas mostradas deben ser accesibles para el usuario en el momento en que se le muestran.
		\end{enumerate}
		\item \textit{Flujo de operaciones}
		\begin{enumerate}
			\item Se muestra la información destacada de cada oferta.
			\item El cliente puede seleccionar una oferta para ver su información detallada.
		\end{enumerate}
		\item \textit{Respuesta a situaciones no previstas}
		\begin{enumerate}
			\item Si no se puede acceder a la base de datos, no se muestra ningún listado de ofertas y el cliente permanece en el menú actual.
		\end{enumerate}
	\end{enumerate}

				
				% FUNCIÓN: ACCEDER A UNA OFERTA
				\srsfuncion{Acceder a una oferta}
	Esta función debe mostrar una oferta específica elegida por el cliente y podrá dar la opción de comprar lo ofertado.
	
\begin{enumerate}
	\item \textit{Prioridad}: media.
	\item \textit{Entradas}
	\begin{enumerate}
		\item Los datos detallados de la oferta tendrán que estar en el mismo idioma en el que estaba el resumen de la oferta.
	\end{enumerate}
	\item \textit{Flujo de operaciones}
	\begin{enumerate}
		\item La oferta aparecerá descrita detalladamente especificando el/los artículo/s ofertados, por lo que si el usuario estuviese interesado en ella podrá comprar y pagar la oferta.
		\item Si la oferta consta de varias opciones de compra, el cliente podrá elegir entre todas las que haya.
		\item Si el cliente pulsa el botón \verb|Comprar|, se le redireccionará al proceso de realizar pago para que lo efectúe.
	\end{enumerate}
	\item \textit{Respuesta a situaciones no previstas}
	\begin{enumerate}
		\item Si al intentar acceder a una oferta el sistema falla, se mostrará un mensaje de error por pantalla informando de que los datos de esta oferta pueden ser que estén dañados o que simplemente no esté ya disponible. 
	\end{enumerate}
\end{enumerate}


				% FUNCIÓN: INICIAR PAGO BILLETES
				% Caso de uso: iniciar pago billetes de vuelo.
% Obs: para escribir comas en el texto del primer parámetro se han de encerrar entre {}.

\casodeuso{
	% Nombre del caso de uso
	nombre=Iniciar el pago de los billetes de vuelo,
	% Objetivo
	objetivo=Iniciar el proceso de pago del billete para finalizar el proceso de compra.,
	% Entradas
	entradas=La información detallada de los billetes de vuelo.,
	% Precondiciones
	precondiciones={Haberse registrado en la página web de la compañía aérea y haber seleccionado todos los campos detallados de los billetes.},
	% Salidas
	salidas={Se procede a realizar el pago mediante tarjeta de crédido o débito, si procede}.,
	% Postcondiciones en caso de éxito
	postexito=El proceso de compra será redirigido para finalizar el pago mediante tarjeta de crédito o débito.,
	% Postcondiciones en caso de error
	posterror=La compra no se ha realizado y la base de datos no ha sido modificada.,
	% Actores
	actores={El usuario, la base de datos y las entidades financieras.},
}{
	% Tabla de secuencia normal del caso de uso
	\begin{tablasecuencias}
		1 & Mostrar todos los datos de la compra. Si error S-1.\\
		2 & Indicar con claridad las claúsulas de las leyes de protección de datos para que el usuario las acepte y pueda seguir con la compra. Si error S-2.\\ 
		3 & Se redirecciona según a \textbf{Realizar Pago con Tarjeta}.
	\end{tablasecuencias}
}{
	% Tabla de secuencia con errores del caso de uso
	\begin{tablasecuencias}
		S-1 & La reserva de los billetes ha expirado porque ha pasado mucho tiempo desde que se añadieron los billetes al carrito de la compra o por algún error en su almacenamiento. Se mostrará un mensaje indicando que se vuelva a realizar la compra desde el principio.\\3
		S-2 & El cliente no ha aceptado las claúsulas de leyes de protección de datos. Se aborta la operación, se muestra un mensaje por pantalla indicándolo y se vuelve a la página principal de la aplicación.
	\end{tablasecuencias}
}



				% FUNCIÓN: PAGO TARJETA
				\srsfuncion{Realizar pago con tarjeta} \label{fun:pagotarjeta}
	Esta función debe permitir al usuario finalizar el proceso de compra del billete realizando el pago con trajeta de crédito o débito.

\begin{enumerate}
	\item \textit{Entradas}
	\begin{enumerate}
		\item El nombre del usuario (nombre y apellidos) deberán contener únicamente carácteres alfabéticos latinos, acentuados o no, y espacios.
		\item El número de la tarjeta deberá ser una secuencia de 4 bloques de 4 dígitos, todos ellos enteros mayores o iguales que 0.
		\item El código IBAN deberá ser una secuencia de 3 dígitos mayores o iguales que 0.
		\item La fecha de caducidad deberá ser una secuencia de 5 caracteres compuesta por: 2 dígitos para indicar el mes (en el rango 01-12) , una barra `/'  y otros 2 dígitos para indicar el año, que serán las dos últimas cifras del mismo.
	\end{enumerate}
	\item \textit{Flujo de operaciones}
	\begin{enumerate}
		\item Se muestra por pantalla una tabla con los datos de la tarjeta a completar (nombre y apellidos, número, código IBAN -International Bank Account Number- y fecha de caducidad). Una vez completos, se habilita un botón \verb|Confirmar|.
		\item Se transfieren los datos de la tarjeta a la empresa emisora de las mismas para que compruebe si los datos son correctos y la tarjeta está operativa. En este caso, se enviará al instante un mensaje a la compañía indicando que los datos introducidos por el usuario son válidos.
		\item Si está todo correcto, se da la opción de imprimir en el momento la tarjeta de embarque o guardarla como pdf para imprimirla en otro momento.
	\end{enumerate}
	\item \textit{Respuesta a situaciones no previstas}
	\begin{enumerate}
		\item Si no se puede acceder a la base de datos para almacenar la información: se muestra un mensaje de error por pantalla informando de que el proceso de pago se ha interrumpido. Se vuelve a la página anterior.
		\item Si alguno de los datos no es válido: se muestran los campos erróneos y se da la opción de editarlos de nuevo.
		\item Si algún campo no se ha rellenado: se muestra un mensaje indicando que es obligatorio completarlo.
		\item Si la tarjeta está inhabilitada por algún motivo: se muestra un error indicando que el pago no ha podido completarse porque la tarjeta está bloqueada. Se cancela la operación y se vuelve a la página principal de la aplicación.
		\item Si los datos de la tarjeta no son válidos: se muestra un error indicando que el pago no ha podido completarse y se da la opción al usuario de modificar los datos introducidos al principio de la operación.
	\end{enumerate}
\end{enumerate}


				% FUNCIÓN: PRESENTAR RECLAMACIÓN
				% Caso de uso: Presentar reclamacion
% Obs: para escribir comas en el texto del primer parámetro se han de encerrar entre {}.

% Revisado por Juanan el día 12/03/2013

\casodeuso{
	% Nombre del caso de uso
	nombre=Presentar reclamaciones,
	% Objetivo
	objetivo= Presentar una reclamación por parte del cliente.,
	% Entradas
	entradas= Motivo de la queja y detallado de la misma.,
	% Precondiciones
	precondiciones=No hay precondiciones,
	% Salidas
	salidas=Confirmación del envio y número asociado a la reclamación.,
	% Postcondiciones en caso de éxito
	postexito=La notificación se envia al departamento correspondiente.,
	% Postcondiciones en caso de error
	posterror=No se realiza ningún cambio en el sistema.,
	% Actores
	actores=Cliente-usuario de interfaz web.,
}{
	% Tabla de secuencia normal del caso de uso
	\begin{tablasecuencias}
		1 & El usuario completa los datos requeridos. Si error S-1. \\
		2 & Muestra número asignado a la reclamación junto a la confirmación de que la reclamación ha sido tramitada y en breve será atendia por el departamento correspondiente. Si error S-2. 
		
	\end{tablasecuencias}
}{
	% Tabla de secuencia con errores del caso de uso
	\begin{tablasecuencias}
		S-1 & Los campos de datos están incompletos. Se vuelve a 1 de la secuencia normal de uso y se incican los campos erróneos o vacíos.\\
		S-2 & Si no puede transmitir los datos de la reclamación a la base de datos se indica en un mensaje y se insta a que se vuelva a intentar.
	\end{tablasecuencias}
}


						
				% FUNCIÓN: COMPRAR BILLETE
				% Caso de uso: Comprar billete.
% Obs: para escribir comas en el texto del primer parámetro se han de encerrar entre {}.

% Revisado por Cristina el día 11/03/2013

\casodeuso{
	% Nombre del caso de uso
	nombre=Comprar billete,
	% Objetivo
	objetivo=Realizar la compra de un billete.,
	% Entradas
	entradas={Nombre y apellidos del(los) pasajero(s), así como su NIF o equivalentes, reserva de asientos, datos de la persona que paga e identificación del billete a comprar.},
	% Precondiciones
	precondiciones=No hay precondiciones.,
	% Salidas
	salidas={Información detallada de la compra (itinerario de vuelo, precio total a pagar, desglose del precio (pasajero, tarifa, impuestos, tasas y suplementos de de transporte) y datos de los pasajeros).},
	% Postcondiciones en caso de éxito
	postexito=Se registra la compra y se redirecciona a \textit{Iniciar Pago Billetes de Vuelo}.,
	% Postcondiciones en caso de error
	posterror=No se realiza ningún cambio en el sistema.,
	% Actores
	actores=Clientes de la compañía y la base de datos.,
}{
	% Tabla de secuencia normal del caso de uso
	\begin{tablasecuencias}
		1 & El cliente introduce los datos de los pasajeros y de la persona que realiza el pago. Si error S-1.\\
		2 & El cliente selecciona los asientos. Si error S-2. \\
		3 & Se almacenan los datos de la compra en la base de datos. Si error S-3.
	\end{tablasecuencias}
}{
	% Tabla de secuencia con errores del caso de uso
	\begin{tablasecuencias}
		S-1 & El sistema vuelve a 1 de la secuencia normal de uso e indica los campos erróneos.\\
		S-2 & No se puede realizar la reserva de asientos. Se muestra un mensaje de error por pantalla dando la opción de reintentar o volver al menú principal de la aplicación.\\
		S-3 & No se puede conectar con la base de datos, se muestra un mensaje de error por pantalla dando la opción de reintentar o volver al menú principal de la aplicación.
	\end{tablasecuencias}
}


				% FUNCIÓN: CONSULTAR VUELO
				% Caso de uso: consultar vuelos.
% Obs: para escribir comas en el texto del primer parámetro se han de encerrar entre {}.

\casodeuso{
	% Nombre del caso de uso
	nombre=Consultar vuelos,
	% Objetivo
	objetivo={Mostrar al cliente la relación de vuelos operados por la compañía, pudiendo filtrar resultados y buscar por diferentes criterios; permitiendo además obtener información detallada de los vuelos seleccionados.},
	% Entradas
	entradas={Opcionalmente las que correspondan a los filtros (aeropuertos de origen y destino, número de escalas, fecha y hora, precio del billete\ldots). En última instancia, vuelo seleccionado.},
	% Precondiciones
	precondiciones={La información de vuelos ha sido previamente introducida en el sistema, así como los criterios configurados.},
	% Salidas
	salidas={Una lista filtrada de vuelos y, al seleccionar uno de ellos, información detallada del mismo.},
	% Postcondiciones en caso de éxito
	postexito=El cliente puede acceder a la compra de billetes del vuelo seleccionado.,
	% Postcondiciones en caso de error
	posterror={Una pantalla de notificación de error, en la medida de lo posible.},
	% Actores
	actores={Clientes de la compañía, base de datos.},
}{
	% Tabla de secuencia normal del caso de uso
	\begin{tablasecuencias}
		1 & Se muestra una lista de vuelos ordenados por un criterio asignado por defecto. Si error S-1. \\
		2 & El usuario puede filtrar los resultados según diferentes reglas, aparecerá una lista reducida de vuelos (incluso nula).\\
		3 & Seleccionando uno de ellos se accederá a la información especializada en ese servicio, dando acceso a la adquisición de billetes. 
	\end{tablasecuencias}
}{
	% Tabla de secuencia con errores del caso de uso
	\begin{tablasecuencias}
		S-1 & Si no se puede acceder al servidor central o no se puede obtener la información de vuelos, informar al usuario.
	\end{tablasecuencias}
}


		% Termina el índice parcial para funciones
		\stopcontents[tocfunciones]

		\subsection{Requisitos de rendimiento}
			% Los números son completamente inventados.
			El sistema interno deberá reconocer y soportar hasta 4.000 terminales, si bien en determinadas circunstancias agrupaciones de terminales podrían comportarse como uno sólo, gestionándose \textit{in situ} esa eventualidad. El sistema deberá soportar hasta 1.000 usuarios conectados simultáneamente a aquellos servicios del \software de gestión interna que requieran una conexión latente (\textit{KeepAlive}) con el servidor central, ampliándose ese límite a 2.000 al considerar las conexiones de corta duración. Además, el 90\% de las transacciones deberán realizarse en menos de un minuto. La interfaz de gestión interna o servidor web admitirá hasta 6.000 conexiones simultáneas para el contenido estático, si bien el número de conexiones para contenido dinámico estará sometido a los límites marcados por el sistema de gestión interna.\\

			% Los requisitos provienen de los requisitos de Windows Xp.
			Los requisitos de rendimiento de los equipos sobre los que se ejecute la aplicación de \textit{Gestión Interna} demandan los siguientes requisitos de \hardware{}:
				\begin{itemize}
					\item Procesador \textit{Pentium} o equivalente a un mínimo de 233 MHz.
					\item 128 Mb de memoria RAM con velocidad de transferencia de más de 250 MHz.
					\item 10 GB disponibles en el disco duro a una tasa de transferencia superior a 10 MBit/s.
					\item Adaptador de vídeo igual o superior a \textit{Nvidea GeForce 256} o equivalente.
				\end{itemize}

			Bastará con una intensidad media de conexión a Internet para la \textit{Gestión Externa} de la compañía.\\

			La base de datos es un elemento crítico en el rendimiento del sistema. Se prevén realizar miles de interacciones al día, por lo que debe ofrecer alta disponibilidad.

		\subsection{Requisitos lógicos de la base de datos}
			Los requisitos lógicos para las funciones que trabajan con la base de datos son:
			\begin{itemize}
				\item \textbf{Tipos de información usada por diversas funciones: }
				Existen funciones que recibirán cadenas de texto, unas que las usarán para mostrar los datos por pantalla para que el usuario de la aplicación pueda observar la información, y otras que las recibirán para cubrir las necesidades ya que la acción que realiza requiere de dichos datos. En el primer tipo de funciones, existen varias que podrán recibir además imágenes como complemento para mostrar la información. \\ 
				
				Además, especificando en el ámbito de programación, algunas funciones como \nameref{fun:modinventario}, \nameref{fun:EntrMat} y \nameref{fun:ConsulInv} (entre otras), tendrán que manejar objetos de una ``clase Item'' para poder trabajar con ellos en la base de datos y almacenarlos ahí en \gls{arrays}. 
				\item \textbf{Frecuencia de uso: } Habrá funciones que la frecuencia con la que accedan a la base de datos sea muy baja debido a que se accederá a dichas funciones en escasas ocasiones y en momentos específicos, como pueden ser~\nameref{fun:configSystem}, Organizar puestos de trabajo, Modificar empleados, etc. 
				\item \textbf{Capacidades de acceso: } Si comparamos una función que tenga que buscar un cliente específico entre todos los que están en la base de datos, y una función que tenga que mostrar la información económica de la empresa, podemos observar la gran diferencia de capacidades que tienen que tener unas funciones y otras, pues el coste de lo primero es mucho más grande que el coste de lo segundo. Esto se debe a que por lo dicho anteriormente en el ámbito de la programación, no tiene el mismo coste buscar en un array un objeto de la clase Item en concreto a cargar unas cadenas de texto que te pasa la base de datos. Por ello una información tendrá que ser más accesible y fácil de encontrar para unas funciones, y así facilitarle el trabajo por ejemplo al que tenga que realizar la tarea de buscar un cliente/empleado específico. 
				% Eso de lo arrays de la clase Item: demasiado informal y TPístico
				\item \textbf{Entidades de datos y sus relaciones: } Existen datos que deben estar ``incluidos'' en otros datos, esto es por ejemplo, los datos personales de un empleado/cliente deben de estar incluidos en los datos del conjunto de empleados/clientes. Es por eso que en algunos casos para buscar algo específico debes buscar primero en lo general (a no ser que por algún motivo los datos que busques estén copiados temporalmente en otra sección de la base de datos). 
			\end{itemize}

		\subsection{Restricciones de diseño}
			El diseño de este producto está condicionado por la naturaleza y los usos del cliente, de lo que se derivan algunas limitaciones. La existencia previa de un parque de equipos informáticos distribuidos en las instalaciones de la compañía que el cliente quiere conservar imponen la necesidad de acomodar el \software a las posibilidades que ofrecen dichas máquinas. Este en particular, no plantea grandes problemas, ya que la aplicación cliente para la gestión interna --que será instalada en esos equipos-- no exige atributos que no estén presentes en la mayoría de las máquinas; si bien la obsolescencia de algunos equipos puede dejar sin efecto los requisitos de rendimiento indicados en el texto. Por otro lado, el amplio uso de ordenadores compartidos en ciertos sectores de la compañía ha impuesto limitar el almacenamiento local de información. Así mismo, la expansión internacional del cliente y las diferentes legislaciones sobre tratamiento de datos de carácter personal restringen posibles características del producto.
			\subsubsection{Cumplimiento de estándares}
				Este producto ha sido diseñado de acuerdo a ciertos estándares sobre materias diversas.\\
				
				En el terreno económico y financiero, toda operación realizada por la aplicación es conforme a las \gls{NIC} \cite{NIC2006} y especialmente al \textit{Plan General de Contabilidad} de España \cite{PGC2007}.\\

				En el ámbito de la actividad del cliente, se han seguido recomendaciones y estándares relevantes de la \gls{IATA}.
				
		\subsection{Atributos del sistema software}
			El diseño del sistema se ha basado en características como:
			\begin{itemize}
				\item \textit{Seguridad}: implementando sistemas para evitar fraude, sistemas seguros y de encriptado para cobro, procesos de identificación de empleados basados en niveles de acceso y secciones.
				\item \textit{Estabilidad}: es fundamental la protección del sistema frente a errores, manteniendo en todo momento copias de seguridad de las bases de datos y registrando accesos y modificaciones.
				\item \textit{\gls{Usabilidad}}: pone al alcance del usuario una interfaz gráfica sencilla e intuitiva pero a la vez completa.
				\item \textit{Escalabilidad}: se desarrolla el sistema facilitando el mantenimiento del mismo así como la posibilidad de ampliaciones y nuevas funcionalidades futuras.
			\end{itemize}
	
	% Apéndice: Glosario
	\newpage
	\appendix
	\section{Apéndice: Glosario} \label{srs:glosario}
	\printglossary 

	% Subíndice de funciones
	\newpage
	\printcontents[tocfunciones]{}{2}{\section{Lista de funciones}\setcounter{tocdepth}{4}}

	\newpage
	\listoffigures

	\newpage
	\nocite{IEEE:1074}
	\nocite{IEEE:830}
	\bibliography{srs}
	\bibliographystyle{plain}
	
	\listoftodos
\end{document}
