\srsfuncion{Editar cliente} \label{fun:editarcliente}
	Esta función debe mostrar al usuario su perfil y permitirle la modificación del mismo.

\begin{enumerate}
	\item \textit{Entradas}
	\begin{enumerate}
		\item Las opciones que admiten modificación son: Contraseña, domicilio, dirección de correo electrónico, tarjeta de crédito asociada.
		\item La contraseña estará formada por entre 8 y 16 caracteres alfanúmericos.
		\item El código postal del domicilio ha de estar formado por 5 cifras.
		\item Se comprobará la correción del correo electrónico mediante un e-mail de verificación.
		\item Se verificará algoritmicamente los 20 digitos de numeración de la tarjeta de crédito, así como la validez de la misma junto con la fecha de caducidad y el código \gls{CVV2}.
	\end{enumerate}
	\item \textit{Flujo de operaciones}
	\begin{enumerate}
		\item Se muestra un formulario con los campos modificables actuales del usuario.
		\item El usuario modifica al menos uno de los diferentes campos. La válided de los campos modificados se comprueba al pulsar el botón Guardar Cambios.
		\item Si se encuentra algún dato erroneo se informa de ello. Si se verificán los datos modificados se confirma la operación, actualizando la \gls{base_de_datos}.
		\item Se muestra el perfil de usuario resultado de la modificación.
	\end{enumerate}
	\item \textit{Respuesta a situaciones no previstas}
	\begin{enumerate}
		\item Si no se puede acceder o modificar a la base de datos: informa de la no disponibilidad temporal al usuario.
	\end{enumerate}
\end{enumerate}
