\srsfuncion{Realizar mantenimiento}
	Esta función debe permitir registrar la información sobre un mantenimiento realizado.

\begin{enumerate}
	\item \textit{Prioridad}: alta.
	\item \textit{Entradas}
	\begin{enumerate}
		\item La información introducida deberá componerse únicamente de carácteres alfabéticos latinos, acentuados o no, dígitos, espacios y otros signos de puntuación.
		\item La cantidad de items del inventario utilizados deberá ser, para cada tipo de item, menor o igual que el número de items disponibles.
	\end{enumerate}
	\item \textit{Flujo de operaciones}
	\begin{enumerate}
		\item Se muestra por pantalla un listado de los mantenimientos programados para el usuario, ordenados por fecha por defecto.
		\item El usuario selecciona un mantenimiento e introduce el informe completo de la operación y de los resultados de ésta. Además, deberá seleccionar una opción \verb|Mantenimiento completado| en caso de que el mantenimiento haya podido finalizarse con éxito. Una vez que haya introducido toda la información detallada del mantenimiento, deberá pulsar en el botón \verb|Guardar|.
		\item A continuación, se muestra por pantalla el listado del material mecánico disponible en el inventario de la empresa. El usuario seleccionará los items que haya utilizado en el mantenimiento e indicará, para cada uno de ellos, la cantidad utilizada. Cuando haya terminado, deberá pulsar en el botón \verb|Guardar|.
		\item Se muestra un mensaje confirmando el registro del mantenimiento.
	\end{enumerate}
	\item \textit{Respuesta a situaciones no previstas}
	\begin{enumerate}
		\item Si no se puede acceder a la base de datos: se muestra un mensaje de error por pantalla y regresa a la página anterior.
	\end{enumerate}

\end{enumerate}
