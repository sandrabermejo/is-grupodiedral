\documentclass[draft]{ltxdoc}
\usepackage[utf8x]{inputenc}
\usepackage[T1]{fontenc}
\usepackage{lmodern}
\usepackage[spanish]{babel}
\usepackage{xcolor}
\usepackage[none]{hyphenat}
\usepackage{fancyvrb}
\usepackage{multicol}
\usepackage{hyperref}
\usepackage{ifpdf}
\usepackage{attachfile2}

\begin{document}
\title{Documentación del paquete \textsf{isdiedral}}
\author{Grupo diedral\\(\small{ }Ingeniería del Software / Doble Grado de Ingeniería Informática y Matemáticas )\\
\texttt{\color{gray}alguien at ucm dot es}}
\date{2012/2311 \qquad v0.1}

% \DeclareOption{tight}{\@tightshtoctrue}

\maketitle

\begin{abstract}
	Un paquete \LaTeX{} que implementa una serie de herramientas para la creación de la documentación del proyecto de Ingeniería del Sofware desarrollado por el \textit{Grupo Diedral} en el Doble Grado de Ingeniería Informática y Matemáticas.
\end{abstract}

\tableofcontents

\section{Manual de utilización}
\subsection{Casos de uso}
\subsubsection{El comando \texttt{casosdeuso}} \label{comandocasos}
	El comando \verb|casosdeuso| proporciona una plantilla para descripciones tabuladas de casos de uso, basada en el ejemplo de la página 40 de los apuntes de la asignatura.

	\begin{Verbatim}[frame=single]
\casosdeuso{parámetros clave-valor}{secuencia de ejecución normal} 
{secuencias de ejecución errónea}
	\end{Verbatim}

	En el primer parámetro del comando, se ha de introducir una serie de asignaciones clave-valor de acuerdo al formato
fijado por \textsf{keyval}. Los subparámetros admitidos son los siguientes:
	\begin{multicols}{2}
		\begin{itemize}
			\item \verb|nombre|: nombre del caso de uso.
			\item \verb|objetivo|: objetivos o descripción del caso de uso.
			\item \verb|entradas|
			\item \verb|precondiciones|
			\item \verb|salidas|
			\item \verb|postexito|: postcondición en caso de éxito.
			\item \verb|posterror|: postcondición en caso de error.
			\item \verb|actores|
		\end{itemize}
	\end{multicols}

	El segundo y tercer parámetro se corresponden con descripciones de secuencias de funcionamiento del caso de uso. Para introducir esas secuencias se recomienda recurrir al entorno \verb|listasecuencias| (véase ~\ref{entornosecuencias}).

\subsubsection{El entorno \texttt{listasecuencias}} \label{entornosecuencias}
	El entorno \verb|listasecuencias| no es más que una tabla con ciertas adaptaciones para mejor integración en las celdas de secuencias de la tabla de casos de uso generada por \verb|casosdeuso| (véase ~\ref{comandocasos}). Carece de sentido su utilización fuera de este contexto. Cada fila de la tabla consta de un identificador o enumerador en la primera columna y una descripción en la segunda.

\subsubsection{Ejemplo de uso}
\begin{Verbatim}[numbers=left, label=Ejemplo]
% Ejemplo de caso de uso.
		
\casosdeuso{
	% Nombre del caso de uso
	nombre=Registrar,
	% Otros datos:
	objetivo=Registrar un usuario,
	entradas=Datos del futuro usuario,
	precondiciones=El usuario no existe previamente,
	salidas=Identificador de usuario asignado,
	% Postcondiciones tras funcionamiento correcto
	postexito=El usuario ha quedado registrado,
	% Postcondiciones si error
	posterror=No se ha producido cambio alguno,
	actores=Personal administrativo de RRHH.
}
{
	% Tabla de secuencia normal
	\begin{tablasecuencias}
		1 & Introducir los datos. Si error S-1.\\
		2 & Confirmar los datos.
	\end{tablasecuencias}
}
{
	% Tabla de secuencias de error
	\begin{tablasecuencias}
		S-1 & Se da a escoger entre abortar y reintentar.
	\end{tablasecuencias}
}
\end{Verbatim}

<<Ubicar aquí el resultado>>

\subsection{Especificación de requisitos software}
	\paragraph{srsfuncion}comando de sección de orden inferior a la \verb|subsubsection| para organizar las funciones en el documento.
	\paragraph{software}comando que imprime el término software en cursiva.
	\paragraph{hardware}comando que imprime el término hardware en cursiva.

\subsection{Control de cambios}
	\paragraph{Entorno tablacambios}{entorno de lista para homogeneizar la lista de cambios. La tabla consta de cuatro columnas: \textit{versión}, \textit{fecha} , \textit{autores} y \textit{descripción}.}


	% Adjunta el código fuente del documento
	\ifpdf
		\vspace*{\fill}
		\begin{center}
			\color{gray} \hrule
			\textattachfile[description={Código fuente del documento}, color=gray]{isdiedral.tex}{Código fuente del documento}
			\hrule
		\end{center}
	\fi
\end{document}
