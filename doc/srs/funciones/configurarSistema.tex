\srsfuncion{Configurar el sistema}  \label{fun:configSystem}
	Esta función debe permitir modificar la configuración general del sistema restringiéndose a los servicios contratados por la empresa para que algunos datos sean modificables o no.

	\begin{enumerate}
		\item \textit{Entradas}
		\begin{enumerate}
			\item No se podrán establecer configuraciones inválidas o que vayan a causar error en otras funciones.
			\item Sólo se podrán modificar aquellas cosas que la empresa haya elegido al contratar el producto.
			\item La configuración del sistema no puede quedar dañada por la modificación del empleado.
		\end{enumerate}
		\item \textit{Flujo de operaciones}
		\begin{enumerate}
			\item Acceder a la base de datos del sistema para cargar todos sus datos en la página y sólo dejar disponibles de modificación los que tengan autorización de ello.
			\item Se modifican los campos que se quieren reconfigurar en el sistema.
			\item Para mandar a validar los datos se pulsará el botón \verb|Guardar configuración|. 
			\item Una vez que la configuración ha sido reestablecida con éxito, se mostrará un mensaje por pantalla indicando que se ha modificado la configuración del sistema general.
		\end{enumerate}
		\item \textit{Respuesta a situaciones no previstas}
		\begin{enumerate}
		  \item Si no se ha podido acceder a la base de datos del sistema para poder modificar así la configuración, se mostrará un mensaje por pantalla indicando que en estos momentos no se puede realizar una configuración del sistema. Se le mandará al menú principal.
			\item Si los datos introducidos en los campos son incorrectos, se marcarán dichos incorrectos y se mostrará un mensaje de error indicando un breve resumen de los detalles por los que esos campos son incorrectos.
		\end{enumerate}

	\end{enumerate}
