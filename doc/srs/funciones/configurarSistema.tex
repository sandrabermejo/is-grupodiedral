\srsfuncion{Configurar el sistema}  \label{fun:configSystem}
	Esta función permite modificar ciertos parámetros de la configuración general del sistema.

	\begin{enumerate}
		\item \textit{Prioridad}: media
		\item \textit{Usuarios}: personal administrativo o de los servicios informáticos debidamente autorizado.

		\item \textit{Entradas} \\

			Los parámetros configurables se restringirán a los indicados a continuación. Téngase en cuenta que algunos conllevan una considerables cambios difícilmente aplicables, son de uso totalmente infrecuente y requieren la supervisión activa del usuario.
			\begin{enumerate}
				\item Fecha y hora del servidor central. Su aplicación no será inmediata ya que requerirá procesos de sincronía con los equipos conectados y probablemente el reinicio de ciertos componentes del sistema.
				\item Nombre de la compañía. El nombre que aparece en los informes generados por la aplicación, así como en las ventanas de su interfaz.
				\item Moneda. Tipo monetario en el que se expresan los importes comerciales y la información económica de la empresa. Su implantación no es inmediata pues puede requerir conversión de los datos en uso o históricos. La guía del usuario será necesaria para la aplicación de este cambio.
				\item Criterio de ordenación por defecto para cada caso de uso que presente información ordenada\footnote{\textit{Consultar ficha empleado}, \textit{Consultar ficha cliente}, \textit{Establecer organización laboral}\ldots}. Permitirá seleccionar entre los criterios de ordenación disponibles para cada caso de uso aquél que se tomará por omisión. De aplicación inmediata.
			\end{enumerate}

			Las entradas deberán atenerse a lo siguiente:
			\begin{enumerate}
				\item No se podrán establecer configuraciones inválidas o que vayan a causar error en otras funciones.
				\item La configuración del sistema no puede quedar dañada por la modificación del empleado.
			\end{enumerate}

		\item \textit{Flujo de operaciones}
		\begin{enumerate}
			\item Mostrar los valores previos de los parámetros configurables.
			\item Se modifican los campos que se quieren reconfigurar en el sistema.
			\item El usuario, una vez satisfecho con los cambios realizados, confirma su modificación. Se comprueba la coherencia de los datos antes del envío. 
			\item Una vez que la configuración ha sido establecida con éxito, se mostrará un mensaje por pantalla indicando que se ha modificado la configuración del sistema general.
			\item Los cambios que así lo requieran presentarán al usuario pantallas para afinar el proceso de consolidación de los ajustes. Además si la implantación requiriese un determinado tiempo para llevarse a cabo se monitorizará el proceso. Algunos cambios pueden requerir el reinicio del sistema o alguno de sus componentes para llegar a efecto.
		\end{enumerate}
		\item \textit{Respuesta a situaciones no previstas}
		\begin{enumerate}
		  \item Si no se ha podido acceder a la base de datos del sistema para poder modificar así la configuración, se mostrará un mensaje por pantalla indicando que en estos momentos no se puede realizar una configuración del sistema. Se le mandará al menú principal.
			\item Si los datos introducidos en los campos son incorrectos, se marcarán dichos incorrectos y se mostrará un mensaje de error indicando un breve resumen de los detalles por los que esos campos son incorrectos.
		\end{enumerate}

	\end{enumerate}
