% Revisado por Juanan el día 12/03/2013

\srsfuncion{Acceder gestión interna}
	Función que permite al empleado de la compañía aérea identificarse en el sistema para así poder hacer uso de las funciones que ofrece de acuerdo a las limitaciones establecidas para su puesto de trabajo (\gls{Login}).
		
	\begin{enumerate}
		\item \textit{Prioridad}: alta.
		\item \textit{Entradas}
		\begin{enumerate}
			\item El identificador de usuario deberá introducirse para poder acceder, además de la contraseña personal de cada usuario.
			\item En el campo contraseña serán válidos los caracteres ASCII imprimibles.
		\end{enumerate}
		\item \textit{Flujo de operaciones}
		\begin{enumerate}
			\item Se muestran por pantalla dos campos a rellenar: uno para introducir el id del usuario y otro para escribir la contraseña.
			\item El usuario para identificarse y acceder al sistema despues de haber introducido los datos requeridos selecciona la opción acceder.
		\end{enumerate}
		\item \textit{Respuesta a situaciones no previstas}
		\begin{enumerate}
			\item Si algún campo introducido no es válido, se indica y se da la opción de introducirlo de nuevo. Existe un límite de 5 intentos de acceso fallido en un periodo de tiempo corto (15 minutos).
		\end{enumerate}
	
\end{enumerate}
