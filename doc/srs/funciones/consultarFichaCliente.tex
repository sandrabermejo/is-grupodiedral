
% Revisado por Juanan el día 12/03/2013

\srsfuncion{Consultar ficha de clientes} \label{fun:consultarcliente}
	Esta función permite consultar la relación de clientes de la compañía, así como obtener detalles sobre los servicios contratados y los datos personales del usuario. Esto le otorga a los diseñadores --y finalmente a los usuarios-- de esta función una fuerte responsabilidad en el tratamiento de la información.

	\begin{enumerate}
		\item \textit{Prioridad}: alta.
		\item \textit{Entradas}\\
			Al menos uno de los siguientes: identificador personal del cliente (ver \nameref{srs:idpersonal}); nombre o apellidos; o número de vuelo.
			\begin{enumerate}
				\item El identificador personal será validado de acuerdo a las especificaciones de su formato. El tipo de identificador ha de ser seleccionado previamente.
				\item El nombre y los apellidos han de cumplir las condiciones especificadas en este documento, así como el número de vuelo.
			\end{enumerate}
		\item \textit{Flujo de operaciones}
			\begin{enumerate}
				\item Se muestra un formulario de búsqueda, permitiendo al usuario rellenar los campos y ejecutar la petición.
				\item Si no hay resultados se muestra un mensaje; en caso contrario se muestra una pantalla con lista de clientes incluyendo identificador, nombre y apellidos y fecha del último vuelo (si hubiese).
				\item Cuando el usuario selecciona un cliente se muestra una pantalla con la información detallada del mismo.
			\end{enumerate}
		\item \textit{Respuesta a situaciones no previstas}
			\begin{enumerate}
				\item Si no se puede realizar la búsqueda en la base de datos: se informa del error y restablece el sistema para que se pueda reintentar la operación.
				\item Si no se puede obtener la información de la base de datos: informar al cliente y quedarse en la última pantalla funcional.
			\end{enumerate}
		\item \textit{Relación con otras funciones}\\
		La función está relacionada con \nameref{fun:editarcliente}, \nameref{fun:registrarse} (en la gestión interna) y \nameref{fun:bajacliente}. Se puede utilizar los formularios de consulta para la edición en el \verb|Editar cliente| interno.
	\end{enumerate}
