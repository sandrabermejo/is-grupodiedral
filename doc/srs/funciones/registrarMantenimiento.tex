
% Revisado por Cristina el día 12/03/2013

\srsfuncion{Registrar mantenimiento}
	Esta función permite registrar la información sobre un mantenimiento realizado.

	\begin{enumerate}
		\item \textit{Prioridad}: alta.
		\item \textit{Entradas}
		\begin{enumerate}
			\item La información introducida debe componerse únicamente de carácteres alfabéticos latinos, acentuados o no, dígitos, espacios y otros signos de puntuación.
			\item La cantidad de items del inventario utilizados debe ser, para cada tipo de item, menor o igual que el número de items disponibles.
		\end{enumerate}
		\item \textit{Flujo de operaciones}
		\begin{enumerate}
			\item Se muestra por pantalla un listado de los mantenimientos programados para el usuario, ordenados por fecha por defecto.
			\item El empleado mecánico selecciona un mantenimiento.
			\item El empleado introduce el informe completo de la operación y de los resultados de ésta. Además, selecciona una opción \verb|Mantenimiento completado| en caso de que el mantenimiento esté finalizado con éxito. 
			\item El empleado confirma el registro.
			\item Se muestra por pantalla el listado del material mecánico disponible en el inventario de la empresa. El usuario selecciona los items que haya utilizado en el mantenimiento e indica, para cada uno de ellos, la cantidad utilizada.
			\item El empleado confirma el registro.
			\item Se muestra un mensaje confirmando el registro del mantenimiento.
		\end{enumerate}
		\item \textit{Respuesta a situaciones no previstas}
		\begin{enumerate}
			\item Si no se puede acceder a la base de datos, se muestra un mensaje de error por pantalla dando la opción de reintentar o volver al menú principal de la aplicación.
		\end{enumerate}
	\end{enumerate}
