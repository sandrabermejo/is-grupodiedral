% Revisado por Cristina el día 12/03/2013

\srsfuncion{Programar horario}
	Esta función permite programar los horarios de los empleados, actualizando la base de datos.

	\begin{enumerate}
		\item \textit{Prioridad}: media.
		\item \textit{Entradas}
		\begin{enumerate}
			\item El horario introducido debe estar en el rango entre una jornada mínima previamente\break establecida y la jornada laboral máxima legal.
			\item El calendario laboral debe fijar el período de descanso diario, semanal y un período vacacional en relación a los días trabajados, asi como los días festivos, todo ello de acuerdo a la ley vigente.
		\end{enumerate}
		\item \textit{Flujo de operaciones}
		\begin{enumerate}
			\item Se muestra por pantalla el listado de personal, ordenados por orden alfabético según el primer apellido. Junto al nombre de cada empleado se indica el puesto de trabajo dentro de la empresa.
			\item El personal administrativo selecciona un empleado y un nuevo horario para el mismo (mañana, tarde o noche), indicando la fecha a partir de la cual ese horario pasa a estar vigente. 
			\item El adminisitrativo confirma los cambios.
			\item Se muestra el nuevo horario y se envia automáticamente un correo a la cuenta de correo personal del usuario para notificarle su nuevo horario.
		\end{enumerate}
		\item \textit{Respuesta a situaciones no previstas}
		\begin{enumerate}
			\item Si no se puede acceder a la base de datos del personal, se muestra un mensaje de error por pantalla y se vuelve a la página principal del sistema.
			\item Si no se puede conectar con la base de datos para almacenar la información, se muestra un mensaje de error por pantalla y se vuelve a la página principal del sistema.
		\end{enumerate}
	\end{enumerate}
