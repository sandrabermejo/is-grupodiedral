% Revisado por Cristina el día 12/03/2013

\srsfuncion{Consultar vuelo}
	Esta función muestra al cliente la relación de vuelos operados por la compañía, pudiendo filtrar resultados y buscar por diferentes criterios, y permitiendo además obtener información detallada de los vuelos seleccionados.
	
	\begin{enumerate}
		\item \textit{Prioridad}: media.
		\item \textit{Entradas}
		\begin{enumerate}
			\item Opcionalmente las que correspondan a los filtros (aeropuertos de origen y destino, número de escalas, fecha y hora, precio del billete\ldots). En última instancia, vuelo seleccionado..
			\item Los aeropuertos de origen y destino son un conjunto finito y han de haber sido configurados previamente. Internamente se componen de nombre, ciudad y código \gls{IATA}; externamente se visualizan como una secuencia de texto configurada. El usuario podrá seleccionar uno entre ellos para cada entrada (operación que se puede abreviar introduciendo el código IATA).
			\item Las fechas y horas introducidas deben ser válidas.
			\item El precio del billete viene dado por defecto en euros.
		\end{enumerate}
		\item \textit{Flujo de operaciones}
		\begin{enumerate}
			\item El usuario filtra los vuelos según los criterios anteriormente descritos, completando al menos uno de los diferentes campos.
			\item Si no se obtiene ningún resultado se informa de ello. 
			\item Si se producen varias coincidencias se muestra una lista (indicando precios, fechas y\break aeropuertos) que permite la selección de alguno de ellos.
			\item El usuario selecciona un vuelo y accede a la información detallada del mismo, dando acceso a la adquisición de billetes.
		\end{enumerate}
		\item \textit{Respuesta a situaciones no previstas}
		\begin{enumerate}
			\item Si no se puede acceder a la base de datos, se muestra un mensaje de error y se da la opción de reintentar o abortar el proceso.
		\end{enumerate}
	\end{enumerate}
