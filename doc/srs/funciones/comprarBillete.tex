
% Revisado por Juanan el día 12/03/2013

\srsfuncion{Comprar billete} \label{fun:comprarbillete}
	Esta función permite al usuario iniciar el proceso de compra de billetes de un vuelo seleccionado.		

	\begin{enumerate}
		\item \textit{Prioridad}: alta.	
		\item \textit{Entradas}
			\begin{enumerate}
				\item El nombre y apellidos de los pasajeros debe contener únicamente carácteres alfabéticos latinos, acentuados o no, y espacios.
				\item La selección de asientos se realiza en función de los asientos disponibles en el momento de la compra de los billetes.
				\item Los datos de la persona que paga son, de momento, nombre y apellidos,  que deberá contener únicamente carácteres alfabéticos latinos, acentuados o no, y espacios.							
			\end{enumerate}
		\item \textit{Flujo de operaciones}
			\begin{enumerate}
				\item Se muestra por pantalla el itinerario del vuelo a comprar, el precio total a pagar, desglosándolo en la parte correspondiente al pasajero, la tarifa, los impuestos, tasas y suplementos de transporte.
				\item Se muestran los asientos disponibles del avión en el momento de la compra, dando la opción al cliente de reservar los asientos que desee en su vuelo.
				\item El cliente debe introducir los datos (nombre y apellidos) de la persona que realizará el pago de billetes (no tiene porqué ser un pasajero). Si todo ha ido correctamente se habilita una opción para \verb|Continuar| y se redirecciona a~\ref{fun:iniciarpago}.
			\end{enumerate}
		\item \textit{Respuesta a situaciones no previstas}
			\begin{enumerate}
				\item Si no se puede mostrar la información del vuelo seleccionado: se muestra un mensaje de error por pantalla y se vuelve a la página anterior.
				\item Si alguno de los datos no es válido: se muestran los campos erróneos y se da la opción de editarlos de nuevo.
				\item Si algún campo no se ha rellenado: se muestra un mensaje indicando que es obligatorio completarlo.
				\item Si no se puede realizar la reserva de asientos: se muestra el error y sigue la secuencia normal.
				\item Si no se puede conectar con la base de datos y, por tanto, no se puede realizar la compra,  se mostrará un mensaje indicando que el proceso de pago ha sido interrumpido. Se vuelve a la página principal de la aplicación.
			\end{enumerate}
	\end{enumerate}
								
