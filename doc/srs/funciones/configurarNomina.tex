
% Revisado por Juanan el día 12/03/2013

\srsfuncion{Configurar nómina}
	Función que permite confeccionar y almacenar las nóminas mensuales de cada empleado de la compañía según las incidencias que se hayan producido en el último mes.
						
	\begin{enumerate}
		\item \textit{Prioridad}: alta.
		\item \textit{Entradas}
			\begin{enumerate}
				\item El usuario introduce en el campo de incidencias los motivos por los cuales este mes se produce una modificación en la nómina como, por ejemplo, aumentos o disminuciones del salario a causa de horas extras, comisiones, sustituciones, huelgas\ldots
				\item El sueldo del mes será el resultante de sumar el sueldo base y el correspondiente a las incidencias.
			\end{enumerate}
		\item \textit{Flujo de operaciones}
			\begin{enumerate}
				\item Mediante \verb{Consultar empleado} se accede a la configuración de la nómina de cada empleado. 
				\item A continuación, el usuario rellena de forma obliglatoria el campo de incidencias y confirma que quiere aplicar los cambios en la nómina del cliente seleccionado.
				\item Por último, se registra la nómina en la base de datos.
			\end{enumerate}
		\item \textit{Respuesta a situaciones no previstas}
			\begin{enumerate}
				\item Si no se puede establecer conexión con la base de datos: se muestra un mensaje de error y se da la opción de reintentar o abortar el proceso.
				\item Si no se puede registrar la nómina: anular la operación y volver a la página principal del sistema.
			\end{enumerate}				
		\item \textit{Relación con otras funciones}\\
		La función está relacionada con \nameref{fun:consultarnomina} y \nameref{fun:consultarempleado}.
	\end{enumerate}
								
