%
%	SRS: Glosario
%

\PrerenderUnicode{ñ}
\PrerenderUnicode{ó}
\PrerenderUnicode{í}

\newglossaryentry{la_app}{
	name=la aplicación,
	description={producto sofware descrito en el documento: \textit{Airline Common Environment}.},
	sort=aplicación
}

\newglossaryentry{el_programa}{
	name=el programa,
	description={véase \gls{la_app}.},
	sort=aplicación
}

\newglossaryentry{Internet}{
	name=Internet,
	description={conjunto descentralizado y globalizado de redes de comunicación interconectadas, cuyo origen se sitúa en la puesta en marcha de la \textit{Advanced Research Projects Agency Network} (ARPANET) en 1969 y que se popularizó a finales del siglo XX y principios de XXI}
}

\newglossaryentry{CNED_3}{
	name=superior a educación secundaria,
	description={\textit{referido al nivel de estudios}, en términos de la \gls{CNED} y la \gls{CINE} (niveles 3 y 4), habiendo cursado 2ª etapa de educación secundaria o postsecundaria no superior. En el texto, salvo que se diga lo contrario, incluye también niveles superiores}
}

\newglossaryentry{CNED_5}{
	name=educación superior o doctorado,
	description={\textit{referido al nivel de estudios}, en términos de la \gls{CNED} y la \gls{CINE} (niveles 5 y 6), habiendo cursado 1º o 2º ciclo de educación superior, o doctorado}
}

\newglossaryentry{CVV2}{
	name=código CVV2/CVC2,
	description={El CVV2/CVC2 es un código de seguridad de tres dígitos que se encuentra impreso al dorso de las tarjetas de crédito. Este código se utiliza para determinar que usted está en posesión de la tarjeta empleada para el pago. Todas las tarjetas MasterCard y Visa, tanto de crédito como de débito, deben llevar un CVV2.}
}

\newglossaryentry{base_de_datos}{
	name=base de datos,
	description={Una base de datos o banco de datos es un conjunto de datos pertenecientes a un mismo contexto y almacenados sistemáticamente para su posterior uso}	
}

\newglossaryentry{numero_de_vuelo}{
	name=número de vuelo,
	description={código de vuelo acorde el formato definido por la \gls{IATA} en \cite{SSIM} omitiendo la\break identificación de la compañía}	
}

% ¿Deberíamos usar este concepto?
\newglossaryentry{combobox}{
	name=cuadro desplegable,
	description={control de interfaz gráfica que muestra un valor comprendido en una colección finita, editable o no, y permite escoger otros valores de la colección mostrando una lista\break desplegable}
}

\newglossaryentry{captcha}{
	name=captcha,
	description={(\textit{Completely Automated Public Turing test to tell Computers and Humans Apart}, Prueba de Turing pública y automática para diferenciar máquinas y humanos) prueba para comprobar que un usuario es humano que consiste en hacerle escribir el texto representado en una imagen distorsionada de forma que para una máquina sería difícil descifrar}
}

\newglossaryentry{Portable_Document_Format}{
	name=Portable Document Format,
	description={(Formato de Documento Portátil) formato de almacenamiento de\break documentos digitales independiente de plataformas de \software o \hardware diseñado originalmente por \textit{Adobe Systems} y estandarizado en el ISO 32000-1:2008}
}
\newglossaryentry{C++}{
	name=C++,
	description={C++ es un lenguaje de programación diseñado a mediados de los años 1980 por Bjarne Stroustrup (1950). La intención de su creación fue el extender al exitoso lenguaje de programación C con mecanismos que permitan la manipulación de objetos}
}
\newglossaryentry{Qt}{
	name=Qt,
	description={Qt es una biblioteca multiplataforma ampliamente usada para desarrollar aplicaciones con interfaz gráfica de usuario, así como también para el desarrollo de programas sin interfaz gráfica, como herramientas para la línea de comandos y consolas para servidores},
}
\newglossaryentry{HTML5}{
	name=HTML5,
	description={HTML5 (HyperText Markup Language, versión 5) es la quinta revisión importante del lenguaje básico de la World Wide Web, HTML}
}
\newglossaryentry{Usabilidad}{
	name=Usabilidad,
	description={ Neologismo usabilidad (del inglés usability -facilidad de uso-)}
}
\newglossaryentry{arrays}{
	name=array,
	description={En programación, una matriz o vector (llamados en inglés arrays) es una zona de almacenamiento continuo, que contiene una serie de elementos del mismo tipo, los elementos de la matriz}
}
\newglossaryentry{RAE}{
	name=RAE,
	description={(Real Academia Española) es una institución cultural con sede en Madrid. Junto con otras veintiuna Academias correspondientes en sendos países donde se habla español, conforman la Asociación de Academias de la Lengua Española}
}
\newglossaryentry{PayPal}{
	name=PayPal,
	description={PayPal es una empresa estadounidense, propiedad de eBay, perteneciente al sector del comercio electrónico por Internet que permite la transferencia de dinero entre usuarios que tengan correo electrónico, una alternativa al tradicional método en papel como los cheques o giros postales.}
}
\newglossaryentry{QuickPay}{
	name=Quick Pay,
	description={Quick Pay es un servicio que permite a clientes nacionales e internacionales pagar\break directamente por vía electrónica, transferiendo pagos electrónicamente y dando la opción, cuando sea posible, de convertir automáticamente el dinero a la moneda local}
}
\newglossaryentry{WesternUnion}{
	name=Western Union,
	description={Western Union es una compañía estadounidense que ofrece servicios financieros y de comunicación}
}
\newglossaryentry{Login}{
	name=Login,
	description = {En el ámbito de seguridad informática, login o logon (en español ingresar o entrar) es el proceso mediante el cual se controla el acceso individual a un sistema informático mediante la identificación del usuario utilizando credenciales provistas por el usuario}
}

\newglossaryentry{Linux}{
	name=Linux,
	description = {En el texto, Linux no hace referencia al núcleo Linux, sino a distribuciones basadas en Linux generalmente utilizadas (frecuentemente con el \software del proyecto GNU), como Debian,\break Mandrake, Red Hat o SuSE.}
}

\newglossaryentry{Microsoft Windows}{
	name={Microsoft Windows},
	description = {Familia de sistemas operativos desarrollados y comercializados por multinacional estadounidense Microsoft, que comienza con la versión 1.0 (1985) y actualmente termina con la versión 8 (2012)}
}


\newacronym{ACE}{ACE}{Airline Common Environment}
\newacronym{CNED}{CNED}{\textit{Clasificación Nacional de Educación} (España)}
\newacronym{IATA}{IATA}{\textit{Asociación Internacional de Transporte Aéreo}}
\newacronym{INE}{INE}{\textit{Instituto Nacional de Estadística} (España)}
\newacronym{CINE}{CINE-97}{\textit{Clasificación Internacional Normalizada de la Educación}}
\newacronym{NIC}{NIC}{\textit{Normas Internacionales de Contabilidad}}
\newacronym{PTLA}{PTLA}{Piloto de Tranporte de Línea Aérea}
\newacronym{ATPL}{ATPL}{\textit{Airline Transport Pilot License}, veáse \gls{PTLA}}
\newacronym{UTF8}{UTF-8}{\textit{Unicode Tranformation Format 8-bit}}
\newacronym{DNI}{DNI}{\textit{Documento Nacional de Identidad} (España)}
\newacronym{NIF}{NIF}{\textit{Número de Identificación Fiscal} (España)}
\newacronym{NIE}{NIE}{\textit{Número de Identidad de Extranjero} (España)}
\newacronym{PDF}{PDF}{\textit{\gls{Portable_Document_Format}}}
\newacronym{GCC}{GCC}{\textit{GNU Compiler Collection}}
