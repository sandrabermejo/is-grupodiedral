%
%	Documento de análisis (diagramas)
%

\documentclass[11pt, a4paper, twoside, titlepage]{article}
\usepackage[utf8x]{inputenc}
\usepackage[T1]{fontenc}
\usepackage[spanish, es-ucroman]{babel}
\usepackage{lmodern}
\usepackage{anysize}
\usepackage{fancyhdr}
\usepackage[none]{hyphenat}
\usepackage[colorlinks, linkcolor=red]{hyperref}
\usepackage{float}
\usepackage{lscape}
\usepackage{pdflscape}
\usepackage[doc=analisis]{isdiedral}

% Nombre del documento (para futuras referencias)
\newcommand*{\doctitle}{Análisis}
\newcommand*{\docversion}{1.0}


%%% Configuraciones %%%
\marginsize{2.5cm}{2cm}{2cm}{2cm}

% Usa como familia tipográfica por defecto "Sans"
\renewcommand{\familydefault}{\sfdefault}

% Establece la profundidad hasta la cual se numeran los elementos de sección
\setcounter{secnumdepth}{4}

% Establece la profundidad de niveles de sección que aparece en el TOC
\setcounter{tocdepth}{4}

% Configuración de los encabezados
\encabezadodiedral{\doctitle{} \docversion}
\pagestyle{fancy}

\renewcommand*{\thepage}{\sffamily \roman{page}}

\title{\doctitle\\\textsl{Airline Common Environment}}
\author{Grupo Diedral}

% Metadatos del pdf
\hypersetup{
pdfinfo={
	Author={Grupo Diedral},
	Title={\doctitle{} \docversion},
	Subject={Airline Common Environment},
	Keywords={análisis, UML, Airline Common Environment, Ingeniería del Software}
}
}

\begin{document}
	% Tabla de cambios
	\begin{tablacambios}
		1.0 & 28 de abril de 2013 & Todos & Versión inicial \\ \hline
		1.1 & 13 de mayo de 2013  & Todos & Correcciones puntuales
	\end{tablacambios}

	% Cita inicial
	\fijacitainicial{Para mostrar este diagrama adecuadamente, necesitaría una pantalla de cuatro dimensiones. Sin embargo, debido a los recortes del gobierno, nos tendremos que conformar con una pantalla bidimensional}{\emph{Historia del tiempo}, Stephen Hawking}

	% Portada
	\portadaace{\doctitle}{\docversion}

	\tableofcontents
	\newpage

	\iniciarnumeraciondiedral

	\begin{prologo}
		Este documento recoge la documentación generada durante la fase de análisis del proyecto. En primer lugar se incluye el diagrama de modelo de dominio que presenta y relaciona los conceptos generales relacionados con el dominio del producto. A continuación se introduce el diagrama de paquetes. Finalmente se muestra el análisis detallado del paquete {\itshape Gestión Externa}, incluyendo el diagrama de clases y los diferentes diagramas de comunicación correspondientes a cada caso de uso.\\

		Esta información ha sido especificada por medio del \itshape{Lenguaje Unificado de Modelado} (UML).

	\paragraph*{Nota sobre herramientas empleadas:} para la elaboración de los diagramas se ha utilizado el\break programa {\normalfont BoUML} en su versión {\normalfont 4.23 patch 7 `ultimate'}.
	\end{prologo}

	\section{Diagrama de modelo de dominio}
		\begin{figure}[H]\centering
			\hspace*{-.5cm}
			\includegraphics[scale=.722]{diagramas/modelodominio.pdf}
		\end{figure}	
	\newpage

\begin{landscape}
	\section{Diagrama de paquetes}
		\vfill
		\begin{figure}[H]\centering
			\hspace*{-.5cm}
			\includegraphics[scale=1]{diagramas/paquetes.pdf}
		\end{figure}
		\vfill
\end{landscape}
	\newpage

\begin{landscape}
	\section{El paquete {\itshape Gestión Interna}}
		\subsection{Diagrama de clases}

			\begin{figure}[H]\centering
				\vspace{2cm}
				\hspace{-2cm}
				\includegraphics[scale=1]{diagramas/diagramaclases.pdf}
			\end{figure}		
\end{landscape}
\begin{landscape}
			\begin{figure}[H]\centering
				\includegraphics[scale=1]{diagramas/sitioweb.pdf}
			\end{figure}
\end{landscape}

		\subsection{Diagramas de comunicación}

			\subsubsection{Acceder web}
				\begin{figure}[H]\centering
					\includegraphics[scale=.75]{diagramas/accederweb.pdf}
				\end{figure}

			\subsubsection{Comprar billete}
				\begin{figure}[H]\centering
					\includegraphics[scale=.72]{diagramas/comprarbillete.pdf}
				\end{figure}

			\subsubsection{Consultar oferta}
				\begin{figure}[H]\centering
					\includegraphics[scale=.85]{diagramas/consultaroferta.pdf}
				\end{figure}

			\subsubsection{Consultar vuelos}
				\begin{figure}[H]\centering
					\includegraphics[scale=.71]{diagramas/consultarvuelos.pdf}
				\end{figure}
			
			\subsubsection{Editar datos personales}
				\begin{figure}[H]\centering
					\includegraphics[scale=.86]{diagramas/editardatospersonales.pdf}
				\end{figure}


			\subsubsection{Iniciar pago billetes}
				\begin{figure}[H]\centering
					\includegraphics[scale=.76]{diagramas/iniciarpagobilletes.pdf}
				\end{figure}

			\subsubsection{Mostrar ofertas}
				\begin{figure}[H]\centering
					\includegraphics[scale=.72]{diagramas/mostrarofertas.pdf}
				\end{figure}

			\subsubsection{Realizar pago con tarjeta}
				\begin{figure}[H]\centering
					\includegraphics[scale=.7]{diagramas/pagotarjeta.pdf}
				\end{figure}

			\subsubsection{Presentar reclamación}
				\begin{figure}[H]\centering
					\includegraphics[scale=.82]{diagramas/presentarreclamacion.pdf}
				\end{figure}

			\subsubsection{Registrarse}
				\begin{figure}[H]\centering
					\includegraphics[scale=.8]{diagramas/registrarse.pdf}
				\end{figure}

			\subsubsection{Restablecer contraseña}
				\begin{figure}[H]\centering
					\includegraphics[scale=.77]{diagramas/restablecercontrasena.pdf}
				\end{figure}

			\subsubsection{Ver información de vuelo contratado}
				\begin{figure}[H]\centering
					\includegraphics[scale=.8]{diagramas/verinfovuelocontratado.pdf}
				\end{figure}
\end{document}
