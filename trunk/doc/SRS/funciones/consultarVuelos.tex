\srsfuncion{Consultar vuelos}
	Esta función debe mostrar al cliente la relación de vuelos operados por la compañía, pudiendo filtrar resultados y buscar por diferentes criterios; permitiendo además obtener información detallada de los vuelos seleccionados.
	
\begin{enumerate}
	\item \textit{Prioridad}: media.
	\item \textit{Entradas}
	\begin{enumerate}
		\item Opcionalmente las que correspondan a los filtros (aeropuertos de origen y destino, número de escalas, fecha y hora, precio del billete\ldots). En última instancia, vuelo seleccionado..
		\item Los aeropuertos de origen y destino son un conjunto finito y han de haber sido configurados previamente. Internamente se componen de nombre, ciudad y código \gls{IATA}; externamente se visualizan como una secuencia de texto configurada. El usuario podrá seleccionar uno entre ellos para cada entrada (operación que se puede abreviar introduciendo el código IATA).
		\item No se debe permitir introducir fechas u horas no válidas.
		\item El precio del billete será indicado en la moneda correspondiente al pais de origen del cliente.
	\end{enumerate}
	\item \textit{Flujo de operaciones}
	\begin{enumerate}
		\item Se muestra un formulario con los campos anteriormente descritos y un botón \verb|Buscar|.
		\item El usuario completa al menos uno de los diferentes campos. Ningún campo permite entradas erróneas por definición.
		\item Si no se obtiene ningún resultado se informa de ello con un cuadro de diálogo.Si se producen varias coincidencias se muestra una lista (indicando precios, fechas y aeropuertos) que permite la selección de alguno de ellos.
		\item Seleccionando uno de ellos se accederá a la información especializada en ese servicio, dando acceso a la adquisición de billetes.
	\end{enumerate}
	\item \textit{Respuesta a situaciones no previstas}
	\begin{enumerate}
		\item Si no se puede acceder al servidor central o no se puede obtener la información de vuelos, informar al usuario.
		\item Si la respuesta de búsqueda en la base datos no tiene lugar o es errónea: informar al usuario y permanecer en la pantalla actual como si no se hubiese buscado.
		\item Si la información de vuelos no puede ser obtenida: informar al usuario y permanecer en la pantalla actual como si no se hubiese buscado o seleccionado un vuelo.
	\end{enumerate}
\end{enumerate}
