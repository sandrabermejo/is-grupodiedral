\srsfuncion{Modificar inventario} \label{fun:modinventario}
	Esta función debe permitir modificar los diversos datos almacenados sobre los elementos del inventario.

\begin{enumerate}
	\item \textit{Entradas}
	\begin{enumerate}
		\item Las entradas del inventario se ordenarán en primera instancia por el criterio configurado como por defecto. Se admitirán los siguientes criterios: número de registro, fecha de entrada en almacén, cantidad, destino de uso (oficina, mecánica, edificios, aeronáutico\ldots) y valor de adquisición.
		\item La fecha será introducida por medio de un campo autoverificado de fecha acorde al formato de visualización de fechas especificado en este documento.
		\item El número de elementos introducido debe ser un número entero mayor o igual que cero. Se deberá escoger la unidad de medida asociada, entre las que se cuenta \textit{número de elementos} además de las configuradas por la aplicación (pueden ser \textit{litros}, \textit{metros}, \textit{pies}\ldots).
		\item El destino de uso será uno de los admitidos en una lista finita fijada con antelación.
		\item El valor de adquisición vendrá dado por un número real positivo en los términos especificados para ese formato en este documento.	% Muchas referencias a la nada (pero debería existir esa información sobre formatos).
	\end{enumerate}
	\item \textit{Flujo de operaciones}
	\begin{enumerate}
		\item Se muestra por pantalla una tabla con la lista de elementos disponibles en el inventario ordenada por el criterio por defecto. Se da la opción al usuario de ordenarla siguiendo diferentes criterios, cada uno de los cuales aparece en una pestaña en la parte superior de la tabla.
		\item Se podrán añadir o eliminar elementos, introduciendo en cada caso una explicación al efecto, además de los datos necesarios en el caso de la inclusión. Se mostrará una pantalla para introducir los datos requeridos.
		\item Para cada elemento, se podrán modificar cualquiera de sus campos salvo el número de registro.
		\item Cuando se modifique la cantidad de un elemento, con un dato válido, se guardarán los cambios y se mostrará la tabla actualizada.
		% Como haya muchos objetos el programa se va a quedar cargando tablas.
	\end{enumerate}
	\item \textit{Respuesta a situaciones no previstas}
	\begin{enumerate}
		\item Si no se puede acceder a la base de datos del inventario: se muestra por pantalla un mensaje informando al usuario tras un somero diagnóstico de las causas. Se intenta enviar informe de fallo al servidor central.
		\item Si no se pueden modificar los datos de un determinado elemento: mostrar un mensaje por pantalla informando al usuario del error.
		\item Si algún dato introducido no es válido: se muestra un mensaje con el error y no se permite aplicar los cambios hasta que no haya sido corregido.
	\end{enumerate}
\end{enumerate}
