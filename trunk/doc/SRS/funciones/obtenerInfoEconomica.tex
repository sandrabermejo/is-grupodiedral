\srsfuncion{Obtener información económica}
	Esta función debe mostrar la información económica de la empresa de forma personalizada para cada perfil de usuario. Entre los datos que puede mostrar se incluyen resúmenes actualizados del inventario e infraestructura de la empresa, de los recursos humanos, de los gastos e ingresos, de la cotización en bolsa e inversiones financieras, además de las cuentas anuales actuales y la de años anteriores. Esta visualización puede incluir enlaces a documentos externos para los que haría falta una aplicación aparte. Se considera la opción de incluir gráficos en la pantalla mostrada.

	\begin{enumerate}
		\item \textit{Prioridad}: media.
		\item \textit{Entradas}\\
			Esta función no recibe parámetros de entrada.
			\item \textit{Flujo de operaciones}
				\begin{enumerate}
					\item Se cargará de la base de datos la información económica de la empresa que puede visualizar el usuario que ha accedido en estos momentos al sistema.
					\item Se mostrará esta información económica de la empresa cargada de la base de datos.
			\end{enumerate}
		\item \textit{Respuesta a situaciones no previstas}
			\begin{enumerate}
				\item Si no se puede obtener la información económica de la base de datos: se informa al usuario del error y se aborta la carga de la pantalla.
				\item Si no se pueden generar las visualizaciones: intentar omitirlas en la medida de lo posible y si no fuese posible aborta la carga de la pantalla.
			\end{enumerate}
		\item \textit{Relación con otras funciones}\\
			Esta función puede dar acceso a otras como \verb|Consultar inventario| y depende indirectamente de \nameref{fun:editareconomica}.
	\end{enumerate}
