\srsfuncion{Programar horarios}
	Esta función debe permitir programar los horarios de los empleados, actualizando la base de datos.

\begin{enumerate}
	\item \textit{Entradas}
	\begin{enumerate}
		\item El horario introducido deberá estar en el rango entre una jornada mínima previamente establecida y la jornada laboral máxima legal.
		\item El calendario laboral deberá fijar el período de descanso diario, semanal y un período vacacional en relación a los días trabajados, asi como los días festivos, todo ello de acuerdo a la ley vigente.
	\end{enumerate}
	\item \textit{Flujo de operaciones}
	\begin{enumerate}
		\item Se muestra por pantalla el listado de personal con horario flexible, ordenados por orden alfabético según el primer apellido. Junto al nombre de cada empleado se indicará el puesto de trabajo dentro de la empresa.
		\item Tras seleccionar un empleado, el encargado del registro deberá entonces seleccionar un nuevo horario para el empleado (mañana, tarde o noche), indicando la fecha a partir de la cual ese horario pasa a estar vigente. Deberá confirmar los cambios pulsando en el botón \verb|Programar horario|.
		\item Cuando los datos se hayan modificado, se mostrará el nuevo horario y se enviará automáticamente un correo a la cuenta de correo personal del usuario para notificarle su nuevo horario.
	\end{enumerate}
	\item \textit{Respuesta a situaciones no previstas}
	\begin{enumerate}
		\item Si no se puede acceder a la base de datos del personal: se muestra un mensaje de error por pantalla y se vuelve a la página principal del sistema.
		\item Si no se puede conectar con la base de datos para almacenar la información: se muestra un mensaje de error por pantalla informando de que la horario no ha podido actualizarse y se vuelve a la página principal del sistema.
		\item Si no se ha podido ordenar en orden alfabético: mostrar la información desordenada e indicar que no se ha podido ordenar.
	\end{enumerate}

\end{enumerate}
