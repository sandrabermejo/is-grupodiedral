%% Casos de uso %%

\documentclass[11pt, a4paper, twoside]{report}
\usepackage[utf8x]{inputenc}
\usepackage[T1]{fontenc}
\usepackage{lmodern}
\usepackage[spanish]{babel}
\usepackage[none]{hyphenat}
\usepackage{fancyhdr}
\usepackage{isdiedral}
\usepackage{anysize}
\usepackage{graphicx}
\usepackage{float}
\usepackage{rotating}
\usepackage[colorlinks, linkcolor=black]{hyperref}


% Configuraciones
\marginsize{2cm}{2cm}{2cm}{2cm}

% Usa como familia tipográfica por defecto "Sans"
\renewcommand{\familydefault}{\sfdefault}

% Para las clases "book" o semejantes desactiva la impresión del número de capítulo
\renewcommand*\thesection{\arabic{section}}

% Cambia el nombre del TOC a "Índice" (pues en "report" se le denomina por defecto "Índice general")
\addto{\captionsspanish}{\renewcommand*{\contentsname}{Índice}}

% Establece la profundidad de niveles de sección que aparece en el TOC
\setcounter{tocdepth}{3}

% Configura los encabezados y pies de paǵina
\encabezadodiedral{Casos de uso}
\pagestyle{fancy}
\renewcommand*{\thepage}{\sffamily \roman{page}}

\title{Casos de uso\\\textsl{Airline Common Environment}}
\author{Grupo Diedral}

% Metadatos del pdf
\hypersetup{
pdfinfo={
	Author={Grupo Diedral},
	Title={Casos de uso},
	Subject={Airline Common Environment},
	Keywords={Casos de uso;Airline Common Environment;Ingeniería del Software}
}
}

\begin{document}
	% Página de título
	\portadaace{Casos de uso}
	
	% Índice
	\tableofcontents
	\listoffigures
	\newpage

	% Historial de cambios
%	\begin{tablacambios}
%		1 & 13 de diciembre de 2013 & \begin{tabular}{l}Cristina Alonso\\Natalia Angulo\\Sandra Bermejo\\Juan Andrés Claramunt\\Rubén Rafael Rubio\end{tabular} & Continuación
%	\end{tablacambios}
%	\newpage

	% Cada caso de uso se puede escribir en un archivo aparte e importar
	% usando  "include"  que  utiliza  necesariamente  una  página (para
	% evitarlo se puede usar "input" en su lugar).
	\section{Casos de uso}
	
	% Gestión interna
	\subsection{Gestión interna} \vspace{.5cm}
	\iniciarnumeraciondiedral

	\begin{sidewaysfigure}	% LaTex lo coloca donde le parece (como debe hacer)
		\hspace*{-2cm}\includegraphics[scale=.9]{diagramas/gestioninterna1.pdf}
		\caption{Diagrama de gestión interna (administrativo)}
	\end{sidewaysfigure}

	\begin{sidewaysfigure}
		\hspace*{0cm}\includegraphics[scale=.9]{diagramas/gestioninterna2.pdf}
		\caption{Diagrama de gestión interna (mecánico y personal de a bordo)}
	\end{sidewaysfigure}

	\srsfuncion{Acceder}
	\todo[inline]{No es bueno que aparezcan referencias demasiado tecnológicas. El objetivo del prototipo no es detallar hasta este punto los requisitos, aunque en otra parte pueda servir para hacer un diseño detallado de la interfaz (pero, insisto, no en los requisitos).}
	Función que debe permitir hacer \textit{\gls{Login}} al usuario (en este caso empleado de la compañía aérea) para poder acceder al sistema y observar los datos que su puesto de trabajo le permite ver.
		
	\begin{enumerate}
		\item \textit{Prioridad}: alta.
		\item \textit{Entradas}
		\begin{enumerate}
			\item El id de usuario (correo de un empleado asociado a la empresa) deberá introducirse para poder acceder, además de la contraseña personal de cada usuario.
			\item En el campo contraseña serán inválidos los caracteres que no sean ni alfanuméricos ni otros como `.', `,', `?', `¿', `!', y `¡'.
		\end{enumerate}
		\item \textit{Flujo de operaciones}
		\begin{enumerate}
			\item Se muestran por pantalla dos campos a rellenar: uno para introducir el id del usuario y otro para escribir la contraseña.
			\item Para que el usuario pueda acceder al sistema, además de rellenar los datos deberá pulsar el botón \verb|Aceptar| situado justo abajo de los dos campos nombrados anteriormente.
		\end{enumerate}
		\item \textit{Respuesta a situaciones no previstas}
		\begin{enumerate}
			\item La contraseña o el usuario han sido inválidos. Se mostrará en rojo los campos que tienen error junto con un mensaje que indicará que se ha introducido mal el usuario o la contraseña. Se le da la opción de volver a introducir los datos.
		\end{enumerate}
	
\end{enumerate}
								

	% Caso de uso: registrar empleado.
% Obs: para escribir comas en el texto del primer parámetro se han de encerrar entre {}.

\casodeuso{
	% Nombre del caso de uso
	nombre=Registrar empleado,
	% Objetivo
	objetivo=Añadir un nuevo usuario a la base de datos del personal de la compañía con unos determinados permisos de acceso.,
	% Entradas
	entradas={Nombre de usuario, datos personales y puesto de trabajo al que se incorpora dentro de la compañía.},
	% Precondiciones
	precondiciones={Que el futuro usuario haya sido contratado por la empresa y que éste haya firmado la LOPD (Ley Orgánica de Protección de Datos), y que el encargado del registro, personal de Recursos Humanos, haya accedido al sistema y a la opción \textit{Registrar empleado}.},
	% Salidas
	salidas={Confirmación de la creación de la cuenta o exposición de los datos incorrectos, según proceda.},
	% Postcondiciones en caso de éxito
	postexito=El usuario queda registrado en la base de datos pero su cuenta no estará activada hasta que sea verificada por el departamento de intervención.,
	% Postcondiciones en caso de error
	posterror=La base de datos no ha sido alterada.,
	% Actores
	actores={El personal administrativo encargado de registrar nuevos empleados, el usuario a registrar y la base de datos.},
}{
	% Tabla de secuencia normal del caso de uso
	\begin{tablasecuencias}
		1 & El usuario registrador inserta los datos del futuro usuario siendo el nombre de usuario el que se asignará a la cuenta de correo de la compañía. Si error S-1.\\
		2 & El sistema genera una contraseña aleatoria. \\
		3 & Se crea una cuenta de correo. \\
		4 & Se vuelcan los datos a la base de datos. Si error S-2.\\
		5 & Se muestra un mensaje de confirmación del registro. \\
		6 & Se imprime un documento con los datos de acceso para el usuario.
	\end{tablasecuencias}
}{
	% Tabla de secuencia con errores del caso de uso
	\begin{tablasecuencias}
		S-1 & Alguno de los datos no es válido. El sistema vuelve al paso 1 de la secuencia normal de uso e indica los campos erróneos.\\
		S-2 & No se puede conectar con la base de datos. Se cancela la operación, se muestra un mensaje de error por pantalla y se vuelve a la página principal del sistema.
	\end{tablasecuencias}
}



	% Caso de uso: verificar registro empleado.
% Obs: para escribir comas en el texto del primer parámetro se han de encerrar entre {}.

% Revisado por Cristina y Juanan el día 11/03/2013

\casodeuso{
	% Nombre del caso de uso
	nombre=Verificar el registro de un empleado,
	% Objetivo
	objetivo=Establecer una nueva cuenta de usuario como válida.,
	% Entradas
	entradas={Los datos del usuario previamente registrado.},
	% Precondiciones
	precondiciones={El operador de la aplicación tiene credenciales que le habilitan para realizar dicha operación. El nuevo usuario está registrado correctamente en el sistema.},
	% Salidas
	salidas=Se confirma la operación.,
	% Postcondiciones en caso de éxito
	postexito={La cuenta del usuario queda verificada y, por tanto, pasa a estar activa y totalmente operativa.},
	% Postcondiciones en caso de error
	posterror=La cuenta del usuario permanece inactiva.,
	% Actores
	actores=El personal del departamento de intervención y la base de datos.,
}{
	% Tabla de secuencia normal del caso de uso
	\begin{tablasecuencias}
		1 & Se extrae de la base de datos del sistema el listado de usuarios pendientes de verificación. Si error S-1.\\
		2 & El empleado selecciona el usuario.\\
		3 & Se muestran los datos por pantalla.\\
		4 & Se comprueba que sean correctos. Si alguno de los datos no es correcto, S-2.\\
		5 & Se verifica la cuenta del usuario y se modifica en la base de datos. Si error S-3.
	\end{tablasecuencias}
}{
	% Tabla de secuencia con errores del caso de uso
	\begin{tablasecuencias}
		S-1 & No se puede conectar con la base de datos, se muestra un mensaje de error por pantalla dando la opción de reintentar o volver al menú principal de la aplicación.\\
		S-2 & Se cancela la operación y vuelve al menú principal.\\
		S-3 & No se puede acceder a la base de datos. Se cancela la operación, se muestra un mensaje de error por pantalla dando la opción de reintentar o volver al menú principal de la aplicación.
	\end{tablasecuencias}
}



	% Caso de uso: consultar ficha empleado
% Obs: para escribir comas en el texto del primer parámetro se han de encerrar entre {}.

% Revisado por Cristina y Juanan el día 11/03/2013

\casodeuso{
	% Nombre del caso de uso
	nombre=Consultar ficha empleado,
	% Objetivo
	objetivo={Mostrar la lista de empleados de la compañía, permitiendo buscar y filtrar resultados, así como información detallada de cada empleado en particular.},
	% Entradas
	entradas={Opcionalmente, campos de búsqueda.},
	% Precondiciones
	precondiciones=El operador de la aplicación tiene credenciales que le habilitan para realizar dicha operación.,
	% Salidas
	salidas=Lista de empleados e información detallada sobre el seleccionado.,
	% Postcondiciones en caso de éxito
	postexito=No se realiza ningún cambio en el sistema.,
	% Postcondiciones en caso de error
	posterror=No se realiza ningún cambio en el sistema.,
	% Actores
	actores={Los directivos, empleados de recursos humanos y la base de datos.}.
}{
	% Tabla de secuencia normal del caso de uso
	\begin{tablasecuencias}
		1 & Se extrae de la base de datos del sistema el listado de empleados. Si error S-1.\\
		2 & Se muestra por pantalla la lista de empleados.\\
		3 & El usuario puede filtrar los resultados y buscar empleados según diferentes criterios, como tipo de empleado, duración en la empresa, departamentos, etc.\\
		4 & Se selecciona un empleado.\\
		5 & Se muestra la información detallada del empleado. Si error S-1.
	\end{tablasecuencias}
}{
	% Tabla de secuencia con errores del caso de uso
	\begin{tablasecuencias}
		S-1 & No se puede conectar con la base de datos, se muestra un mensaje de error por pantalla dando la opción de reintentar o volver al menú principal de la aplicación.
	\end{tablasecuencias}
}

	% Caso de uso: editar empleado.
% Obs: para escribir comas en el texto del primer parámetro se han de encerrar entre {}.

\casodeuso{
	% Nombre del caso de uso
	nombre=Modificar ficha de empleado.,
	% Objetivo
	objetivo=Permite actualizar la información referente a uno de los empleados.,
	% Entradas
	entradas=Los datos nuevos a actualizar.,
	% Precondiciones
	precondiciones={Haber accedido al sistema con un usuario válido y los permisos necesarios.},
	% Salidas
	salidas=La ficha de empleado con los datos modificados actualizados.,
	% Postcondiciones en caso de éxito
	postexito=Los cambios efectuados se guardan en la base de datos.,
	% Postcondiciones en caso de error
	posterror={El sistema central no ha sufrido cambios y, por tanto, no se modifica la ficha del empleado.},
	% Actores
	actores={El personal administrativo de la compañía con permisos para realizar estas modificaciones, base de datos.},
}{
	% Tabla de secuencia normal del caso de uso
	\begin{tablasecuencias}
		1 & Muestra lista de empleados. Si error S-1. \\
		2 & Selecciona ficha de empleado. \\
		3 & Muestra ficha seleccionada.\\
		4 & Selecciona campos a modificar. \\
		5 & Introducir datos nuevos. \\
		6 & Comprueba corrección en los datos introducidos. Si error S-2. \\
		7 & Almacenar los cambios en la base de datos. Si error S-3. \\
		8 & Muestra confirmación y a continuación vuelve al paso 3.
	\end{tablasecuencias}
}{
	% Tabla de secuencia con errores del caso de uso
	\begin{tablasecuencias}
		S-1 & Si no se puede conectar con la base de datos se muestra mensaje  de tipo \textit{información no disponible temporalmente} y se vuelve al paso 1 de la secuencia normal.\\
		S-2 & Alguno de los datos introducidos no es válido. Volver al paso 3 de la secuencia normal de uso indicando los campos erróneos. \\
		S-3 & No se han podido almacenar los cambios en la base de datos. Se cancela la operación, se informa y se vuelve a la página principal del sistema.
	\end{tablasecuencias}
}


	\srsfuncion{Dar de baja empleado}
	Función que debe permitir dar de baja a un empleado, eliminando su información personal de acuerdo a la legislación vigente y derogando las autorizaciones de acceso al sistema y las instalaciones.
							
	\begin{enumerate}
		\item \textit{Prioridad}: alta.
		\item \textit{Entradas}
			\begin{enumerate}
				\item El usuario debe introducir en el campo de causa los motivos por los cuales se ha decidido llevar a cabo esta operación.
			\end{enumerate}
		\item \textit{Flujo de operaciones}
			\begin{enumerate}
				\item Una vez que el usuario ha pulsado el botón de \verb|Dar de baja| en la información detallada del empleado, rellenará de forma obliglatoria el campo de causa y confirmará que quiere realizar la operación dar de baja con el cliente seleccionado.
				\item El empleado dado de baja pierde inmediatamente todo acceso al sistema y su información personal se elimina de la base de datos de acuerdo a la legislación vigente.
				\item Por último, se registra la operación en la \textit{Cola de Supervisión}.
			\end{enumerate}
		\item \textit{Respuesta a situaciones no previstas}
			\begin{enumerate}
				\item Si no se puede establecer conexión con la base de datos: se muestra un mensaje de error y se da la opción de reintentar o abortar el proceso.
				\item Si no se puede registrar  la operación en la \textit{Cola de Supervisión} se anula la operación o se solicita al operador que informe a dicho departamento manualmente.
			\end{enumerate}
	\end{enumerate}
								

	% Caso de uso: Configurar sistema general.
% Obs: para escribir comas en el texto del primer parámetro se han de encerrar entre {}.
% A peticionario de Cristina ->
% Revisado por Rubén el día xx/0x/2013
\casodeuso{
	% Nombre del caso de uso
	nombre=Configurar sistema general,
	% Objetivo
	objetivo=Configurar el sistema.,
	% Entradas
	entradas={Cambios que se quieran realizar en la configuración del sistema.},
	% Precondiciones
	precondiciones={Que un personal administrativo autorizado acceda a la configuración del sistema para realizar los cambios que desee dentro de las opciones posibles.},
	% Salidas
	salidas={Cambiar la configuración.},
	% Postcondiciones en caso de éxito
	postexito={La configuración que ha elegido el personal administrativo habrá sido restablecida por lo que haya indicado.},
	% Postcondiciones en caso de error
	posterror={No se ha realizado ningún cambio en el sistema y la configuración sigue igual que estaba.},
	% Actores
	actores=El personal administrativo cuyo rol implique cambiar la configuración del sistema de la empresa.,
}{
	% Tabla de secuencia normal del caso de uso
	\begin{tablasecuencias}
		1 & El usuario registrado accede a la configuración del sistema.\\
		2 & Introduce las modificaciones que desea realizar en el sistema.\\
		3 & Se modifican los datos de la configuración indicados previamente. Si error S-1.\\
		3 & Se confirma que los datos han sido modificados con éxito.
	\end{tablasecuencias}
}{
	% Tabla de secuencia con errores del caso de uso
	\begin{tablasecuencias}
		S-1 & Los datos indicados no se pueden eliminar. Si quiere modificar esos datos asegúrese de que son datos modificables. Si lo son, ha ocurrido un fallo en el sistema que debe de comunicar al personal técnico para su revisión.
	\end{tablasecuencias}
}



	% Caso de uso: establecer organización laboral
% Obs: para escribir comas en el texto del primer parámetro se han de encerrar entre {}.

% Revisado por Cristina el día 12/03/2013

\casodeuso{
	% Nombre del caso de uso
	nombre=Establecer organización laboral,
	% Objetivo
	objetivo={Establecer las configuraciones generales sobre la organización laboral de la compañía, como secciones y puestos de trabajo, detallando la información relacionada (salario base\dots) y fijando los privilegios de acceso dentro de la aplicación.},
	% Entradas
	entradas=Los valores a configurar.,
	% Precondiciones
	precondiciones=El operador de la aplicación tiene credenciales que le habilitan para realizar esta operación.,
	% Salidas
	salidas=El valor final de los parámetros configurados.,
	% Postcondiciones en caso de éxito
	postexito=Los cambios efectuados se guardan en la base de datos.,
	% Postcondiciones en caso de error
	posterror=No se realiza ningún cambio en el sistema.,
	% Actores
	actores={Personal administrativo, \textit{Recursos Humanos} y \textit{Servicios Informáticos}.},
}{
	% Tabla de secuencia normal del caso de uso
	\begin{tablasecuencias}
		1 & Se muestran sendas listas de categorías obtenidas del servidor central, permitiendo su edición general con precaución.\\
		2 & El usuario edita los detalles de cada configuración. Si error S-1.\\
		3 & Se almacenan los cambios en la base de datos. Si error S-2.
	\end{tablasecuencias}
}{
	% Tabla de secuencia con errores del caso de uso
	\begin{tablasecuencias}
		S-1 & Alguno de los datos introducidos no es válido. Vuelve a 1 de la secuencia normal de uso indicando los campos erróneos.\\
		S-2 & No se puede conectar con la base de datos, se muestra un mensaje de error por pantalla dando la opción de reintentar o volver al menú principal de la aplicación.
	\end{tablasecuencias}
}

	\srsfuncion{Acceder horarios}
	Esta función permite al empleado consultar sus horarios.

\begin{enumerate}
	\item \textit{Prioridad}: media.
	\item \textit{Entradas}
	\begin{enumerate}
		\item Fecha o rango de fechas del que se quiere consultar la nómina.
		\item Se comprobará que estas fechas marcadas sean válidas.
	\end{enumerate}
	\item \textit{Flujo de operaciones}
	\begin{enumerate}
		\item Se seleccionarán fechas en las que se desea consultar la nómina.
		\item Se mostrará una plantilla con los turnos de trabajo del empleado distinguiendo turnos de mañana, tarde o noche, y resaltando los días festivos.
	\end{enumerate}
	\item \textit{Respuesta a situaciones no previstas}
	\begin{enumerate}
		\item Si no se encuentran en el sistema datos relativos a la búsqueda se notifica al usuario y se vuelve al menú principal del sistema.
	\end{enumerate}
\end{enumerate}

	% Caso de uso: programar horarios.
% Obs: para escribir comas en el texto del primer parámetro se han de encerrar entre {}.

\casodeuso{
	% Nombre del caso de uso
	nombre=Programar horarios.,
	% Objetivo
	objetivo=Añadir a la base de datos del sistema los horarios de los empleados.,
	% Entradas
	entradas=Los nuevos horarios y el personal al que afectan.,
	% Precondiciones
	precondiciones={Haber accedido al sistema con un usuario válido perteneciente al Personal de planificación de operaciones y elegir la opción \textit{Programar horarios}.},
	% Salidas
	salidas={Los horarios de los empleados quedan registrados o modificados, si procede.},
	% Postcondiciones en caso de éxito
	postexito=Los horarios de los empleados han sido actualizados en la base de datos.,
	% Postcondiciones en caso de error
	posterror=El sistema no ha sufrido ningún cambio.,
	% Actores
	actores=El Personal de planificación de operaciones y la base de datos.,
}{
	% Tabla de secuencia normal del caso de uso
	\begin{tablasecuencias}
		1 & Extraer de la base de datos el listado de personal de la compañía cuyo horario pueda ser modificado. Si error S-1. \\
		2 & Seleccionar el empleado cuyo horario vaya a ser modificado. \\
		3 & Seleccionar un nuevo horario para el empleado y una fecha a partir de la cual será vigente. Si error S-2. \\
		4 & Almacenar los cambios en la base de datos. Si error S-3. \\
		5 & Mostrar un mensaje de confirmación de la operación.
	\end{tablasecuencias}
}{
	% Tabla de secuencia con errores del caso de uso
	\begin{tablasecuencias}
		S-1 & No se ha podido extraer la información de la base de datos. Mostrar un mensaje de error por pantalla y volver a la página principal del sistema.\\
		S-2 & El horario o la fecha introducidos no son válido por algún motivo. Vuelve al paso 3 de la secuencia normal de uso indicando que el horario no es válido. \\
		S-3 & No se ha podido conectar con la base de datos. Se cancela la operación, se muestra el error por pantalla y vuelve a la página principal del sistema.
	\end{tablasecuencias}
}


	\srsfuncion{Consultar plan de vuelo}
	\todo[inline]{¿De qué manera depende de ``Introducir plan de vuelo''?}
	Esta función debe mostrar una formulario de búsqueda de vuelos y devolver al usuario una lista de vuelos según sus restricciones para así obtener el plan de vuelo detallado de cada servicio.

\begin{enumerate}
	\item \textit{Prioridad}: alta.
	\item \textit{Entradas}
	\begin{enumerate}
		\item Las opciones que admite el formulario de búsqueda son: \gls{numero_de_vuelo}, fecha y hora de origen y llegada, aeropuerto de origen y destino, y personal que forma parte de la tripulación.
		\item El número de vuelo caracteriza e identifica unívocamente a todos los vuelos operados por la compañía. Si dicho parámetro es introducido, todos los demás deben ser ignorados en la búsqueda. Se ha de comprobar que el formato del número de vuelo se corresponde con una sucesión de 4 caracteres numéricos (si el número de caracteres es menor que 4 se completará con \verb|0| por la izquierda).
		\item No se debe permitir introducir fechas u horas no válidas.
		\item Los aeropuertos de origen y destino son un conjunto finito y han de haber sido configurados previamente. Internamente se componen de nombre, ciudad y código \gls{IATA}; externamente se visualizan como una secuencia de texto configurada. El usuario podrá seleccionar uno entre ellos para cada entrada (operación que se puede abreviar introduciendo el código IATA).
		\item En los circunstancias que se estime oportuno, aparecerá una lista editable de personal que formará parte de la tripulación del vuelo para filtrar resultados.
	\end{enumerate}
	\item \textit{Flujo de operaciones}
	\begin{enumerate}
		\item Se muestra un formulario con los campos anteriormente descritos y un botón \verb|Buscar|.
		\item El usuario completa al menos uno de los diferentes campos. Ningún campo permite entradas erróneas por definición, salvo el de código de aeropuertos, que tras introducir un código de aeropuerto válido lo selecciona en el \gls{combobox} de aeropuertos y si no es válido no tiene efecto alguno, restaurándose su valor original.
		\item Si no se obtiene ningún resultado se informa de ello con un cuadro de diálogo. Si el resultado es único se muestra el plan de vuelo. Si se producen varias coincidencias se muestra una lista (indicando número de vuelo, fechas y aeropuertos) que permite la selección de alguno de ellos.
		\item Se accede a una nueva pantalla donde la información del plan de vuelo es distribuida de forma organizada. Desde esta pantalla se puede volver a la pantalla anterior de resultados y búsqueda.
	\end{enumerate}
	\item \textit{Respuesta a situaciones no previstas}
	\begin{enumerate}
		\item Si no se puede acceder a la base de datos de configuraciones: no mostrar la pantalla de inicio de la función e informar del error al usuario.
		\item Si la respuesta de búsqueda en la base datos no tiene lugar o es errónea: informar al usuario y permanecer en la pantalla actual como si no se hubiese buscado.
		\item Si la información de vuelos no puede ser obtenida: informar al usuario y permanecer en la pantalla actual como si no se hubiese buscado o seleccionado un vuelo.
	\end{enumerate}
	\item \textit{Relación con otras funciones}
		Depende indirectamente de la función \verb|Introducir plan de vuelo|.
\end{enumerate}

\begin{figure}[ht]\centering
\includegraphics[scale=.6]{imagenes/BuscarPlanVuelo.png}
\caption{Pantalla aproximada de la búsqueda del plan de vuelo}
\end{figure}

	% Caso de uso: introducir plan de vuelo
% Obs: para escribir comas en el texto del primer parámetro se han de encerrar entre {}.

% Revisado por Cristina el día 11/03/2013

\casodeuso{
	% Nombre del caso de uso
	nombre=Introducir plan de vuelo,
	% Objetivo
	objetivo=Añadir un plan de vuelo a la lista de vuelos de la compañía aérea.,
	% Entradas
	entradas={La información detallada del vuelo: número de vuelo (4 dígitos), fecha, hora de salida y de llegada, terminal de salida y de llegada, modelo del avión, precio total según preferencias\ldots},
	% Precondiciones
	precondiciones=El operador de la aplicación tiene credenciales que le habilitan para realizar dicha operación.,
	% Salidas
	salidas=Confirmación del registro del nuevo vuelo.,
	% Postcondiciones en caso de éxito
	postexito=El plan de vuelo queda registrado en la lista de vuelos de la compañía aérea.,
	% Postcondiciones en caso de error
	posterror=No se realiza ningún cambio en el sistema.,
	% Actores
	actores=El personal administrativo y la base de datos.,
}{
	% Tabla de secuencia normal del caso de uso
	\begin{tablasecuencias}
		1 & El usuario introduce la información detallada del vuelo. Si error S-1.\\
		2 & Se almacenan los datos del nuevo vuelo en la base de datos de la compañía. Si error S-2.
	\end{tablasecuencias}
}{
	% Tabla de secuencia con errores del caso de uso
	\begin{tablasecuencias}
		S-1 & Algún dato no es válido. El sistema vuelve a 1 de la secuencia normal de uso e indica los campos erróneos.\\
		S-2 & No se puede conectar con la base de datos. Se cancela la operación, se muestra un mensaje de error por pantalla y se vuelve al menú principal de la aplicación.
	\end{tablasecuencias}
}

	\srsfuncion{Editar información económica} \label{fun:editareconomica}
	Esta función debe permitir añadir, editar o eliminar elementos patrimoniales que forman parte de los gastos e ingresos de la compañía aérea, de acuerdo a los resultados del último ejercicio.

\begin{enumerate}
	\item \textit{Entradas}
	\begin{enumerate}
		\item Los nombres de los conceptos deberán contener únicamente carácteres alfabéticos latinos, acentuados o no, y espacios.
		\item Los datos tanto de los gastos como de los ingresos de la compañía aérea en el último ejercicio serán números reales positivos, con dos decimales a lo sumo separados por una coma (se completará con \verb|0| los dos dígitos decimales en caso de no especificarse).
		\item Se utilizará el euro como unidad monetaria.
	\end{enumerate}
	\item \textit{Flujo de operaciones}
	\begin{enumerate}
		\item Se mostrarán dos tablas: la primera de ella con los gastos de la compañía y la segunda con los ingresos. Ambas tablas aparecerán ordenadas por defecto en orden alfabético por concepto. Se dará la opción al usuario de ordenarlas según diferentes criterios (de mayor a menor importe y de mayor a menor porcentaje que suponen dentro de la masa patrimonial correspondiente), los cuales aparecen en pestañas en la parte superior de la tabla.
		\item Al final de cada tabla aparecerá una fila adicional con el importe total. En un cuadro separado en la parte inferior aparecerá el resultado del ejercicio, el cual se calculará como la resta de ingresos menos gastos, y se mostrará en negro si tiene saldo positivo o cero y en rojo si tiene saldo negativo.
		\item En cada tabla, aparecerá en la parte superior un botón \verb|Añadir| que permitirá añadir un nuevo elemento a la masa patrimonial. Al añadirlo, deberá indicarse un nombre y un importe válidos. Además, cada elemento patrimonial de la tabla podrá modificarse (debiendo introducirse nuevos datos válidos) y eliminarse de la tabla. Cada vez que se produzca una modificación en los datos, se actualizará el total de la tabla así como el resultado del ejercicio.
	\end{enumerate}
	\item \textit{Respuesta a situaciones no previstas}
	\begin{enumerate}
		\item Si no se puede conectar con la base de datos para obtener la información económica: se muestra un mensaje de error por pantalla y regresa a la página principal del sistema.
		\item Si no se puede conectar con la base de datos para almacenar la información: se muestra un mensaje de error por pantalla informando de que la información económica no ha podido actualizarse y se vuelve a la página principal del sistema.
		\item El nombre introducido no es válido: se muestra un mensaje avisando del error y se da la opción de editarlo de nuevo.
		\item El importe introducido no es válido: se muestra un mensaje avisando del error y se da la opción de editarlo de nuevo.
		\item Si no se ha podido ordenar en orden alfabético: mostrar la información desordenada e indicar que no se ha podido ordenar.
	\end{enumerate}

\end{enumerate}

	% Caso de uso: obtener información económica
% Obs: para escribir comas en el texto del primer parámetro se han de encerrar entre {}.

\casodeuso{
	% Nombre del caso de uso
	nombre=Obtener información económica,
	% Objetivo
	objetivo={Mostrar la información económica de la empresa, de acuerdo a las funciones del usuario. Entre la información recogida se encuentran los balances de la empresa, cuentas de resultados, inversiones en bolsa\ldots},
	% Entradas
	entradas=,
	% Precondiciones
	precondiciones={El operador de la aplicación está debidamente registrado y ocupar el puesto de directivo o empleado de asuntos económicos e infraestructura. La información económica ha sido introducida y procesada con anterioridad.},
	% Salidas
	salidas={La información correspondiente debidamente organizada.},
	% Postcondiciones en caso de éxito
	postexito=El usuario puede explorar los datos económicos mostrados.,
	% Postcondiciones en caso de error
	posterror={El sistema central no ha sufrido cambios.},
	% Actores
	actores={El usuario (personal de asuntos económicos e infraestructura, directivos\dots) y la base de datos.},
}{
	% Tabla de secuencia normal del caso de uso
	\begin{tablasecuencias}
		1 & Extraer de la base de datos la información económica de la empresa. Si error S-1. \\
		2 & Mostrar la visualización de los datos obtenidos (podría incluir gráficos). Si error S-2.
	\end{tablasecuencias}
}{
	% Tabla de secuencia con errores del caso de uso
	\begin{tablasecuencias}
		S-1 & No se puede acceder a la base de datos o no se pueden obtener los datos necesarios. Informar al usuario y abortar la operación.\\
		S-2 & No se ha podido generar la visualización de los datos por algún error imprevisto o por la carencia de datos de la información económica. Infomar al usuario, omitir los elementos erróneos en la medida de lo posible o abortar la operación.
	\end{tablasecuencias}
}

	% Caso de uso: Consultar inventario
% Obs: para escribir comas en el texto del primer parámetro se han de encerrar entre {}.

\casodeuso{
	% Nombre del caso de uso
	nombre=Consultar inventario,
	% Objetivo
	objetivo=Muestra el inventario de la empresa. Este material es tanto para reparaciones como de otros tipos que necesite la empresa.,
	% Entradas
	entradas=,
	% Precondiciones
	precondiciones={El operador de la aplicación está debidamente registrado y posee credenciales que le habilitan para realizar esta operación. El servidor que hospeda la base de datos de inventario está operativo.},
	% Salidas
	salidas=El registro de material en el inventario de la empresa.,
	% Postcondiciones en caso de éxito
	postexito=El empleado quedará informado sobre el material del que dispone la empresa en el inventario.,
	% Postcondiciones en caso de error
	posterror=El empleado no habrá podido recibir la información del inventario de la empresa y desconocerá el material disponible.,
	% Actores
	actores=El empleado con permisos para acceder a la información requerida y la base de datos.
}{
	% Tabla de secuencia normal del caso de uso
	\begin{tablasecuencias}
		1 & Extraer de la base de datos el inventario. Si error S-1. \\
		2 & Realizar una lista con la información organizada por el criterio configurado. Si error S-2. \\
		3 & Mostrar por pantalla estos datos permitiendo ser filtrados.\\
		4 & Al seleccionar un elemento, mostrar información detallada del mismo. Si error S-1.
	\end{tablasecuencias}
}{
	% Tabla de secuencia con errores del caso de uso
	\begin{tablasecuencias}
		S-1 & No se pudo conectar con la base de datos o no se pudo obtener la información. Informar al usuario y abandonar la carga.\\
		S-2 & Si no se encuentra configuración sobre criterios de ordenación, ordenar por orden lexicográfico o numérico del primer campo si lo hubiera.
	\end{tablasecuencias}
}


	
% Revisado por Juanan el día 12/03/2013

\srsfuncion{Consultar nómina} \label{fun:consultarnomina}
	Esta función muestra la nómina del usuario correspondiente a un mes seleccionado.

\begin{enumerate}
	\item \textit{Prioridad}: media.
	\item \textit{Entradas}
	\begin{enumerate}
		\item Mes del que se quiere realizar la consulta.
		\item El sueldo del mes (cobrado o a cobrar) aparece expresado en la moneda que haya sido configurada.
	\end{enumerate}
	\item \textit{Flujo de operaciones}
	\begin{enumerate}
		\item Se selecciona el mes.
		\item Se muestra en pantalla la información requerida.
	\end{enumerate}
	\item \textit{Respuesta a situaciones no previstas}
	\begin{enumerate}
		\item Si no existe nómina del mes seleccionado o no se ha podido acceder a ella se informa de lo sucedido y se permite seleccionar otro.
	\end{enumerate}
\end{enumerate}	

	
% Revisado por Juanan el día 12/03/2013

\srsfuncion{Configurar nómina}
	Función que permite confeccionar y almacenar las nóminas mensuales de cada empleado de la compañía según las incidencias que se hayan producido en el último mes.
						
	\begin{enumerate}
		\item \textit{Prioridad}: alta.
		\item \textit{Entradas}
			\begin{enumerate}
				\item El usuario introduce en el campo de incidencias los motivos por los cuales este mes se produce una modificación en la nómina como, por ejemplo, aumentos o disminuciones del salario a causa de horas extras, comisiones, sustituciones, huelgas\ldots
				\item El sueldo del mes será el resultante de sumar el sueldo base y el correspondiente a las incidencias.
			\end{enumerate}
		\item \textit{Flujo de operaciones}
			\begin{enumerate}
				\item Mediante \verb{Consultar empleado} se accede a la configuración de la nómina de cada empleado. 
				\item A continuación, el usuario rellena de forma obliglatoria el campo de incidencias y confirma que quiere aplicar los cambios en la nómina del cliente seleccionado.
				\item Por último, se registra la nómina en la base de datos.
			\end{enumerate}
		\item \textit{Respuesta a situaciones no previstas}
			\begin{enumerate}
				\item Si no se puede establecer conexión con la base de datos: se muestra un mensaje de error y se da la opción de reintentar o abortar el proceso.
				\item Si no se puede registrar la nómina: anular la operación y volver a la página principal del sistema.
			\end{enumerate}				
		\item \textit{Relación con otras funciones}\\
		La función está relacionada con \nameref{fun:consultarnomina} y \nameref{fun:consultarempleado}.
	\end{enumerate}
								

	% Caso de uso: consultar ficha cliente.
% Obs: para escribir comas en el texto del primer parámetro se han de encerrar entre {}.

\casodeuso{
	% Nombre del caso de uso
	nombre=Consultar ficha cliente,
	% Objetivo
	objetivo={Mostrar la lista de clientes de la compañía, permitiendo buscar y filtrar resultados, así como información detallada de cada cliente en particular, respetando en todo momento la privacidad del sujeto.},
	% Entradas
	entradas=-,
	% Precondiciones
	precondiciones={Haber accedido al sistema con un usuario válido pertenenciente al Personal de atención al cliente o al Personal en aeropuerto y elegir la opción \textit{Consultar ficha cliente} de la aplicación. Potencialmente existen datos sobre clientes previamente registrados.},
	% Salidas
	salidas=Lista de clientes y información sobre cada uno.,
	% Postcondiciones en caso de éxito
	postexito={Se pueden acceder a otros usos relaccionadas con los clientes (como eliminar un cliente, modificar sus datos\dots).},
	% Postcondiciones en caso de error
	posterror={Una pantalla de notificación de error, en la medida de lo posible.},
	% Actores
	actores=El Personal de atención al cliente o Personal en aeropuerto y la base de datos.,
}{
	% Tabla de secuencia normal del caso de uso
	\begin{tablasecuencias}
		1 & Se muestra una lista de clientes ordenados por un criterio asignado por defecto. Si error S-1. \\
		2 & El usuario puede filtrar los resultados y buscar según diferentes reglas, aparecerá una lista reducida de clientes (incluso nula).\\
		3 & Seleccionando uno de ellos se accederá a la información detallada del cliente, pudiendo dar acceso a otras operaciones. 
	\end{tablasecuencias}
}{
	% Tabla de secuencia con errores del caso de uso
	\begin{tablasecuencias}
		S-1 & Si no se puede acceder al servidor central o no se puede obtener la información de clientes, informar al usuario.
	\end{tablasecuencias}
}

	% Caso de uso: modificar items inventario.
% Obs: para escribir comas en el texto del primer parámetro se han de encerrar entre {}.

\casodeuso{
	% Nombre del caso de uso
	nombre=Modificar inventario,
	% Objetivo
	objetivo=Permite modificar diversos datos sobre el material registrado en el inventario de la empresa.,
	% Entradas
	entradas=La nueva información sobre el material.,
	% Precondiciones
	precondiciones={El operador de la aplicación está debidamente registrado y posee credenciales que le habilitan para realizar esta operación. El servidor que hospeda la base de datos de inventario está operativo.},
	% Salidas
	salidas=El inventario de la empresa modificado en función de los cambios registrados.
	% Postcondiciones en caso de éxito
	postexito=El material disponible de la empresa en el inventario se habrá actualizado de acuerdo a los datos introducidos.,
	% Postcondiciones en caso de error
	posterror={El sistema central no ha sufrido cambios y, por tanto, no se actualiza el material disponible de la empresa.},
	% Actores
	actores=El personal de la compañía con permisos para realizar estas modificaciones y la base de datos.,
}{
	% Tabla de secuencia normal del caso de uso
	\begin{tablasecuencias}
		1 & Extraer de la base de datos el inventario con los elementos disponibles. Si error S-1. \\
		2 & Mostrar una lista ordenada según el criterio configurado y permitir la búsqueda. Si error S-2. \\
		3 & Permitir añadir o eliminar elementos del inventario. Si error S-3.\\
		4 & Al seleccionar un objeto, mostrar una vista editable de sus propiedades. Si error S-4.
	\end{tablasecuencias}
}{
	% Tabla de secuencia con errores del caso de uso
	\begin{tablasecuencias}
		S-1 & La base de datos está dañada y no se han podido extraer los datos, o ha habido un error en la aplicación. Mostrar por pantalla un mensaje para que el usuario se ponga en contacto con el personal técnico de la empresa y le manifieste el error, disculparse por las molestias y dar las gracias por el aviso.\\
		S-2 & Si no se ha podido generar la lista, mostrar información del error al usuario y abortar la operación. \\
		S-3 & Si no se puede acometer la transacción con la base de datos, informar al usuario y permitir reintento. \\
		S-4 & Si no se puede obtener la información de la base de datos, informar al usuario y cancelar la pantalla de edición. Si no se puede introducir el contenido editado, S-3.
	\end{tablasecuencias}
}


	% Caso de uso: realizar mantenimiento.
% Obs: para escribir comas en el texto del primer parámetro se han de encerrar entre {}.

\casodeuso{
	% Nombre del caso de uso
	nombre=Realizar mantenimiento,
	% Objetivo
	objetivo=Permite registrar la información sobre un mantenimiento realizado.,
	% Entradas
	entradas=Los datos del mantenimiento realizado.,
	% Precondiciones
	precondiciones=Haber accedido al sistema con un usuario válido y elegir la opción \textit{Realizar mantenimiento}.,
	% Salidas
	salidas=El mantenimiento queda registrado.
	% Postcondiciones en caso de éxito
	postexito={Se ha registrado el mantenimiento y, si procede, se ha actualizado el material disponible de la empresa en el inventario.},
	% Postcondiciones en caso de error
	posterror=El mantenimiento no ha quedado registrado y el sistema central no ha sufrido cambios.,
	% Actores
	actores=El personal mecánico de la compañía y la base de datos.,
}{
	% Tabla de secuencia normal del caso de uso
	\begin{tablasecuencias}
		1 & Extraer de la base de datos el listado de los mantenimientos programados para el usuario. Si error S-1. \\
		2 & Seleccionar un mantenimiento.\\
		3 & Introducir la información detallada del mantenimiento realizado (informe de la operación, y si ha podido completarse o no). Si error S-2.\\
		4 & Se muestra el listado del material mecánico disponible en el inventario. Si error S-3 \\
		5 & El usuario selecciona el material empleado en el mantenimiento, así como el cantidad de items de cada tipo utilizados. Si error S-4 \\
		6 & Mostrar un mensaje confirmando el registro del mantenimiento.
	\end{tablasecuencias}
}{
	% Tabla de secuencia con errores del caso de uso
	\begin{tablasecuencias}
		S-1 & No se ha podido extraer la información de la base de datos. Mostrar un mensaje de error por pantalla y volver a la página principal del sistema.\\
		S-2 & No se ha podido almacenar la información introducida. Mostrar un mensaje por pantalla indicándolo y regresar a la página anterior. \\
		S-3 & No se ha podido cargar el listado del material del inventario. Mostrar un mensaje de error por pantalla y volver a la página anterior. \\
		S-4 & No se ha podido almacenar los cambios introducidos. Mostrar un mensaje por pantalla indicándolo y regresar a la página anterior.
	\end{tablasecuencias}
}

	
% Revisado por Juanan el día 12/03/2013

\srsfuncion{Programar revisión}
	Esta función permite programar una revisión a un vehículo determinado.

\begin{enumerate}
	\item \textit{Prioridad}: alta.
	\item \textit{Entradas}
	\begin{enumerate}
		\item Fecha y hora que será coherente con el horario del personal seleccionado, personal, material y herramientas necesarias.
	\end{enumerate}
	\item \textit{Flujo de operaciones}
	\begin{enumerate}
		\item Se elige fecha y hora para la revisión. 
		\item Se selecciona el vehículo a reparar. Además, se rellena el resto del formulario indicando el material y las herramientas necesarias para la reparación, así como el motivo por el que necesita una reparación especificando qué se va a realizar en ella.
		\item Se muestra por pantalla el listado de personal mecánico disponible en esos momentos, ordenados por orden alfabético según el primer apellido. 
		\item Se elegien los trabajadores para que realice la tarea. Para ello se seleccionan los empleados de la lista.
		\item Cuando los datos se han modificado, se muestra confirmación detallada y se envia automáticamente notificación al personal afectado.
	\end{enumerate}
	\item \textit{Respuesta a situaciones no previstas}
	\begin{enumerate}
		\item Si no se puede acceder a la base de datos del personal: se muestra un mensaje de error por pantalla y se vuelve a la página principal del sistema.
		\item Si no se puede conectar con la base de datos para almacenar la información de la revisión: se muestra un mensaje de error por pantalla informando de que la revisión no ha podido darse de alta y se vuelve a la página principal del sistema.
		\item Si no existe ningún empleado disponible para la fecha y hora indicadas: Mostrar un mensaje indicando que la revisión no se puede programar por falta de personal disponible en esa fecha elegida. Dar la opción de modificar la fecha y hora elegidas.
	\end{enumerate}

\end{enumerate}

	
\srsfuncion{Registrar entrada material}
	Esta función debe permitir registrar un nuevo material en el inventario de la empresa.

\begin{enumerate}
	\item \textit{Entradas}
	\begin{enumerate}
		\item Al añadirlo, los items del inventario deberán seguir quedando ordenados por defecto por orden alfabético, ordenando posteriormente el sistema si se desea por otros campos como fecha de entrada en almacén, cantidad de items, destino de uso (oficina, mecánica, edificios, aeronáutico\ldots) y valor de adquisición.
		\item El número de registro del item debe de ser un número mayor o igual que cero (numeración de los items).
	\end{enumerate}
	\item \textit{Flujo de operaciones}
	\begin{enumerate}
		\item Se muestra por pantalla el formulario a rellenar con los datos específicos del item que se van a introducir en la base de datos para poder identificarlo posteriormente y poder mostrar la información sobre él.
		\item Habrá un botón \verb|Añadir|, que registrará la entrada de material en el sistema.
		\item Si el item existe se incrementará la cantidad de items en el inventario.
		\item Si el item no existe, cuando se añada se modificará la lista añadiéndolo en el lugar correcto.
	\end{enumerate}
	\item \textit{Respuesta a situaciones no previstas}
	\begin{enumerate}
		\item Si se ha introducido algún campo incorrecto en el formulario a rellenar de los datos específicos del item, marcar los campos erróneos y mostrar un mensaje por pantalla indicando cuáles de estos campos son erróneos (por escrito) y una nota con una breve descripción diciendo el por qué.
		\item Si no se ha podido ordenar en orden alfabético: mostrar la información desordenada e indicar que no se ha podido ordenar.
	\end{enumerate}

\end{enumerate}
\label{fun:EntrMat}

	% Caso de uso: Dar de baja cliente.
% Obs: para escribir comas en el texto del primer parámetro se han de encerrar entre {}.

\casodeuso{
	% Nombre del caso de uso
	nombre= Dar de baja cliente,
	% Objetivo
	objetivo={Permitir a un cliente que pueda darse de baja en nuestro sistema borrando todos sus datos de él. Esta acción es obligatoria en cualquier sistema donde se pueda registrar un cliente por la LOPD \textit{(Ley Orgánica de Protección de Datos)}.},
	% Entradas
	entradas={Un personal administrativo acceda a la opción del sistema para dar de baja a un cliente.},
	% Precondiciones
	precondiciones={Que el cliente se haya puesto en contacto con la empresa indicándole que quiere darse de baja en el sistema y haberse registrado como un usuario válido de la aplicación perteneciente al Personal de atención al cliente. Elegir la opción \textit{Dar de baja cliente}.},
	% Salidas
	salidas= {Eliminar al cliente de la base de datos de la empresa.},
	% Postcondiciones en caso de éxito
	postexito={Los datos del cliente habrán desaparecido de la base de datos del sistema, borrándose todo por completo.},
	% Postcondiciones en caso de error
	posterror={Los datos del cliente no se habrán eliminado y seguirán estando en nuestro sistema.},
	% Actores
	actores={El Personal de atención al cliente, el cliente a dar de baja y la base de datos.},
}{
	% Tabla de secuencia normal del caso de uso
	\begin{tablasecuencias}
		1 & El cliente notifica a la empresa que desea darse de baja en su sistema.\\
		2 & El personal administrativo accede al sistema y da a la opción de dar de baja al cliente.\\
		3 & El sistema accede a la base de datos para encontrar al cliente. Si error S-1\\
		4 & Se borran del sistema los datos del cliente. Si error S-2.\\
		5 & Se notifica al cliente de que ha sido dado de baja con éxito.
	\end{tablasecuencias}
}{
	% Tabla de secuencia con errores del caso de uso
	\begin{tablasecuencias}
		S-1 & No se ha encontrado al cliente en la base de datos, por lo que se muestra un mensaje de que el cliente no existe.\\
		S-2 & Si no se ha podido eliminar los datos del cliente se le notifica indicándole que si quiere darse de baja vuelva a realizar la operación.
	\end{tablasecuencias}
}



	\srsfuncion{Programar oferta}
	Esta función debe permitir crear una oferta de un item a promocionar que posteriormente se mostrará a los clientes y podrán disponer de ella.
	
\begin{enumerate}
	\item \textit{Entradas}
	\begin{enumerate}
		\item Se programarán las ofertas que la empresa crea convenientes.
		\item Todas las palabras del lenguaje en el que se programe la oferta deberán estar en la \gls{RAE} para posteriormente poder traducirlas de un idioma a otro.
		\item El título y resumen de la oferta deberán tener un máximo de caracteres de 60 y 200 respectivamente.
	\end{enumerate}
	\item \textit{Flujo de operaciones}
	\begin{enumerate}
		\item El empleado con el rol de programar las ofertas introducirá los siguientes datos de la oferta:
		
			\begin{enumerate}
				\item Título
				\item Resumen
				\item Descripción detallada
				
					\begin{itemize}
						\item Artículo/s promocionado/s
						\item Precio anterior de cada artículo
						\item Precio de cada artículo con esta oferta
						\item Fecha límite de caducidad de la oferta
					\end{itemize}
				
			\end{enumerate}
		\item El programador validará pulsando el botón \verb|Validar| la oferta para introducirla en la base de datos del programa.
	\end{enumerate}
	\item \textit{Respuesta a situaciones no previstas}
	\begin{enumerate}
		\item En caso de no haber completado algún campo del formulario, marcar los campos erróneos e indicar una breve descripción indicando que no se han completado el formulario entero.
		\item Si al intentar validar la oferta esta no puede ser incluida en la base de datos, mostrar un mensaje indicando el error por no poder ser añadida a la base de datos del sistema. El programador podrá volver a pulsar el botón \verb|Validar| para intentar volver a ivalidar la oferta.
	\end{enumerate}
\end{enumerate}

	% Caso de uso: Efectuar embarque
% Obs: para escribir comas en el texto del primer parámetro se han de encerrar entre {}.

% Comentario de Aitor, si un cliente no se presenta a un embarque se cancelan todos los billetes asociados a ese cliente.
% Revisado por Juanan el día 12/03/2013

\casodeuso{
	% Nombre del caso de uso
	nombre=Efectuar embarque,
	% Objetivo
	objetivo=Registrar el embarque del pasajero.,
	% Entradas
	entradas=El número de vuelo y el número de reserva del pasajero.,
	% Precondiciones
	precondiciones=El operador de la aplicación tiene credenciales que le habilitan para realizar dicha operación y un billete válido,
	% Salidas
	salidas=Confirmación de que el pasajero pasa el control de seguridad con éxito y ha embarca en el avión.,
	% Postcondiciones en caso de éxito
	postexito=Se registra el embarque del pasajero.,
	% Postcondiciones en caso de error
	posterror=No se realiza ningún cambio en el sistema.,
	% Actores
	actores=El personal de la compañía presente en el aeropuerto y la base de datos.
}{
	% Tabla de secuencia normal del caso de uso
	\begin{tablasecuencias}
		1 & El usuario introduce el número de vuelo y el número de reservar del pasajero. Si error S-1.\\
		2 & Se comprueba la reserva y se actualiza la base de datos. Si error S-2.		
	\end{tablasecuencias}
}{
	% Tabla de secuencia con errores del caso de uso
	\begin{tablasecuencias}
		S-1 & Alguno de los campos introducidos por el usuario no es válido. Se muestra un mensaje de error y se vuelve a 1 de la secuencia normal de uso indicando los datos erroneos. \\
		S-2 & No se ha podido conectar con la base de datos. Se muestra un mensaje de error y se ofrece la posibilidad de reintentar.
	\end{tablasecuencias}
}


	% Caso de uso: ver incidencias del sistema.
% Obs: para escribir comas en el texto del primer parámetro se han de encerrar entre {}.

\casodeuso{
	% Nombre del caso de uso
	nombre=Ver incidencias del sistema,
	% Objetivo
	objetivo={Permite a los supervisores informáticos del sistema inspeccionar el correcto funcionamiento del \software y responder a comportamientos erróneos del mismo que hayan sido detectados, a partir de los registros que este genera.},
	% Entradas
	entradas={El nombre de registro que se quiere consultar.},
	% Precondiciones
	precondiciones={El operador de la aplicación está debidamente registrado y posee credenciales que le habilitan para realizar esta operación.},
	% Salidas
	salidas={Archivos de registro del sistema solicitados.},
	% Postcondiciones en caso de éxito
	postexito={},
	% Postcondiciones en caso de error
	posterror={El sistema central no habrá sufrido cambios.},
	% Actores
	actores={Personal de \textit{Servicios Informáticos} y supervisores del sistema con autorización para ello.},
}{
	% Tabla de secuencia normal del caso de uso
	\begin{tablasecuencias}
		1 & Los archivos de registro del servidor central y los errores reportados por las aplicaciones cliente componen una serie de archivos de texto en el servidor central. Esos archivos podrán ser consultados dando acceso a su ubicación en el sistema de archivos del servidor central.
		% De momento, ¿para qué complicarlo?
	\end{tablasecuencias}
}{
	% Tabla de secuencia con errores del caso de uso
	\begin{tablasecuencias}
		S-1 & Las secuencias alternativas son las que determine el método de acceso al servidor.
	\end{tablasecuencias}
}


	% Gestión externa
	\subsection{Gestión externa} \vspace{.5cm}

	\begin{sidewaysfigure}
		\includegraphics[scale=.93]{diagramas/gestionexterna.pdf}
		\caption{Diagrama de gestión externa}
	\end{sidewaysfigure}

	% Revisado por Cristina el día 12/03/2013

\srsfuncion{Acceder web} \label{fun:accederge}
	Función que permite hacer \textit{\gls{Login}} al usuario (en este caso cliente de la compañía aérea) para poder acceder al sistema.
		
	\begin{enumerate}
		\item \textit{Prioridad}: media.
		\item \textit{Entradas}
		\begin{enumerate}
			\item El nombre de usuario y la contraseña son campos obligatorios a introducir.
			\item En el campo contraseña son válidos los caracteres ASCII imprimibles.
		\end{enumerate}
		\item \textit{Flujo de operaciones}
		\begin{enumerate}
			\item Se muestran por pantalla dos campos a rellenar: uno para introducir el id del usuario y otro para escribir la contraseña.
			\item El usuario selecciona la opción de acceder al sistema.
		\end{enumerate}
		\item \textit{Respuesta a situaciones no previstas}
		\begin{enumerate}
			\item Si algún campo introducido no es válido, se indica y se da la opción de introducirlo de nuevo. Existe un límite de 5 intentos de acceso fallido en un periodo de tiempo corto (15 minutos).
		\end{enumerate}
	
\end{enumerate}

	% Caso de uso: Registrarse
% Obs: para escribir comas en el texto del primer parámetro se han de encerrar entre {}.

% Revisado por Cristina el día 11/03/2013

\casodeuso{
	% Nombre del caso de uso
	nombre=Registrarse,
	% Objetivo
	objetivo=Registrar un nuevo cliente de la compañía en la base de datos.,
	% Entradas
	entradas={Datos personales del cliente (nombre y apellidos, NIF o equivalente, dirección, teléfono y una dirección de correo electrónico).},
	% Precondiciones
	precondiciones=El visitante que pretende registrarse no está previamente registrado.,
	% Salidas
	salidas=Confirmación del registro del nuevo cliente.,
	% Postcondiciones en caso de éxito
	postexito=El cliente queda registrado en la base de datos.,
	% Postcondiciones en caso de error
	posterror=No se realiza ningún cambio en el sistema.,
	% Actores
	actores=El visitante de la interfaz externa y la base de datos.,
}{
	% Tabla de secuencia normal del caso de uso
	\begin{tablasecuencias}
		1 & El usuario introduce sus datos personales. Si error S-1.\\
		2 & Se almacenan los datos del nuevo cliente en la base de datos. Si error S-2.
	\end{tablasecuencias}
}{
	% Tabla de secuencia con errores del caso de uso
	\begin{tablasecuencias}
		S-1 & El sistema vuelve al paso 1 de la secuencia normal de uso e indica los campos erróneos.\\
		S-2 & No se puede conectar con la base de datos, se muestra un mensaje de error por pantalla dando la opción de reintentar o volver al menú principal de la aplicación.
	\end{tablasecuencias}
}

	% Caso de uso: editar cliente.
% Obs: para escribir comas en el texto del primer parámetro se han de encerrar entre {}.

% Revisado por Juanan el día 12/03/2013

\casodeuso{
	% Nombre del caso de uso
	nombre=Editar cliente,
	% Objetivo
	objetivo={Editar la información de un cliente, bien él mismo o desde la gestión interna de la aplicación.},
	% Entradas
	entradas=Los datos que se quieren modificar.,
	% Precondiciones
	precondiciones={El operador de la aplicación tiene credenciales que le habilitan para realizar dicha operación y tiene una ficha seleccionada.},
	% Salidas
	salidas=La información de perfil actualizada.,
	% Postcondiciones en caso de éxito
	postexito=Los cambios efectuados se guardan en la base de datos.,
	% Postcondiciones en caso de error
	posterror=No se realiza ningún cambio en el sistema.,
	% Actores
	actores={Cliente-usuario de interfaz web o el personal administrativo y la base de datos},
}{
	% Tabla de secuencia normal del caso de uso
	\begin{tablasecuencias}
		1 & Muestra los campos de datos personales de usuario. Si error S-1.\\
		2 & El usuario modifica los datos deseados. Si error S-2.\\
		3 & Se almacenan los cambios en la base de datos. Si error S-3.
	\end{tablasecuencias}
}{
	% Tabla de secuencia con errores del caso de uso
	\begin{tablasecuencias}
		S-1 & Si no se puede conectar con la base de datos se muestra mensaje  de tipo \textit{información no disponible temporalmente} y se vuelve a 1 de la secuencia normal de uso.\\
		S-2 & Alguno de los datos introducidos no es válido. Vuelve a 1 de la secuencia normal de uso indicando los campos erróneos.\\
		S-3 & No se puede conectar con la base de datos. Se cancela la operación, se muestra un mensaje de error por pantalla y se vuelve a la ficha del cliente.
	\end{tablasecuencias}
}


	% Caso de uso: consultar vuelos.
% Obs: para escribir comas en el texto del primer parámetro se han de encerrar entre {}.

\casodeuso{
	% Nombre del caso de uso
	nombre=Consultar vuelos,
	% Objetivo
	objetivo={Mostrar al cliente la relación de vuelos operados por la compañía, pudiendo filtrar resultados y buscar por diferentes criterios; permitiendo además obtener información detallada de los vuelos seleccionados.},
	% Entradas
	entradas={Opcionalmente las que correspondan a los filtros (aeropuertos de origen y destino, número de escalas, fecha y hora, precio del billete\ldots). En última instancia, vuelo seleccionado.},
	% Precondiciones
	precondiciones={La información de vuelos ha sido previamente introducida en el sistema, así como los criterios configurados.},
	% Salidas
	salidas={Una lista filtrada de vuelos y, al seleccionar uno de ellos, información detallada del mismo.},
	% Postcondiciones en caso de éxito
	postexito=El cliente puede acceder a la compra de billetes del vuelo seleccionado.,
	% Postcondiciones en caso de error
	posterror={Una pantalla de notificación de error, en la medida de lo posible.},
	% Actores
	actores={Clientes de la compañía, base de datos.},
}{
	% Tabla de secuencia normal del caso de uso
	\begin{tablasecuencias}
		1 & Se muestra una lista de vuelos ordenados por un criterio asignado por defecto. Si error S-1. \\
		2 & El usuario puede filtrar los resultados según diferentes reglas, aparecerá una lista reducida de vuelos (incluso nula).\\
		3 & Seleccionando uno de ellos se accederá a la información especializada en ese servicio, dando acceso a la adquisición de billetes. 
	\end{tablasecuencias}
}{
	% Tabla de secuencia con errores del caso de uso
	\begin{tablasecuencias}
		S-1 & Si no se puede acceder al servidor central o no se puede obtener la información de vuelos, informar al usuario.
	\end{tablasecuencias}
}

	% Caso de uso: Mostrar ofertas.
% Obs: para escribir comas en el texto del primer parámetro se han de encerrar entre {}.

\casodeuso{
	% Nombre del caso de uso
	nombre=Mostrar ofertas,
	% Objetivo
	objetivo=Mostrar las ofertas actuales a los clientes con el fin de incentivar su compra.,
	% Entradas
	entradas=El cliente podrá elegir una oferta entre las que se muestran.,
	% Precondiciones
	precondiciones=Haber accedido a la página web de la compañía.,
	% Salidas
	salidas=Las ofertas actuales.,
	% Postcondiciones en caso de éxito
	postexito=Muestra las ofertas actuales de la compañía.,
	% Postcondiciones en caso de error
	posterror=No se modifica la página web.,
	% Actores
	actores=Cualquier usuario que acceda a la página web y la base de datos.
}{
	% Tabla de secuencia normal del caso de uso
	\begin{tablasecuencias}
		1 & Acceder a la base de datos. Si error S-1.\\
		2 & Muestra por pantalla las ofertas de la compañía.\\
		3 & Si el cliente accede a una oferta en concreto, se rediccionará a esa oferta a través de \textit{Acceder a la oferta}. Si error S-2.
	\end{tablasecuencias}
}{
	% Tabla de secuencia con errores del caso de uso
	\begin{tablasecuencias}
		S-1 & No se puede acceder a la base de datos. No se cargan las ofertas y se muestra la pantalla de la página web sin las ofertas.\\
		S-2 & Si no se ha podido rediccionar a la oferta elegida se mostrará un mensaje por pantalla indicando que la oferta no es accesible en estos momentos.
	\end{tablasecuencias}
}

	\srsfuncion{Acceder a una oferta}
	Esta función debe mostrar una oferta específica elegida por el cliente y podrá dar la opción de comprar lo ofertado.
	
\begin{enumerate}
	\item \textit{Prioridad}: media.
	\item \textit{Entradas}
	\begin{enumerate}
		\item Los datos detallados de la oferta tendrán que estar en el mismo idioma en el que estaba el resumen de la oferta.
	\end{enumerate}
	\item \textit{Flujo de operaciones}
	\begin{enumerate}
		\item La oferta aparecerá descrita detalladamente especificando el/los artículo/s ofertados, por lo que si el usuario estuviese interesado en ella podrá comprar y pagar la oferta.
		\item Si la oferta consta de varias opciones de compra, el cliente podrá elegir entre todas las que haya.
		\item Si el cliente pulsa el botón \verb|Comprar|, se le redireccionará al proceso de realizar pago para que lo efectúe.
	\end{enumerate}
	\item \textit{Respuesta a situaciones no previstas}
	\begin{enumerate}
		\item Si al intentar acceder a una oferta el sistema falla, se mostrará un mensaje de error por pantalla informando de que los datos de esta oferta pueden ser que estén dañados o que simplemente no esté ya disponible. 
	\end{enumerate}
\end{enumerate}

	% Caso de uso: Presentar reclamacion
% Obs: para escribir comas en el texto del primer parámetro se han de encerrar entre {}.

% Revisado por Juanan el día 12/03/2013

\casodeuso{
	% Nombre del caso de uso
	nombre=Presentar reclamaciones,
	% Objetivo
	objetivo= Presentar una reclamación por parte del cliente.,
	% Entradas
	entradas= Motivo de la queja y detallado de la misma.,
	% Precondiciones
	precondiciones=No hay precondiciones,
	% Salidas
	salidas=Confirmación del envio y número asociado a la reclamación.,
	% Postcondiciones en caso de éxito
	postexito=La notificación se envia al departamento correspondiente.,
	% Postcondiciones en caso de error
	posterror=No se realiza ningún cambio en el sistema.,
	% Actores
	actores=Cliente-usuario de interfaz web.,
}{
	% Tabla de secuencia normal del caso de uso
	\begin{tablasecuencias}
		1 & El usuario completa los datos requeridos. Si error S-1. \\
		2 & Muestra número asignado a la reclamación junto a la confirmación de que la reclamación ha sido tramitada y en breve será atendia por el departamento correspondiente. Si error S-2. 
		
	\end{tablasecuencias}
}{
	% Tabla de secuencia con errores del caso de uso
	\begin{tablasecuencias}
		S-1 & Los campos de datos están incompletos. Se vuelve a 1 de la secuencia normal de uso y se incican los campos erróneos o vacíos.\\
		S-2 & Si no puede transmitir los datos de la reclamación a la base de datos se indica en un mensaje y se insta a que se vuelva a intentar.
	\end{tablasecuencias}
}


	% Caso de uso: Comprar billete.
% Obs: para escribir comas en el texto del primer parámetro se han de encerrar entre {}.

% Revisado por Cristina el día 11/03/2013

\casodeuso{
	% Nombre del caso de uso
	nombre=Comprar billete,
	% Objetivo
	objetivo=Realizar la compra de un billete.,
	% Entradas
	entradas={Nombre y apellidos del(los) pasajero(s), así como su NIF o equivalentes, reserva de asientos, datos de la persona que paga e identificación del billete a comprar.},
	% Precondiciones
	precondiciones=No hay precondiciones.,
	% Salidas
	salidas={Información detallada de la compra (itinerario de vuelo, precio total a pagar, desglose del precio (pasajero, tarifa, impuestos, tasas y suplementos de de transporte) y datos de los pasajeros).},
	% Postcondiciones en caso de éxito
	postexito=Se registra la compra y se redirecciona a \textit{Iniciar Pago Billetes de Vuelo}.,
	% Postcondiciones en caso de error
	posterror=No se realiza ningún cambio en el sistema.,
	% Actores
	actores=Clientes de la compañía y la base de datos.,
}{
	% Tabla de secuencia normal del caso de uso
	\begin{tablasecuencias}
		1 & El cliente introduce los datos de los pasajeros y de la persona que realiza el pago. Si error S-1.\\
		2 & El cliente selecciona los asientos. Si error S-2. \\
		3 & Se almacenan los datos de la compra en la base de datos. Si error S-3.
	\end{tablasecuencias}
}{
	% Tabla de secuencia con errores del caso de uso
	\begin{tablasecuencias}
		S-1 & El sistema vuelve a 1 de la secuencia normal de uso e indica los campos erróneos.\\
		S-2 & No se puede realizar la reserva de asientos. Se muestra un mensaje de error por pantalla dando la opción de reintentar o volver al menú principal de la aplicación.\\
		S-3 & No se puede conectar con la base de datos, se muestra un mensaje de error por pantalla dando la opción de reintentar o volver al menú principal de la aplicación.
	\end{tablasecuencias}
}

	% Caso de uso: iniciar pago billetes de vuelo.
% Obs: para escribir comas en el texto del primer parámetro se han de encerrar entre {}.

\casodeuso{
	% Nombre del caso de uso
	nombre=Iniciar el pago de los billetes de vuelo,
	% Objetivo
	objetivo=Iniciar el proceso de pago del billete para finalizar el proceso de compra.,
	% Entradas
	entradas=La información detallada de los billetes de vuelo.,
	% Precondiciones
	precondiciones={Haberse registrado en la página web de la compañía aérea y haber seleccionado todos los campos detallados de los billetes.},
	% Salidas
	salidas={Se procede a realizar el pago mediante tarjeta de crédido o débito, si procede}.,
	% Postcondiciones en caso de éxito
	postexito=El proceso de compra será redirigido para finalizar el pago mediante tarjeta de crédito o débito.,
	% Postcondiciones en caso de error
	posterror=La compra no se ha realizado y la base de datos no ha sido modificada.,
	% Actores
	actores={El usuario, la base de datos y las entidades financieras.},
}{
	% Tabla de secuencia normal del caso de uso
	\begin{tablasecuencias}
		1 & Mostrar todos los datos de la compra. Si error S-1.\\
		2 & Indicar con claridad las claúsulas de las leyes de protección de datos para que el usuario las acepte y pueda seguir con la compra. Si error S-2.\\ 
		3 & Se redirecciona según a \textbf{Realizar Pago con Tarjeta}.
	\end{tablasecuencias}
}{
	% Tabla de secuencia con errores del caso de uso
	\begin{tablasecuencias}
		S-1 & La reserva de los billetes ha expirado porque ha pasado mucho tiempo desde que se añadieron los billetes al carrito de la compra o por algún error en su almacenamiento. Se mostrará un mensaje indicando que se vuelva a realizar la compra desde el principio.\\3
		S-2 & El cliente no ha aceptado las claúsulas de leyes de protección de datos. Se aborta la operación, se muestra un mensaje por pantalla indicándolo y se vuelve a la página principal de la aplicación.
	\end{tablasecuencias}
}


	% Revisado por Juanan el día 12/03/2013

\srsfuncion{Realizar pago con tarjeta} \label{fun:pagotarjeta}
	Esta función permite al usuario finalizar el proceso de compra del billete realizando el pago con tarjeta de crédito o débito.

\begin{enumerate}
	\item \textit{Prioridad}: alta.
	\item \textit{Entradas}
	\begin{enumerate}
		\item El nombre del usuario (nombre y apellidos) deberán contener únicamente carácteres\break alfabéticos latinos, acentuados o no, y espacios.
		\item El número de la tarjeta deberá ser una secuencia de 4 bloques de 4 dígitos, todos ellos enteros mayores o iguales que 0.
		\item El código CCV deberá ser una secuencia de 3 dígitos mayores o iguales que 0.
		\item La fecha de caducidad  ser una secuencia de 5 caracteres compuesta por: 2 dígitos para indicar el mes (en el rango 01-12) , una barra `/'  y otros 2 dígitos para indicar el año, que serán las dos últimas cifras del mismo.
	\end{enumerate}
	\item \textit{Flujo de operaciones}
	\begin{enumerate}
		\item Se muestra por pantalla una tabla con los datos de la tarjeta a completar (nombre y apellidos, número, código CCV y fecha de caducidad). Una vez completos, se habilita la opción \verb|Confirmar|.
		\item Se transfieren los datos de la tarjeta a la empresa emisora de las mismas para que compruebe si los datos son correctos y la tarjeta está operativa. En este caso, se enviará al instante un mensaje a la compañía indicando que los datos introducidos por el usuario son válidos.
		\item Si está todo correcto, se da la opción de imprimir en el momento la tarjeta de embarque o guardarla como pdf para imprimirla en otro momento.
	\end{enumerate}
	\item \textit{Respuesta a situaciones no previstas}
	\begin{enumerate}
		\item Si no se puede acceder a la base de datos para almacenar la información: se muestra un mensaje de error por pantalla informando de que el proceso de pago se ha interrumpido. Se vuelve a la página anterior.
		\item Si alguno de los datos no es válido: se muestran los campos erróneos y se da la opción de editarlos de nuevo.
		\item Si algún campo no se ha rellenado: se muestra un mensaje indicando que es obligatorio completarlo.
		\item Si la tarjeta está inhabilitada por algún motivo: se muestra un error indicando que el pago no ha podido completarse porque la tarjeta está bloqueada. Se cancela la operación y se vuelve a la página principal de la aplicación.
		\item Si los datos de la tarjeta no son válidos: se muestra un error indicando que el pago no ha podido completarse y se da la opción al usuario de modificar los datos introducidos al principio de la operación.
	\end{enumerate}
	\item \textit{Relación con otras funciones}\\
		Esta función está relacionada con \nameref{fun:consultaroferta}, \nameref{fun:mostrarofertas}, \nameref{fun:iniciarpago} y \nameref{fun:comprarbillete}.	
\end{enumerate}

	
% Revisado por Juanan el día 12/03/2013

\srsfuncion{Ver información de vuelo contratado}
	
	Esta función muestra a los clientes que han adquirido billetes información sobre sus reservas. Muestra una lista de todos los vuelos pendientes contratados con la compañía, permitiendo imprimir un informe sobre cada uno de ellos o sobre la totalidad. No muestra información sobre vuelos pasados ni permite filtrar los resultados obtenidos.

	\begin{enumerate}
		\item \textit{Prioridad}: media.
		\item \textit{Entradas}\\
			Esta función no acepta ningún parámetro.

		\item \textit{Precondiciones}: el usuario ha iniciado sesión correctamente en la interfaz externa.
		
		\item \textit{Flujo de operaciones}
			\begin{enumerate}
				\item Se muestra una sucesión de cuadros describiendo cada uno de los vuelos contratados. Si no hay ningún vuelo contratado mostrar ese mensaje.
				\item El usuario puede seleccionar --en cada vuelo o en general-- la opción imprimir que formateará la información adecuadamente para ser impresa o generará un documento \gls{PDF} para ser descargado.
			\end{enumerate}
		\item \textit{Respuesta a situaciones no previstas}
			\begin{enumerate}
				\item Si no se puede conectar con la base de datos u obtener la información: informar del error al usuario.
				\item Si no se puede generar la vista de impresión o el documento \gls{PDF}: informar al usuario e indicarle que puede volver a intentarlo en otro momento.
			\end{enumerate}
		\item \textit{Relación con otras funciones}\\
			Esta función requiere indirectamente de las funciones \verb|Comprar billete| y \verb|Acceder|.
	\end{enumerate}

	
\end{document}
