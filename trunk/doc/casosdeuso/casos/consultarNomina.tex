% Caso de uso:Consultar nómina.
% Obs: para escribir comas en el texto del primer parámetro se han de encerrar entre {}.

% Revisado por Cristina el día 11/03/2013

\casodeuso{
	% Nombre del caso de uso
	nombre=Consultar nómina,
	% Objetivo
	objetivo=Consultar la nómina de un empleado.,
	% Entradas
	entradas=Mes del que se desea obtener la nómina.,
	% Precondiciones
	precondiciones=El operador de la aplicación tiene credenciales que le habilitan para realizar dicha operación y tiene una ficha seleccionada.,
	% Salidas
	salidas=El desglose detallado de la nómina.,
	% Postcondiciones en caso de éxito
	postexito=No se realiza ningún cambio en el sistema.,
	% Postcondiciones en caso de error
	posterror=No se realiza ningún cambio en el sistema.,
	% Actores
	actores=Empleados y base de datos.,
}{
	% Tabla de secuencia normal del caso de uso
	\begin{tablasecuencias}
		1 & El usuario selecciona el mes del que se desea consultar la nómina. Si error S-1.
	\end{tablasecuencias}
}{
	% Tabla de secuencia con errores del caso de uso
	\begin{tablasecuencias}
		S-1 & Si no se puede conectar con la base de datos, se muestra un mensaje de error y vuelve a la ficha de empleado.
	\end{tablasecuencias}
}
