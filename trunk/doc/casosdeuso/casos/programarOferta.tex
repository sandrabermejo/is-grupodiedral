% Caso de uso: Programar oferta.
% Obs: para escribir comas en el texto del primer parámetro se han de encerrar entre {}.

% Revisado por Cristina el día 11/03/2013

\casodeuso{
	% Nombre del caso de uso
	nombre=Programar oferta,
	% Objetivo
	objetivo=Crear una nueva oferta para los clientes de la compañía.,
	% Entradas
	entradas=Tipo de vuelos y clientes a los que se aplica.,
	% Precondiciones
	precondiciones=El operador de la aplicación tiene credenciales que le habilitan para realizar dicha operación.,
	% Salidas
	salidas=Confirmación de la operación.,
	% Postcondiciones en caso de éxito
	postexito=Se actualiza el listado de ofertas incluyendo la oferta programada.,
	% Postcondiciones en caso de error
	posterror=No se realiza ningún cambio en el sistema.,
	% Actores
	actores=El personal administrativo y la base de datos., 
}{
	% Tabla de secuencia normal del caso de uso
	\begin{tablasecuencias}
		1 & Se extrae de la base de datos el listado de vuelos. Si error S-1.\\
		2 & El administrativo selecciona los vuelos afectados por la oferta.\\
		3 & Se extrae de la base de datos el listado de clientes. Si error S-1.\\
		4 & El administrativo selecciona los clientes afectados por la oferta.\\
		5 & Se almacenan los cambios en la base de datos. Si error S-1.
	\end{tablasecuencias}
}{
	% Tabla de secuencia con errores del caso de uso
	\begin{tablasecuencias}
		S-1 & No se puede conectar con la base de datos, se muestra un mensaje de error por pantalla dando la opción de reintentar o volver al menú principal de la aplicación.
	\end{tablasecuencias}
}
