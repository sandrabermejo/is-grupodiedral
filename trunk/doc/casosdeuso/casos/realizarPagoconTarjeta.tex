% Caso de uso: realizar pago con tarjeta
% Obs: para escribir comas en el texto del primer parámetro se han de encerrar entre {}.

% Revisado por Cristina el día 11/03/2013

\casodeuso{
	% Nombre del caso de uso
	nombre=Realizar pago con tarjeta,
	% Objetivo
	objetivo=Efectuar el pago mediante tarjeta de crédito o débito.,
	% Entradas
	entradas=Datos de la tarjeta bancaria del usuario.,
	% Precondiciones
	precondiciones=El cliente inicia el pago mediante \textbf{Iniciar Pago Billetes}.,
	% Salidas
	salidas=Tarjeta de embarque.,
	% Postcondiciones en caso de éxito
	postexito=El usuario completa la compra y la empresa recibe el dinero por la venta.,
	% Postcondiciones en caso de error
	posterror=No se realiza ningún cambio en el sistema.,
	% Actores
	actores={El usuario, la base de datos y las entidades financieras.},
}{
	% Tabla de secuencia normal del caso de uso
	\begin{tablasecuencias}
		1 & El cliente rellena los campos obligatorios con los datos de la tarjeta de crédito o débito desde la se quiere efectuar el pago. Si error S-1.\\
		2 & Se muestran los detalles del pago.\\
		3 & El usuario confirma el pago. Si no S-2.\\
		3 & Se transfieren los datos de la tarjeta a la empresa emisora de la misma para que compruebe si los datos son correctos y la tarjeta está operativa. En este caso, se envia un mensaje a la compañía indicando que los datos introducidos por el usuario son válidos. Si no S-3.\\
		4 & Se da la opción de imprimir en el momento la tarjeta de embarque o guardarla como pdf para imprimirla en otro momento.
	\end{tablasecuencias}
}{
	% Tabla de secuencia con errores del caso de uso
	\begin{tablasecuencias}
		S-1 & Algún dato no es válido o no se ha rellenado algún campo obligatorio. El sistema vuelve a 1 de la secuencia normal de uso e indica los campos erróneos.\\
		S-2 & El usuario cancela el proceso de compra. Se muestra un mensaje por pantalla informando de la cancelación y se vuelve al menú principal de la aplicación.\\
		S-3 & El sistema vuelve a 1 de la secuencia normal de uso e indica los campos erróneos.
	\end{tablasecuencias}
}

