% Caso de uso: realizar pago con tarjeta.
% Obs: para escribir comas en el texto del primer parámetro se han de encerrar entre {}.

\casodeuso{
	% Nombre del caso de uso
	nombre=Realizar pago con tarjeta.,
	% Objetivo
	objetivo=Efectuar el pago mediante tarjeta de crédito o débito.,
	% Entradas
	entradas=Datos de la tarjeta bancaria del usuario.,
	% Precondiciones
	precondiciones=Haber iniciado el pago mediante \textbf{Iniciar Pago Billetes}.,
	% Salidas
	salidas={La compañía aérea recibe el dinero por la venta, si procede.},
	% Postcondiciones en caso de éxito
	postexito=El usuario completa la compra y la empresa recibirá el dinero por la venta.,
	% Postcondiciones en caso de error
	posterror=La compra no se ha efectuado y la base de datos no ha sido alterada.,
	% Actores
	actores={El usuario, la base de datos y las entidades financieras.},
}{
	% Tabla de secuencia normal del caso de uso
	\begin{tablasecuencias}
		1 & Rellenar los campos obligatorios introduciendo todos los datos de la tarjeta de crédito o débito del usuario desde la se quiere efectuar el pago. Si error S-1. \\
		2 & Mostrar los detalles del pago y pedir confirmación al usuario. Si error S-2. \\
		3 & Se transfieren los datos de la tarjeta a la empresa emisora de las mismas para que compruebe si los datos son correctos y la tarjeta está operativa. En este caso, se enviará al instante un mensaje a la compañía indicando que los datos introducidos por el usuario son válidos. Si error S-3.\\
		4 & Se da la opción de imprimir en el momento la tarjeta de embarque o guardarla como pdf para imprimirla en otro momento.
	\end{tablasecuencias}
}{
	% Tabla de secuencia con errores del caso de uso
	\begin{tablasecuencias}
		S-1 & Algún dato no es válido o no se ha rellenado algún campo obligatorio. Volver al paso 1 de la secuencia normal de uso indicando los campos incorrectos para que el usuario los pueda modificar.\\
		S-2 & El usuario no ha confirmado el pago. Se cancela el proceso de compra, se muestra un mensaje por pantalla informando de la cancelación y se vuelve a la página principal de la aplicación. \\
		S-3 & Los datos introducidos por el usuario no son válidos. Volver al paso 1 de la secuencia normal de uso indicando los campos erróneos.
	\end{tablasecuencias}
}

