% Caso de uso: Comprar billete.
% Obs: para escribir comas en el texto del primer parámetro se han de encerrar entre {}.

\casodeuso{
	% Nombre del caso de uso
	nombre=Comprar billete,
	% Objetivo
	objetivo=Realizar la compra de un billete.,
	% Entradas
	entradas={Nombre y apellidos del(los) pasajero(s), reserva de asientos, datos de la persona que paga, identificación del billete a comprar},
	% Precondiciones
	precondiciones={Estar correctamente registrado en el sistema y que el billete a comprar sea válido.},
	% Salidas
	salidas={Comprobación del itinerario de vuelo, precio total a pagar, desglose del precio (pasajero, tarifa, impuestos, tasas y suplementos de de transporte), opción de reservar asientos y campos a rellenar con la informacion de los pasajeros y la persona que realizará el pago.},
	% Postcondiciones en caso de éxito
	postexito={Se registra la compra, se muestran por pantalla los detalles de la misma, se da la opción al cliente de imprimirlos y se redirecciona a \textit{Iniciar Pago Billetes de Vuelo}.},
	% Postcondiciones en caso de error
	posterror=No se puede realizar la compra del billete del vuelo seleccionado.,
	% Actores
	actores={Clientes de la compañía, previamente registrados en la aplicación, y la base de datos}.
}{
	% Tabla de secuencia normal del caso de uso
	\begin{tablasecuencias}
		1 & Se muestra la información del itinerario de vuelo, así como los detalles del precio de compra. Si error S-1. \\
		2 & El cliente introduce los datos de los pasajeros y de la persona que realiza el pago. Si error S-2.\\
		3 & El cliente selecciona los asientos que desee. Si error S-3. \\
		3 & Se almacenan los datos de la compra en la base de datos. Si error S-4.\\
		4 & Se muestran por pantalla los detalles completos de la compra y se da la opción de imprimir la tarjeta de embarque.
	\end{tablasecuencias}
}{
	% Tabla de secuencia con errores del caso de uso
	\begin{tablasecuencias}
		S-1 & Se muestra el error por pantalla y se vuelve a la página principal de la aplicación. \\
		S-2 & Alguno de los datos introducidos por el cliente no es válido.Se pide de nuevo que introduzca los datos.\\
		S-3 & No se puede realizar la reserva de asientos. Se muestra el error y sigue la secuencia normal. \\
		S-4 & No se ha podido conectar con la base de datos ni realizarse la compra. Se muestra el error por pantalla y se vuelve a la página principal de la aplicación.
	\end{tablasecuencias}
}
