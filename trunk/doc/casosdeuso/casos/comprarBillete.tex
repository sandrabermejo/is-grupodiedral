% Caso de uso: Comprar billete.
% Obs: para escribir comas en el texto del primer parámetro se han de encerrar entre {}.

% Revisado por Cristina el día 11/03/2013

\casodeuso{
	% Nombre del caso de uso
	nombre=Comprar billete,
	% Objetivo
	objetivo=Realizar la compra de un billete.,
	% Entradas
	entradas={Nombre y apellidos del(los) pasajero(s), así como su NIF o equivalentes, reserva de asientos, datos de la persona que paga e identificación del billete a comprar.},
	% Precondiciones
	precondiciones=No hay precondiciones.,
	% Salidas
	salidas={Información detallada de la compra (itinerario de vuelo, precio total a pagar, desglose del precio (pasajero, tarifa, impuestos, tasas y suplementos de de transporte) y datos de los pasajeros).},
	% Postcondiciones en caso de éxito
	postexito=Se registra la compra y se redirecciona a \textit{Iniciar Pago Billetes de Vuelo}.,
	% Postcondiciones en caso de error
	posterror=No se realiza ningún cambio en el sistema.,
	% Actores
	actores=Clientes de la compañía y la base de datos.,
}{
	% Tabla de secuencia normal del caso de uso
	\begin{tablasecuencias}
		1 & El cliente introduce los datos de los pasajeros y de la persona que realiza el pago. Si error S-1.\\
		2 & El cliente selecciona los asientos. Si error S-2. \\
		3 & Se almacenan los datos de la compra en la base de datos. Si error S-3.
	\end{tablasecuencias}
}{
	% Tabla de secuencia con errores del caso de uso
	\begin{tablasecuencias}
		S-1 & El sistema vuelve a 1 de la secuencia normal de uso e indica los campos erróneos.\\
		S-2 & No se puede realizar la reserva de asientos. Se muestra un mensaje de error por pantalla dando la opción de reintentar o volver al menú principal de la aplicación.\\
		S-3 & No se puede conectar con la base de datos, se muestra un mensaje de error por pantalla dando la opción de reintentar o volver al menú principal de la aplicación.
	\end{tablasecuencias}
}
