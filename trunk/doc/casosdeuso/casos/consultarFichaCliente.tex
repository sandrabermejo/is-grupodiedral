% Caso de uso: consultar ficha cliente.
% Obs: para escribir comas en el texto del primer parámetro se han de encerrar entre {}.

\casodeuso{
	% Nombre del caso de uso
	nombre=Consultar ficha cliente,
	% Objetivo
	objetivo={Mostrar la lista de clientes de la compañía, permitiendo buscar y filtrar resultados, así como información detallada de cada cliente en particular, respetando en todo momento la privacidad del sujeto.},
	% Entradas
	entradas=-,
	% Precondiciones
	precondiciones={Haber accedido al sistema con un usuario válido pertenenciente al Personal de atención al cliente o al Personal en aeropuerto y elegir la opción \textit{Consultar ficha cliente} de la aplicación. Potencialmente existen datos sobre clientes previamente registrados.},
	% Salidas
	salidas=Lista de clientes y información sobre cada uno.,
	% Postcondiciones en caso de éxito
	postexito={Se pueden acceder a otros usos relaccionadas con los clientes (como eliminar un cliente, modificar sus datos\dots).},
	% Postcondiciones en caso de error
	posterror={Una pantalla de notificación de error, en la medida de lo posible.},
	% Actores
	actores=El Personal de atención al cliente o Personal en aeropuerto y la base de datos.,
}{
	% Tabla de secuencia normal del caso de uso
	\begin{tablasecuencias}
		1 & Se muestra una lista de clientes ordenados por un criterio asignado por defecto. Si error S-1. \\
		2 & El usuario puede filtrar los resultados y buscar según diferentes reglas, aparecerá una lista reducida de clientes (incluso nula).\\
		3 & Seleccionando uno de ellos se accederá a la información detallada del cliente, pudiendo dar acceso a otras operaciones. 
	\end{tablasecuencias}
}{
	% Tabla de secuencia con errores del caso de uso
	\begin{tablasecuencias}
		S-1 & Si no se puede acceder al servidor central o no se puede obtener la información de clientes, informar al usuario.
	\end{tablasecuencias}
}
