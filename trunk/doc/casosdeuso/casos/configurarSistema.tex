% Caso de uso: configurar sistema general
% Obs: para escribir comas en el texto del primer parámetro se han de encerrar entre {}.

% Revisado por Rubén el día 13/03/2013

\casodeuso{
	% Nombre del caso de uso
	nombre=Configurar sistema general,
	% Objetivo
	objetivo={Configurar ciertos parámetros de configuración general del  sistema como nombre de la compañía, moneda utilizada, fecha y hora, ordenación de configuración por defecto\ldots\footnote{Para mayor concreción consúltese la {\itshape Especificación de Requisitos Software}.}}.,
	% Entradas
	entradas={Valores de los nuevos parámetros cuyo cambio se quiera aplicar al sistema.},
	% Precondiciones
	precondiciones={El operador de la aplicación tiene credenciales que le habilitan para realizar dicha operación.},
	% Salidas
	salidas={Confirmación de los cambios aplicados.},
	% Postcondiciones en caso de éxito
	postexito={Las variaciones en la configuración se habrán aplicado sobre el sistema o se implantarán, si así lo requieren, en el próximo reinicio.},
	% Postcondiciones en caso de error
	posterror={No se ha realizado ningún cambio en el sistema.},
	% Actores
	actores={El personal administrativo o, por delegación, el personal de los servicios informáticos debidamente autorizado.},
}{
	% Tabla de secuencia normal del caso de uso
	\begin{tablasecuencias}
		1 & Se presentan los valores actuales de la configuración general del sistema. \\
		2 & El usuario introduce los nuevos valores de las propiedades que desee modificar. \\
		3 & Se almacenan los ajustes introducidos en el sistema central. Si error S-1.\\
		4 & Se aplican los ajustes si su naturaleza permite su aplicación inmediata. En caso contrario se inician los procesos correspondientes para su implantación o se pospone ésta al próximo reinicio de los componentes implicados. Si error S-2. \\
		5 & Se confirma que los datos han sido modificados con éxito.
	\end{tablasecuencias}
}{
	% Tabla de secuencia con errores del caso de uso
	\begin{tablasecuencias}
		S-1 & No se puede conectar con el servidor central. Se informa al usuario y se le permite reintentar la petición o abandonar la operación.\\
		S-2 & Los datos no han podido ser aplicados o no se han podido iniciar los procesos correspondientes. Se informa al usuario y se le permite reintentar o abandonar la aplicación revirtiendo los efectos del punto 3 que no hayan podido ser aplicados.
	\end{tablasecuencias}
}


