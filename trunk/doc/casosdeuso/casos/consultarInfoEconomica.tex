% Caso de uso: consultar información económica
% Obs: para escribir comas en el texto del primer parámetro se han de encerrar entre {}.
% Revisado por Cristina el día 11/03/2013
\casodeuso{
	% Nombre del caso de uso
	nombre=COnsultar información económica.,
	% Objetivo
	objetivo=Mostrar la información económica de la empresa.,
	% Entradas
	entradas=No hay entradas.,
	% Precondiciones
	precondiciones=El operador de la aplicación tiene credenciales que le habilitan para realizar dicha operación.,
	% Salidas
	salidas=La información requerida.,
	% Postcondiciones en caso de éxito
	postexito=No se realiza ningún cambio en el sistema.,
	% Postcondiciones en caso de error
	posterror=No se realiza ningún cambio en el sistema.,
	% Actores
	actores={El personal de asuntos económicos e infraestructura, directivos\dots) y la base de datos.},
}{
	% Tabla de secuencia normal del caso de uso
	\begin{tablasecuencias}
		1 & Se extrae de la base de datos la información económica de la empresa. Si error S-1.
	\end{tablasecuencias}
}{
	% Tabla de secuencia con errores del caso de uso
	\begin{tablasecuencias}
		S-1 & Si no se puede conectar con la base de datos se muestra un mensaje de tipo \textit{información no disponible temporalmente} y se vuelve al menú principal de la aplicación.
	\end{tablasecuencias}
}
