% Caso de uso: modificar items inventario.
% Obs: para escribir comas en el texto del primer parámetro se han de encerrar entre {}.

% Revisado por Juanan el día 11/03/2013

\casodeuso{
	% Nombre del caso de uso
	nombre=Modificar inventario,
	% Objetivo
	objetivo={Permite modificar diversos datos(Cantidad, precio, imagen, \ldots) , sobre el material registrado en el inventario de la empresa.},
	% Entradas
	entradas=La nueva información sobre el material.,
	% Precondiciones
	precondiciones={El operador de la aplicación tiene credenciales que le habilitan para realizar dicha operación. El servidor que hospeda la base de datos de inventario está operativo.},
	% Salidas
	salidas=El inventario de la empresa modificado en función de los cambios registrados.
	% Postcondiciones en caso de éxito
	postexito=El material registrado en el inventario se actualiza de acuerdo a los datos introducidos.,
	% Postcondiciones en caso de error
	posterror=No se realiza ningún cambio en el sistema.,
	% Actores
	actores=Personal de administación y mecánico.,
}{
	% Tabla de secuencia normal del caso de uso
	\begin{tablasecuencias}
		1 & Se extrae de la base de datos el inventario con los elementos disponibles. Si error S-1. \\
		2 & Se muestra una lista ordenada según el criterio configurado y permite la búsqueda y filtrado.\\
		3 & El usuario añade, elimina elementos del inventario o modifica sus propiedades. Si error S-2.\\
		4 & El usuario confirma que desea hacer permanentes los cambios realizados. Si error S-3.
	\end{tablasecuencias}
}{
	% Tabla de secuencia con errores del caso de uso
	\begin{tablasecuencias}
		S-1 & Si no se puede conectar con la base de datos, se muestra un mensaje de error y vuelve al menú principal.\\
		S-2 & Si algun campo modificado contiene un dato o formato erroneo se avisa y se vuelve a 2 de la secuencia normal de uso. \\
		S-3 & Si no se puede conectar con la base de datos, se muestra un mensaje de error. Se ofrece la posibilidad de reintentar o volver al listado de inventario.
	\end{tablasecuencias}
}

