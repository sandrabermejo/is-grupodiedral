% Caso de uso: Restablecer Contraseña Cliente
% Obs: para escribir comas en el texto del primer parámetro se han de encerrar entre {}.

% Revisado por Juanan el día 12/03/2013

\casodeuso{
	% Nombre del caso de uso
	nombre=Restablecer contraseña,
	% Objetivo
	objetivo=Cambiar la contraseña de acceso asociada a una cuenta.,
	% Entradas
	entradas=Dirección de correo electrónico asociada a la cuenta.,
	% Precondiciones
	precondiciones=El usuario dispone de una cuenta de cliente de la compañía aérea.,
	% Salidas
	salidas=Un e-mail con un enlace que permite establecer una nueva contraseña.,
	% Postcondiciones en caso de éxito
	postexito=La nueva contraseña queda almacenada sustituyendo a la anterior.,
	% Postcondiciones en caso de error
	posterror=No se realiza ningún cambio en el sistema.,
	% Actores
	actores=Cliente-usuario de interfaz web,
}{
	% Tabla de secuencia normal del caso de uso
	\begin{tablasecuencias}
		1 & El cliente introduce su dirección de correo electrónico. Si error S-1. \\
		2 & El sistema muestra mensaje de confirmación.
	\end{tablasecuencias}
}{
	% Tabla de secuencia con errores del caso de uso
	\begin{tablasecuencias}
		S-1 & Si la dirección de correo no se encuentra asociada a una cuenta registrada se avisa de ello mediante un mensaje y se vuelve a 1 de la secuencia normal de uso.
	\end{tablasecuencias}
}
