% Caso de uso: Acceder horarios
% Obs: para escribir comas en el texto del primer parámetro se han de encerrar entre {}.

\casodeuso{
	% Nombre del caso de uso
	nombre=Acceder Horarios,
	% Objetivo
	objetivo=Mostrar los horarios de trabajo del empleado.,
	% Entradas
	entradas=Fecha o rango de fechas.,
	% Precondiciones
	precondiciones=Haber accedido al sistema con un usuario válido y acceder a la opción \textit{Consultar horarios},
	% Salidas
	salidas=Muestra por pantalla los horarios del empleado.,
	% Postcondiciones en caso de éxito
	postexito=El empleado puede acceder a los horarios en la fecha seleccionada.,
	% Postcondiciones en caso de error
	posterror={Una pantalla de notificación de error, en la medida de lo posible.},
	% Actores
	actores={Empleados de la compañía, base de datos.},
}{
	% Tabla de secuencia normal del caso de uso
	\begin{tablasecuencias}
		1 & Selecciona fecha o rango de fechas de las que consultar horario. Si error S-1. \\
		2 & Extrae de la base de datos los horarios del empleado. Si error S-1. \\
		2 & Muestra por pantalla horario pedido. 
	\end{tablasecuencias}
}{
	% Tabla de secuencia con errores del caso de uso
	\begin{tablasecuencias}
		S-1 & Si el rango de fechas no es válido o no hay datos registrados correspondientes se vuelve al paso 1 notificando del error. \\
		S-2 & Si no se puede conectar con la base de datos se muestra mensaje  de tipo \textit{información no disponible temporalmente} y se vuelve al paso 1 de la secuencia normal.
	\end{tablasecuencias}
}

