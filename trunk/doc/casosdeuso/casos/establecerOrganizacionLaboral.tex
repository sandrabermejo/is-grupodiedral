% Caso de uso: establecer organización laboral
% Obs: para escribir comas en el texto del primer parámetro se han de encerrar entre {}.

% Revisado por Cristina el día 12/03/2013

\casodeuso{
	% Nombre del caso de uso
	nombre=Establecer organización laboral,
	% Objetivo
	objetivo={Establecer las configuraciones generales sobre la organización laboral de la compañía, como secciones y puestos de trabajo, detallando la información relacionada (salario base\dots) y fijando los privilegios de acceso dentro de la aplicación.},
	% Entradas
	entradas=Los valores a configurar.,
	% Precondiciones
	precondiciones=El operador de la aplicación tiene credenciales que le habilitan para realizar esta operación.,
	% Salidas
	salidas=El valor final de los parámetros configurados.,
	% Postcondiciones en caso de éxito
	postexito=Los cambios efectuados se guardan en la base de datos.,
	% Postcondiciones en caso de error
	posterror=No se realiza ningún cambio en el sistema.,
	% Actores
	actores={Personal administrativo, \textit{Recursos Humanos} y \textit{Servicios Informáticos}.},
}{
	% Tabla de secuencia normal del caso de uso
	\begin{tablasecuencias}
		1 & Se muestran sendas listas de categorías obtenidas del servidor central, permitiendo su edición general con precaución.\\
		2 & El usuario edita los detalles de cada configuración. Si error S-1.\\
		3 & Se almacenan los cambios en la base de datos. Si error S-2.
	\end{tablasecuencias}
}{
	% Tabla de secuencia con errores del caso de uso
	\begin{tablasecuencias}
		S-1 & Alguno de los datos introducidos no es válido. Vuelve a 1 de la secuencia normal de uso indicando los campos erróneos.\\
		S-2 & No se puede conectar con la base de datos, se muestra un mensaje de error por pantalla dando la opción de reintentar o volver al menú principal de la aplicación.
	\end{tablasecuencias}
}
