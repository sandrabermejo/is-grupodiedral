% Caso de uso: establecer organización laboral.
% Obs: para escribir comas en el texto del primer parámetro se han de encerrar entre {}.
% discutir

\casodeuso{
	% Nombre del caso de uso
	nombre=Establecer organización laboral,
	% Objetivo
	objetivo={Establecer las configuraciones generales sobre la organización laboral de la compañía, como secciones y puestos de trabajo, detallando la información relacionada (salario base\dots) y fijando los privilegios de acceso dentro de la aplicación.},
	% Entradas
	entradas={Las valores correspondientes a esas configuraciones, como por ejemplo: nombre, descripción, salario base, privilegios\dots},
	% Precondiciones
	precondiciones={El operador de la aplicación está debidamente registrado y posee credenciales que le habilitan para realizar esta operación.},
	% Salidas
	salidas={Valor final de los parámetros configurados.},
	% Postcondiciones en caso de éxito
	postexito={Las configuraciones establecidas podrán ser utilizadas en los contextos que las requieran (registro, edición y consulta de ficha de empleado, nóminas\ldots).},
	% Postcondiciones en caso de error
	posterror={El sistema central no habrá sufrido cambios.},
	% Actores
	actores={Personal de administración, \textit{Recursos Humanos} y \textit{Servicios Informáticos}.},
}{
	% Tabla de secuencia normal del caso de uso
	\begin{tablasecuencias}
		1 & Se muestran sendas listas de categorías obtenidas del servidor central, permitiendo su edición general con precaución.\\
		2 & Se podrán editar los detalles de cada configuración.\\
		3 & Las información editada se sincronizará en el servidor central.
	\end{tablasecuencias}
}{
	% Tabla de secuencia con errores del caso de uso
	\begin{tablasecuencias}
		S-1 & Si no se puede acceder al servidor central o no se puede obtener la información requerida, informar al usuario.
	\end{tablasecuencias}
}
