% Caso de uso: registrar mantenimiento.
% Obs: para escribir comas en el texto del primer parámetro se han de encerrar entre {}.

% Revisado por Cristina el día 11/03/2013
% Nombre antiguo: realizarMantenimiento

\casodeuso{
	% Nombre del caso de uso
	nombre=Registrar mantenimiento,
	% Objetivo
	objetivo=Registrar la información sobre un mantenimiento realizado.,
	% Entradas
	entradas={Los datos del mantenimiento realizado (tipo y cantidad de items utilizados, observaciones, si ha podido completarse o no, etc.).},
	% Precondiciones
	precondiciones=El operador de la aplicación tiene credenciales que le habilitan para realizar dicha operación.,
	% Salidas
	salidas=Confirmación del registro del mantenimiento.,
	% Postcondiciones en caso de éxito
	postexito={Se registra el mantenimiento y, si procede, se actualiza el material disponible de la empresa en el inventario.},
	% Postcondiciones en caso de error
	posterror=No se realiza ningún cambio en el sistema.,
	% Actores
	actores=El personal mecánico de la compañía y la base de datos.,
}{
	% Tabla de secuencia normal del caso de uso
	\begin{tablasecuencias}
		1 & Se extrae de la base de datos el listado de los mantenimientos programados para el usuario. Si error S-1.\\
		2 & El empleado selecciona un mantenimiento.\\
		3 & El empleado introduce la información detallada del mantenimiento realizado. Si error S-1.\\
		4 & Se extrae el listado del material mecánico disponible en el inventario. Si error S-1.\\
		5 & Se muestra el inventario por pantalla.\\
		6 & El usuario selecciona el material empleado en el mantenimiento, así como la cantidad de items de cada tipo utilizados. Si error S-1. 
	\end{tablasecuencias}
}{
	% Tabla de secuencia con errores del caso de uso
	\begin{tablasecuencias}
		S-1 & No se puede conectar con la base de datos, se muestra un mensaje de error por pantalla dando la opción de reintentar o volver al menú principal de la aplicación.
	\end{tablasecuencias}
}
