% Caso de uso: Dar de baja cliente.
% Obs: para escribir comas en el texto del primer parámetro se han de encerrar entre {}.

\casodeuso{
	% Nombre del caso de uso
	nombre= Dar de baja cliente,
	% Objetivo
	objetivo={Permitir a un cliente que pueda darse de baja en nuestro sistema borrando todos sus datos de él. Esta acción es obligatoria en cualquier sistema donde se pueda registrar un cliente por la LOPD \textit{(Ley Orgánica de Protección de Datos)}.},
	% Entradas
	entradas={Un personal administrativo acceda a la opción del sistema para dar de baja a un cliente.},
	% Precondiciones
	precondiciones={Que el cliente se haya puesto en contacto con la empresa indicándole que quiere darse de baja en el sistema y haberse registrado como un usuario válido de la aplicación perteneciente al Personal de atención al cliente. Elegir la opción \textit{Dar de baja cliente}.},
	% Salidas
	salidas= {Eliminar al cliente de la base de datos de la empresa.},
	% Postcondiciones en caso de éxito
	postexito={Los datos del cliente habrán desaparecido de la base de datos del sistema, borrándose todo por completo.},
	% Postcondiciones en caso de error
	posterror={Los datos del cliente no se habrán eliminado y seguirán estando en nuestro sistema.},
	% Actores
	actores={El Personal de atención al cliente, el cliente a dar de baja y la base de datos.},
}{
	% Tabla de secuencia normal del caso de uso
	\begin{tablasecuencias}
		1 & El cliente notifica a la empresa que desea darse de baja en su sistema.\\
		2 & El personal administrativo accede al sistema y da a la opción de dar de baja al cliente.\\
		3 & El sistema accede a la base de datos para encontrar al cliente. Si error S-1\\
		4 & Se borran del sistema los datos del cliente. Si error S-2.\\
		5 & Se notifica al cliente de que ha sido dado de baja con éxito.
	\end{tablasecuencias}
}{
	% Tabla de secuencia con errores del caso de uso
	\begin{tablasecuencias}
		S-1 & No se ha encontrado al cliente en la base de datos, por lo que se muestra un mensaje de que el cliente no existe.\\
		S-2 & Si no se ha podido eliminar los datos del cliente se le notifica indicándole que si quiere darse de baja vuelva a realizar la operación.
	\end{tablasecuencias}
}


