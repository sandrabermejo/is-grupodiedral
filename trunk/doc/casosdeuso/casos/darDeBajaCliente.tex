% Caso de uso: Dar de baja cliente.
% Obs: para escribir comas en el texto del primer parámetro se han de encerrar entre {}.

% Revisado por Juanan el día 12/03/2013

\casodeuso{
	% Nombre del caso de uso
	nombre=Dar de baja cliente,
	% Objetivo
	objetivo={Causar la baja de un cliente en el sistema a petición del mismo.},
	% Entradas
	entradas=No hay entradas.,
	% Precondiciones
	precondiciones=El operador de la aplicación tiene credenciales que le habilitan para realizar dicha operación y la petición expresa del cliente y la ficha de cliente seleccionada.,
	% Salidas
	salidas=Confirmación de la operación.,
	% Postcondiciones en caso de éxito
	postexito=Los datos del cliente almacenados en el sistema son eliminados.,
	% Postcondiciones en caso de error
	posterror=No se realiza ningún cambio en el sistema.,
	% Actores
	actores=Personal administrativo y la base de datos., %quizá personal de atencion al cliente?
}{
	% Tabla de secuencia normal del caso de uso
	\begin{tablasecuencias}
		1 & El personal administrativo confirma la operación. Si no S-1.\\
		2 & Se hacen efectivos los cambios en la base de datos. Si error S-2.
	\end{tablasecuencias}
}{
	% Tabla de secuencia con errores del caso de uso
	\begin{tablasecuencias}
		S-1 & Se cancela la operación y se vuelve a la ficha del cliente.\\
		S-2 & No se puede conectar con la base de datos. Se cancela la operación, se muestra un mensaje de error por pantalla y se vuelve a la ficha del cliente.
	\end{tablasecuencias}
}


