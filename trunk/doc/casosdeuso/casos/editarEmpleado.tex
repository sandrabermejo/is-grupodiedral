% Caso de uso: editar empleado.
% Obs: para escribir comas en el texto del primer parámetro se han de encerrar entre {}.
% Revisado por Cristina y Juanan el día 11/03/2013
%  Incluir o extender de consultar ficha empleado?
\casodeuso{
	% Nombre del caso de uso
	nombre=Editar empleado.,
	% Objetivo
	objetivo=Actualizar la información referente a uno de los empleados.,
	% Entradas
	entradas=Los datos del empleado a actualizar.,
	% Precondiciones
	precondiciones=El empleado cuenta con los permisos necesarios y tiene una ficha seleccionada.,
	% Salidas
	salidas=La ficha de empleado con los datos modificados actualizados.,
	% Postcondiciones en caso de éxito
	postexito=Los cambios efectuados se guardan en la base de datos.,
	% Postcondiciones en caso de error
	posterror=No se realiza ningún cambio en el sistema.,
	% Actores
	actores={El personal administrativo y base de datos.},
}{
	% Tabla de secuencia normal del caso de uso
	\begin{tablasecuencias}
		1 & El empleado introduce los datos nuevos. Si algún dato no es correcto S-1. \\
		2 & Se almacenan los cambios en la base de datos. Si error S-2.
	\end{tablasecuencias}
}{
	% Tabla de secuencia con errores del caso de uso
	\begin{tablasecuencias}
		S-1 & El sistema vuelve a 1 de la secuencia normal de uso e indica los campos erróneos.\\
		S-2 & No se puede conectar con la base de datos. Se cancela la operación, se muestra un mensaje de error por pantalla y se vuelve a la página principal del sistema.
	\end{tablasecuencias}
}

