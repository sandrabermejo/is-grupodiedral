% Caso de uso: ver información de vuelo
% Obs: para escribir comas en el texto del primer parámetro se han de encerrar entre {}.

\casodeuso{
	% Nombre del caso de uso
	nombre=Ver información de vuelo contratado,
	% Objetivo
	objetivo=Mostrar al usuario la información de sus vuelos contratados.,
	% Entradas
	entradas=No hay entradas.,
	% Precondiciones
	precondiciones=El usuario ha iniciado sesión correctamente en la interfaz web.,
	% Salidas
	salidas=Información detallada de la reserva(número de vuelo, fecha y hora, número de plaza, terminales de salida y de llegada\ldots)},
	% Postcondiciones en caso de éxito
	postexito=No se realiza ningún cambio en el sistema.,
	% Postcondiciones en caso de error
	posterror=No se realiza ningún cambio en el sistema.,
	% Actores
	actores=Cliente-usuario de interfaz web.
}{
	% Tabla de secuencia normal del caso de uso
	\begin{tablasecuencias}
		1 & Muestra listado detallo de los vuelos contratados. Si no disponibles S-1.\\
		2 & Ofrece la opción de seleccionar e imprimir la información de los vuelos mostrados.
	\end{tablasecuencias}
}{
	% Tabla de secuencia con errores del caso de uso
	\begin{tablasecuencias}
		S-1 & Si no se puede conectar con la base de datos se muestra mensaje  de tipo \textit{información no disponible temporalmente} y se vuelve al menú principal de la aplicación.
	\end{tablasecuencias}
}
