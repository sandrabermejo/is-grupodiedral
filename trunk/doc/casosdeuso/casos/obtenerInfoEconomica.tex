% Caso de uso: obtener información económica
% Obs: para escribir comas en el texto del primer parámetro se han de encerrar entre {}.

\casodeuso{
	% Nombre del caso de uso
	nombre=Obtener información económica,
	% Objetivo
	objetivo={Mostrar la información económica de la empresa, de acuerdo a las funciones del usuario. Entre la información recogida se encuentran los balances de la empresa, cuentas de resultados, inversiones en bolsa\ldots},
	% Entradas
	entradas=,
	% Precondiciones
	precondiciones={El operador de la aplicación está debidamente registrado y ocupar el puesto de directivo o empleado de asuntos económicos e infraestructura. La información económica ha sido introducida y procesada con anterioridad.},
	% Salidas
	salidas={La información correspondiente debidamente organizada.},
	% Postcondiciones en caso de éxito
	postexito=El usuario puede explorar los datos económicos mostrados.,
	% Postcondiciones en caso de error
	posterror={El sistema central no ha sufrido cambios.},
	% Actores
	actores={El usuario (personal de asuntos económicos e infraestructura, directivos\dots) y la base de datos.},
}{
	% Tabla de secuencia normal del caso de uso
	\begin{tablasecuencias}
		1 & Extraer de la base de datos la información económica de la empresa. Si error S-1. \\
		2 & Mostrar la visualización de los datos obtenidos (podría incluir gráficos). Si error S-2.
	\end{tablasecuencias}
}{
	% Tabla de secuencia con errores del caso de uso
	\begin{tablasecuencias}
		S-1 & No se puede acceder a la base de datos o no se pueden obtener los datos necesarios. Informar al usuario y abortar la operación.\\
		S-2 & No se ha podido generar la visualización de los datos por algún error imprevisto o por la carencia de datos de la información económica. Infomar al usuario, omitir los elementos erróneos en la medida de lo posible o abortar la operación.
	\end{tablasecuencias}
}
