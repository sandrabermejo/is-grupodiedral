% Caso de uso: consultar ficha empleado.
% Obs: para escribir comas en el texto del primer parámetro se han de encerrar entre {}.
% Revisado por Cristina y Juanan el día 11/03/2013
\casodeuso{
	% Nombre del caso de uso
	nombre=Consultar ficha empleado.,
	% Objetivo
	objetivo={Mostrar la lista de empleados de la compañía, permitiendo buscar y filtrar resultados, así como información detallada de cada empleado en particular.},
	% Entradas
	entradas=Opcionalmente, campos de búsqueda.,
	% Precondiciones
	precondiciones=El operador de la aplicación tiene credenciales que le habilitan para realizar dicha operación.,
	% Salidas
	salidas=Lista de empleados e información detallada sobre el seleccionado.,
	% Postcondiciones en caso de éxito
	postexito=No se realiza ningún cambio en el sistema.,
	% Postcondiciones en caso de error
	posterror=No se realiza ningún cambio en el sistema.,
	% Actores
	actores={Los directivos, empleados de recursos humanos y la base de datos.}.
}{
	% Tabla de secuencia normal del caso de uso
	\begin{tablasecuencias}
		1 & Se extrae de la base de datos del sistema el listado de empleados. Si error S-1.\\
		2 & Se muestra por pantalla la lista de empleados.\\
		3 & El usuario puede filtrar los resultados y buscar empleados según diferentes criterios, como tipo de empleado, duración en la empresa, departamentos, etc.\\
		4 & Se selecciona un empleado.\\
		5 & Se muestra la información detallada del empleado. Si error S-1.
	\end{tablasecuencias}
}{
	% Tabla de secuencia con errores del caso de uso
	\begin{tablasecuencias}
		S-1 & No se puede conectar con la base de datos, se muestra un mensaje de error por pantalla dando la opción de reintentar o volver al menú principal de la aplicación.\\
	\end{tablasecuencias}
}
