% Caso de uso:Consultar ficha empleado.
% Obs: para escribir comas en el texto del primer parámetro se han de encerrar entre {}.

\casodeuso{
	% Nombre del caso de uso
	nombre=Consultar ficha empleado,
	% Objetivo
	objetivo={Mostrar la lista de empleados de la compañía, permitiendo buscar y filtrar resultados, así como información detallada de cada empleado en particular.},
	% Entradas
	entradas=,
	% Precondiciones
	precondiciones=Haber accedido al sistema siendo un usuario directivo o de Recursos Humanos y acceder a la opción \textit{Consultar Ficha Empleado}.,
	% Salidas
	salidas=Lista de empleados e información detallada sobre cada uno.,
	% Postcondiciones en caso de éxito
	postexito=Se puede acceder a la ficha de un empleado con todos los datos que la empresa tiene almacenados de él.,
	% Postcondiciones en caso de error
	posterror=La información sobre la ficha de los empleados no ha podido mostrarse al usuario.,
	% Actores
	actores={Los empleados de la compañía, los usuarios que han accedido al sistema(directivos y empleados de recursos humanos), y la base de datos}.
}{
	% Tabla de secuencia normal del caso de uso
	\begin{tablasecuencias}
		1 & Obtener los datos de los empleados de la base de datos. Si error S-1.\\
		2 & Mostrar por pantalla una lista de empleados ordenados por un criterio asignado por defecto.\\
		3 & El usuario puede filtrar los resultados y buscar empleados según diferentes criterios, como tipo de empleado, duración en la empresa, departamentos, etc.\\
		4 & Seleccionando uno de ellos se accederá a la información detallada del empleado. Si error S-2.
	\end{tablasecuencias}
}{
	% Tabla de secuencia con errores del caso de uso
	\begin{tablasecuencias}
		S-1 & Si no se puede acceder a la base de datos, mostrar un mensaje de error por pantalla y volver a la página principal del sistema.\\
		S-2 & No se ha podido obtener la ficha con los datos del empleado por algún error en el servidor. Mostrar un mensaje de error por pantalla y volver a la página principal del sistema.
	\end{tablasecuencias}
}
