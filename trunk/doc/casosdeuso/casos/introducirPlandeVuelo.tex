% Caso de uso: introducir plan de vuelo
% Obs: para escribir comas en el texto del primer parámetro se han de encerrar entre {}.

% Revisado por Cristina el día 11/03/2013

\casodeuso{
	% Nombre del caso de uso
	nombre=Introducir plan de vuelo,
	% Objetivo
	objetivo=Añadir un plan de vuelo a la lista de vuelos de la compañía aérea.,
	% Entradas
	entradas={La información detallada del vuelo: número de vuelo (4 dígitos), fecha, hora de salida y de llegada, terminal de salida y de llegada, modelo del avión, precio total según preferencias\ldots},
	% Precondiciones
	precondiciones=El operador de la aplicación tiene credenciales que le habilitan para realizar dicha operación.,
	% Salidas
	salidas=Confirmación del registro del nuevo vuelo.,
	% Postcondiciones en caso de éxito
	postexito=El plan de vuelo queda registrado en la lista de vuelos de la compañía aérea.,
	% Postcondiciones en caso de error
	posterror=No se realiza ningún cambio en el sistema.,
	% Actores
	actores=El personal administrativo y la base de datos.,
}{
	% Tabla de secuencia normal del caso de uso
	\begin{tablasecuencias}
		1 & El usuario introduce la información detallada del vuelo. Si error S-1.\\
		2 & Se almacenan los datos del nuevo vuelo en la base de datos de la compañía. Si error S-2.
	\end{tablasecuencias}
}{
	% Tabla de secuencia con errores del caso de uso
	\begin{tablasecuencias}
		S-1 & Algún dato no es válido. El sistema vuelve a 1 de la secuencia normal de uso e indica los campos erróneos.\\
		S-2 & No se puede conectar con la base de datos. Se cancela la operación, se muestra un mensaje de error por pantalla y se vuelve al menú principal de la aplicación.
	\end{tablasecuencias}
}
