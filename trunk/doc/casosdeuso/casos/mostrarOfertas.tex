% Caso de uso: Mostrar ofertas.
% Obs: para escribir comas en el texto del primer parámetro se han de encerrar entre {}.

% Revisado por Juanan el día 12/03/2013

\casodeuso{
	% Nombre del caso de uso
	nombre=Mostrar ofertas,
	% Objetivo
	objetivo=Mostrar las ofertas actuales.,
	% Entradas
	entradas=Palabras clave para filtrar la busqueda y/o criterio de ordenación.,
	% Precondiciones
	precondiciones=No hay precondiciones.,
	% Salidas
	salidas=Las ofertas actuales acordes al filtro especificado y ordenadas según criterio.,
	% Postcondiciones en caso de éxito
	postexito=No se realiza ningún cambio en el sistema.,
	% Postcondiciones en caso de error
	posterror=No se realiza ningún cambio en el sistema.,
	% Actores
	actores=Cualquier usuario que acceda a la página web y la base de datos.
}{
	% Tabla de secuencia normal del caso de uso
	\begin{tablasecuencias}
		1 & Se extrae de la base de datos del sistema el listado de ofertas. Si error S-1.\\
		2 & Se muestran listado de ofertas que responde a la busqueda si lo hay o aviso de que los criterios seleccionados no produjeron resultados.\\
		3 & El cliente selecciona y accede a los detalles de una oferta se rediccionará a esa oferta a través de \textit{Acceder a la oferta}. Si error S-2.
	\end{tablasecuencias}
}{
	% Tabla de secuencia con errores del caso de uso
	\begin{tablasecuencias}
		S-1 & No se puede acceder a la base de datos. Se muestra mensaje y se vuelve al menú principal.\\
		S-2 & Si no se puede rediccionar a la oferta elegida se muestra un mensaje indicando que la oferta no es accesible en estos momentos.
	\end{tablasecuencias}
}
