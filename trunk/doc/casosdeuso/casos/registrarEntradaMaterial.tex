% Caso de uso: registrar entrada de material
% Obs: para escribir comas en el texto del primer parámetro se han de encerrar entre {}.

% Revisado por Cristina el día 11/03/2013

\casodeuso{
	% Nombre del caso de uso
	nombre=Registrar entrada de material,
	% Objetivo
	objetivo=Registrar una entrada de material en el inventario de la empresa.,
	% Entradas
	entradas={Nombre del producto o identificador si es un material previamente registrado, descripción, sección que lo recibe (si procede) y ubicación final.},
	% Precondiciones
	precondiciones=El operador de la aplicación tiene credenciales que le habilitan para realizar dicha operación.,
	% Salidas
	salidas=Adhesivo de inventario con la clave de identificación del material en el sistema (si procede).,
	% Postcondiciones en caso de éxito
	postexito=Se actualiza el inventario de la empresa de acuerdo al registro realizado.,
	% Postcondiciones en caso de error
	posterror=No se realiza ningún cambio en el sistema.,
	% Actores
	actores=El personal de la compañía y la base de datos.,
}{
	% Tabla de secuencia normal del caso de uso
	\begin{tablasecuencias}
		1 & El empleado introduce los datos necesarios del registro. Si error S-1.\\
		2 & Se registran los cambios en el sistema correspondiente. Si error S-2.\\
		3 & Se obtiene una adhesivo impreso para identificar el objeto catalogado (si procede).
	\end{tablasecuencias}
}{
	% Tabla de secuencia con errores del caso de uso
	\begin{tablasecuencias}
		S-1 & El sistema vuelve a 1 de la secuencia normal de uso e indica los campos erróneos.\\
		S-2 & No se puede conectar con la base de datos, se muestra un mensaje de error por pantalla dando la opción de reintentar o volver al menú principal de la aplicación.
	\end{tablasecuencias}
}
