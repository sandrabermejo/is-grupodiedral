% Caso de uso: Consultar el plan de vuelo.
% Obs: para escribir comas en el texto del primer parámetro se han de encerrar entre {}.
% Revisado por Juanan el día 11/03/2013
\casodeuso{
	% Nombre del caso de uso
	nombre=Consultar el plan de vuelo.,
	% Objetivo
	objetivo=Mostrar planes de vuelo de los servicios operados por la compañía,
	% Entradas
	entradas={Alguno de las siguientes: un número de vuelo (4 dígitos); fecha, hora y aeropuertos implicados; tripulación asignada\dots},
	% Precondiciones
	precondiciones=El operador de la aplicación tiene credenciales que le habilitan para realizar dicha operación. Los planes de vuelo han sido introducidos con anterioridad.,
	% Salidas
	salidas={El plan de vuelo detallado o, en caso de que la entrada no determine unívocamente un servicio, una lista de vuelos.},
	% Postcondiciones en caso de éxito
	postexito=No se realiza ningún cambio en el sistema.,
	% Postcondiciones en caso de error
	posterror=No se realiza ningún cambio en el sistema.,
	% Actores
	actores={Tripulación de vuelo, el personal administrativo y la base de datos.}
}{
	% Tabla de secuencia normal del caso de uso
	\begin{tablasecuencias}
		1 & El empleado introduce un número de vuelo u otras restricciones. Si el número de vuelo no es válido S-1.\\
		2 & Accede a la base de datos y extrae la información. Si error S-2.\\
		3 & Si el criterio no es unívoco, mostrar una lista de vuelos (permite pasar a 4 con cada uno de ellos) o informa de que ningún vuelo cumple los requisitos establecidos. Si lo es, 4.\\
		4 & Muestra el plan de vuelo por pantalla.
	\end{tablasecuencias}
}{
	% Tabla de secuencia con errores del caso de uso
	\begin{tablasecuencias}
		S-1 & Muestra mensaje de error y vuelve a 1 de la secuencia normal de uso.\\
		S-2 & No se puede acceder a la base de datos. Muestra error por pantalla y vuekve a 1 de la secuencia normal de uso.\\
	\end{tablasecuencias}
}
