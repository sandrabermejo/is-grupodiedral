% Caso de uso: Efectuar embarque
% Obs: para escribir comas en el texto del primer parámetro se han de encerrar entre {}.

% Comentario de Aitor, si un cliente no se presenta a un embarque se cancelan todos los billetes asociados a ese cliente.
% Revisado por Juanan el día 12/03/2013

\casodeuso{
	% Nombre del caso de uso
	nombre=Efectuar embarque,
	% Objetivo
	objetivo=Registrar el embarque del pasajero.,
	% Entradas
	entradas=El número de vuelo y el número de reserva del pasajero.,
	% Precondiciones
	precondiciones=El operador de la aplicación tiene credenciales que le habilitan para realizar dicha operación y un billete válido,
	% Salidas
	salidas=Confirmación de que el pasajero pasa el control de seguridad con éxito y ha embarca en el avión.,
	% Postcondiciones en caso de éxito
	postexito=Se registra el embarque del pasajero.,
	% Postcondiciones en caso de error
	posterror=No se realiza ningún cambio en el sistema.,
	% Actores
	actores=El personal de la compañía presente en el aeropuerto y la base de datos.
}{
	% Tabla de secuencia normal del caso de uso
	\begin{tablasecuencias}
		1 & El usuario introduce el número de vuelo y el número de reservar del pasajero. Si error S-1.\\
		2 & Se comprueba la reserva y se actualiza la base de datos. Si error S-2.\\

		
	\end{tablasecuencias}
}{
	% Tabla de secuencia con errores del caso de uso
	\begin{tablasecuencias}
		S-1 & Alguno de los campos introducidos por el usuario no es válido. Se muestra un mensaje de error y se vuelve a 1 de la secuencia normal de uso indicando los datos erroneos. \\
		S-2 & No se ha podido conectar con la base de datos. Se muestra un mensaje de error y se ofrece la posibilidad de reintentar.
	\end{tablasecuencias}
}

