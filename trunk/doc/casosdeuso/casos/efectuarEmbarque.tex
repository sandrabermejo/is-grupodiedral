% Caso de uso: Efectuar embarque
% Obs: para escribir comas en el texto del primer parámetro se han de encerrar entre {}.

\casodeuso{
	% Nombre del caso de uso
	nombre=Efectuar embarque.,
	% Objetivo
	objetivo=Controlar el embarque del pasajero.,
	% Entradas
	entradas=El número de vuelo y el número de reserva del pasajero.,
	% Precondiciones
	precondiciones=Haber accedido al sistema con un usuario válido y elegir la opción \textit{Efectuar embarque} de la aplicación.,
	% Salidas
	salidas=Confirmación de que el pasajero ha pasado el control de seguridad con éxito y ha embarcado en el avión correctamente.,
	% Postcondiciones en caso de éxito
	postexito=El pasajero ha embarcado correctamente en su vuelo.,
	% Postcondiciones en caso de error
	posterror=Ha habido algún error en el proceso de embarque y el pasajero se queda en tierra.,
	% Actores
	actores=El personal de la compañía presente en el aeropuerto y la base de datos.
}{
	% Tabla de secuencia normal del caso de uso
	\begin{tablasecuencias}
		1 & El usuario introduce el número de vuelo y el número de reservar del pasajero. Si error S-1.\\
		2 & Se comprueba que el usuario debe embarcar en ese vuelo. Si error S-2.\\
		3 & Se mostrará un mensaje de embarque completado con éxito.
		
	\end{tablasecuencias}
}{
	% Tabla de secuencia con errores del caso de uso
	\begin{tablasecuencias}
		S-1 & Alguno de los campos introducidos por el usuario no es válido. Se muestra un mensaje de error y se pide de nuevo la introducción de los datos. \\
		S-2 & No se ha podido conectar con la base de datos. Se muestra un mensaje de error y se pide de nuevo la introducción de los datos.
	\end{tablasecuencias}
}

