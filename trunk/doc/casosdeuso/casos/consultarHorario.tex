% Caso de uso: consultar horario
% Obs: para escribir comas en el texto del primer parámetro se han de encerrar entre {}.
% Revisado por Cristina y Juanan el día 11/03/2013
%  Incluir o extender de consultar ficha empleado?
\casodeuso{
	% Nombre del caso de uso
	nombre=Consultar horario.,
	% Objetivo
	objetivo=Mostrar el horario de trabajo del empleado.,
	% Entradas
	entradas=Fecha o rango de fechas.,
	% Precondiciones
	precondiciones=El operador de la aplicación tiene credenciales que le habilitan para realizar dicha operación y tiene una ficha seleccionada.,
	% Salidas
	salidas=El horario del empleado en el rango seleccionado.,
	% Postcondiciones en caso de éxito
	postexito=No se realiza ningún cambio en el sistema.,
	% Postcondiciones en caso de error
	posterror=No se realiza ningún cambio en el sistema.,
	% Actores
	actores=Empleados y base de datos.,
}{
	% Tabla de secuencia normal del caso de uso
	\begin{tablasecuencias}
		1 & El empleado selecciona la fecha o el rango de fechas a consultar. Si error S-1.\\
		2 & Se extrae de la base de datos el horario del empleado. Si error de conexión S-2. Si otro error S-3.
	\end{tablasecuencias}
}{
	% Tabla de secuencia con errores del caso de uso
	\begin{tablasecuencias}
		S-1 & La fecha no es válida por algún motivo. Se muestra un mensaje de error por pantalla dando la opción de reintentar o volver al menú principal de la aplicación.\\
		S-2 & Si no se puede conectar con la base de datos se muestra un mensaje de tipo \textit{información no disponible temporalmente} y se vuelve a la ficha del empleado.\\
		S-3 & Si no hay datos registrados correspondientes se vuelve a la ficha del empleado y se notifica el error. 
	\end{tablasecuencias}
}