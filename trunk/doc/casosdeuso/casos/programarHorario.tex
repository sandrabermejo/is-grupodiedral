% Caso de uso: programar horario
% Obs: para escribir comas en el texto del primer parámetro se han de encerrar entre {}.
% Revisado por Cristina el día 11/03/2013
%  Incluir o extender de consultar horario?
\casodeuso{
	% Nombre del caso de uso
	nombre=Programar horario.,
	% Objetivo
	objetivo=Añadir a la base de datos del sistema el horario de un empleado.,
	% Entradas
	entradas=El nuevo horario.,
	% Precondiciones
	precondiciones=El operador de la aplicación tiene credenciales que le habilitan para realizar dicha operación.,
	% Salidas
	salidas=El horario del empleado modificado.,
	% Postcondiciones en caso de éxito
	postexito=El horario del empleado ha sido actualizado en la base de datos.,
	% Postcondiciones en caso de error
	posterror=No se realiza ningún cambio en el sistema.,
	% Actores
	actores=El personal administrativo y la base de datos.,
}{
	% Tabla de secuencia normal del caso de uso
	\begin{tablasecuencias}
		1 & Se selecciona un nuevo horario para el empleado y una fecha a partir de la cual será vigente. Si error S-1.\\
		2 & Se almacenan los cambios en la base de datos. Si error S-2.
	\end{tablasecuencias}
}{
	% Tabla de secuencia con errores del caso de uso
	\begin{tablasecuencias}
		S-1 & El horario o la fecha introducidos no son válidos por algún motivo. Vuelve a 1 de la secuencia normal de uso indicando que el horario no es válido. \\
		S-2 & No se puede conectar con la base de datos. Se cancela la operación, se muestra un mensaje de error por pantalla y se vuelve a la ficha del empleado.
	\end{tablasecuencias}
}