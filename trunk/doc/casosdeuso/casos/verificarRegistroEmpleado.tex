% Caso de uso: verificar registro empleado.
% Obs: para escribir comas en el texto del primer parámetro se han de encerrar entre {}.

% Revisado por Cristina y Juanan el día 11/03/2013

\casodeuso{
	% Nombre del caso de uso
	nombre=Verificar registro empleado,
	% Objetivo
	objetivo=Establecer una nueva cuenta de usuario como válida.,
	% Entradas
	entradas={Los datos del usuario previamente registrado.},
	% Precondiciones
	precondiciones={El operador de la aplicación tiene credenciales que le habilitan para realizar dicha operación. El nuevo usuario está registrado correctamente en el sistema.},
	% Salidas
	salidas=Se confirma la operación.,
	% Postcondiciones en caso de éxito
	postexito={La cuenta del usuario queda verificada y, por tanto, pasa a estar activa y totalmente operativa.},
	% Postcondiciones en caso de error
	posterror=La cuenta del usuario permanece inactiva.,
	% Actores
	actores=El personal del departamento de intervención y la base de datos.,
}{
	% Tabla de secuencia normal del caso de uso
	\begin{tablasecuencias}
		1 & Se extrae de la base de datos del sistema el listado de usuarios pendientes de verificación. Si error S-1.\\
		2 & El empleado selecciona el usuario.\\
		3 & Se muestran los datos por pantalla.\\
		4 & Se comprueba que sean correctos. Si alguno de los datos no es correcto, S-2.\\
		5 & Se verifica la cuenta del usuario y se modifica en la base de datos. Si error S-3.
	\end{tablasecuencias}
}{
	% Tabla de secuencia con errores del caso de uso
	\begin{tablasecuencias}
		S-1 & No se puede conectar con la base de datos, se muestra un mensaje de error por pantalla dando la opción de reintentar o volver al menú principal de la aplicación.\\
		S-2 & Se cancela la operación y vuelve al menú principal.\\
		S-3 & No se puede acceder a la base de datos. Se cancela la operación, se muestra un mensaje de error por pantalla dando la opción de reintentar o volver al menú principal de la aplicación.
	\end{tablasecuencias}
}


