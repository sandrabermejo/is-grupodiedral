% Caso de uso: Configurar nómina
% Obs: para escribir comas en el texto del primer parámetro se han de encerrar entre {}.

% Revisado por Juanan el día 11/03/2013

\casodeuso{
	% Nombre del caso de uso
	nombre=Configurar nómina,
	% Objetivo
	objetivo=Confeccionar y almacenar las nóminas mensuales de un empleado.,
	% Entradas
	entradas={Incidencias relativas a la nómina de un empleado correspondientes a un mes (horas extras, comisiones, sustituciones\ldots)},
	% Precondiciones
	precondiciones=El operador de la aplicación tiene credenciales que le habilitan para realizar dicha operación y tiene una ficha seleccionada..,
	% Salidas
	salidas=Confirmación de la operación,
	% Postcondiciones en caso de éxito
	postexito=La nómina queda archivada en la base de datos.,
	% Postcondiciones en caso de error
	posterror=No se realiza ningún cambio en el sistema.,
	% Actores
	actores=Personal de administración y la base de datos.
}{
	% Tabla de secuencia normal del caso de uso
	\begin{tablasecuencias}
		1 & Se extraen los datos de la base de datos. Si error S-1. \\
		2 & Se introducen los conceptos e incidencias correspondientes.\\
		3 & Se almacenan los cambios en la base de datos. Si error S-2.
	\end{tablasecuencias}
}{
	% Tabla de secuencia con errores del caso de uso
	\begin{tablasecuencias}
		S-1 & Si no se puede conectar con la base de datos, se muestra un mensaje de error y vuelve a la ficha de empleado. \\
		S-2 & No se puede conectar con la base de datos. Se cancela la operación, se muestra un mensaje de error por pantalla y se vuelve a la ficha del empleado.
	\end{tablasecuencias}
}
