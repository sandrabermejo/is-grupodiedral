% Caso de uso: Acceder Gestión Interna
% Obs: para escribir comas en el texto del primer parámetro se han de encerrar entre {}.

\casodeuso{
	% Nombre del caso de uso
	nombre=Acceder,
	% Objetivo
	objetivo=Abrir la aplicación de gestión interna de la compañía.,
	% Entradas
	entradas=Nombre del usuario y contraseña.,
	% Precondiciones
	precondiciones=Que el usuario esté registrado como empleado de la compañía.,
	% Salidas
	salidas=Las diversas opciones de gestión interna de la aplicación.,
	% Postcondiciones en caso de éxito
	postexito={El usuario tiene acceso a las diversas opciones de la gestión interna de la compañía, dependiendo de su función dentro de la empresa.},
	% Postcondiciones en caso de error
	posterror=El usuario no ha podido acceder a la aplicación de gestión interna.,
	% Actores
	actores=El usuario y la base de datos.,
}{
	% Tabla de secuencia normal del caso de uso
	\begin{tablasecuencias}
		1 & El usuario introduce su nombre y contraseña y elige la opción \textit{Acceder}.\\
		2 & Se verifican los datos introducidos y se cargan todos los datos accesibles al usuario que ha accedido. Si error S-2. 
	\end{tablasecuencias}
}{
	% Tabla de secuencia con errores del caso de uso
	\begin{tablasecuencias}
		S-1 & Si el usuario o la contraseña no son válidos, el sistema vuelve a la secuencia normal de uso e indica los campos erróneos. Se le da la opción de volver a introducir los datos de nuevo.\\
	\end{tablasecuencias}
}



