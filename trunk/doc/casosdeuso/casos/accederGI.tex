% Caso de uso: acceder gestión interna
% Obs: para escribir comas en el texto del primer parámetro se han de encerrar entre {}.
% Revisado por Cristina y Juanan el día 11/03/2013
\casodeuso{
	% Nombre del caso de uso
	nombre=Acceder gestión interna,
	% Objetivo
	objetivo=Abrir la aplicación de gestión interna de la compañía.,
	% Entradas
	entradas=Nombre del usuario y contraseña.,
	% Precondiciones
	precondiciones=El empleado debe estar registrado como usuario del sistema.,
	% Salidas
	salidas=Las diversas opciones de gestión interna de la aplicación.,
	% Postcondiciones en caso de éxito
	postexito={El empleado tiene acceso a las diversas opciones de la gestión interna de la compañía, dependiendo de su función dentro de la empresa.},
	% Postcondiciones en caso de error
	posterror=No se realiza ningún cambio en el sistema.,
	% Actores
	actores=El empleado y la base de datos.,
}{
	% Tabla de secuencia normal del caso de uso
	\begin{tablasecuencias}
		1 & El empleado introduce su nombre y contraseña y elige la opción \textit{Acceder}.\\
		2 & El sistema verifica los datos introducidos y se cargan todos los datos accesibles al usuario que ha accedido. Si error S-1. 
	\end{tablasecuencias}
}{
	% Tabla de secuencia con errores del caso de uso
	\begin{tablasecuencias}
		S-1 & Si el usuario o la contraseña no son válidos, el sistema vuelve a 1 de la secuencia normal e indica los campos erróneos.
	\end{tablasecuencias}
}



