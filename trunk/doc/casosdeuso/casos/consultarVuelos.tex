% Caso de uso: consultar vuelos.
% Obs: para escribir comas en el texto del primer parámetro se han de encerrar entre {}.

\casodeuso{
	% Nombre del caso de uso
	nombre=Consultar vuelos,
	% Objetivo
	objetivo={Mostrar al cliente la relación de vuelos operados por la compañía, pudiendo filtrar resultados y buscar por diferentes criterios; permitiendo además obtener información detallada de los vuelos seleccionados.},
	% Entradas
	entradas={Opcionalmente las que correspondan a los filtros (aeropuertos de origen y destino, número de escalas, fecha y hora, precio del billete\ldots). En última instancia, vuelo seleccionado.},
	% Precondiciones
	precondiciones={La información de vuelos ha sido previamente introducida en el sistema, así como los criterios configurados.},
	% Salidas
	salidas={Una lista filtrada de vuelos y, al seleccionar uno de ellos, información detallada del mismo.},
	% Postcondiciones en caso de éxito
	postexito=El cliente puede acceder a la compra de billetes del vuelo seleccionado.,
	% Postcondiciones en caso de error
	posterror={Una pantalla de notificación de error, en la medida de lo posible.},
	% Actores
	actores={Clientes de la compañía, base de datos.},
}{
	% Tabla de secuencia normal del caso de uso
	\begin{tablasecuencias}
		1 & Se muestra una lista de vuelos ordenados por un criterio asignado por defecto. Si error S-1. \\
		2 & El usuario puede filtrar los resultados según diferentes reglas, aparecerá una lista reducida de vuelos (incluso nula).\\
		3 & Seleccionando uno de ellos se accederá a la información especializada en ese servicio, dando acceso a la adquisición de billetes. 
	\end{tablasecuencias}
}{
	% Tabla de secuencia con errores del caso de uso
	\begin{tablasecuencias}
		S-1 & Si no se puede acceder al servidor central o no se puede obtener la información de vuelos, informar al usuario.
	\end{tablasecuencias}
}
