% Caso de uso: Facturar
% Obs: para escribir comas en el texto del primer parámetro se han de encerrar entre {}.

\casodeuso{
	% Nombre del caso de uso
	nombre=Facturar,
	% Objetivo
	objetivo=Controlar la facturación del equipaje del viajero.,
	% Entradas
	entradas={El número de vuelo, el número de reserva del pasajero y el peso del equipaje.},
	% Precondiciones
	precondiciones=Haber accedido al sistema con un usuario válido y elegir la opción \textit{Facturar} de la aplicación.,
	% Salidas
	salidas={Confirmación de que el equipaje ha sido facturado, en caso de que proceda.},
	% Postcondiciones en caso de éxito
	postexito=El pasajero ha facturado su equipaje.,
	% Postcondiciones en caso de error
	posterror=Ha habido algún error en el proceso de facturación y el pasajero no puede facturar su equipaje.,
	% Actores
	actores=El personal de la compañía presente en el aeropuerto y la base de datos.
}{
	% Tabla de secuencia normal del caso de uso
	\begin{tablasecuencias}
		1 & El usuario introduce el número de vuelo y el número de reserva del pasajero. Si error S-1. \\
		2 & El usuario introduce el peso de la maleta del pasajero y, en caso de exceso de peso, se mostrará un mensaje indicando la cantidad de dinero a pagar en el momento. Si error S-2. \\
		3 & Se mostrará un mensaje de facturación completada con éxito.	
	\end{tablasecuencias}
}{
	% Tabla de secuencia con errores del caso de uso
	\begin{tablasecuencias}
		S-1 & Alguno de los campos introducidos por el usuario no es válido. Se muestra un mensaje de error y se pide de nuevo la introducción de los datos. \\
		S-2 & El peso introducido no está en el rango admitido por la compañía. Se muestra un mensaje de error y se pide de nuevo la introducción del peso.
	\end{tablasecuencias}
}

