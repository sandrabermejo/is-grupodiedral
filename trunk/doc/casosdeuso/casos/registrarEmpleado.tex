% Caso de uso: registrar empleado
% Obs: para escribir comas en el texto del primer parámetro se han de encerrar entre {}.

% Revisado por Cristina y Juanan el día 11/03/2013

\casodeuso{
	% Nombre del caso de uso
	nombre=Registrar empleado,
	% Objetivo
	objetivo=Añadir un nuevo usuario a la base de datos del personal de la compañía con unos determinados permisos de acceso.,
	% Entradas
	entradas={Nombre de usuario, datos personales y puesto de trabajo al que se incorpora dentro de la compañía.},
	% Precondiciones
	precondiciones={El operador de la aplicación tiene credenciales que le habilitan para realizar dicha operación. El futuro usuario debe estar contratado por la empresa y haber firmado la LOPD (Ley Orgánica de Protección de Datos).},
	% Salidas
	salidas=Se confirma la creación de la cuenta.,
	% Postcondiciones en caso de éxito
	postexito=El usuario se registra en la base de datos y su cuenta queda pendiente de verificación.,
	% Postcondiciones en caso de error
	posterror=No se realiza ningún cambio en el sistema.,
	% Actores
	actores=El personal administrativo y la base de datos.,
}{
	% Tabla de secuencia normal del caso de uso
	\begin{tablasecuencias}
		1 & El administrativo inserta los datos del futuro usuario. Si error S-1.\\
		2 & El sistema genera una contraseña aleatoria.\\
		3 & Se vuelcan los datos a la base de datos. Si error S-2.\\
		4 & Se muestra un mensaje de confirmación del registro. \\
		5 & Se imprime un documento con los datos de acceso para el usuario.
	\end{tablasecuencias}
}{
	% Tabla de secuencia con errores del caso de uso
	\begin{tablasecuencias}
		S-1 & Alguno de los datos no es válido. El sistema vuelve a 1 de la secuencia normal de uso e indica los campos erróneos.\\
		S-2 & No se puede conectar con la base de datos. Se cancela la operación, se muestra un mensaje de error por pantalla y se vuelve a la página principal del sistema.
	\end{tablasecuencias}
}


