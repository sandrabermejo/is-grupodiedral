% Hoja de descripción de riesgo: incumplimiento de lo acordado a la fecha de entrega

\hojariesgo{Incumplimiento de lo acordado a la fecha de entrega}{
	% Identificador del riesgo
	id=P6,
	% Persona o departamento que ha identificado el riesgo
	identificador=Departamento de Planificación,
	% Fecha de identificación del riesgo
	fecha=20 de febrero de 2013,
	% Descripción
	descripcion={Es posible que a la fecha de entrega al cliente no se haya cumplido con las expectativas o compromisos acordados. Este hecho tiene como consecuencia más inmediata la denegación por el parte del cliente de las contraprestaciones acordadas (una nota decente).},
	P6 & \nameref{Riesgos:P6} & L & S & O	& CRI	& ALTO	\\ \hline % incumplimiento de lo acordado
	% Influencia (C->coste, S->calendario, R->rendimiento, Q->calidad)
	influye=S,
	% Consecuencia (C->crítico, S->serio, M->moderado, N->menor)
	consecuencia=C,
	% Impacto 
	impacto=ALTO,
	% Probabilidad (A->alta, M->media, B->baja, %)
	probabilidad=O,
	% Periodo de previsión (C->corto plazo, L->largo plazo)
	periodo=L,
	% Estrategia de prevención
	estrategia={Emplear un proceso de desarrollo incremental para mitigar las consecuencias del retraso, al estar lista al menos parte de la funcionalidad pretendida. Planificar de forma realista (o sensiblemente pesimista) para evitar compromisos no asequibles.},
	% Plan de contigencia ante el acontecimiento
	contingencia={Si próximo a la consumación del plazo, se descubriese que los compromisos son imposibles de cumplir, se convocará una reunión urgente que decidirá, en función de las circunstancias, las acciones a llevar a cabo. Entre las opciones consideradas está la realización de un \textit{sprint} para conseguir un mayor nivel de avance para la entrega, el planteamiento de negociaciones con el cliente para postergar la fecha de entrega, el envío o no de material de calidad no verificada correspondiente a la parte incompleta\ldots},
	% Grupo de riesgos al que pertenece
	grupo=Proyecto
}
