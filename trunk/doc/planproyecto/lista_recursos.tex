% Airline Common Environment (ACE)
% Proyecto de Ingeniería del Software
% Grupo Diedral 2013

% Plan de proyecto - Lista de recursos

\begin{itemize}
	\item Personal \\

		\recurso{1}	% id
		{Cristina Alonso Fernández}	% nombre
		{Miembro del equipo de desarrollo}	% descripción
		{}	% origen
		{Tiempo parcial}	% disponibilidad
		{}	% uso

		\recurso{2}
		{Natalia Angulo Herrera}
		{Miembro del equipo de desarrollo}
		{}
		{Tiempo parcial}
		{}

		\recurso{3}
		{Sandra Bermejo Cañadas}
		{Miembro del equipo de desarrollo}
		{}
		{Tiempo parcial}
		{}

		\recurso{4}
		{Juan Andrés Claramunt Pérez}
		{Miembro del equipo de desarrollo}
		{}
		{Tiempo parcial}
		{}

		\recurso{5}
		{Rubén Rafael Rubio Cuéllar}
		{Miembro del equipo de desarrollo}
		{}
		{Tiempo parcial}
		{}

	\item Componentes de software reutilizables \\

		\recurso{6}
		{Biblioteca gráfica \textit{Qt} 4.8 GPL}
		{Biblioteca para interfaces gráficas multiplataforma para C++. Disponible para otros lenguajes de programación.}
		{\textit{Digia} (anteriormente \textit{Nokia}). Licencia \textit{GPL}.}
		{Simultánea e ilimitada.}
		{Codificación y/o prototipado.}


		\recurso{6.5}
		{Iconos \textit{Tango}.}
		{Iconos para uso general dentro de una aplicación.}
		{\textit{Tango Desktop Project} (más detalles en \url{http://tango.freedesktop.org/}).}
		{Simultáneo e ilimitada.}
		{Codificación y/o prototipado.}

	\item Entorno de desarrollo: elaboración  de la documentación \\

		\recurso{7}
		{Sistema tipográfico {\rmfamily \TeX{}} y {\rmfamily\LaTeX{}}}
		{Distribuciones {\rmfamily\TeX{} Live} y {\rmfamily Mik\TeX{}}.}
		{Proyecto {\rmfamily\LaTeX{}3}. Licencia \textit{LPPL}.}
		{Simultánea y no limitada.}
		{Elaboración de la documentación durante todo el proyecto.}

		\recurso{8}
		{Paquetes {\rmfamily\LaTeX{}} de terceros}
		{Los paquetes {\rmfamily color}, {\rmfamily babel}, {\rmfamily xkeyval}, {\rmfamily ifthen}, {\rmfamily ifpdf}, {\rmfamily environ}, {\rmfamily amsmath},  {\rmfamily tabularx}, {\rmfamily supertabular}, {\rmfamily multirow}, {\rmfamily fancyhdr}, {\rmfamily tikz}, {\rmfamily hyphenat}, {\rmfamily anysize}, {\rmfamily graphicsx}, {\rmfamily float}, {\rmfamily rotating}, {\rmfamily hyperref}, {\rmfamily titletoc}, {\rmfamily glossaries} y {\rmfamily todonotes}.}
		{Autores y más información en \url{www.ctan.org}.}
		{Disponible en los equipos personales de los desarrolladores. Algunos miembros del equipo no cuentan con la distribución correctamente configurada. No disponible en los laboratorios de las facultades, salvo configuración \textit{ad hoc}.}
		{Elaboración de la documentación durante todo el proyecto.}

		\recurso{9}
		{Paquete {\rmfamily isdiedral}}
		{Paquete de {\rmfamily \LaTeX{}} elaborado para facilitar la generación de documentación durante el proyecto.}
		{Elaboración propia.}
		{Simultánea y no limitada.}
		{Elaboración de la documentación durante todo el proyecto.}

		\recurso{10}
		{Clases de terceros para {\rmfamily \LaTeX{}}}
		{Las clases {\rmfamily ltxdoc} y {\rmfamily beamer}, para elaboración de documentación de paquetes y de presentaciones de diapositivas respectivamente.}
		{Autores y más información en \url{www.ctan.org}.}
		{Simultánea y no limitada.}
		{Elaboración de la documentación durante todo el proyecto.}

		\recurso{11}
		{Familia de fuentes {\rmfamily Latin Modern}}
		{Familia de fuentes serifadas derivada de {\rmfamily Computer Modern}, familia tipográfica por defecto de {\rmfamily \TeX{}}.}
		{Mantenido por Bogusław Jackowski y Janusz M. Nowacki.}
		{Simultánea y no limitada.}
		{Elaboración de la documentación durante todo el proyecto.}

		\recurso{12}
		{\textit{ArgoUML}}
		{Herramienta de modelado UML.}
		{Licencia \textit{EPL} 1.0. Página oficial: \url{http://argouml.tigris.org}.}
		{Simultánea e ilimitada. Ejecutable en los laboratorios de la Facultad de Informática.}
		{Elaboración de los diagramas de casos de uso y clases en el diseño.}


	\item Entorno de desarrollo: herramientas para la comunicación \\
		
		\recurso{13}
		{\textit{RiouxSVN}}
		{Repositorio de control de versiones basado en \textit{Subversion}.}
		{Ver \url{http://riouxsvn.com}.}
		{Continua y simultánea. Cortes de servicio puntuales por labores de mantenimiento.}
		{Gestión de la configuración, durante todo el proyecto.}

		\recurso{14}
		{\textit{Assembla}}
		{Repositorio de control de versiones basado en \textit{Subversion} como copia de respaldo del
		repositorio corriente.}
		{Ver \url{http://www.assembla.com}.}
		{Continua y simultánea.}
		{Gestión de la configuración, durante todo el proyecto.}

		\recurso{15}
		{Correo electrónico de la Universidad Complutense de Madrid}
		{Correo institucional de la universidad.}
		{}
		{Continua y simultánea.}
		{Comunicación del equipo de desarrollo.}

		\recurso{16}
		{\textit{Google Drive}}
		{Sistema de almacenamiento de documentos. Accedido a través de las \textit{Google Apps} de la Universidad Complutense de Madrid.}
		{}
		{Continua y simultánea.}
		{Comunicación del equipo de desarrollo.}

	\item Entorno de desarrollo: herramientas CASE \\

		\recurso{17}
		{\textit{Microsoft Project}}
		{Programa de administración de proyectos.}
		{\textit{Microsoft Corportation}, a través del programa \textit{DreamSpark} asociado a la Facultad de Informática de la UCM.}
		{Continua y simultánea.}
		{Planificación y gestión del proyecto.}

	\item Entorno de desarrollo: equipo físico \\

		\recurso{17}
		{Uso de los laboratorios}
		{Uso de los laboratorios de la \textit{Facultad de Informática} y la \textit{Facultad de Ciencias Matemáticas} de la UCM.}
		{}
		{Restringida al horario y disponibilidad de plazas.}
		{Durante todo el proyecto.}

		\recurso{18}
		{Ordenadores personales de los miembros del equipo de desarrollo}
		{Ordenadores personales para trabajo por separado.}
		{}
		{No determinada.}
		{Durante todo el proyecto.}

		\recurso{19}
		{Equipo utilizado en las exposiciones}
		{Proyector, pizarra, ordenador del profesor\ldots}
		{}
		{Cuando corresponde.}
		{En las exposiciones públicas en clase.}

	
	\item Entorno de desarrollo: programación de prototipos y versiones del producto \\
		
		\recurso{20}
		{Colección de Compiladores de GNU (GCC)}
		{Colección de compiladores multiplataforma. Especialmente usados los de C y C++.}
		{\textit{Free Software Fundation}. Licencia \textit{GPL}.}
		{Simultánea y no limitada.}
		{Elaboración de prototipos y/o versión final.}

		\recurso{21}
		{Compilador de \textit{Microsoft Visual C++} v11}
		{Compilador de C++ para \textit{Windows}.}
		{\textit{Microsoft Corporation}.}
		{Simultánea y no limitada.}
		{Elaboración de prototipos y/o versión final.}

		\recurso{22}
		{Java}
		{Máquina virtual y entorno de desarrollo de Java, por ejemplo \textit{OpenJDK} 7.}
		{\textit{Oracle Corporation}.}
		{Simultánea y no limitada.}
		{Elaboración de prototipos y/o versión final. Elaboración de herramientas.}

		
	\item Entorno de desarrollo: material de referencia

		\recurso{23}
		{Libro \textit{Ingeniería del Software: un enfoque práctico}}
		{El libro \cite{PSMAN}.}
		{Roger S. Pressman}
		{Sujeta a la disponibilidad de ejemplares en la biblioteca de la facultad.}
		{Todo el proyecto.}

		\recurso{24}
		{Estándares de Ingeniería de Software}
		{Estándares accesibles a través de \textit{IEEEXplore\marcaregistrada} mediante la cuenta de la Universidad Complutense de Madrid.}
		{IEEE.}
		{Simultánea e ilimitada al conjunto de estándares necesarios.}
		{Todo el proyecto.}

		\recurso{25}
		{Documentación de C++}
		{Documentación de C++ disponible en los sitios web \url{www.cplusplus.com} y \url{http://es.cppreference.com/}. Estándares disponibles en \url{www.open-std.com} y el libro \emph{El lenguaje de programación C++} de Bjarne Stroustrup \cite{STROUSTRUP}.}
		{Sus correspondientes autores.}
		{Simultánea y no limitada. Para el libro, sujeta a la disponibilidad de ejemplares en la biblioteca aunque disponible
		en inglés de forma ilimitada y simultánea por medio de la biblioteca de la universidad.}
		{Codificación y/o prototipado.}

		\recurso{26}
		{Documentación de Java}
		{Documentación de la API de Java disponible en \url{http://docs.oracle.com/javase/7/docs/api/} y los apuntes de la asignatura \textit{Tecnología de la Programación} \cite{AP_TP}.}
		{Sus correspondientes autores.}
		{Simultánea y no limitada.}
		{Codificación y/o prototipado.}
\end{itemize}
