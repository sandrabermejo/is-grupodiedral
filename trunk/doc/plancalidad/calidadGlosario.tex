%
%	Plan SQA: Glosario
%

\PrerenderUnicode{ñ}
\PrerenderUnicode{ó}
\PrerenderUnicode{í}

\newglossaryentry{linea_base}{
	name=Línea Base,
	description={Una especificación o producto que se ha revisado formalmente y sobre los que se ha llegado a un acuerdo, y que de ahí en adelante sirve como base para un desarrollo posterior y que puede cambiarse solamente a través de procedimientos formales de control de cambios.},
}
\newglossaryentry{Gestor_superior}{
	name=Gestor superior del proyecto,
	description={Definen los aspectos de negocios que a menudo tienen una significativa influencia en el proyecto.},
}
\newglossaryentry{Gestor_tecnico}{
	name=Gestor técnico del proyecto,
	description={Deben planificar, organizar y controlar a los profesionales que realizan el trabajo del software.},
}
\newglossaryentry{profesional}{
	name=Profesional,
	description={Proporcionan las capacidades técnicas para la ingeniería de un producto.},
}
\newglossaryentry{desarrollador}{
	name=Desarrollador,
	description={Construcción de prototipos y generación de plantillas. Colaboración en la elaboración de las pruebas funcionales y en las validaciones con el usuario.}
}
\newglossaryentry{ing_software}{
	name=Ingeniero de Software,
	description={Gestión de requisitos, gestión de configuración y cambios, preparación de las pruebas funcionales, elaboración de la documentación.}
}
\newglossaryentry{coordinador}{
	name=Coordinador,
	description={El coordinador asigna los recursos, gestiona las prioridades, establece las interacciones con los clientes y usuarios, y mantiene al equipo del proyecto enfocado en los objetivos. El coordinador también intenta asegurar la calidad de los artefactos del proyecto. Además, se encargará de supervisar el establecimiento de la arquitectura del sistema. Gestión de riesgos. Planificación y control del proyecto.}
}
\newglossaryentry{analista_sistemas}{
	name=Analista de Sistemas,
	description={Captura, especificación y validación de requisitos, interactuando con el cliente y los usuarios mediante reuniones. Elaboración del modelo de diseño. Colaboración en la elaboración de las pruebas funcionales. }
}

