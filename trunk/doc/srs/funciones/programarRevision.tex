% Revisado por Juanan el día 12/03/2013

\srsfuncion{Programar revisión}
	Esta función permite programar una revisión a un vehículo determinado.

\begin{enumerate}
	\item \textit{Prioridad}: alta.
	\item \textit{Entradas}
	\begin{enumerate}
		\item Fecha y hora que será coherente con el horario del personal seleccionado, personal, material y herramientas necesarias.
	\end{enumerate}
	\item \textit{Flujo de operaciones}
	\begin{enumerate}
		\item Se elige fecha y hora para la revisión. 
		\item Se selecciona el vehículo a reparar. Además, se rellena el resto del formulario indicando el material y las herramientas necesarias para la reparación, así como el motivo por el que necesita una reparación especificando qué se va a realizar en ella.
		\item Se muestra por pantalla el listado de personal mecánico disponible en esos momentos, ordenados por orden alfabético según el primer apellido. 
		\item Se eligen los trabajadores para que realice la tarea. Para ello se seleccionan los empleados de la lista.
		\item Cuando los datos se han modificado, se muestra confirmación detallada y se envía\break automáticamente notificación al personal afectado.
	\end{enumerate}
	\item \textit{Respuesta a situaciones no previstas}
	\begin{enumerate}
		\item Si no se puede acceder a la base de datos del personal: se muestra un mensaje de error por pantalla y se vuelve a la página principal del sistema.
		\item Si no se puede conectar con la base de datos para almacenar la información de la revisión: se muestra un mensaje de error por pantalla informando de que la revisión no ha podido darse de alta y se vuelve a la página principal del sistema.
		\item Si no existe ningún empleado disponible para la fecha y hora indicadas: Mostrar un mensaje indicando que la revisión no se puede programar por falta de personal disponible en esa fecha elegida. Dar la opción de modificar la fecha y hora elegidas.
	\end{enumerate}

\end{enumerate}
