
% Revisado por Juanan el día 12/03/2013

\srsfuncion{Realizar pago con tarjeta} \label{fun:pagotarjeta}
	Esta función permite al usuario finalizar el proceso de compra del billete realizando el pago con trajeta de crédito o débito.

\begin{enumerate}
	\item \textit{Prioridad}: alta.
	\item \textit{Entradas}
	\begin{enumerate}
		\item El nombre del usuario (nombre y apellidos) deberán contener únicamente carácteres alfabéticos latinos, acentuados o no, y espacios.
		\item El número de la tarjeta deberá ser una secuencia de 4 bloques de 4 dígitos, todos ellos enteros mayores o iguales que 0.
		\item El código CCV deberá ser una secuencia de 3 dígitos mayores o iguales que 0.
		\item La fecha de caducidad  ser una secuencia de 5 caracteres compuesta por: 2 dígitos para indicar el mes (en el rango 01-12) , una barra `/'  y otros 2 dígitos para indicar el año, que serán las dos últimas cifras del mismo.
	\end{enumerate}
	\item \textit{Flujo de operaciones}
	\begin{enumerate}
		\item Se muestra por pantalla una tabla con los datos de la tarjeta a completar (nombre y apellidos, número, código CCV y fecha de caducidad). Una vez completos, se habilita la opción \verb|Confirmar|.
		\item Se transfieren los datos de la tarjeta a la empresa emisora de las mismas para que compruebe si los datos son correctos y la tarjeta está operativa. En este caso, se enviará al instante un mensaje a la compañía indicando que los datos introducidos por el usuario son válidos.
		\item Si está todo correcto, se da la opción de imprimir en el momento la tarjeta de embarque o guardarla como pdf para imprimirla en otro momento.
	\end{enumerate}
	\item \textit{Respuesta a situaciones no previstas}
	\begin{enumerate}
		\item Si no se puede acceder a la base de datos para almacenar la información: se muestra un mensaje de error por pantalla informando de que el proceso de pago se ha interrumpido. Se vuelve a la página anterior.
		\item Si alguno de los datos no es válido: se muestran los campos erróneos y se da la opción de editarlos de nuevo.
		\item Si algún campo no se ha rellenado: se muestra un mensaje indicando que es obligatorio completarlo.
		\item Si la tarjeta está inhabilitada por algún motivo: se muestra un error indicando que el pago no ha podido completarse porque la tarjeta está bloqueada. Se cancela la operación y se vuelve a la página principal de la aplicación.
		\item Si los datos de la tarjeta no son válidos: se muestra un error indicando que el pago no ha podido completarse y se da la opción al usuario de modificar los datos introducidos al principio de la operación.
	\end{enumerate}
	\item \textit{Relación con otras funciones}\\
		Esta función está relacionado con \nameref{fun:consultaroferta}, \nameref{fun:mostrarofertas}, \nameref{fun:iniciarpago} y \nameref{fun:comprarbillete}.	
\end{enumerate}
