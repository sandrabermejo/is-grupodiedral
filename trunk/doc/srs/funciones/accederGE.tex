
% Revisado por Cristina el día 12/03/2013

\srsfuncion{Acceder}
	Función que permite hacer \textit{\gls{Login}} al usuario (en este caso cliente de la compañía aérea) para poder acceder al sistema.
		
	\begin{enumerate}
		\item \textit{Prioridad}: media.
		\item \textit{Entradas}
		\begin{enumerate}
			\item El nombre de usuario y la contraseña son campos obligatorios a introducir.
			\item En el campo contraseña son válidos los caracteres ASCII imprimibles.
		\end{enumerate}
		\item \textit{Flujo de operaciones}
		\begin{enumerate}
			\item Se muestran por pantalla dos campos a rellenar: uno para introducir el id del usuario y otro para escribir la contraseña.
			\item El usuario selecciona la opción de acceder al sistema.
		\end{enumerate}
		\item \textit{Respuesta a situaciones no previstas}
		\begin{enumerate}
			\item Si algún campo introducido no es válido, se indica y se da la opción de introducirlo de nuevo. Existe un límite de 5 intentos de acceso fallido en un periodo de tiempo corto (15 minutos).
		\end{enumerate}
	
\end{enumerate}
