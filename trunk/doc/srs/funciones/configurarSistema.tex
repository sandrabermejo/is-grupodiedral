
% Revisado por Cristina el día 13/03/2013

\srsfuncion{Configurar sistema general}  \label{fun:configSystem}
	Esta función permite modificar ciertos parámetros de la configuración general del sistema.

	\begin{enumerate}
		\item \textit{Prioridad}: media
		\item \textit{Usuarios}: personal administrativo o de los servicios informáticos debidamente autorizado.
		\item \textit{Entradas} \\

			Los parámetros configurables se restringirán a los indicados a continuación. Téngase en cuenta que algunos conllevan una considerables cambios difícilmente aplicables, son de uso totalmente infrecuente y requieren la supervisión activa del usuario.
			\begin{enumerate}
				\item Fecha y hora del servidor central. Su aplicación no es inmediata ya que requiere procesos de sincronía con los equipos conectados y probablemente el reinicio de ciertos componentes del sistema.
				\item Nombre de la compañía. El nombre que aparece en los informes generados por la aplicación, así como en las ventanas de su interfaz.
				\item Moneda. Tipo monetario en el que se expresan los importes comerciales y la información económica de la empresa. Su implantación no es inmediata pues puede requerir conversión de los datos en uso o históricos. La guía del usuario es necesaria para la aplicación de este cambio.
				\item Criterio de ordenación por defecto para cada caso de uso que presente información ordenada\footnote{\textit{Consultar ficha empleado}, \textit{Consultar ficha cliente}, \textit{Establecer organización laboral}\ldots}. Permite seleccionar entre los criterios de ordenación disponibles para cada caso de uso aquél que se tomará por omisión. De aplicación inmediata.
			\end{enumerate}

			Las entradas deben atenerse a lo siguiente:
			\begin{enumerate}
				\item No se pueden establecer configuraciones inválidas o que causen error en otras funciones.
				\item La configuración del sistema no puede quedar dañada por la modificación del empleado.
			\end{enumerate}

		\item \textit{Flujo de operaciones}
		\begin{enumerate}
			\item Se muestran los valores previos de los parámetros configurables.
			\item Se modifican los campos que se quieren reconfigurar en el sistema.
			\item El usuario, una vez satisfecho con los cambios realizados, confirma su modificación. Se comprueba la coherencia de los datos antes del envío. 
			\item Una vez que la configuración se establece con éxito, se muestra un mensaje por pantalla indicando las modificaciones de la configuración del sistema general.
			\item Los cambios que así lo requieran deben presentar al usuario pantallas para afinar el proceso de consolidación de los ajustes. Además si la implantación requiere un determinado tiempo para llevarse a cabo se monitoriza el proceso. Algunos cambios pueden requerir el reinicio del sistema o alguno de sus componentes para llegar a efecto.
		\end{enumerate}
		\item \textit{Respuesta a situaciones no previstas}
		\begin{enumerate}
			\item Si no se puede acceder a la base de datos del sistema para poder modificar así la configuración, se muestra un mensaje de error y se da la opción de reintentar o abortar el proceso..
			\item Si algún campo introducido no es válido, se indica y se da la opción de introducirlo de nuevo.
		\end{enumerate}

	\end{enumerate}
