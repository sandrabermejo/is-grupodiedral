
% Revisado por Cristina el día 13/03/2013

\srsfuncion{Registrar empleado} \label{fun:registrarempleado}
	Esta función permite añadir un nuevo empleado a la base de datos del sistema de la compañía aérea.

	\begin{enumerate}
		\item \textit{Prioridad}: alta.
		\item \textit{Entradas}
		\begin{enumerate}
			\item El nombre y apellidos del empleado deben contener únicamente carácteres alfabéticos latinos, acentuados o no, y espacios.
			\item El código de identificación personal debe cumplir los requisitos que se establecen en \nameref{srs:idpersonal}.
			\item La dirección debe cumplir los requisitos establecidos en \nameref{srs:direccionpostal}.
			\item El número de teléfono debe ser una secuencia de 9 dígitos.
			\item El puesto de trabajo debe seleccionarse de una lista finita fijada con anterioridad.
		\end{enumerate}
		\item \textit{Flujo de operaciones}
		\begin{enumerate}
			\item Se muestran por pantalla los datos personales del empleado que deben ser completados (nombre, apellidos, código de identificación peronal, dirección, teléfono, puesto de trabajo que ocupará dentro de la empresa y otros datos personales opcionales para completar el registro).
			\item El encargado del registro debe confirmar el alta.
			\item Se genera un nuevo nombre de usuario y una contraseña, que se asignan a la cuenta de correo interna del usuario.
			\item Se imprime una ficha con los datos de la cuenta para el empleado.
		\end{enumerate}
		\item \textit{Respuesta a situaciones no previstas}
		\begin{enumerate}
			\item Si algún campo introducido no es válido, se indica y se da la opción de introducirlo de nuevo.
			\item Si no se puede acceder o modificar a la base de datos, se muestra un mensaje de error y se da la opción de reintentar o abortar el proceso.
		\end{enumerate}

	\end{enumerate}
