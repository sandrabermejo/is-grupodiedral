\srsfuncion{Dar de baja empleado}
	Función que debe permitir dar de baja a un empleado, eliminando su información personal de acuerdo a la legislación vigente y derogando las autorizaciones de acceso al sistema y las instalaciones.
							
	\begin{enumerate}
		\item \textit{Prioridad}: alta.
		\item \textit{Entradas}
			\begin{enumerate}
				\item El usuario debe introducir en el campo de causa los motivos por los cuales se ha decidido llevar a cabo esta operación.
			\end{enumerate}
		\item \textit{Flujo de operaciones}
			\begin{enumerate}
				\item Una vez que el usuario ha pulsado el botón de \verb|Dar de baja| en la información detallada del empleado, rellenará de forma obliglatoria el campo de causa y confirmará que quiere realizar la operación dar de baja con el cliente seleccionado.
				\item El empleado dado de baja pierde inmediatamente todo acceso al sistema y su información personal se elimina de la base de datos de acuerdo a la legislación vigente.
				\item Por último, se registra la operación en la \textit{Cola de Supervisión}.
			\end{enumerate}
		\item \textit{Respuesta a situaciones no previstas}
			\begin{enumerate}
				\item Si no se puede establecer conexión con la base de datos: se muestra un mensaje de error y se da la opción de reintentar o abortar el proceso.
				\item Si no se puede registrar  la operación en la \textit{Cola de Supervisión} se anula la operación o se solicita al operador que informe a dicho departamento manualmente.
			\end{enumerate}
	\end{enumerate}
								
