
% Revisado por Cristina el día 12/03/2013

\srsfuncion{Editar cliente} \label{fun:editarcliente}
	Esta función permite la modificación de los datos de un cliente, ya sea por él mismo o por algún empleado de la compañía.

	\begin{enumerate}
		\item \textit{Prioridad}: alta.
		\item \textit{Entradas}
		\begin{enumerate}
			\item Las opciones que admiten modificación son: Contraseña, domicilio, dirección de correo electrónico, tarjeta de crédito asociada.
			\item La contraseña está formada por entre 8 y 16 caracteres alfanúmericos.
			\item El código postal del domicilio está formado por 5 cifras.
			\item Se debe comprobar la correción del correo electrónico mediante un e-mail de verificación.
			\item Se deben verificar algorítmicamente los 20 dígitos de numeración de la tarjeta de crédito, así como la validez de la misma junto con la fecha de caducidad y el código \gls{CVV2}.
		\end{enumerate}
		\item \textit{Flujo de operaciones}
		\begin{enumerate}
			\item Se muestra un formulario con los campos modificables actuales del usuario.
			\item El usuario modifica al menos uno de los diferentes campos.
			\item Si se verifican los datos modificados se confirma la operación, actualizando la \gls{base_de_datos}.
			\item Se muestra el perfil de usuario actualizado.
		\end{enumerate}
		\item \textit{Respuesta a situaciones no previstas}
		\begin{enumerate}
			\item Si algún campo introducido no es válido, se indica y se da la opción de introducirlo de nuevo.
			\item Si no se puede acceder a la base de datos, se muestra un mensaje de error y se da la opción de reintentar o abortar el proceso.
		\end{enumerate}
	\end{enumerate}
