% Revisado por Cristina el día 12/03/2013

\srsfuncion{Editar empleado} \label{fun:editarempleado}
	Esta función permite modificar la información de la ficha de un empleado.

	\begin{enumerate}
		\item \textit{Prioridad}: media.
		\item \textit{Entradas}
		\begin{enumerate}
			\item Las opciones que admiten modificación son: nombre, apellidos, foto, número de teléfono, NIF, número de seguridad social, contraseña de acceso, domicilio, dirección de correo electrónico y número de cuenta bancaria.
			\item La contraseña está formada por entre 8 y 16 caracteres alfanuméricos.
			\item El \gls{NIF} debe ser verificado algorítmicamente.
			\item La dirección debe cumplir los requisitos establecidos en \nameref{srs:direccionpostal}.
			\item Se verifica mediante el dígito de control la validez del número de cuenta.
		\end{enumerate}
		\item \textit{Flujo de operaciones}
		\begin{enumerate}
			\item El administrativo modifica al menos uno de los diferentes campos.
			\item El empleado confirma los cambios.
			\item Si se verifican los datos modificados se confirma la operación, actualizando la \gls{base_de_datos}.
			\item Se muestra la ficha de empleado actualizada.
		\end{enumerate}
		\item \textit{Respuesta a situaciones no previstas}
		\begin{enumerate}
			\item Si algún campo introducido no es válido, se indica y se da la opción de introducirlo de nuevo.
			\item Si no se puede acceder o modificar a la base de datos, se muestra un mensaje de error y se da la opción de reintentar o abortar el proceso.
		\end{enumerate}
	\end{enumerate}
