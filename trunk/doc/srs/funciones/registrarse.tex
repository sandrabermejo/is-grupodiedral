
% Revisado por Juanan el día 12/03/2013

\srsfuncion{Registrarse} \label{fun:registrarse}
	Esta función crea una nueva cuenta de cliente asociada a un visitante de la página web de la compañía.

	\begin{enumerate}
		\item \textit{Prioridad}: alta.
		\item \textit{Entradas}\\
			Será necesario introducir los siguiente datos: nombre y apellidos, código de identificación personal y una cuenta de correo electrónico. En cambio, la introducción de los siguientes datos es opcional: teléfono, dirección.

			\begin{enumerate}
				\item El código de idenficación personal será validado de acuerdo a las especificaciones de su formato.
				\item El nombre y los apellidos deberán contener únicamente carácteres alfabéticos latinos, acentuados o no, y espacios.
				\item La dirección de correo ha de seguir el formato \verb|nombre@dominio.com|. Se comprobará, en la medida de los posible, la autenticidad del servicio de correo introducido.
				\item El número de teléfono debe ser una secuencia de 9 dígitos. % Telefono español?
				\item La dirección se compondrá de un tipo de vía (calle, avenida, paseo), un nombre válido, un número de portal (entero mayor estricto que 0), escalera (izquierda, derecha, centro o nada), piso (entero mayor estricto que 0 o letra B) y puerta (letra o entero mayor estricto 0).
			\end{enumerate}
		
		\item \textit{Flujo de operaciones}
			\begin{enumerate}
				\item Se muestra un formulario con los distintos campos anteriormente detallados para que el usuario los rellene. 
				\item Se solicita además el paso de un \gls{captcha} por motivos de seguridad. Hasta que éste no sea superado no se permitirá continuar con el registro.
				\item Se muestran los derechos del usuario y las condiciones del servicio que deberán ser aceptadas, no pudiendo seguir con el registro hasta entonces.
				\item Cuando ha realizado correctamente todos los pasos anteriores debe \verb|Confirmar registro| y se procede al registro en la base de datos central.
			\end{enumerate}
		\item \textit{Respuesta a situaciones no previstas}
			\begin{enumerate}
				\item Si alguno de los datos no es válido o se ha dejado sin rellenar algún campo obligatorio se mostrará como erróneo y se dará la opción al usuario de modificarlo.
				\item Si no se puede registrar al usuario en la base de datos: informar al usuario e indicarle que puede volverlo a intentar en otro momento.
			\end{enumerate}
		\item \textit{Relación con otras funciones}\\
			Este caso es útil para todos aquellos que requieran un usuario registrado, como \verb|Acceder web| o aunque no lo requieran como \nameref{fun:iniciarpago}.
	\end{enumerate}
