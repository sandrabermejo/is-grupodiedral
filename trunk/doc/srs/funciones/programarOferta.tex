
% Revisado por Juanan el día 12/03/2013

\srsfuncion{Programar oferta} \label{fun:programaroferta}
	Esta función permite crear una oferta a promocionar que posteriormente se mostrará a los clientes y podrán disponer de ella.
	
\begin{enumerate}
	\item \textit{Prioridad}: alta.
	\item \textit{Entradas}
	\begin{enumerate}
		\item Se programan las ofertas que la empresa crea convenientes.
		\item Se expresarán en un lenguaje formal para facilitar posibles traducciones.
		\item El título y resumen de la oferta tienen un máximo de 60 y 200 caracteres respectivamente.
	\end{enumerate}
	\item \textit{Flujo de operaciones}
	\begin{enumerate}
		\item El empleado con el rol de programar las ofertas introduce los siguientes datos de la oferta:
		
			\begin{enumerate}
				\item Título
				\item Resumen
				\item Descripción detallada
				
					\begin{itemize}
						\item Artículo/s promocionado/s
						\item Precio anterior de cada artículo si la hubiese
						\item Precio de cada artículo con esta oferta
						\item Periodo de validez de la oferta
					\end{itemize}
				
			\end{enumerate}
		\item El programador valida la operación y la oferta se almacena en la base de datos del programa.
	\end{enumerate}
	\item \textit{Respuesta a situaciones no previstas}
	\begin{enumerate}
		\item En caso de no haber completado algún campo del formulario, marca los campos erróneos e indica una breve descripción indicando que no se han completado el formulario adecuadamente.
		\item Si al intentar validar la oferta esta no puede ser incluida en la base de datos, muestra un mensaje indicando el error por no poder ser añadida a la base de datos del sistema.
	\end{enumerate}
\end{enumerate}
