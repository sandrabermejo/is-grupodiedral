\srsfuncion{Programar oferta}
	Esta función debe permitir crear una oferta de un item a promocionar que posteriormente se mostrará a los clientes y podrán disponer de ella.
	
\begin{enumerate}
	\item \textit{Entradas}
	\begin{enumerate}
		\item Se programarán las ofertas que la empresa crea convenientes.
		\item Todas las palabras del lenguaje en el que se programe la oferta deberán estar en la \gls{RAE} para posteriormente poder traducirlas de un idioma a otro.
		\item El título y resumen de la oferta deberán tener un máximo de caracteres de 60 y 200 respectivamente.
	\end{enumerate}
	\item \textit{Flujo de operaciones}
	\begin{enumerate}
		\item El empleado con el rol de programar las ofertas introducirá los siguientes datos de la oferta:
		
			\begin{enumerate}
				\item Título
				\item Resumen
				\item Descripción detallada
				
					\begin{itemize}
						\item Artículo/s promocionado/s
						\item Precio anterior de cada artículo
						\item Precio de cada artículo con esta oferta
						\item Fecha límite de caducidad de la oferta
					\end{itemize}
				
			\end{enumerate}
		\item El programador validará pulsando el botón \verb|Validar| la oferta para introducirla en la base de datos del programa.
	\end{enumerate}
	\item \textit{Respuesta a situaciones no previstas}
	\begin{enumerate}
		\item En caso de no haber completado algún campo del formulario, marcar los campos erróneos e indicar una breve descripción indicando que no se han completado el formulario entero.
		\item Si al intentar validar la oferta esta no puede ser incluida en la base de datos, mostrar un mensaje indicando el error por no poder ser añadida a la base de datos del sistema. El programador podrá volver a pulsar el botón \verb|Validar| para intentar volver a ivalidar la oferta.
	\end{enumerate}
\end{enumerate}
