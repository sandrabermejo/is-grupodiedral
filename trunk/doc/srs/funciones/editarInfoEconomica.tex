
% Revisado por Cristina el día 12/03/2013

\srsfuncion{Editar información económica} \label{fun:editareconomica}
	Esta función permite añadir, editar o eliminar diferentes elementos patrimoniales pertenecientes a la gestión económica de la compañía aérea.

	\begin{enumerate}
		\item \textit{Prioridad}: media.
		\item \textit{Entradas}
		\begin{enumerate}
			\item Los nombres de los conceptos deberán contener únicamente carácteres alfabéticos latinos, acentuados o no, y espacios.
			\item Los datos tanto de los gastos como de los ingresos de la compañía aérea en el último ejercicio son números reales positivos, con dos decimales a lo sumo separados por una coma (se completará con \verb|0| los dos dígitos decimales en caso de no especificarse).
			\item Se utiliza el euro como unidad monetaria por defecto.
		\end{enumerate}
		\item \textit{Flujo de operaciones}
		\begin{enumerate}
			\item Se muestran la información económica de la compañía ordenada por defecto en orden alfabético por concepto. Se da la opción al usuario de ordenarlas según diferentes criterios (de mayor a menor importe y de mayor a menor porcentaje que suponen dentro de la masa patrimonial correspondiente).
			\item Se muestra el importe total de cada masa patrimonial así como el resultado del ejercicio, el cual se calcula como la resta de ingresos menos gastos.
			\item El personal administrativo puede añadir nuevos elemento a cada masa patrimonial. Al añadirlo, debe indicar un nombre y un importe válidos. Además, cada elemento patrimonial de la tabla podrá modificarse (debiendo introducirse nuevos datos válidos) y eliminarse de la tabla. Cada vez que se produzca una modificación en los datos, se actualiza el total de la tabla así como el resultado del ejercicio.
		\end{enumerate}
		\item \textit{Respuesta a situaciones no previstas}
		\begin{enumerate}
			\item Si no se puede conectar con la base de datos para obtener la información económica, se muestra un mensaje de error por pantalla y regresa a la página principal del sistema.
			\item Si no se puede conectar con la base de datos para almacenar la información, se muestra un mensaje de error por pantalla informando de que la información económica no ha podido actualizarse y se vuelve a la página principal del sistema.
			\item El algún dato introducido no es válido, se indica y se da la opción de editarlo de nuevo.
		\end{enumerate}
	\end{enumerate}
