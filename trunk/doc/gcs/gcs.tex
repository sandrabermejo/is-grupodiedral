%
%	Plan de Gestión de Configuración del Software (GCS)
%

\documentclass[11pt, a4paper, twoside, titlepage]{article}
\usepackage[utf8x]{inputenc}
\usepackage[T1]{fontenc}
\usepackage[spanish, es-ucroman]{babel}
\usepackage{lmodern}
\usepackage{anysize}
\usepackage{multicol}
\usepackage[none]{hyphenat}
\usepackage[colorlinks, linkcolor=red]{hyperref}
\usepackage{glossaries}
\usepackage{glossaries-babel}
\usepackage{tikz}
\usepackage{tikz-qtree}
\usepackage{float}
\usepackage[doc=gcs]{isdiedral}

% Nombre del documento (para futuras referencias)
% Obs: el nombre del encabezado no usa \doctitle
\newcommand*{\doctitle}{Plan de gestión de configuración del software}
\newcommand*{\docversion}{3.0}


%%% Configuraciones %%%
\marginsize{2.5cm}{2cm}{2cm}{2cm}

% Usa como familia tipográfica por defecto "Sans"
\renewcommand{\familydefault}{\sfdefault}

% Establece la profundidad hasta la cual se numeran los elementos de sección
\setcounter{secnumdepth}{4}

% Establece la profundidad de niveles de sección que aparece en el TOC
\setcounter{tocdepth}{4}

% Fija que la entrada del glosario no genere una sección
\renewcommand*{\glossarysection}[2][]{%
	\setlength\glsdescwidth{0.6\linewidth}%
	\glossarymark{Glosario}%
}

% Configuración de los encabezados
\encabezadodiedral{Plan GCS \docversion}
\pagestyle{fancy}

\renewcommand*{\thepage}{\sffamily \roman{page}}


% Modelo copiado de los apuntes del tema 8 (páginas 93 a 95) IEEE Std. 730-2002

\title{\doctitle\\\textsl{Airline Common Environment}}
\author{Grupo Diedral}

% Metadatos del pdf
\hypersetup{
pdfinfo={
	Author={Grupo Diedral},
	Title={\doctitle{} \docversion},
	Subject={Airline Common Environment},
	Keywords={SQA, Airline Common Environment, Ingeniería del Software}
}
}

% Inclusión del glosario (gracias a David Peñas)
%
%	Plan GCS: Glosario
%

\PrerenderUnicode{ñ}
\PrerenderUnicode{ó}
\PrerenderUnicode{í}


\makeglossaries

\begin{document}
	% Historial de cambios
	\begin{tablacambios}
		3.0 & 21 de marzo de 2013 & NAH, SBC, CAF, RRC & Tercera entrega
	\end{tablacambios}

	% Cita inicial
	\fijacitainicial{Nada es permanente a excepción del cambio}{Heráclito de Éfeso, $\sim$ 500 a.C.}

	% Portada
	\portadaace{\doctitle}{\docversion}

	\tableofcontents
	\newpage

	\iniciarnumeraciondiedral
	
	\section{Introducción} % Natalia
		\subsection{Propósito}
			El propósito del Plan de \gls{config_software} es establecer y mantener la integridad de los productos a desarrollar a través del proceso de software asociado. 
		\subsection{Alcance}
		\begin{itemize}
			\item Se \textit{identificarán y definirán los elementos del sistema}, controlando el posible cambio de estos durante todo su ciclo de vida, y se verificará que sean correctos y completos.
			\item Se establecerá un protocolo de \textit{gestión de cambios} de estos elementos.
			\item Será la base de partida para que asegurar una buena \textit{calidad de software}.
			\item Ayudará a la comunicación y organización en el grupo de desarrollo o con personas ajenas al proyecto.
		\end{itemize}
			
		%\subsection{Definición de términos clave} Esto es el Glosario
		\subsection{Referencia}
			\nocite{IEEE828-1998}
			\nocite{PSMAN}
			\nocite{ESAPSS-05-0}
			Ver sección de {\itshape Referencias} al final del documento.
		
	\section{Gestión de la GCS} % Sandra
		\subsection{Organización}
			En este apartado describimos el contexto organizativo en el cual se aplicará la gestión de configuración de software. Para llevar a cabo dicha gestión hemos definido los siguientes roles:

			\begin{itemize}
				\item Coordinador de Proyecto: Rubén Rafael Rubio Cuéllar.
				\item Responsables de la Configuración: Natalia Angulo Herrera, Sandra Bermejo Cañadas.
				\item Ingenieros: Cristina Alonso Fernández, Juan Andrés Claramunt Pérez.
			\end{itemize}

			Si la situación lo requiriese, se podrían realizar cambios en la estructura descrita con el objetivo de asegurar que todos los productos a desarrollar y sus cambios sean debidamente justificados y de una calidad conveniente.
		\subsection{Responsabilidades GCS}
			A continuación, se establecen las tareas asignadas a cada rol, garantizando así que todas las actividades esenciales se lleven a cabo y además que las realice sólo un integrante del equipo.
			\begin{itemize}
				\item Coordinador de Proyecto
					\begin{itemize}
						\item Controlar la gestión de configuración.
						\item Informar al equipo de desarrollo acerca de los planes de gestión de configuración de software.
						\item Asegurar que los documentos se generen de acuerdo con las políticas establecidas en la siguiente sección.
					\end{itemize}
				\item Responsables de la Configuración
					\begin{itemize}
						\item Organizar y realizar la gestión de configuración.
						\item Mantener el documento de \textit{Gestión de Configuración de Software}.
						\item Realizar informes (si procede).
					\end{itemize}
				\item Ingenieros
					\begin{itemize}
						\item Decidir si un cambio se lleva a cabo y evaluarlo.
						\item Seguir los procedimientos de gestión de configuración que publiquen los Responsables de la Configuración.
						\item Almacenar los cambios realizados.
					\end{itemize}
			\end{itemize}

		\subsection{Políticas, directivas y procedimientos aplicables}
			Se llevarán a cabo las políticas establecidas en el estándar \textit{ESA PSS-05-0} \cite{ESAPSS-05-0} que tengan relación con la gestión de configuración de software. \\

			Además, se incluyen los procedimientos definidos por el coordinador del proyecto y los responsables de la gestión de configuración.

	\section{Actividades de la GCS}
		\subsection{Identificación de la configuración} % Natalia
			En la siguiente sección se va a realizar la identificación, nombrado y adquisición de \gls{ECS}.

			\subsubsection{Casos de Uso}
				\begin{enumerate}
					\item {\itshape \bfseries Descripción.}
						\begin{itemize}
							\item \textit{Tipo de ECS:} documento.
							\item \textit{Identificador proyecto:} \verb|CU|.
							\item \textit{Información de la versión y/o cambio:} es el documento más antiguo del proyecto y ha sufrido más cambios que cualquier otro. Paulatinamente se han ido añadiendo nuevos casos de uso al comprender su necesidad, se han refinado los existentes y se han corregido algunos errores en la exposición de los casos.
						\end{itemize}

					\item {\itshape \bfseries Lista de recursos.}
						El principal recurso de este documento es la información recabada en las conversaciones que el equipo de desarrollo ha venido manteniendo con el cliente y los distintos usuarios. Siguiendo el modelo de casos de uso, las etapas siguientes del proyecto tendrán su base en este documento. Este es el motivo por el que se ha hecho un gran esfuerzo en la gestión de configuración y de calidad del mismo.

				\end{enumerate}

			\subsubsection{Especificación de requisitos \software}
				\begin{enumerate}
					\item {\itshape \bfseries Descripción.}
						\begin{itemize}
							\item \textit{Tipo de ECS:} documento.
							\item \textit{Identificador proyecto:} \verb|SRS|.
							\item \textit{Información de la versión y/o cambio:} los cambios en los casos de uso (documento CU) se reflejado también en este documento. Además, se han corregido errores de redacción y conflictos entre los diferentes epígrafes del documento.
						\end{itemize}

					\item {\itshape \bfseries Lista de recursos.}	
						Principalmente el documento de la {\itshape Especificación de Requisitos Software} tiene como entidades requeridas al documento de {\itshape Casos de Uso} y el {\itshape Prototipo}, es decir, depende directamente de ellos.
				\end{enumerate}

			\subsubsection{Prototipo Gestión Externa}
			\begin{enumerate}
				\item {\itshape \bfseries Descripción.}
						\begin{itemize}
							\item \textit{Tipo de ECS:} programa.
							\item \textit{Identificador proyecto:} \verb|Prototipo Externa|.
							\item \textit{Información de la versión y/o cambio:} no ha habido cambios desde la primera versión entregada.
						\end{itemize}

					\item {\itshape \bfseries Lista de recursos.}
						La principal fuente de referencia del \textit{Prototipo de Gestión Externa} es la \textit{Especificación de Requisitos Software} puesto el prototipo pretende ilustrar las especificaciones recogidas en dicho documento.
				\end{enumerate}

			\subsubsection{Prototipo Gestión Interna}
			\begin{enumerate}
				\item {\itshape \bfseries Descripción.}
						\begin{itemize}
							\item \textit{Tipo de ECS:} programa.
							\item \textit{Identificador proyecto:} \verb|Prototipo Interna|.
							\item \textit{Información de la versión y/o cambio:} modificaciones menores en la visualización del prototipo y cambios en la estructura del código de éste. No se han realizado grandes cambios desde la última entrega por la proximidad que tiene para ser desechado y reemplazado por la implementación final del producto.
						\end{itemize}

					\item {\itshape \bfseries Lista de recursos.}
						\textit{(Ver lista de recursos de Prototipo Gestión Externa)}
				\end{enumerate}

			\subsubsection{Plan de Proyecto}
			\begin{enumerate}
				\item {\itshape \bfseries Descripción.}
						\begin{itemize}
							\item \textit{Tipo de ECS:} documento.
							\item \textit{Identificador proyecto:} \verb|PR|.
							\item \textit{Información de la versión y/o cambio:} se están acometiendo diferentes cambios referentes a la \textit{Estimación de Puntos de Función}. En la última versión entregada se omitió por error el cálculo de los PF totales, lo cual motivó un cambio para recoger tal apartado.
						\end{itemize}

					\item {\itshape \bfseries Lista de recursos.}
						El Plan de Proyecto está compuesto por los siguientes subrecursos internos: el \textit{Plan de Gestión de Riesgo}, la \textit{Estimación} y el \textit{Documento de Planificación temporal}.
				\end{enumerate}

			\subsubsection{Plan de Gestión de Configuración}
			\begin{enumerate}
				\item {\itshape \bfseries Descripción.}
						\begin{itemize}
							\item \textit{Tipo de ECS:} documento.
							\item \textit{Identificador proyecto:} \verb|GCS|.
							\item \textit{Información de la versión y/o cambio:} existe una primera y única versión del documento.
						\end{itemize}

					\item {\itshape \bfseries Lista de recursos.}
						Este documento está ligado al \textit{Plan de Calidad}, pues ambos tiene cometidos comunes relacionados con la supervisión de los diferentes elementos del proyecto.
				\end{enumerate}

			\subsubsection{Plan de Gestión de Calidad}
			\begin{enumerate}
				\item {\itshape \bfseries Descripción.}
						\begin{itemize}
							\item \textit{Tipo de ECS:} documento.
							\item \textit{Identificador proyecto:} \verb|PGC|.
							\item \textit{Información de la versión y/o cambio:} la primera versión de este documento incluye el desarrollo de la política de supervisión y corrección de los elementos del proyecto. Extiende el contenido, que sobre calidad del \software, incluía el {\itshape Plan del Proyecto}. Estará sujeto a continua revisión durante el transcurso del proyecto.
						\end{itemize}

					\item {\itshape \bfseries Lista de recursos.}
						Este documento está ligado al \textit{Plan de Gestión de Configuración} como se afirma en la descripción de éste.
				\end{enumerate}

			\subsubsection{Otros}
				Algunos ECS importantes que no se han desarrollado aún y que por tanto no se van a detallar en este documento son: diagramas de colaboración de análisis, el documento de diseño, el manual de usuario, el plan de pruebas del sistema, documentos de diseño de base de datos, especificaciones de prueba del sistema\ldots
	% ¿Por qué hablamos de lo que no vamos a hacer?
	%N: Es que sino no sabía que poner en Otros, si no borramos este punto!

		\subsection{Contabilidad de estado de configuración} % Sandra
			El objetivo de la contabilidad del estado de configuración es registrar la configuración de las líneas de referencia y todos los cambios que se produzcan dede ese momento. Por tanto, la contabilidad se debe iniciar desde que se generen los primeros datos de la configuración y continuar a lo largo de proyecto. \\			

			La contabilidad del estado de configuración se llevará a cabo cada dos semanas (un día establecido por todos los miembros del equipo) para lograr un seguimiento del avance del trabajo de cada integrante del Grupo Diedral.
			
			\begin{itemize}
				\item \textit{Registros a mantener}

				El cliente se asegurará de que se mantengan los registros de la contabilidad. El registro debería incluir, entre otros
					\begin{itemize}
					\item El nombre del cliente y del equipo de desarrollo 
					\item Los ECS. Para cada ECS, el registro debe contener el número de ECS afectados y especificaciones como la fecha de revisión y publicación.
					\end{itemize}

				\item \textit{Informes a generar}

				Al alcanzar una Línea Base, se generará un informe de los elementos que la componen, detallando los autores y versiones que correspondan. Además, para cada elemento se suministrará la siguiente información: nombre y tipo del elemento y cambios pedidos (descripción y estado). En caso de encontrar algún error, se registrarán el tipo de error y la acción correctiva.

				\item \textit{Procedimientos de captura, almacenamiento y procesamiento de información}

				Por cada solicitud de cambio, se especifca el elemento cuestionado, con una breve descripción, la fecha y quien lo solicita. Una vez realizada esta solicitud, el equipo de desarrollo evaluará su impacto sobre el proyecto. Según estas discusiones, el cambio podrá ser aprobado o rechazado y se llevarán a cabo las modificaciones que se consideren convenientes. 

			\end{itemize}


	\section{Recursos de la GCS} % Natalia

		\subparagraph{Herramientas técnicas y \software.} La principal herramienta en la que se va a basar la \textit{Gestión de Configuración del Software} será en el repositorio de control de versiones que sirve de entorno de trabajo a los desarrolladores. El servidor que anteriormente albergaba el repositorio --\textit{RiouxSVN}-- quedó inoperativo, circunstancia prevista en el \textit{Plan de Gestión de Riesgo}\footnote{Véase riesgo T2 `\textit{Repositorio inoperativo}' en el \textit{Plan de Proyecto}.} recién elaborado, por lo que se tomaron las acciones de contingencia necesarias, lo que no permitió salvar el historial de versiones como preveía el plan de prevención. A pesar de ello, no han surgido grandes conflictos de documentos. Así, la organización del nuevo repositorio hospedado en el servidor \textit{Assembla} ha quedado de la siguiente manera: \\
			
			\begin{figure}[H] \centering
				\tikzset{edge from parent/.style={draw, edge from parent path={(\tikzparentnode.south) -- +(0,-8pt) -| (\tikzchildnode)}}}
				\tikzset{sibling distance=20pt}
				\Tree [.{\itshape Repositorio} [.trunk doc presentacion actas util ].trunk [.branches prototipo\_interna prototipo\_externa ].branches ].{\itshape Repositorio}

			\caption{Árbol de estructura del repositorio}
			\end{figure}

			El repositorio cuenta con dos directorios en el directorio raíz \verb|trunk| y \verb|branches|. El directorio \verb|trunk| a su vez se organiza en 4 directorios: \verb|doc| (que alberga toda la documentación del proyecto), \verb|presentacion| (que contiene los documentos elaborados para las exposiciones en clase), \verb|actas| (que incluye las actas generadas en las diferentes actividades del proyecto) y \verb|util| (que recoge herramientas elaboradas internamente como ayuda para el desarrollo del proyecto). En el directorio \verb|branches| residen los prototipos.

			\begin{figure}[H] \centering
				\tikzset{edge from parent/.style={draw, edge from parent path={(\tikzparentnode.south) -- +(0,-8pt) -| (\tikzchildnode)}}}
				\tikzset{sibling distance=15pt}
				\Tree [.doc casosdeuso diagramas plancalidad srs common gcs planproyecto ].doc

			\caption{Árbol de estructura de la documentación}
			\end{figure}

	\section{Glosario}
		\printglossaries

	\newpage
	\part*{Apéndices}
		\appendix
		\section{Actividad del repositorio}
			\tiny

% Usa como impresor de fechas el del paquete "isdiedral"
% Otra opción es "isodate" (pero no soporta español aún)
\let \printdate = \imprimefecha

% Comando de uso particular para imprimir revisiones
\newcommand*{\reporev}[3]{\item #1 (#2, \printdate{#3}).}

\begin{multicols}{2}
\begin{enumerate}
	\renewcommand*{\theenumi}{\bfseries r\arabic{enumi}}

	\reporev{Creación de la estructura general del repositorio}{Automático}{22.11.2012}
	\reporev{Primera subida de archivos}{NAH}{22.11.2012}
	\reporev{Añadido diagrama de gestión interna y documentación del paquete isdiedral}{RRC}{24.11.2012}
	\reporev{Añadidos 3 casos de uso del primer reparto}{NAH}{24.11.2012}
	\reporev{Borrador de 2 casos de uso. Inclusión de los casos de uso en el documento}{RRC}{25.11.2012}
	\reporev{Subidos casos de uso: crear nómina y programar revisión, y actualizado fichero casosdeuso}{JCP}{25.11.2012}
	\reporev{Dos casos de uso Gestión Externa (presentar reclamaciones y realizar pago billetes)}{NAH}{28.11.2012}
	\reporev{Solución temporal a las tablas multipágina}{RRC}{28.11.2012}
	\reporev{Cambios técnicos: separación de secuencias, índice y desactivación de separación de palabras al final de línea}{RRC}{2.12.2012}
	\reporev{Nuevo caso de uso: consulta vuelos}{RRC}{2.12.2012} % r10
	\reporev{Más casos de uso}{SBC}{7.12.2012}
	\reporev{Proto-prototipo del sistema de gestión interna}{RRC}{10.12.2012}
	\reporev{Casos de uso: comprar billete y mostrar ofertas de GE; consultar nómina y consultar plan de vuelo de GI}{CAF}{10.12.2012}
	\reporev{Subida tabla ver info vuelo}{JCP}{10.12.2012}
	\reporev{Modificados casos de uso: consultar inventario, modificar items inventario, consultar plan de vuelo y entrada material}{NAH}{11.12.2012}
	\reporev{Modificados casos de uso que hice y he vuelto a subir los diagramas de gestión interna pues no los había subido correctamente}{NAH}{12.12.2012}
	\reporev{Nuevos casos de uso de gestión externa}{NAH}{12.12.2012}
	\reporev{Vuelvo a subir el diagrama gestión interna añadido un caso de uso que se nos había olvidado (crear nómina). Hecho el diagrama de gestión externa. Dividido el caso de uso de realizar pago en tres partes}{NAH}{12.12.2012}
	\reporev{Corregidos errores LaTeX en casos de uso. Esquema de SRS. Esquema de diagramas en LaTeX}{RRC}{12.12.2012}
	\reporev{Cambios en el prototipo GI. Prototipo GE}{RRC}{12.12.2012} % r20
	\reporev{Orden alfabético en casos de uso. Arreglados errores}{RRC}{12.12.2012}
	\reporev{Casos de uso (comprar billete, mostrar ofertas, consultar nómina y consultar plan de vuelo) modificados}{CAF}{12.12.2012}
	\reporev{Modificado el diagrama añadiendo el nuevo caso de uso consultar ficha empleado. Incluido en fichero casos de uso}{NAH}{13.12.2012}
	\reporev{Corregido error de sintaxis en consultar plan de vuelo}{NAH}{13.12.2012}
	\reporev{Modificacion casos de uso: accederGE, accederGI, registrarse y ModificarItemsInventario}{SBC}{13.12.2012}
	\reporev{{}[QPrototipo] Añadida pantalla de inicio}{RRC}{13.12.2012}
	\reporev{Cambios en laboratorio. Comienzo de SRS}{todos [NAH]}{13.12.2012}
	\reporev{Revisión de casos de uso, pequeños cambios}{JCP}{14.12.2012}
	\reporev{Revisados y añadidos casos de uso. Algo de las Srs}{RRC}{14.12.2012}
	\reporev{{}[SRS] Añadido soporte para bibliografía con BibTex, añadido srsfuncion (comando de sección de nivel 4), control de versiones e intento de espeficación. [Casos de uso] Algún cambio}{RRC}{15.12.2012}
	\reporev{{}[Casos de uso] Revisión de los que me correspondían (bastantes cambios)}{RRC}{15.12.2012} % r30
	\reporev{Nuevos casos de uso: dar de baja cliente y configurar sistema general}{NAH}{15.12.2012}
	\reporev{Revisados casos de uso: comprarBillete, consultarInventario, consultarFichaEmpleado, configurarNomina; y añadidos:\break introducirPlanDeVuelo, efectuarEmbarque}{SBC}{15.12.2012}
	\reporev{Casos de uso revisados: editarInformacionEconomica,\break IniciarPagoBilletesDeVuelo, ModificarItemsInventario, programarHorarios, realizarPagoconTarjeta, registrarEmpleado, verificarRegistroEmpleado. He eliminado pago con PayPal pues no es frecuente en compra de billetes de vuelo}{CAF}{15.12.2012}
	\reporev{Cambio de nombres [erróneo]}{CAF}{15.12.2012}
	\reporev{Revisiones y añadidos dos casos de uso}{JCP}{15.12.2012}
	\reporev{Renombrados ya que causaban conflicto [ver r35]}{JCP}{15.12.2012}
	\reporev{{}[SRS] Modificada la función Ver plan de vuelo y completados otros epígrafes}{RRC}{13.12.2012}
	\reporev{{}[SRS] funciones del producto}{JCP}{15.12.2012}
	\reporev{SRS de mis casos de uso (llevan el nombre del caso de uso al que corresponden con un SRS delante).}{CAF}{16.12.2012} % r40
	\reporev{{}[SRS] añadido editar cliente en funciones de gestión interna, añadida sección atributos del sistema de software y completada sección alcance}{JCP}{16.12.2012}
	\reporev{{}[SRS] añadidos: consultar nómina, acceder gestión externa, editar empleado, acceder horarios, presentar reclamaciones, consultar vuelos}{JCP}{16.12.2012}
	\reporev{He añadido las funciones de los SRS. Además he modificado también con los campos que me tocaban realizar la SRS (documento general). También he subido los casos de uso revisados}{NAH}{14.12.2012}
	\reporev{{}[SRS] Unidas las diferentes modificaciones manualmente (si alguien nota que falta algo...). Extraídas todas las funciones del archivo principal. Hay 2 iniciar sesión interno y ninguno externo. El significado de registrar entrada material no es adecuado}{RRC}{16.12.2012}
	\reporev{Renombrado un archivo}{NAH}{16.12.2012}
	\reporev{Cambios derivados del renombramiento anterior en el documento general}{NAH}{16.12.2012}
	\reporev{Eliminado el caso de uso pago con PayPal}{NAH}{16.12.2012}
	\reporev{Cambios en registrar entradas de material}{NAH}{16.12.2012}
	\reporev{Diagramas de casos de uso con argoUML}{todos (NAH)}{16.12.2012}
	\reporev{Revisión de las SRS e inclusiones en el glosario}{NAH}{16.12.2012} % r50
	\reporev{Introducido el generador de plantillas para pantallas}{RRC}{16.12.2012}
	\reporev{{}[Prototipo Interno] Añadidas viejas pantallas olvidadas, ultraesqueleto de FListado y recursos (iconos Tango)}{RRC}{16.12.2012}
	\reporev{Modifico la parte de configurar sistema en los srs.tex y el glosario por algunas modificaciones realizadas ahí}{NAH}{16.12.2012}
	\reporev{Modificación del commit anterior [r52]}{NAH}{16.12.2012}
	\reporev{Movidas las pantallas a la carpeta pantallas. Ahora el prototipo NO compila temporalente}{RRC}{16.12.2012}
	\reporev{Cambiado ref por nameref. Líos varios}{RRC}{16.12.2012}
	\reporev{Pantallas: dar de baja cliente, programar revisión y registrar entrada material; glosario y SRS; casos de uso: mostrar ofertas, programar oferta y acceder a una oferta (SRS y diagramas)}{NAH}{17.12.2012}
	\reporev{+1 pantallas Programaroferta subida}{NAH}{17.12.2012}
	\reporev{Pantalla Editar Empleado}{JAP}{17.12.2012}
	\reporev{Pantallas casos de uso de Gestión Interna (modificarItemsInventario, registrarEmpleado, verificarRegistroEmpleado, programarHorarios y editarInformacionEconomica)}{CAF}{17.12.2012} % r60
	\reporev{Pequeñas modificaciones}{CAF}{17.12.2012}
	\reporev{Pequeñas modificaciones en SRS de casos uso}{CAF}{17.12.2012}
	\reporev{Pantalla ConsultaNomina}{JAP}{17.12.2012}
	\reporev{Modificaciones en los casos de uso}{CAF}{17.12.2012}
	\reporev{Añadidas pantallas: consultar inventario, consultar ficha empleado, dar de baja empleado, configurar nómina, introducir plan de vuelo y efectuar embarque}{SBC}{17.12.2012}
	\reporev{Añado también el formato imagen de las pantallas}{SBC}{17.12.2012}
	\reporev{Especificación de requisitos de mis casos de uso}{SBC}{17.12.2012}
	\reporev{Implementación básica del prototipo. Ya no subáis más pantallas}{RRC}{18.12.2012}
	\reporev{Añadidas algunas imágenes al documento srs}{NAH}{18.12.2012}
	\reporev{Prototipo web}{NAH}{18.12.2012} % r70
	\reporev{Arreglados errores en las SRS}{RRC}{18.12.2012}
	\reporev{Añadida presentación-exposición}{RRC}{18.12.2012}
	\reporev{Mejoradas (o no) algunas pantallas del prototipo}{RRC}{18.12.2012}
	\reporev{Quitado comentario ya modificado}{NAH}{19.12.2012}
	\reporev{Subo el prototipo web con todo más o menos hecho}{NAH}{19.12.2012}
	\reporev{Resubo Realizar pago final que se había colado una versión antigua}{NAH}{19.12.2012}
	\reporev{Subo el SRS compilado con todos los archivos que genera a petición de Juan Andrés}{NAH}{19.12.2012}
	\reporev{He añadido a la SRS: resumen, supuestos y dependencias, requisitos de rendimiento}{SBC}{19.12.2012}
	\reporev{Añadidos SRS finales}{CAF}{19.12.2012}
	\reporev{Eliminados archivos innecesarios y altamente conflictivos}{RRC}{19.12.2012} % r80
	\reporev{Mejorada considerablemente la apariencia de los diagramas de la presentación. Logo transparente}{RRC}{19.12.2012}
	\reporev{{}[Casos de uso] Diagramas añadidos al fichero casos de uso}{RRC}{19.12.2012}
	\reporev{Cambios en el prototipo web}{NAH}{19.12.2012}
	\reporev{Ligera modificación en la web (logo sin fondo) y tabla de vuelos de modificadas las terminales}{NAH}{19.12.2012}
	\reporev{Error al subirlo. Esta es la versión buena.}{NAH}{19.12.2012}
	\reporev{Eliminada copia de logotipo duplicada}{NAH}{19.12.2012}
	\reporev{Añadidas palabras al glosario}{NAH}{19.12.2012}
	\reporev{Pequeña correción}{JAP}{19.12.2012}
	\reporev{Añadida función perdida de procedencia desconocida}{RRC}{19.12.2012}
	\reporev{Cambios de estilo en el prototipo interno}{RRC}{19.12.2012} % r90
	\reporev{Otra subida del prototipo de gestión externa}{NAH}{19.12.2012}
	\reporev{Incluidos todos los casos de uso}{NAH}{20.12.2012}
	\reporev{Corregidos errores e incluidos casos de uso}{RRC}{20.12.2012}
	\reporev{Arreglados los conflictos con el commit de Natalia. Salen 35 casos de uso, antes salían 34. Latex dice que no hay repes}{RRC}{20.12.2012}
	\reporev{Actualizado el árbol de directorios para aislar los casos de uso}{RRC}{21.12.2012}
	\reporev{Modelo de plan de proyecto y tabla de descripción de riesgo}{RRC}{24.12.2012}
	\reporev{Subo los casos de uso revisados de Personal de Recursos Humanos y Directivos.}{NAH}{25.12.2012}
	\reporev{Ligeras modificaciones en casos de uso del Personal de planificación de operaciones y del Personal de atención al cliente}{SBC}{26.12.2012}
	\reporev{Completando el commit anterior.}{SBC}{26.12.2012}
	\reporev{Revisión de SRS: acceder, configurarNomina, consultarFichaEmpleado, consultarHorarios, consultarNomina, darDeBajaEmpleado,\break obtenerInfoEconomica, registrarEmpleado, verificarRegistroEmpleado. He añadido a cada SRS un nuevo item como hizo Rubén que indica la prioridad de la función}{NAH}{27.12.2012} % r100
	\reporev{Revisado en srs: Resumen, Características del usuario (modificado algo muy mínimo) y supuestos y dependencias (he añadido un párrafo)}{NAH}{27.12.2012}
	\reporev{Solucionados algunos errores en los documentos (ausencia de corchetes y presencia de comas)}{RRC}{28.12.2012}
	\reporev{{}[Prototipo] Muchos cambios, muchos más archivos, pocas mejoras}{RRC}{30.12.2012}
	\reporev{{}[Prototipo] Arreglados ciertos errores y algunas mejoras}{RRC}{31.12.2012}
	\reporev{Cambios en los apartados de las SRS que me tocaron. Ortografía. Añadidos metadatos PDF}{RRC}{03.1.2013}
	\reporev{Eliminado directorio `prototipo web dreamweaver' porque es copia de `prototipo\_externa'}{RRC}{3.1.2013}
	\reporev{Revisión de casos de uso y sus SRS correspondientes. Añadido caso de uso realizar mantenimiento y dividido el antiguo caso de uso efectuar embarque en: facturar y realizar embarque}{CAF}{4.1.2013}
	\reporev{Corregidos errores LaTeX. Cambios en los casos de uso. Nuevo caso de uso `incidencias sistema'}{RRC}{4.1.2013}
	\reporev{Realizar mantenimiento actualizado, sin que redirija a Modificar inventario}{CAF}{4.1.2013}
	\reporev{Añadidas funciones Realizar mantenimiento y Facturar, así como pequeños cambios en SRS: Perspectiva del producto, Supuestos y dependencias}{CAF}{4.1.2013} % r110
	\reporev{{}[Prototipo] Corregidos algunos graves errores y añadida cierta funcionalidad aparente con persistencia}{RRC}{6.1.2013}
	\reporev{Cambios en la SRS y el plan de proyecto}{RRC}{10.1.2013}
	\reporev{Cambios en los documentos para incluir encabezados}{RRC}{10.1.2013}
	\reporev{Añadido un cuadro de riesgo (Pr) y especificación del formato de fecha (SRS)}{RRC}{23.1.2013}
	\reporev{Reorganización de archivos y portadas}{RRC}{16.2.2013}
	\reporev{Índice de funciones en SRS y otros cambios}{RRC}{16.2.2013}
	\reporev{Añadida estructura del programa de gestión interna}{RRC}{18.2.2013}
	\reporev{Subido riesgo Inactividad por periodo examenes. Duda en la parte de autor/departamento que realiza el riesgo}{NAH}{19.2.2013}
	\reporev{Repito lo anterior, que no se ha subido.}{NAH}{19.2.2013}
	\reporev{Un nuevo riesgo: softwarenoreutilizable}{NAH}{19.2.2013} % r120
	\reporev{Corregidos algunos fallos}{NAH}{19.2.2013} % r120	
	\reporev{Última tabla de riesgo: faltacomunicacioncliente. Comentamos algunas dudas mañana}{NAH}{19.2.2013}
	\reporev{Cambio mínimo de acuerdo a la correción}{RRC}{19.2.2013}
	\reporev{planproyecto modificado. Hay cosas que hay que revisar mañana}{NAH}{20.2.2013}
	\reporev{3 riesgos añadidos. Cambios en obsolescencia de tecnologias utilizadas en: - Impacto de N a M - Probabilidad de M a B}{CAF}{20.2.2013}
	\reporev{Plan de proyecto actualizado}{CAF}{20.2.2013}
	\reporev{Subidos otros 3 riesgos y añadidos en planproyecto}{SBC}{20.2.2013}
	\reporev{Riesgos}{JCP}{20.2.2013}
	\reporev{Corregidos errores de sintaxis. Abreviados nombres de archivo. Anexo de control de versiones. Añadida plantilla para la presentación}{RRC}{20.2.2013}
	\reporev{Algunos cambios por correciones}{RRC}{21.2.2013} % r130
	\reporev{Tabla de enumeración de riesgos. Algunos cambios más}{RRC}{21.2.2013} 
	\reporev{Cambiada la tabla de descripción de riesgos}{RRC}{21.2.2013}
	\reporev{Modificaciones en hoja de riesgos}{RRC}{21.2.2013}
	\reporev{Nuevos riesgos}{RRC}{21.2.2013}
	\reporev{Riesgo añadido: ensamblar}{JCP}{21.2.2013}
	\reporev{Lista de materiales y otros cambios}{RRC}{21.2.2013}
	\reporev{Pequeñas modificaciones en riesgos y añadido nuevo riesgo desconocimientotecnologias}{CAF}{22.2.2013}
	\reporev{Plan de proyecto. Añadidas declaración y funciones principales en Ámbito y estructura del equipo en Organización personal}{CAF}{22.2.2013}
	\reporev{Añadidos archivos de Cocomo y Project}{RRC}{22.2.2013}
	\reporev{id riesgos modificados}{NAH}{22.2.2013} % r140
	\item[$\star$] {\itshape Actualizado a 23 de febrero de 2013}


\end{enumerate}
\end{multicols}

\normalsize

% fin


	\newpage
	\bibliography{gcs}
	\bibliographystyle{plain}
\end{document}
