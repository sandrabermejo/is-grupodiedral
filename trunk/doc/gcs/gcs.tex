%
%	Plan de Gestión de Configuración del Software (GCS)
%

\documentclass[11pt, a4paper, twoside, titlepage]{article}
\usepackage[utf8x]{inputenc}
\usepackage[T1]{fontenc}
\usepackage[spanish]{babel}
\usepackage{lmodern}
\usepackage{anysize}
\usepackage[none]{hyphenat}
\usepackage[colorlinks, linkcolor=red]{hyperref}
\usepackage{glossaries}
\usepackage{glossaries-babel}
\usepackage{isdiedral}

% Nombre del documento (para futuras referencias)
\newcommand*{\doctitle}{Plan de gestión de configuración del software}


%%% Configuraciones %%%
\marginsize{2.5cm}{2cm}{2cm}{2cm}

% Usa como familia tipográfica por defecto "Sans"
\renewcommand{\familydefault}{\sfdefault}

% Establece la profundidad hasta la cual se numeran los elementos de sección
\setcounter{secnumdepth}{4}

% Establece la profundidad de niveles de sección que aparece en el TOC
\setcounter{tocdepth}{4}

% Fija que la entrada del glosario se comporte como una subsección
\setglossarysection{subsection}

% Configuración de los encabezados
\encabezadodiedral{\doctitle}
\pagestyle{fancy}

\renewcommand*{\thepage}{\sffamily \roman{page}}


% Modelo copiado de los apuntes del tema 8 (páginas 93 a 95) IEEE Std. 730-2002

\title{\doctitle\\\textsl{Airline Common Environment}}
\author{Grupo Diedral}

% Metadatos del pdf
\hypersetup{
pdfinfo={
	Author={Grupo Diedral},
	Title={\doctitle},
	Subject={Airline Common Environment},
	Keywords={SQA;Airline Common Environment;Ingeniería del Software}
}
}

% Inclusión del glosario (gracias a David Peñas)
%
%	Plan GCS: Glosario
%

\PrerenderUnicode{ñ}
\PrerenderUnicode{ó}
\PrerenderUnicode{í}


\makeglossaries

\begin{document}

	% Portada
	\portadaace{\doctitle}{2.0}

	\tableofcontents
	\newpage

	\iniciarnumeraciondiedral
	
	\section{Introducción}
		\subsection{Propósito}
		\subsection{Alcance}
		\subsection{Definición de términos clave}
		\subsection{Referencia}
	\section{Gestión de la GCS}
		\subsection{Organización}
		\subsection{Responsabilidades GCS}
		\subsection{Políticas, directivas y procedimientos aplicables}
	\section{Actividades de la GCS}
		\subsection{Identificación de la configuración}
			\subsubsection{Identificación de ECS}
			\subsubsection{Nombrado de ECS}
			\subsubsection{Adquisición de ECS}
		\subsection{Control de configuración}
			\subsubsection{Petición de cambios}
			\subsubsection{Evaluación de cambios}
			\subsubsection{Aprobación o desaprobación de cambios}
			\subsubsection{Implementación de cambios}
		\subsection{Contabilidad de estado de configuración}
		\subsection{Auditorías y revisiones de la configuración}
		\subsection{Control de interfaz}
		\subsection{Control de la subcontratación/compra}
	\section{Planificación de la GCS}
	\section{Recursos de la GCS}
	\section{Mantenimiento del plan de GCS}
	\section{Glosario}
		\printglossaries

	%\newpage
	%\bibliography{gcs}
	%\bibliographystyle{plain}
\end{document}
