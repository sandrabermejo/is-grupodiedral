%
%	Plan de Gestión de Configuración del Software (GCS)
%

\documentclass[11pt, a4paper, twoside, titlepage]{article}
\usepackage[utf8x]{inputenc}
\usepackage[T1]{fontenc}
\usepackage[spanish]{babel}
\usepackage{lmodern}
\usepackage{anysize}
\usepackage[none]{hyphenat}
\usepackage[colorlinks, linkcolor=red]{hyperref}
\usepackage{glossaries}
\usepackage{glossaries-babel}
\usepackage{tikz}
\usepackage{tikz-qtree}
\usepackage{float}
\usepackage[doc=gcs]{isdiedral}


% Nombre del documento (para futuras referencias)
% Obs: el nombre del encabezado no usa \doctitle
\newcommand*{\doctitle}{Plan de gestión de configuración del software}
\newcommand*{\docversion}{2.0}


%%% Configuraciones %%%
\marginsize{2.5cm}{2cm}{2cm}{2cm}

% Usa como familia tipográfica por defecto "Sans"
\renewcommand{\familydefault}{\sfdefault}

% Establece la profundidad hasta la cual se numeran los elementos de sección
\setcounter{secnumdepth}{4}

% Establece la profundidad de niveles de sección que aparece en el TOC
\setcounter{tocdepth}{4}

% Fija que la entrada del glosario no genere una sección
\renewcommand*{\glossarysection}[2][]{%
	\setlength\glsdescwidth{0.6\linewidth}%
	\glossarymark{Glosario}%
}

% Configuración de los encabezados
\encabezadodiedral{Plan GCS \docversion}
\pagestyle{fancy}

\renewcommand*{\thepage}{\sffamily \roman{page}}


% Modelo copiado de los apuntes del tema 8 (páginas 93 a 95) IEEE Std. 730-2002

\title{\doctitle\\\textsl{Airline Common Environment}}
\author{Grupo Diedral}

% Metadatos del pdf
\hypersetup{
pdfinfo={
	Author={Grupo Diedral},
	Title={\doctitle{} \docversion},
	Subject={Airline Common Environment},
	Keywords={SQA;Airline Common Environment;Ingeniería del Software}
}
}

% Inclusión del glosario (gracias a David Peñas)
%
%	Plan GCS: Glosario
%

\PrerenderUnicode{ñ}
\PrerenderUnicode{ó}
\PrerenderUnicode{í}


\makeglossaries

\begin{document}

	% Cita inicial
	\fijacitainicial{Nada es permanente a excepción del cambio}{Heráclito de Éfeso, $\sim$ 500 a.C.}

	% Portada
	\portadaace{\doctitle}{\docversion}

	\tableofcontents
	\newpage

	\iniciarnumeraciondiedral
	
	\section{Introducción} % Natalia
		\subsection{Propósito}
			El propósito del Plan de \gls{config_software} es establecer y mantener la integridad de los productos a desarrollar a través del proceso de software asociado. 
		\subsection{Alcance}
		\begin{itemize}
			\item Se \textit{identificarán y definirán los elementos del sistema}, controlando el posible cambio de estos durante todo su ciclo de vida, y se verificará que sean correctos y completos.
			\item Se establecerá un protocolo de \textit{gestión de cambios} de estos elementos.
			\item Será la base de partida para que asegurar una buena \textit{calidad de software}.
			\item Ayudará a la comunicación y organización en el grupo de desarrollo o con personas ajenas al proyecto.
		\end{itemize}
			
		%\subsection{Definición de términos clave} Esto es el Glosario
		\subsection{Referencia}
			\nocite{IEEE828-1998}
			\nocite{PSMAN}
			\nocite{ESAPSS-05-0}
			Ver sección de {\itshape Referencias} al final del documento.
		
	\section{Gestión de la GCS} % Sandra
		\subsection{Organización}
			En este apartado describimos el contexto organizativo en el cual se aplicará la gestión de configuración de software. Para llevar a cabo dicha gestión hemos definido los siguientes roles:

			\begin{itemize}
				\item Coordinador de Proyecto: Rubén Rafael Rubio Cuéllar.
				\item Responsables de la Configuración: Natalia Angulo Herrera, Sandra Bermejo Cañadas.
				\item Ingenieros: Cristina Alonso Fernández, Juan Andrés Claramunt Pérez.
			\end{itemize}

			Si la situación lo requiriese, se podrían realizar cambios en la estructura descrita con el obetivo de asegurar que todos los productos a desarrollar y sus cambios sean debidamente justificados y de una calidad conveniente.
		\subsection{Responsabilidades GCS}
			A continuación, se establecen las tareas asignadas a cada rol, garantizando así que todas las actividades esenciales se lleven a cabo y además que las realice sólo un integrante del equipo.
			\begin{itemize}
				\item Coordinador de Proyecto
					\begin{itemize}
						\item Controlar la gestión de configuración.
						\item Informar al equipo de desarrollo acerca de los planes de gestión de configuración de software.
						\item Asegurar que los documentos se generen de acuerdo con las políticas establecidas en la siguiente sección.
					\end{itemize}
				\item Responsables de la Configuración
					\begin{itemize}
						\item Organizar y realizar la gestión de configuración.
						\item Mantener el documento de \textit{Gestión de Configuración de Software}.
						\item Realizar informes (si procede).
					\end{itemize}
				\item Ingenieros
					\begin{itemize}
						\item Decidir si un cambio se lleva a cabo y evaluarlo.
						\item Seguir los procedimientos de gestión de configuración que publiquen los Responsables de la Configuración.
						\item Almacenar los cambios realizados.
					\end{itemize}
			\end{itemize}

		\subsection{Políticas, directivas y procedimientos aplicables}
			Se llevarán a cabo las políticas establecidas en el estándar \textit{ESA PSS-05-0} \cite{ESAPSS-05-0} que tengan relación con la gestión de configuración de software. \\

			Además, se incluyen los procedimientos definidos por el coordinador del proyecto y los responsables de la gestión de configuración.

	\section{Actividades de la GCS}
		\subsection{Identificación de la configuración} % Natalia
			En la siguiente sección se va a realizar la identificación, nombrado y adquisición de \gls{ECS}.

			\subsubsection{Casos de Uso}
				\begin{enumerate}
					\item {\itshape \bfseries Descripción.}
						\begin{itemize}
							\item \textit{Tipo de ECS:} documento.
							\item \textit{Identificador proyecto:} \verb|CU|.
							\item \textit{Información de la versión y/o cambio:} es el documento más antiguo del proyecto y ha sufrido más cambios que cualquier otro. Paulatinamente se han ido añadiendo nuevos casos de uso al comprender su necesidad, se han refinado los existentes y se han corregido algunos errores en la exposición de los casos.
						\end{itemize}

					\item {\itshape \bfseries Lista de recursos.}
						El principal recurso de este documento es la información recabada en las conversaciones que el equipo de desarrollo ha venido manteniendo con el cliente y los distintos usuarios. Siguiendo el modelo de casos de uso, las etapas siguientes del proyecto tendrán su base en este documento. Este es el motivo por el que se ha hecho un gran esfuerzo en la gestión de configuración y de calidad del mismo.

				\end{enumerate}

			\subsubsection{Especificación de requisitos \software}
				\begin{enumerate}
					\item {\itshape \bfseries Descripción.}
						\begin{itemize}
							\item \textit{Tipo de ECS:} documento.
							\item \textit{Identificador proyecto:} \verb|SRS|.
							\item \textit{Información de la versión y/o cambio:} los cambios en los casos de uso (documento CU) se reflejado también en este documento. Además, se han corregido errores de redacción y conflictos entre los diferentes epígrafes del documento.
						\end{itemize}

					\item {\itshape \bfseries Lista de recursos.}	
						Principalmente el documento de la {\itshape Especificación de Requisitos Software} tiene como entidades requeridas al documento de {\itshape Casos de Uso} y el {\itshape Prototipo}, es decir, depende directamente de ellos.
				\end{enumerate}

			\subsubsection{Prototipo Gestión Externa}
			\begin{enumerate}
				\item {\itshape \bfseries Descripción.}
						\begin{itemize}
							\item \textit{Tipo de ECS:} programa.
							\item \textit{Identificador proyecto:} \verb|Prototipo Externa|.
							\item \textit{Información de la versión y/o cambio:} no ha habido cambios desde la primera versión entregada.
						\end{itemize}

					\item {\itshape \bfseries Lista de recursos.}
						La principal fuente de referencia del \textit{Prototipo de Gestión Externa} es la \textit{Especificación de Requisitos Software} puesto el prototipo pretende ilustrar las especificaciones recogidas en dicho documento.
				\end{enumerate}

			\subsubsection{Prototipo Gestión Interna}
			\begin{enumerate}
				\item {\itshape \bfseries Descripción.}
						\begin{itemize}
							\item \textit{Tipo de ECS:} programa.
							\item \textit{Identificador proyecto:} \verb|Prototipo Interna|.
							\item \textit{Información de la versión y/o cambio:} modificaciones menores en la visualización del prototipo y cambios en la estructura del código de éste. No se han realizado grandes cambios desde la última entrega por la proximidad que tiene para ser desechado y reemplazado por la implementación final del producto.
						\end{itemize}

					\item {\itshape \bfseries Lista de recursos.}
						\textit{(Ver lista de recursos de Prototipo Gestión Externa)}
				\end{enumerate}

			\subsubsection{Plan de Proyecto}
			\begin{enumerate}
				\item {\itshape \bfseries Descripción.}
						\begin{itemize}
							\item \textit{Tipo de ECS:} documento.
							\item \textit{Identificador proyecto:} \verb|PR|.
							\item \textit{Información de la versión y/o cambio:} se están acometiendo diferentes cambios referentes a la \textit{Estimación de Puntos de Función}. En la última versión entregada se omitió por error el cálculo de los PF totales, lo cual motivó un cambio para recoger tal apartado.
						\end{itemize}

					\item {\itshape \bfseries Lista de recursos.}
						El Plan de Proyecto está compuesto por los siguientes subrecursos internos: el \textit{Plan de Gestión de Riesgo}, la \textit{Estimación} y el \textit{Documento de Planificación temporal}.
				\end{enumerate}

			\subsubsection{Plan de Gestión de Configuración}
			\begin{enumerate}
				\item {\itshape \bfseries Descripción.}
						\begin{itemize}
							\item \textit{Tipo de ECS:} documento.
							\item \textit{Identificador proyecto:} \verb|GCS|.
							\item \textit{Información de la versión y/o cambio:} existe una primera y única versión del documento.
						\end{itemize}

					\item {\itshape \bfseries Lista de recursos.}
						Este documento está ligado al \textit{Plan de Calidad}, pues ambos tiene cometidos comunes relacionados con la supervisión de los diferentes elementos del proyecto.
				\end{enumerate}

			\subsubsection{Plan de Gestión de Calidad}
			\begin{enumerate}
				\item {\itshape \bfseries Descripción.}
						\begin{itemize}
							\item \textit{Tipo de ECS:} documento.
							\item \textit{Identificador proyecto:} \verb|PGC|.
							\item \textit{Información de la versión y/o cambio:} la primera versión de este documento incluye el desarrollo de la política de supervisión y corrección de los elementos del proyecto. Extiende el contenido, que sobre calidad del \software, incluía el {\itshape Plan del Proyecto}. Estará sujeto a continua revisión durante el transcurso del proyecto.
						\end{itemize}

					\item {\itshape \bfseries Lista de recursos.}
						Este documento está ligado al \textit{Plan de Gestión de Configuración} como se afirma en la descripción de éste.
				\end{enumerate}

			\subsubsection{Otros}
				Algunos ECS importantes que no se han desarrollado aún y que por tanto no se van a detallar en este documento son: diagramas de colaboración de análisis, el documento de diseño, el manual de usuario, el plan de pruebas del sistema, documentos de diseño de base de datos, especificaciones de prueba del sistema\ldots
	% ¿Por qué hablamos de lo que no vamos a hacer?
	%N: Es que sino no sabía que poner en Otros, si no borramos este punto!

		\subsection{Contabilidad de estado de configuración} % Sandra

	\section{Recursos de la GCS} % Natalia

		\subparagraph{Herramientas técnicas y \software.} La principal herramienta en la que se va a basar la \textit{Gestión de Configuración del Software} será en el repositorio de control de versiones que sirve de entorno de trabajo a los desarrolladores. El servidor que anteriormente albergaba el repositorio --\textit{RiouxSVN}-- quedó inoperativo, circunstancia prevista en el \textit{Plan de Gestión de Riesgo}\footnote{Véase riesgo T2 `\textit{Repositorio inoperativo}' en el \textit{Plan de Proyecto}.} recién elaborado, por lo que se tomaron las acciones de contingencia necesarias, lo que no permitió salvar el historial de versiones como preveía el plan de prevención. A pesar de ello, no han surgido grandes conflictos de documentos. Así, la organización del nuevo repositorio hospedado en el servidor \textit{Assembla} ha quedado de la siguiente manera: \\
			
			\begin{figure}[H] \centering
				\tikzset{edge from parent/.style={draw, edge from parent path={(\tikzparentnode.south) -- +(0,-8pt) -| (\tikzchildnode)}}}
				\tikzset{sibling distance=20pt}
				\Tree [.{\itshape Repositorio} [.trunk doc presentacion actas util ].trunk [.branches prototipo\_interna prototipo\_externa ].branches ].{\itshape Repositorio}

			\caption{Árbol de estructura del repositorio}
			\end{figure}

			El repositorio cuenta con dos directorios en el directorio raíz \verb|trunk| y \verb|branches|. El directorio \verb|trunk| a su vez se organiza en 4 directorios: \verb|doc| (que alberga toda la documentación del proyecto), \verb|presentacion| (que contiene los documentos elaborados para las exposiciones en clase), \verb|actas| (que incluye las actas generadas en las diferentes actividades del proyecto) y \verb|util| (que recoge herramientas elaboradas internamente como ayuda para el desarrollo del proyecto). En el directorio \verb|branches| residen los prototipos.

			\begin{figure}[H] \centering
				\tikzset{edge from parent/.style={draw, edge from parent path={(\tikzparentnode.south) -- +(0,-8pt) -| (\tikzchildnode)}}}
				\tikzset{sibling distance=15pt}
				\Tree [.doc casosdeuso diagramas plancalidad srs common gcs planproyecto ].doc

			\caption{Árbol de estructura de la documentación}
			\end{figure}

	\section{Glosario}
		\printglossaries

	\newpage
	\bibliography{gcs}
	\bibliographystyle{plain}
\end{document}
