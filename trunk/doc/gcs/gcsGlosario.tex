%
%	Plan GCS: Glosario
%

%\PrerenderUnicode{ñ}
%\PrerenderUnicode{ó}
%\PrerenderUnicode{í}

\newglossaryentry{config_software}{
	name=Gestión de la Configuración del Software,
	description={La Gestión de la Configuración del Software es el conjunto de actividades y procesos necesarios para identificar y definir los elementos de la configuración de un sistema, controlando la entrega y los cambios de estos elementos a través del ciclo de vida del sistema, almacenando el estado de los elementos de la configuración y de las peticiones de cambio, y verificando que se 
cumple respecto a los requisitos especificados},
}

% Revisar esta definición
\newglossaryentry{ECS}{
	name=ECS,
	description={Se denomina \textit{Elemento de Configuración Software} al conjunto de procesos destinados a asegurar la calidad de todo producto obtenido durante cualquiera de las etapas del desarrollo de un Sistema de Información, a través del estricto control de los cambios realizados sobre los mismos y de la disponibilidad constante de una versión estable de cada elemento para toda persona involucrada en el citado desarrollo},
}

\newglossaryentry{DET}{
	name=DET,
	description={Es un campo único (no repetitivo) reconocible por el usuario. Contar un DET por cada campo no repetitivo, reconocible por el usuario, que se recupera o mantiene desde ILF o EIF a través de un proceso.},
}
